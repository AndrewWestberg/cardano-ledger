\documentclass[11pt,a4paper]{article}
\usepackage{amsmath}
\usepackage{amssymb}
\usepackage{extarrows}
\usepackage{float}
\usepackage[margin=2.5cm]{geometry}
\usepackage[unicode=true,pdftex,pdfa]{hyperref}
\usepackage[capitalise,noabbrev,nameinlink]{cleveref}
\usepackage[utf8]{inputenc}
\usepackage{latexsym}
\usepackage{listings}
\usepackage{mathpazo} % nice fonts
\usepackage{mathtools}
\usepackage{microtype}
\usepackage[colon]{natbib}
%%
%% Package `semantic` can be used for writing inference rules.
%%
\usepackage{semantic}
\usepackage{slashed}
\usepackage{stmaryrd}
\usepackage[colorinlistoftodos,prependcaption,textsize=tiny]{todonotes}

\hypersetup{
  pdftitle={Specification of the Blockchain Layer},
  breaklinks=true,
  bookmarks=true,
  colorlinks=false,
  linkcolor={blue},
  citecolor={blue},
  urlcolor={blue},
  linkbordercolor={white},
  citebordercolor={white},
  urlbordercolor={white}
}
\floatstyle{boxed}
\restylefloat{figure}

%% Setup for the semantic package
\setpremisesspace{20pt}

\DeclareMathOperator{\dom}{dom}
\DeclareMathOperator{\range}{range}

%%
\newcommand{\powerset}[1]{\mathbb{P}~#1}
\newcommand\Set[2]{\{\,#1\mid#2\,\}}
\newcommand{\restrictdom}{\lhd}
\newcommand{\subtractdom}{\mathbin{\slashed{\restrictdom}}}
\newcommand{\restrictrange}{\rhd}
\newcommand{\union}{\cup}
\newcommand{\unionoverride}{\mathbin{\underrightarrow\cup}}
\newcommand{\uniondistinct}{\uplus}
\newcommand{\var}[1]{\mathit{#1}}
\newcommand{\fun}[1]{\mathsf{#1}}
\newcommand{\type}[1]{\mathsf{#1}}
\newcommand{\size}[1]{\left| #1 \right|}
\newcommand{\trans}[2]{\xlongrightarrow[\textsc{#1}]{#2}}
\newcommand{\seqof}[1]{#1^{*}}
\newcommand{\nextdef}{\\[1em]}

%%
%% Types
%%
\newcommand{\Hash}{\type{Hash}}  % hashes of various things, including blocks
\newcommand{\Addr}{\type{Addr}}
\newcommand{\Slot}{\type{Slot}}
\newcommand{\BlockIx}{\type{BlockIx}}
\newcommand{\GBlock}{\type{GBlock}}
\newcommand{\RBlock}{\type{RBlock}}
\newcommand{\Block}{\type{Block}}
\newcommand{\HCert}{\type{C}}
\newcommand{\Queue}{\type{Q}}

\newcommand{\SKey}{\type{SKey}}
\newcommand{\SKeyGen}{\type{SKey_G}}
\newcommand{\VKey}{\type{VKey}}
\newcommand{\VKeyGen}{\type{VKey_G}}
\newcommand{\Sig}{\type{Sig}}
\newcommand{\Data}{\type{Data}}
\newcommand{\DelegState}{\type{DSIState}}

\newcommand{\ProtParams}{\type{ProtParams}} % protocol parameters


%%
%% Function and relation names
%%
\newcommand{\hashname}{bHash}
\newcommand{\bsizename}{bSize}
\newcommand{\signname}{sign}
\newcommand{\verifyname}{verify}
\newcommand{\delegationname}{delegates} % the delegation relation
\newcommand{\signmapname}{\mathcal{M}}
\newcommand{\trimixname}{trimIx}
\newcommand{\incixmapname}{incIxMap}

\newcommand{\hashofblockname}{hashBlock}
\newcommand{\blocksizelimitname}{blockSizeLimit}
\newcommand{\maxblocksizename}{maxBlockSize}
\newcommand{\isebbname}{rbIsEBB}
\newcommand{\rbdataname}{rbData}
\newcommand{\rbcertsname}{rbCerts}
\newcommand{\rbsigname}{rbSig}
\newcommand{\rbixname}{rbIx}
\newcommand{\rbslname}{rbSl}
\newcommand{\rbsignername}{rbSigner}

\newcommand{\qrestrname}{qRestrict}
\newcommand{\qpopname}{qPop}
\newcommand{\qheadname}{qHead}
\newcommand{\qpushname}{qPush}

%% 
%% Functions and relations
%%
\newcommand{\sign}[4]{\fun{\signname}\ #1 ~ #2 ~ #3 ~ #4}
\newcommand{\verify}[3]{\fun{\verifyname} ~ #1 ~ #2 ~ #3}
\newcommand{\hash}[1]{\fun{\hashname}\ #1}
\newcommand{\bsize}[1]{\fun{\bsizename} ~ #1}
\newcommand{\delegation}[3]{\fun{\delegationname} ~ #1 ~ #2 ~#3}
\newcommand{\signmap}[1]{\fun{\signmapname} ~ #1}
\newcommand{\qrestr}[2]{\fun{\qrestrname} ~ #1 ~ #2}
\newcommand{\trimix}[2]{\fun{\trimixname} ~ #1 ~ #2}
\newcommand{\incixmap}[3]{\fun{\incixmapname} ~ #1 ~ #2 ~ #3}

\newcommand{\hashofblock}[1]{\fun{\hashofblockname} ~ #1}
\newcommand{\blocksizelimit}[2]{\fun{\blocksizelimitname} ~ #1 ~ #2}
\newcommand{\maxblocksize}[1]{\fun{\maxblocksizename} ~ #1}
\newcommand{\isebb}[1]{\fun{\isebbname} ~ #1}
\newcommand{\rbdata}[1]{\fun{\rbdataname} ~ #1}
\newcommand{\rbcerts}[1]{\fun{\rbcertsname} ~ #1}
\newcommand{\rbsig}[1]{\fun{\rbsigname} ~ #1}
\newcommand{\rbix}[1]{\fun{\rbixname} ~ #1}
\newcommand{\rbsl}[1]{\fun{\rbslname} ~ #1}
\newcommand{\rbsigner}[1]{\fun{\rbsignername} ~ #1}

\newcommand{\qpop}[1]{\fun{\qpopname} ~ #1}
\newcommand{\qhead}[1]{\fun{\qheadname} ~ #1}
\newcommand{\qpush}[1]{\fun{\qpushname} ~ #1}

% Partial and total function aliases
\newcommand{\totalf}{\to}
\newcommand{\partialf}{\mapsto}

% A type alias for a map from a genesis block verification key to a queue of indices
\newcommand{\mapqueue}{\mathcal{Q}}
% comments
\newcommand{\marko}[1]{\todo[size=\small, color=yellow!40, inline]{Marko: #1}}

\begin{document}


\title{Specification of the Blockchain Layer}

\author{Marko Dimjašević}

\date{November 22, 2018}

\maketitle

\begin{abstract}
  This documents defines inference rules for operations on a blockchain as a
  specification of the blockchain layer of Cardano in the Byron release and in
  a transition to the Shelley release.
  %
  In particular, a block validity definition is given, which is accompanied by
  small-step operational semantics inference rules.
\end{abstract}

\tableofcontents
\listoffigures

\section{Introduction}
\label{sec:introduction}

The idea behind this document is to formalise what it means for a new block to
be added to the blockchain to be valid.
%
The scope of the document is the Byron release and a transition phase to the
Shelley release of the Cardano blockchain platform.


Unless a new block is valid, it cannot be added to the blockchain and thereby
extend it.
%
This is needed for a system that is subscribed to the blockchain and keeps a
copy of it locally.
%
In particular, this document gives a formalisation that should be
straightforward to implement in a programming language, e.g., in Haskell.


\section{Preliminaries}
\label{sec:preliminaries}

\begin{description}
\item[Powerset] Given a set $\type{X}$, $\powerset{\type{X}}$ is the set of all
  the subsets of $X$.
\item[Sequence] Given a set $\type{X}$, $\seqof{\type{X}}$ is a sequence
  having elements taken from $\type{X}$.
  %
  The empty sequence is denoted by $\epsilon$, and given a sequence $\Lambda$,
  $\Lambda; x$ is the sequence that results from appending
  $x \in \type{X}$ to $\Lambda$.
  %
  Furthermore, $\epsilon$ is an identity element for sequence joining:
  $\epsilon; x = x; \epsilon = x$.
\item[Option type] An option type in type $A$ is denoted as $A^? = A + 1$. The
  $A$ case corresponds to a case when there is a value of type $A$ and the $1$
  case corresponds to a case when there is no value.
\item[Union override] The union override operation is defined in
  Figure~\ref{fig:unionoverride}.
  %
  \begin{figure}
    \begin{align*}
      \var{K} \restrictdom \var{M}
      & = \{ i \mapsto o \mid i \mapsto o \in \var{M}, ~ i \in \var{K} \}
      & \text{domain restriction}
      \\
      \var{K} \subtractdom \var{M}
      & = \{ i \mapsto o \mid i \mapsto o \in \var{M}, ~ i \notin \var{K} \}
      & \text{domain exclusion}
      \\
      \var{M} \restrictrange \var{V}
      & = \{ i \mapsto o \mid i \mapsto o \in \var{M}, ~ o \in \var{V} \}
      & \text{range restriction}
      \\
      & \unionoverride \in (A \mapsto B) \to (A \mapsto B) \to (A \mapsto B)
      & \text{union override}\\
      & d_0 \unionoverride d_1 = d_1 \cup (\dom d_1 \subtractdom d_0)
    \end{align*}
    \caption{Definition of the Union Override Operation}
    \label{fig:unionoverride}
  \end{figure}
\end{description}

\section{Basic definitions}
\label{sec:basic-definitions}

This section gives definitions of basic types and functions that operate on
and with blocks and heavyweight delegation certificates.


\subsection{Sets}
\label{sec:sets}

There are several standard sets used in the document:
%
\begin{description}
\item[Booleans] The set of booleans is denoted with $\mathbb{B}$ and has two
  values, $\mathbb{B} = \{\bot, \top\}$.
\item[Natural numbers] The set of natural numbers is denoted with
  $\mathbb{N}$ and defined as $\mathbb{N} = \{0, 1, 2, \dots\}$.
\end{description}

\subsection{Heavyweight Delegation Certificates}
\label{sec:certificates}

A heavyweight delegation certificate is a proof that a genesis block
verification key $vk_s$ is delegating its right to sign a block to a signing
key $sk_d$, which has a corresponding verification key $vk_d$, starting with a
slot $sl$, which is confirmed by a cryptographic signature $\sigma$.
%
The type for a heavyweight delegation certificate is kept abstract in this
document, but it is a product type with projection functions for the
following:
%
\begin{itemize}
\item a delegator key,
\item a delegatee key,
\item a slot, and
\item a cryptographic signature.
\end{itemize}

Basic types and functions pertaining to heavyweight delegation certificates
are given in Figure~\ref{fig:cert-defs}.

\begin{figure}
  \emph{Primitive types}
  %
  \begin{align*}
    sl & \in \Slot    & \text{slot time-stamp}\\
   sig & \in \Sig     & \text{cryptographic signature}\\
    sk & \in \SKey    & \text{signing key}\\
    vk & \in \VKey    & \text{verification key}
  \end{align*}
  %
  \emph{Derived types}
  %
  \begin{align*}
    vk_s & \in \VKeyGen \subseteq \VKey & \text{genesis block's verification key}
  \end{align*}
  %
  \emph{Abstract types}
  %
  \begin{align*}
    c & \in \HCert  & \text{heavyweight delegation certificate}
  \end{align*}
  %
  \emph{Functions}
  %
  \begin{align*}
    \fun{cFrom} & \in \HCert \totalf \VKeyGen & \text{certificate delegator key projection} \\
    \fun{cTo} & \in \HCert \totalf \VKey & \text{certificate delegatee key projection} \\
    \fun{cSl} & \in \HCert \totalf \Slot & \text{certificate slot projection} \\
    \fun{cSig} & \in \HCert \totalf \Sig & \text{certificate signature projection}
  \end{align*}
  %
  \caption{Basic Heavyweight Delegation Certificate-related Types and Functions}
  \label{fig:cert-defs}
\end{figure}


\subsection{Blocks}
\label{sec:blocks}

A block is an element of a blockchain.
%
A blockchain is extended by appending a block.
%
There are two types of blocks.
%
The first type is the genesis block, denoted by $\GBlock$ in this document.
%
Every blockchain starts with a genesis block.
%
A genesis block contains a set of $n$ verification keys.
%
These keys are often referred to as the genesis block verification keys.
%
A genesis block also contains a hash.
%
In this document $\GBlock$ is kept abstract, but it is a product type with
projection functions for the two just mentioned values:
%
\begin{itemize}
\item a set of $n$ verification keys, and
\item a block hash.
\end{itemize}

The second type is for a non-genesis block, denoted by $\RBlock$.
%
In this document $\RBlock$ is kept abstract, but it is a product type with
projection functions for the following:
%
\begin{itemize}
\item a block hash pointing to the block's predecessor,
\item a block index,
\item a verification key of the block signer,
\item a set of heavyweight delegation certificates in the body,
\item a slot it is in,
\item data that is the block body,
\item a cryptographic signature, and
\item a flag indicating if this is an epoch boundary block.
\end{itemize}


For every block, it is possible to compute its hash with the
$\fun{\hashofblockname}$ function.
%
In the case of the genesis block, this simply returns the hash field in the
block.
%
Furthermore, note that the $\fun{\maxblocksizename}$ parameter of a
$\ProtParams$ is the maximum size only of blocks that are not epoch boundary
blocks.
%
Basic types and functions pertaining to blocks are given in
Figure~\ref{fig:block-defs}.

\begin{figure}
  \emph{Primitive types}
  %
  \begin{align*}
    ix & \in \BlockIx & \text{block index}\\
    h  & \in \Hash   & \text{hash}\\
    pp & \in \ProtParams & \text{protocol parameters}
  \end{align*}
  %
  \emph{Abstract types}
  %
  \begin{align*}
    g & \in \GBlock & \text{genesis block} \\
    r & \in \RBlock & \text{non-genesis block}
  \end{align*}
  %
  \emph{Derived types}
  %
  \begin{align*}
    b & \in \Block = \GBlock + \RBlock & \text{block}
  \end{align*}
  %
  \emph{Projection functions}
  %
  \begin{align*}
    \fun{gbKeys} & \in \GBlock \totalf \VKeyGen^n & \text{block keys} \\
    \fun{\rbixname} & \in \RBlock \totalf \BlockIx & \text{block index} \\
    \fun{\hashname} & \in \Block \totalf \Hash
      & \text{previous block's hash} \\
    \fun{\bsizename} & \in \Block \totalf \mathbb{N} & \text{block size in bytes} \\
    \fun{\rbslname} & \in \RBlock \totalf \Slot & \text{block slot} \\
    \fun{\rbsignername} & \in \RBlock \totalf \VKey & \text{block signer} \\
    \fun{\rbcertsname} & \in \RBlock \totalf \powerset{\HCert}
      & \text{block certificates} \\
    \fun{\rbsigname} & \in \RBlock \totalf \Sig & \text{block signature} \\
    \fun{\rbdataname} & \in \RBlock \totalf \Data & \text{block data} \\
    \fun{\isebbname} & \in \RBlock \totalf \mathbb{B} & \text{epoch boundary block check}\\
    \fun{\maxblocksizename} & \in \ProtParams \totalf \mathbb{N} & \text{maximum block size}
  \end{align*}
  %
  \emph{Functions}
  %
  \begin{align*}
    \fun{\hashofblockname} & \in \Block \totalf \Hash & \text{hash of block}
  \end{align*}
  %
  \caption{Basic Block-related Types and Functions}
  \label{fig:block-defs}
\end{figure}


\section{Delegation Interface}
\label{sec:del-interface}

The blockchain layer and the ledger layer need to communicate delegation
certificates for the purpose of signing new blocks.
%
To enable this communication, there is a delegation interface between the two
layers.
%
The interface communicates the minimum of information needed to enable
delegation.


Given that delegation certificates are in the body of a block just like
transactions are, the ledger layer keeps track of the state of certificates.
%
It does so by keeping a set of certificates that are currently active.
%
Furthermore, the ledger layer also needs to maintain a set of delegation
certificates that have been added to the blockchain, but that will become
active and that possibly inactivate other certificates in the future.
%
As part of the delegation interface, the ledger layer exposes functionality
for checking if a particular verification key $vk_d$ has the delegation right
to sign a block in a given slot $sl$.
%
In particular, if the verification key has the right to sign, the
functionality should inform who is the delegator $vk_s$.
%
The functionality can be modeled as a relation
$\fun{\delegationname} \subseteq \VKeyGen \times \VKey \times \Slot$.
%
Note that in the case when $vk_d \in \VKeyGen$, i.e., when the verification
key is a genesis block verification key, and it has not delegated its signing
right to any other key, the element from the relation is $(vk_d, vk_d, sl)$.


New delegation certificates are added to the blockchain in blocks, so the
blockchain layer has to inform the ledger layer of new certificates when a new
block is to be added to the blockchain.
%
Therefore, as part of the delegation interface, the blockchain layer exposes
the new delegation certificates functionality to the ledger layer.
%
This is modeled as an abstract state transition updating the state of type
$\DelegState$.
%
Given that the blockchain layer does not keep the state of delegation
certificates, it has no way of checking if certificates are valid.
%
That is why new certificates can be added to the state only if there are no
conflicts with existing certificates.
%
For example, the blockchain layer cannot know if the delegator already posted
another delegation certificate in the current epoch.
%
Furthermore, the slot in which a certificate is claimed to become active might
not be in K slots, which should disallow such a certificate.
%
In case a certificate is such that the state cannot be updated, the block to
be added to the blockchain is discarded.


\section{State transitions for Blockchain}
\label{sec:state-trans-chain}

\subsection{Block Validity}
\label{sec:block-valid}
% This definition is adopted from shelley-plan.md, which is written by
% Duncan Coutts

A non-genesis block $\var{b}$ is valid if:
%
\begin{enumerate}
\item it is signed by a signing key $sk_d$ for which a valid heavyweight
  delegation certificate $c$ that is active at the time of signing exists.
  %
  The certificate $c$ carries information that signing rights from a genesis
  block verification key $vk_s$ are transferred to a verification key $vk_d$
  starting in a slot $sl$,
\item the corresponding certificate $c$ is signed by a key $sk_s$ that has a
  corresponding verification key in the genesis block.
  %
  The certificate therefore carries a a cryptographic signature $\sigma'$,
  and
\item in the rolling window of the last $K$ blocks, the number of blocks
  signed by keys that $sk_s$ delegated to is no more than a threshold
  $K \cdot t$, where $t$ is a constant\footnote{This is not the same $t$ from
    the Ouroboros BFT paper draft where it denotes the number of Byzantine
    servers, though it should be somehow related to it.}  that will be picked
  in the range $1/5 \leq t \leq 1/4$.
\end{enumerate}

Relationship between the $\text{\signname}$ function and the
$\text{\verifyname}$ relation is given by \eqref{eq:sign-verify}:

\begin{equation}
  \label{eq:sign-verify}
  \forall (sk, vk) \in (\SKey \times \VKey), b = (h, sl, d, \sigma) \in \Block.\
  \verify{vk}{d}{\sigma} \iff \sign{sk}{h}{sl}{d} = \sigma.
\end{equation}


\subsection{Counting Signed Blocks}
\label{sec:counting-signed-blocks}

To make sure no signing key with a corresponding genesis block verification
key gets to sign too many blocks, a state is maintained by keeping a mapping
of keys to a queue of signed blocks in a sliding window of at most K last
blocks in the blockchain.
%
Initially, when the blockchain is made of the genesis block $g_0 \in \GBlock$
only, the initial state is such that no key from the genesis block signed a
block (neither directly or indirectly via delegation) and the last added block
is the genesis block $p = g_0$.
%
The delegation state $ds$ of type $\DelegState$ is kept abstract in this
document;
%
it is defined in the ledger layer specification.
%
The initial state is given in Figure~\ref{fig:init-bc-state}.

\begin{figure}
  \begin{equation*}
    \inference[Initial-state]
    {}
    {\left(
        {\begin{array}{c}
           \signmapname = \emptyset \\
           p = g_0 \\
           ds
         \end{array}}
     \right)
   }
 \end{equation*}
 \caption{The Initial State Rule}
 \label{fig:init-bc-state}
\end{figure}

The types of components in the initial state, namely of $\fun{\signmapname}$
and $ds$ are given in Figure~\ref{fig:block-ext-types-funs}.

Once a new block arrives, the state has to be updated.
%
In particular, if a block with an index $ix$ signed by a signing key
corresponding to a verification key $vk_d$ is to be added to the blockchain,
the sliding window of the last $K$ blocks moves by one block and the map
$\fun{\signmapname}$ has to be updated for the key that signed the
$K^{\text{th}}$ previous block;
%
this is done with $\trimix{\signmapname}{ix}$, as given by
\eqref{eq:trimix}.
%
To finish updating the map, the updated version of $\signmapname$ denoted by
$\signmapname'$ has to push $ix$ to the queue for $vk_s$, which is achieved
with $\incixmap{ix}{vk_s}{\signmapname'}$, as given by
\eqref{eq:incixmap}.
%
Finally, the state $ds$ that is maintained by the ledger layer, is updated;
the details of how the state is updated via a transition is given in a
separate ledger layer specification.
%
Types of both of these functions are given in
Figure~\ref{fig:block-ext-types-funs}.


\begin{align}
  \label{eq:trimix}
  \trimix{\signmapname}{ix} = \Set{(vk_s \partialf q)}{vk_s \in \dom \signmapname.~
  q = \qrestr{ix}{(\signmap{vk_s}})} \\
  \qrestr{ix}{q} = \
  \begin{cases}
    \qpop{q} & \text{if } \size{q} > 0 \wedge \qhead{q} + K < ix \\
    q & \text{otherwise}
  \end{cases}
\end{align}

\begin{equation}
  \label{eq:incixmap}
  \incixmap{ix}{vk_s}{\signmapname} = \signmapname \unionoverride \{vk_s \partialf \qpush{ix}{(\signmap{vk_s})}\}
\end{equation}


\begin{figure}
  \emph{Abstract types}
  %
  \begin{align*}
    q  & \in \Queue_\BlockIx  & \text{block index queue}\\
  data & \in \Data    & \text{data}\\
    ds & \in \DelegState & \text{ledger layer delegation state}
  \end{align*}
  %
  \emph{Derived types}
  %
  \begin{align*}
    \signmapname & \in \VKeyGen \totalf \Queue_\BlockIx & \text{key to block index map}\\
    sk_s & \in \SKeyGen \subseteq \SKey & \text{genesis block's signing key}\\
  \end{align*}
  %
  \emph{Type aliases}
  %
  \begin{align*}
    \mapqueue = \VKeyGen \totalf \Queue_\BlockIx
  \end{align*}
  \emph{Functions and relations}
  %
  \begin{align*}
    \fun{\qheadname} & \in \Queue_\BlockIx \totalf \BlockIx^? & \text{head of queue function} \\
    \fun{\qpushname} & \in \BlockIx \times \Queue_\BlockIx \totalf \Queue_\BlockIx
      & \text{queue push function} \\
    \fun{\qpopname} & \in \Queue_\BlockIx \totalf {\Queue_\BlockIx}^?
      & \text{queue pop function} \\
    \fun{\qrestrname} & \in \BlockIx \times \Queue_\BlockIx \totalf \Queue_\BlockIx
      & \text{restricted queue pop function} \\
    \fun{\trimixname} & \in \mapqueue \times \BlockIx \totalf \mapqueue
      & \text{old block removal function} \\
    \fun{\incixmapname} & \in \BlockIx \times \VKeyGen \times \mapqueue \totalf \mapqueue
      & \text{block count increment function}\\
    \fun{\signname} & \in \SKey \times \Hash \times \Slot \times \Data \partialf \Sig
      & \text{signature function}\\
    \fun{\verifyname} & \in \VKey \times \Data \times \Sig
      & \text{verification relation}\\
    \fun{\delegationname} & \in \DelegState \times \VKeyGen \times \VKey
      & \text{delegation relation}\\
    \fun{\blocksizelimitname} & \in \Block \times \ProtParams \totalf \mathbb{N}
      & \text{block size limit in bytes function}
  \end{align*}
  %
  \caption{Blockchain Extension Types and Functions}
  \label{fig:block-ext-types-funs}
\end{figure}


\subsection{Blockchain Extension}
\label{sec:chain-extension}

A new block $b \in \RBlock$ with an index $ix \in BlockIx$, signed by a
signing key $sk_d$ in a slot $sl$ can extend a blockchain if the following
conditions are met:
%
\begin{enumerate}
\item $sk_d$ has a corresponding verification key $vk_d$ that has delegation
  rights for the slot $sl$,
\item new delegation certificates that are in the body of $b$ are not in
  conflict with earlier delegation certificates, and
\item the size of the block in bytes is within limits, as defined by
  \eqref{eq:blocksizelimit}.
\end{enumerate}

\begin{align}
  \label{eq:blocksizelimit}
  \blocksizelimit{b}{pp} = \
  \begin{cases}
    \maxblocksize{pp} & \text{if $b$ is the genesis block or } \isebb{b} \equiv \bot \\
    2^{21} & \text{otherwise}
  \end{cases}
\end{align}

A rule capturing the blockchain extension is given in
Figure~\ref{fig:blockchain-extension}.

\begin{figure}
  \begin{equation*}
  \inference[Blockchain-extension]
  {
    {\begin{array}{c}
      ix = \rbix{b} \wedge vk_d = \rbsigner{b} \wedge sl = \rbsl{b} \\
      \hash{b} \equiv \hashofblock{p} \\
      \bsize{b} \leq \blocksizelimit{b}{pp} \\
      \delegation{ds}{vk_s}{vk_d} \\
      \verify{vk_d}{(\rbdata{b})}{(\rbsig{b})} \\
      \size{\signmap{vk_s}} \leq K \cdot t \\
      \signmapname' = \incixmap{ix}{vk_s}{(\trimix{\signmapname}{ix})} \\
      p' = b 
     \end{array}}
   & (sl) \vdash ds \trans{\text{DELEG}}{\rbcerts{b}} ds'
  }
  {
    \left(
      {\begin{array}{c}
         pp \\
         sl \\
         K \\
         t
       \end{array}}
    \right)
    \vdash
    \left(
      {\begin{array}{c}
         \signmapname \\
         p \\
         ds
       \end{array}}
    \right)
    \trans{}{b}
    \left(
      {\begin{array}{c}
         \signmapname' \\
         p' \\
         ds'
       \end{array}}
    \right)
 }
  \end{equation*}
  \caption{The Blockchain Extension Rule}
  \label{fig:blockchain-extension}
\end{figure}

\end{document}


%%% Local Variables:
%%% mode: latex
%%% TeX-master: t
%%% End:

\section{UTxO}
\label{sec:utxo}


\begin{figure*}[htb]
  \emph{Derived types}
  %
  \begin{equation*}
    \begin{array}{r@{~\in~}l@{\qquad=\qquad}lr}
      \var{uin}
      & \UTxOIn
      & \TxId \times \Ix
%      & \text{transaction output preference}
      \\
      \var{uout}
      & \UTxOOut
      & (\TxOutND \uniondistinct \TxOutP) \times \Slot
%      & \text{transaction outputs}
      \\
      \var{utxo}
      & \UTxO
      & \UTxOIn \mapsto \UTxOOut
%      & \text{unspent tx outputs}
      \\
      \var{cur}
      & \ScriptPurpose
      & \PolicyID \uniondistinct \UTxOIn \uniondistinct \AddrRWD \uniondistinct \DCert
%      & \text{item the script is validated for}
    \end{array}
  \end{equation*}
  \caption{Definitions used in the UTxO transition system}
  \label{fig:defs:utxo-shelley-1}
\end{figure*}


We make a number of changes to the Shelley UTxO model~\ref{XX} to support the Goguen Extended UTxO model
(see Figure~\ref{fig:defs:utxo-shelley-1}).

\begin{itemize}
\item
  $\UTxO$ entries are stored in the finite map $\UTxOIn\mapsto \UTxOOut$.

\item
  $\UTxOIn$ is the same type as $\TxIn$ in Shelley, but we have changed
  the name because the types of transaction inputs and UTxO keys
  differ in Goguen.

\item
  $\UTxOOut$ is the type of UTxO entries.
  Note that, like in the case of the type of transaction inputs,
  this type differs from the type of the output of a transaction.
  Goguen UTxO entries include a slot number for each output
  that indicates when the output was created.
  This will be used for future functionality.

\item
  $\ScriptPurpose$ is the type of the items that scripts can validate.

\end{itemize}

\subsection{UTxO Transitions}
\label{sec:utxo-trans}

We have added several functions that deal with transaction and UTxO inputs and
outputs as shown in Figure \ref{fig:functions:insouts}. These are used in the definition of the UTxO transition system.

\begin{itemize}
  \item The function $\fun{txinputs_{vf}}$ returns only those transaction inputs
    that were selected to pay transaction fees (we call these ``fee-marked'' inputs).
    These inputs may only contain Ada.
    \begin{note}
      Make it more obvious that we check that the fee inputs only contain Ada, maybe mention it elsewhere. Also, what does vf stand for?
    \end{note}
  \item The predicate $\fun{feesOK}$ checks whether the transaction is
  paying the necessary fees, and that it does it correctly. That is, it checks that:
  \begin{enumerate}[label=({\roman*})]
    \item the fee-marked inputs are not locked by non-native scripts;
    \item all the fee-marked inputs contain strictly Ada and no other kinds of token;
    \item the fee-marked inputs are sufficient to cover the fee amount that is stated
    in the transaction; and
    \item the fee amount that the transaction states it is paying suffices to cover
    the minimum fee that the transaction is obligated to pay.
  \end{enumerate}
  \item The function $\fun{getOut}$ selects the data from a transaction output that
  will be stored in the UTxO, i.e. a $\UTxOOut$ without the slot number.
  \item The function $\fun{txins}$ returns the UTxO keys of transaction inputs.
\end{itemize}

Note that when creating a transaction, the wallet is responsible for
determining the fees. Thus, it also has to execute the non-native scripts
and include the fees for their execution.
\begin{figure}[htb]
  \begin{align*}
    & \fun{txinputs_{vf}} \in \TxBody \to \powerset{\TxId \times \Ix} \\
    & \fun{txinputs_{vf}} ~txb~= \\
    &~~\{ (txid,ix)~\vert~(txid,ix,\var{isfee}) \in
    \fun{txinputs} ~txb,~
     \var{isfee} = \True\}
    \nextdef
    & \fun{feesOK} \in \N \to \PParams \to \GoguenTx \to \UTxO \to \Bool  \\
    & \fun{feesOK} ~n~\var{pp}~tx~utxo~= \\
    &~~\fun{range}~(\fun{txinputs_{vf}}~{txb} \restrictdom \var{utxo}) \subseteq \TxOutND ~ \\
    &~~\wedge~ \var{balance} \in \Coin \\
    &~~      \wedge~ \var{balance} \geq \txfee{txb} ~ \\
    &~~      \wedge~ \minfee~n~{pp}~{tx} \leq \txfee{txb} \\
    &~~      \where \\
    & ~~~~~~~ \var{txb}~=~\txbody{tx} \\
    & ~~~~~~~ \var{balance}~=~\fun{ubalance}~(\fun{txinputs_{vf}}~{txb} \restrictdom \var{utxo})
    \nextdef
    & \fun{getOut} \in \TxOut \to \TxOutND \uniondistinct \TxOutP \\
    & \fun{getOut} ~{txout}~= \begin{cases}
         \var{txout}  & \text{if~} \var{txout} \in \TxOutND \\
              (\fun{getAddr}~\var{txout}, \fun{getValue}~\var{txout},
              \fun{getDataHash}~\var{txout}) & \text{otherwise}
            \end{cases}
    \nextdef
    & \fun{txins} \in \TxBody \to \powerset{\TxId \times \Ix} \\
    & \fun{txins} ~\var{txb} = \{(txid,ix) \mid ((txid,ix),\wcard)\in\fun{txinputs} ~txb\}
  \end{align*}
  \caption{Functions on Tx Inputs and Outputs.}
  \label{fig:functions:insouts}
\end{figure}
%
Figure~\ref{fig:functions:utxo} defines the functions that are needed for the UTxO transition system.
The changes that are needed for Plutus integration are:

\begin{itemize}

  \item The $\fun{getCoin}$ function sums all the Ada in a given output and returns it as a
  $\Coin$ value.

  \item The function $\fun{outs}$ discards the $\HasDV$ tag from a
  transaction output and adds the slot number of the block in which the transaction is
  included.

  \item $\fun{txscriptfee}$ calculates the fee that a transaction must pay for script
  execution based on the amount of $\ExUnits$ it has budgeted, and the prices in the current protocol parameters
  for each component of $\ExUnits$.

  \item The minimum fee calculation, $\fun{minfee}$, includes the script
  fees that the transaction is obligated to pay in order to run its scripts.

  \item The $\fun{produced}$ calculation requires the current slot number as an argument -- this is
  needed to construct the correct UTxO outputs.
\end{itemize}

Since the $\Tx$ type combines both $\ShelleyTx$ and $\GoguenTx$, and
there was already a way of computing an ID from a Shelley transaction, there
is potential for confusion how the ID of a transaction is
computed. Here, $\TxId$ is always computed from values of type $\Tx$,
\emph{and never from the underlying $\ShelleyTx$ or $\GoguenTx$}.  That is, there is a \emph{canonical} ID for each Goguen transaction,
and this is not necessarily the same as the corresponding ID in the underlying Shelley or Goguen transaction type.

\begin{figure}[htb]
  \emph{Helper Functions}
  \begin{align*}
    & \fun{getCoin} \in \UTxOOut \to \Coin \\
    & \fun{getCoin}~{\var{out}} ~=~\sum_{\mathsf{adaID} \mapsto tkns \in \fun{getValue}~out}
       (\sum_{q \in \range~{tkns}} \fun{co}~q)
  \end{align*}
  %
  \emph{Main Calculations}
  \begin{align*}
    & \fun{outs} \in \Slot \to \TxBody \to \UTxO \\
    & \fun{outs} ~ \var{slot}~\var{txb} =
        \left\{
          (\fun{txid} ~ \var{txb}, \var{ix}) \mapsto (\fun{getOut}~\var{txout},\var{slot}) ~
          \middle|~
          \var{ix} \mapsto \var{txout} \in \txouts{txb}
        \right\}
    \nextdef
    & \fun{txscriptfee} \in \N \to \Prices \to \ExUnits \to \Coin \\
    & \fun{txscriptfee}~n~ (\var{pr_{init}, pr_{mem}, pr_{steps}})~ (\var{mem, steps})
    = \var{pr_{init}}*n + \var{pr_{mem}}*\var{mem} + \var{pr_{steps}}*\var{steps}
    \nextdef
    &\fun{minfee} \in \N \to \PParams \to \GoguenTx \to \Coin \\
    &\fun{minfee}  ~n~\var{pp}~\var{tx} = \\
    &~~(\fun{a}~\var{pp}) \cdot \fun{txSize}~\var{tx} + (\fun{b}~\var{pp}) +
    \fun{txscriptfee}~n~(\fun{prices}~{pp})~(\fun{txexunits}~(\fun{txbody}~{tx}))
    \nextdef
    & \fun{produced} \in \Slot \to \PParams \to \StakePools \to \TxBody \to \Value \\
    & \fun{produced}~\var{slot}~\var{pp}~\var{stpools}~\var{txb} = \\
    &~~\ubalance{(\outs{slot}~{txb})} + \fun{coinToValue}(\txfee{txb} + \deposits{pp}{stpools}~{(\txcerts{txb})})
  \end{align*}
  \caption{Functions used in UTxO rules}
  \label{fig:functions:utxo}
\end{figure}

\subsection{Combining Scripts with Their Inputs}
\label{sec:scripts-inputs}

Figure~\ref{fig:functions:script1} shows the helper functions that are needed to
retrieve all the data that is relevant to Plutus script validation.
These include:

\begin{itemize}
\item
  $\fun{indexof}$ finds the index of a given certificate, value, input, or
  withdrawal in a list, finite map, or set of such objects.
  This function assumes there is some ordering on each of these structures.
%  This function is abstract because it assumes there is some ordering rather
%  than giving it explicitly.
  \begin{note}
    $\fun{indexof}$ might need an actual implementation. Also, some
    restructuring in related functions might make it easier.
    Sets and maps don't really have indexes, of course.
  \end{note}
\item
  $\fun{indexedScripts}$ and $\fun{indexedDats}$ create finite maps from sets of the corresponding object.
  The finite maps are indexed by the hashes of the objects that they contain.
%
%  wherein, respectively, all the scripts
%  and datums that a transaction carries as sets, are indexed by their hashes.
\item
  $\fun{findRdmr}$ finds the redeemer in a Goguen transaction
   that corresponds to a given item in the indexed redeemer structure, if it exists.
\end{itemize}


\begin{figure}[htb]
  %
  \emph{Abstract functions}
  \begin{align*}
    &\fun{indexof} \in \DCert \to \seqof{\DCert} \to \Ix\\
    &\fun{indexof} \in \AddrRWD \to \Wdrl \to \Ix\\
    &\fun{indexof} \in \UTxOIn \to \powerset{\TxIn} \to \Ix\\
    &\fun{indexof} \in \PolicyID \to \Value \to \Ix\\
    & \text{get the index of an item in an ordered representation}
  \end{align*}
  %
  \emph{Helper functions}
  \begin{align*}
    &\fun{indexedScripts} \in \GoguenTx \to (\ScriptHash \mapsto \Script) \\
    &\fun{indexedScripts}~{tx} ~=~ \{ h \mapsto s ~\vert~ \fun{hashScript}~{s}~=~h,
     s\in~\fun{txscripts}~(\fun{txwits}~{tx})\}
    \nextdef
    &\fun{indexedDats} \in \GoguenTx \to (\DataHash \mapsto \Data)\\
    &\fun{indexedDats}~{tx} ~=~ \{ h \mapsto d ~\vert~ \fun{hashData}~{d}~=~h,
     d\in~\fun{txdats}~(\fun{txwits}~{tx})\}
    \nextdef
    &\fun{toRdmrPtr} \in \GoguenTx \to \ScriptPurpose \to \RdmrPtr \\
    &\fun{toRdmrPtr}~{tx}~{it} ~=~
      \begin{cases}
        (\mathsf{certTag},\fun{indexof}~\var{it}~(\fun{txcerts}~{txb}))   & \var{it}~\in~\DCert \\
        (\mathsf{wdrlTag},\fun{indexof}~\var{it}~(\fun{txwdrls}~{txb}))   & \var{it}~\in~\AddrRWD \\
        (\mathsf{forgeTag},\fun{indexof}~\var{it}~(\fun{forge}~{txb}))    & \var{it}~\in~\PolicyID \\
        (\mathsf{inputTag},\fun{indexof}~\var{it}~(\fun{txinputs}~{txb})) & \var{it}~\in~\UTxOIn
      \end{cases} \\
    & ~~\where \\
    & ~~~~~~~ \var{txb}~=~\txbody{tx}
    \nextdef
    &\fun{findRdmr} \in \GoguenTx \to (\ScriptPurpose \mapsto \Data)\\
    & \fun{findRdmr}~{tx} ~=~ \{ \var{it} \mapsto \var{d} ~|~
      \var{it} \in \ScriptPurpose,~ \fun{toRdmrPtr}~{tx}~{it} \mapsto \var{d} \in \fun{txrdmrs}~(\fun{txwits}~{tx}) \}
  \end{align*}
  \caption{Combining Script Validators and their Inputs}
  \label{fig:functions:script1}
\end{figure}


\textbf{Plutus Script Validation}
Figure~\ref{fig:defs:functions-valid} shows the abstract functions that are used for script validation.

\begin{itemize}
\item
  $\fun{valContext}$ constructs the validation context.
  This includes all the necessary transaction and chain state data that needs to be passed to the script interpreter.
    It has a $\UTxO$ as its argument to recover the full information of the inputs of the transaction,
    but only the inputs of the transaction are provided to scripts.
\item
    $\fun{hashScript},~ \fun{hashData}$ are abstract hashing functions.
\item
  $\fun{runMSigScript}$ validates multi-signature scripts, exactly as in the Shelley ledger specification (where it is called $\fun{evaluateScript}$).
\item
  $\fun{runPLCScript}$ validates Plutus scripts. It takes the following
  arguments:
  \begin{itemize}
  \item A cost model, that is used to calculate the $\ExUnits$ that are needed for script execution;
  \item A list of terms of type $\Data$ that will be passed to the script; %is given access to.
  \item The execution unit budget.
  \end{itemize}
  It outputs a pair of the validation result (as a Boolean)
  and the remaining execution units (subtracting those that are used for script execution).
  Note that script execution stops if the full budget has been spent before validation is complete.
\end{itemize}
\begin{note}
  Maybe rename runPLCScript.
\end{note}

\begin{note}
  \textbf{Know your contract arguments}
  A Plutus validator script may receive either a list of three terms of type $\Data$, in case it validates the spending of script outputs
  or two terms (redeemer and context, with no datum), for all other uses.
  Script authors must keep this in mind when writing scripts, since the ledger call to the interpreter is oblivious to what
  arguments are required.
\end{note}

\begin{figure*}[htb]
  \emph{Abstract Script Validation Functions}
  %
  \begin{align*}
     &\fun{hashScript} \in  ~\Script\to \ScriptHash \\
     &\text{Compute script hash} \\~\\
     &\fun{hashData} \in  ~\Data \to \DataHash \\
     &\text{Compute hash of data} \\~\\
     &\fun{valContext} \in  \UTxO \to \GoguenTx \to \ScriptPurpose \to \Data \\
     &\text{Build Validation Data} \\~\\
     &\fun{runMSigScript} \in\ScriptMSig\to \GoguenTx \to \IsValidating  \\
     &\text{Validate a multi-signature script} \\~\\
     &\fun{runPLCScript} \in \CostMod \to\ScriptPlutus \to
    \seqof{\Data} \to \ExUnits \to (\IsValidating \times \ExUnits) \\
     &\text{Validate a Plutus script, taking resource limits into account}
  \end{align*}
  %
  \emph{Notation}
  %
  \begin{align*}
    \llbracket \var{script_v} \rrbracket_{\var{cm},\var{exunits}}~\var{d}
    &=& \fun{runPLCScript} ~{cm}~\var{script_v}~\var{d}~\var{exunits}
  \end{align*}
  \caption{Script Validation, cont.}
  \label{fig:defs:functions-valid}
\end{figure*}

Note that no ``checks'' are performed within the functions that match the
scripts with their inputs. Missing validators, missing inputs, incorrect hashes, the wrong type of script etc,
are caught during the application of the UTXOW rule (before these functions are ever applied).
%
Various items of data are involved in building the inputs for script validation:

\begin{itemize}
\item The hash of the validator script;

\item The hash of the required datum, if any;

\item The corresponding full validator and datum object, which are looked up in the finite map
constructed by $\fun{indexedScripts}$ and $\fun{indexedDats}$, respectively;

\item The redeemer, which is contained in the transaction's indexed redeemer structure
and which is located using the $\fun{findRdmr}$ function; and

\item the validation data, built using the UTxO, the transaction itself,
and the current item being validated.
\end{itemize}


\begin{figure}[htb]
  \begin{align*}
    & \fun{getData} \in \GoguenTx \to \UTxO \to \ScriptPurpose \to \seqof{\Data} \\
    & \fun{getData}~{tx}~{utxo}~{it}~=~
      \begin{cases}
        [\var{d}] & \var{it} \mapsto ((a,\_),h_d, \_) \in \var{utxo}, \var{h_d}\mapsto \var{d} \in \fun{indexedDats}~{tx} \\
        \epsilon  & \text{otherwise}
      \end{cases}
    \nextdef
    & \fun{mkPLCLst} \in \PParams \to \GoguenTx \to \UTxO \to \seqof{(\ScriptPlutus \times \seqof{\Data} \times \CostMod)} \\
    & \fun{mkPLCLst} ~\var{pp}~\var{tx}~ \var{utxo} ~=~ \\
    & ~~\fun{toList} \{ (\var{script}, (r; \fun{valContext}~\var{utxo}~\var{tx}~\var{cur}; \fun{getData}~{tx}~{utxo}~{cur}), \var{cm}) \mid \\
    & ~~~~(\var{cur}, \var{scriptHash}) \in \fun{scriptsNeeded}~{tx}~{utxo}, \\
    & ~~~~\var{cur} \mapsto \var{r} \in \fun{findRdmr}~{tx}, \\
    & ~~~~\var{scriptHash}\mapsto \var{script}\in \fun{indexedScripts}~{tx}, \\
    & ~~~~\fun{language}~{script} \mapsto \var{cm} \in \fun{costmdls}~{pp} \}
  \end{align*}
  \caption{Scripts and their Arguments}
  \label{fig:functions:script2}
\end{figure}

\subsection{Two-Phase Transaction Validation for Non-Native Scripts}
\label{sec:two-phase}

\begin{note}
  Make it more obvious somewhere where native vs non-native scripts are processed.
\end{note}

The costs of processing native scripts (those that involve only executing standard ledger rules) are included in the standard transaction fees.
In order to ensure that users pay for the computational resources that are needed to validate non-native scripts, even
if transactions are invalid, transactions are validated in two phases:
the first phase consists of every aspect of transaction validation apart from executing the non-native scripts; and
the second phase involves actually executing those scripts.
%
Our validation approach  uses four transition systems, each with different responsibilities. We
give the details of each below, but to summarize, when a transction is processed,
the processing is done
in the following order:


\begin{tabular}{lp{12cm}}
  \textbf{(UTXOW):} & Verifies that all the necessary witnessing information is present, including
  VKey witnesses, scripts, and all the script input data. It also performs
  key witness checks and runs multisig scripts. It then applies the state changes that are
  computed by the UTXO transition.
  \\
  \textbf{(UTXO):} & Verifies that a transaction satisfies all the accounting requirements
  (including the general accounting property, correct fee payment, etc.),
  and applies the state changes that are computed by the UTXOS transition.
  \\
  \textbf{(UTXOS):} & Performs the appropriate UTxO state changes, based on the
  value of the $\IsValidating$ tag, which it checks using the SVAL transition.
  \\
  \textbf{(SVAL):} & Runs the scripts, verifying that the $\IsValidating$ tag
  is applied correctly.
\end{tabular}

In general, there is no way to check that the budget that has been supplied is sufficient for the transaction,
except by running the scripts. To avoid over-spending the budget, we run them sequentially,
stopping whenever one does not validate, and charging the transaction the corresponding
fees. From the point of view of the ledger, there is no difference
between a script running out of $\ExUnits$ during validation, or not validating.
If a transaction contains an invalid script, the only change to the ledger
as a result of applying this transaction is the fees.

Two-phase validation requires a new transition system
(see Figure \ref{fig:ts-types:utxos}) to sequentially run
scripts and to track spent execution units as part of its state
($\var{remExU}$). The signal here is a sequence of pairs of a validator
script and the corresponding input data.

Note that there is one state variable in the SVAL transition system. The reason
for this is that in the second, script-running validation phase, we separate
the UTxO state update from sequentially running scripts. This transition
system is strictly for running the scripts, and a transition of this type
will be used by another rule to perform the correct UTxO update.

Running scripts sequentially
to verify that they all validate in the allotted $\ExUnits$ budget only requires
the amount of remaining $\ExUnits$ to be included in the state, and nothing else.
In the environment, we need the protocol parameters and the
transaction being validated. All other data needed
to run the scripts comes from the signal.

\begin{figure}[htb]
  \emph{Validation environment}
  \begin{equation*}
    \ValEnv =
    \left(
      \begin{array}{r@{~\in~}lr}
        \var{pp} & \PParams & \text{protocol parameters}\\
        \var{tx} & \GoguenTx & \text{transaction being processed} \\
      \end{array}
    \right)
  \end{equation*}
  %
  \emph{Validation state}
  \begin{equation*}
    \ValState =
    \left(
      \begin{array}{r@{~\in~}lr}
        \var{remExU} & \ExUnits & \text{exunits remaining to spend on validation} \\
      \end{array}
    \right)
  \end{equation*}
  %
  \emph{Script transitions}
  \begin{equation*}
    \_ \vdash
    \var{\_} \trans{sval}{\_} \var{\_}
    \subseteq \powerset (\ValEnv \times \ValState \times \seqof{(\ScriptPlutus\times\seqof{\Data}\times\CostMod)} \times \ValState)
  \end{equation*}
  %
  \caption{UTxO script validation types}
  \label{fig:ts-types:utxos}
\end{figure}

The rules for the second-phase script validation SVAL are given in
Figure~\ref{fig:rules:utxo-scrval}. Again, no UTxO state update
is done in this rule. Its purpose is to verify that the
validation tag ($\fun{txvaltag}$) is applied correctly by the creater of
the block. It does this by running all the scripts.

Note that following the Shelley ledger specification approach, every function
that we define and use in the preconditions or calculations that are used in the ledger rules is
necessarily total.
In this way, all the errors (validation failures) that we encounter always come from
rule applications, i.e. a precondition of a rule is not met.
We mention this here because the SVAL rule looks as if it could be
simply a function. However, we want the incorrect application of the
validation tag to be an error, so this must be an error that comes form
an unmet precondition of a rule.

There are three transition rules.
The first rule, $\mathsf{Scripts\mbox{-}Val}$, applies when:

\begin{enumerate}
\item There
are no scripts left to validate in the signal list (i.e. this is the base case of
induction when all the scripts have validated) -- note that there could be $\ExUnits$ left over; and
\item The validation tag is applied correctly (it is $\True$).
\end{enumerate}

The $\mathsf{Scripts\mbox{-}Stop}$ rule applies when:

\begin{enumerate}
  \item The current script-input pair  does not validate;
  (either because the transaction ran out of $\ExUnits$ or for any other reason); and
  \item The validation tag is correct ($\False$ in this case).
\end{enumerate}

These first two rules require no state change.
The $\mathsf{Scripts\mbox{-}Ind}$ rule applies when:

\begin{enumerate}
  \item The current script being validated has been validated;
  \item There is a non-negative fee which remains to pay for validating
  the rest of the scripts in the list; and
  \item Transition rules apply for the rest of the list (without the current script).
\end{enumerate}

The only state change in this rule is of the variable $\var{remExU}$.
It is decreased by subtracting the cost of the execution of the
current script from its current value.
We use this variable to keep track of the remaining funds for
script execution. If the transaction is overpaying ($\fun{txscriptfee}~{tx}$
is too big), then the whole fee is still taken.

It is always in the interest of the slot leader to have the new block validate,
and for it to contain only valid transactions. This motivates the
slot leader to:

\begin{enumerate}
  \item Correctly apply the $\IsValidating$ tag;
  \item Include all transactions that validate,
  \textit{even in case of a 2nd step script validation failure};
  \item Exclude any transactions that are invalid in some way \textit{other than 2nd step script validation failure}.
\end{enumerate}

We want to
discard the blocks which have transactions with these tags
applied incorrectly.
One of the reasons for having the correct validation tag added by the slot leader
to a transaction is that re-applying blocks would not require repeat
execution of scripts in the transactions inside a block. In fact, when replaying
blocks, all the witnessing information can be thrown away.
We also rely on the correct use of tags in other rules (at this time, only in
the rules that are shown in Figure \ref{fig:rules:ledger}).


\textbf{Non-integral calculations within the Plutus interpreter.} At present, all Plutus calculations use integral types. If there
will be any future non-integral calculations (e.g. from the Actus contracts that are implemented using
the Marlowe interpreter), these should
be performed exactly as in the Shelley ledger (as described in~\cite{non_int}). This is a matter of
ensuring deterministic script validation outcomes: inconsistent rounding, for example, could
result in different validation outcomes running the same script on the same
arguments.


\begin{figure}[htb]
  \begin{equation}
    \inference[Scripts-Val]
    {
    \fun{txvaltag}~\var{tx} = \True  &
    \var{remExU}~\geq~0
    }
    {
    \begin{array}{l}
      \var{pp}\\
      \var{tx}\\
    \end{array}
      \vdash
      \left(
      \begin{array}{r}
        \var{remExU}\\
      \end{array}
      \right)
      \trans{sval}{\epsilon}
      \left(
      \begin{array}{r}
        \var{remExU}\\
      \end{array}
      \right) \\
    }
  \end{equation}
  \begin{equation}
    \inference[Scripts-Stop]
    { \\~\\
    (\var{isVal},\var{remExU'})~:=~ \llbracket sc \rrbracket_
    {cm,\var{remExU}} dt \\
    (sc, dt, cm) := s
    \\
    ~
    \\
    \fun{txvaltag}~\var{tx} = \False &
    (\var{remExU'}~<~0 ~ \lor ~ \var{isVal} = \False)
    }
    {
    \begin{array}{l}
      \var{pp}\\
      \var{tx}\\
    \end{array}
      \vdash
      \left(
      \begin{array}{r}
        \var{remExU}\\
      \end{array}
      \right)
      \trans{sval}{\Gamma;s}
      \left(
      \begin{array}{r}
        \var{remExU}\\
      \end{array}
      \right)
    }
  \end{equation}
  \begin{equation}
    \inference[Scripts-Ind]
    {
    {
    \begin{array}{l}
      \var{pp}\\
      \var{tx}\\
    \end{array}
    }
      \vdash
      \left(
      {
      \begin{array}{r}
        \var{remExU}\\
      \end{array}
      }
      \right)
      \trans{sval}{\Gamma}
      \left(
      {
      \begin{array}{r}
        \var{remExU'}\\
      \end{array}
      }
      \right) \\
    (\var{isVal},\var{remExU''})~:=~ \llbracket sc \rrbracket
    _{cm,\var{remExU'}} dt \\
    (sc, dt, cm) := s & \var{remExU''}~\geq~0
    }
    {
    \begin{array}{l}
      \var{pp}\\
      \var{tx}\\
    \end{array}
      \vdash
      \left(
      \begin{array}{r}
        \var{remExU}\\
      \end{array}
      \right)
      \trans{sval}{\Gamma;s}
      \left(
      \begin{array}{r}
        \varUpdate{remExU''}\\
      \end{array}
      \right)
    }
  \end{equation}
  \caption{Script validation rules}
  \label{fig:rules:utxo-scrval}
\end{figure}


\subsection{Updating the UTxO State}
\label{sec:utxo-state-trans}

We have defined a separate transition system, UTXOS, to represent the two distinct
UTxO state changes, one resulting from all scripts in a transaction validating,
the other - from at least one failing to validate. Its transition types
are all the same as for the for the UTXO transition, see Figure
\ref{fig:ts-types:utxo-scripts}.

\begin{figure}[htb]
  \emph{State transitions}
  \begin{equation*}
    \_ \vdash
    \var{\_} \trans{utxo, utxos}{\_} \var{\_}
    \subseteq \powerset (\UTxOEnv \times \UTxOState \times \GoguenTx \times \UTxOState)
  \end{equation*}
  %
  \caption{UTxO and UTxO script state update types}
  \label{fig:ts-types:utxo-scripts}
\end{figure}

There are two rules, corresponding to the two possible state changes of the
UTxO state in the UTXOS transition system (Figure~\ref{fig:rules:utxo-state-upd}).
%
In both cases, the SVAL transition is called upon to verify that the $\IsValidating$
tag has been applied correctly. The function $\fun{mkPLCLst}$ is used to build
the signal list $\var{sLst}$ for the SVAL transition.
%
The first rule
applies when the validation tag is $\True$.
In this case, the states of the UTxO, fee
  and deposit pots, and updates are updated exactly as in the current Shelley
  ledger spec.
%
  The second rule
  applies when the validation tag is $\False$.
  In this case, the UTxO state changes as follows:

  \begin{enumerate}
    \item All the
    UTxO entries corresponding to the transaction inputs selected for covering
    script fees are removed;

    \item The sum total of the value of the marked UTxO entries
    is added to the fee pot.
  \end{enumerate}


\begin{figure}[htb]
  \begin{equation}
    \inference[Scripts-Yes]
    {
    \var{txb}\leteq\txbody{tx} &
    \fun{txvaltag}~\var{tx} = \True
    \\
    ~
    \\
    \var{sLst} := \fun{mkPLCLst}~\var{pp}~\var{tx}~\var{utxo}
    \\~\\
    {
      \left(
        \begin{array}{r}
          \var{pp} \\
          \var{tx} \\
        \end{array}
      \right)
    }
      \vdash
        \var{\fun{txexunits}~{tx}}
      \trans{sval}{sLst}\var{remExU}
      \\~\\
    {
      \left(
        \begin{array}{r}
          \var{slot} \\
          \var{pp} \\
          \var{genDelegs} \\
        \end{array}
      \right)
    }
    \vdash \var{ups} \trans{\hyperref[fig:rules:update]{up}}{\fun{txup}~\var{tx}} \var{ups'}
    \\~\\
    \var{refunded} \leteq \keyRefunds{pp}{stkCreds}~{txb}
    \\
    \var{depositChange} \leteq
      (\deposits{pp}~{stpools}~{(\txcerts{txb})}) - \var{refunded}
    }
    {
    \begin{array}{l}
      \var{slot}\\
      \var{pp}\\
      \var{stkCreds}\\
      \var{stpools}\\
      \var{genDelegs}\\
    \end{array}
      \vdash
      \left(
      \begin{array}{r}
        \var{utxo} \\
        \var{deposits} \\
        \var{fees} \\
        \var{ups} \\
      \end{array}
      \right)
      \trans{utxos}{tx}
      \left(
      \begin{array}{r}
        \varUpdate{\var{(\txins{txb} \subtractdom \var{utxo}) \cup \outs{slot}~{txb}}}  \\
        \varUpdate{\var{deposits} + \var{depositChange}} \\
        \varUpdate{\var{fees} + \txfee{txb}} \\
        \varUpdate{\var{ups'}} \\
      \end{array}
      \right) \\
    }
  \end{equation}
  \begin{equation}
    \inference[Scripts-No]
    {
    \var{txb}\leteq\txbody{tx} &
    \fun{txvaltag}~\var{tx} = \False
    \\
    ~
    \\
    \var{sLst} := \fun{mkPLCLst}~\var{pp}~\var{tx}~\var{utxo}
    \\~\\
    {
      \left(
        \begin{array}{r}
          \var{pp} \\
          \var{tx} \\
        \end{array}
      \right)
    }
      \vdash
        \var{\fun{txexunits}~{tx}}
      \trans{sval}{sLst}\var{remExU}
    }
    {
    \begin{array}{l}
      \var{slot}\\
      \var{pp}\\
      \var{stkCreds}\\
      \var{stpools}\\
      \var{genDelegs}\\
    \end{array}
      \vdash
      \left(
      \begin{array}{r}
        \var{utxo} \\
        \var{deposits} \\
        \var{fees} \\
        \var{ups} \\
      \end{array}
      \right)
      \trans{utxos}{tx}
      \left(
      \begin{array}{r}
        \varUpdate{\var{\fun{txinputs_{vf}}~{txb} \subtractdom \var{utxo}}}  \\
        \var{deposits} \\
        \varUpdate{\var{fees} + \fun{ubalance}~(\fun{txinputs_{vf}}~{txb}\restrictdom \var{utxo})} \\
        \var{ups} \\
      \end{array}
      \right)
    }
  \end{equation}
  \caption{State update rules}
  \label{fig:rules:utxo-state-upd}
\end{figure}

Figure \ref{fig:rules:utxo-shelley} shows the $\type{UTxO-inductive}$
transition rule for the UTXO transition type. Note that the
signal for this transition is specifically of type $\GoguenTx$, so it does not
work with Shelley transactions. Thus, the UTXOW rule needs to convert a $\Tx$ into
a $\GoguenTx$ to invoke the UTXO rule.
This rule has the following preconditions (plus the relevant ones
from the original Shelley specification):

\begin{enumerate}
  \item The transaction is being processed within its validity interval;

  \item The transaction has at least one input;

  \item All inputs in a transaction correspond to UTxO entries;

  \item The general accounting property holds;

  \item The transaction pays fees correctly;

  \item The transaction is not forging any Ada;

  \item All outputs of the transaction contain only non-negative quantities;

  \item The transaction size does not exceed the maximum limit;

 \item The execution units budget that a transaction gives does not exceed the maximum
  permitted number of units;

  \item The UTXOS state transition is valid.
\end{enumerate}

The resulting state transition is defined entirely by the application of the
UTXOS rule.

\begin{figure}[htb]
  \begin{equation}\label{eq:utxo-inductive-shelley}
    \inference[UTxO-inductive]
    {
      \var{txb}\leteq\txbody{tx} &
      \var{txw}\leteq\fun{txwits}~{tx} \\
      \fun{txfst}~txb \leq \var{slot}
      & \fun{txttl}~txb \geq \var{slot}
      \\
      \txins{txb} \neq \emptyset
      & \txins{txb} \subseteq \dom \var{utxo}
      \\
      \consumed{pp}{utxo}{stkCreds}{rewards}~{txb} = \produced{slot}~{pp}~{stpools}~{txb}
      \\~\\
      \fun{feesOK}~(\vert~ \fun{txscripts}~{tx} \cap \ScriptPlutus ~\vert) ~pp~tx~utxo \\
      \\
      ~
      \\
      \mathsf{adaID}~\notin \dom~{\fun{forge}~tx} \\
      \forall txout \in \txouts{txb}, ~ \fun{getValue}~txout  ~\geq ~ 0 \\~
      \forall txout \in \txouts{txb}, ~ \fun{getCoin}~txout ~\geq \\
      \fun{valueSize}~(\fun{getValue}~txout) * \fun{minUTxOValue}~pp \\~
      \\
      \fun{txsize}~{tx}\leq\fun{maxTxSize}~\var{pp} \\
      \fun{txexunits}~{txb} \leq \fun{maxTxExUnits}~{pp}
      \\
      ~
      \\
      {
        \begin{array}{c}
          \var{slot}\\
          \var{pp}\\
          \var{stkCreds}\\
          \var{stpools}\\
          \var{genDelegs}\\
        \end{array}
      }
      \vdash
      {
        \left(
          \begin{array}{r}
            \var{utxo} \\
            \var{deposits} \\
            \var{fees} \\
            \var{ups}\\
          \end{array}
        \right)
      }
      \trans{utxos}{\var{tx}}
      {
        \left(
          \begin{array}{r}
            \var{utxo'} \\
            \var{deposits'} \\
            \var{fees'} \\
            \var{ups'}\\
          \end{array}
        \right)
      }
    }
    {
      \begin{array}{l}
        \var{slot}\\
        \var{pp}\\
        \var{stkCreds}\\
        \var{stpools}\\
        \var{genDelegs}\\
      \end{array}
      \vdash
      \left(
      \begin{array}{r}
        \var{utxo} \\
        \var{deposits} \\
        \var{fees} \\
        \var{ups}\\
      \end{array}
      \right)
      \trans{utxo}{tx}
      \left(
      \begin{array}{r}
        \varUpdate{\var{utxo'}}  \\
        \varUpdate{\var{deposits'}} \\
        \varUpdate{\var{fees'}} \\
        \varUpdate{\var{ups'}}\\
      \end{array}
      \right)
    }
  \end{equation}
  \caption{UTxO inference rules}
  \label{fig:rules:utxo-shelley}
\end{figure}

\subsection{Witnessing}
\label{sec:wits}

Because of two-phase transaction validation, Plutus script validation is not part of transaction witnessing.
However, native script validation does remain part of transaction witnessing.
When witnessing a transaction, we therefore need to validate only the native scripts.
We have consequently changed the definition of the function
$\fun{scriptsNeeded}$, see Figure~\ref{fig:functions-witnesses} to include both multi-signature and Plutus scripts, plus the scripts that are used for any
validation purpose (forging, outputs, certificates, withdrawals).

\begin{figure}[htb]
  \begin{align*}
    & \hspace{-1cm}\fun{scriptsNeeded} \in \UTxO \to \GoguenTx \to \powerset (\ScriptPurpose \times \ScriptHash) \\
    & \hspace{-1cm}\fun{scriptsNeeded}~\var{utxo}~\var{tx} = \\
    & ~~\{ (\var{i}, \fun{validatorHash}~a) \mid i \mapsto (a, \wcard) \in \var{utxo},\\
    & ~~~~~i\in\fun{txinsScript}~{(\fun{txins~\var{txb}})}~{utxo}\} \\
    \cup & ~~\{ (\var{a}, \fun{stakeCred_{r}}~\var{a}) \mid a \in \dom (\AddrRWDScr
           \restrictdom \fun{txwdrls}~\var{txb}) \} \\
    \cup & ~~\{ (\var{cert}, \var{c}) \mid \var{cert} \in (\DCertDeleg \cup \DCertDeRegKey)\cap\fun{txcerts}~(\txbody{tx}), \\
    & ~~~~~~\var{c} \in \cwitness{cert} \cap \AddrScr\} \\
      \cup & ~~\{ (\var{pid}, \var{pid}) \mid \var{pid} \in \supp~(\fun{forge}~\var{txb}) \} \\
    & \where \\
    & ~~~~~~~ \var{txb}~=~\txbody{tx}
  \end{align*}
  \caption{Scripts Needed}
  \label{fig:functions-witnesses}
\end{figure}

In the Goguen era, we must be able to validate both Shelley transactions
and Goguen transactions. To do this, we transform any Shelley transactions
into Goguen ones, filling any missing data fields with default values.
The only time that we need the original Shelley transaction is to check the signatures
in the hash of the original transaction body, as shown in
Figure~\ref{fig:rules:utxow-goguen}. In addition to the Shelley UTXOW preconditions
that still apply, we have made the following changes and additions:

\begin{itemize}

    \item All the multi-signature scripts in the transaction validate;

    \item The transaction contains exactly those scripts that are required for witnessing and no
    additional ones (this includes all languages of scripts, for all purposes);

    \item The transaction contains a redeemer for every item that needs to be  validated
      by a Plutus script;

    \item The only certificates that are allowed to have scripts as witnesses
    are delegation de-registration certificates;

    \item The transaction has a datum for every Plutus script output that it spends;

    \item The transaction has a datum for every Plutus script output that is
    marked with the $\True$ tag for $\HasDV$;

    \item
    The hash of the subset of protocol parameters in the transaction body is equal to
    the hash of the same subset of protocol parameters that are currently on the ledger;

    \item The hash of the indexed redeemer structure that is attached to the transaction is
    the same as $\fun{rdmrsHash}~{tx}$ (the hash value that is contained in the signed body of
    the transaction).

\end{itemize}

If these conditions are all satisfied, then the resulting UTxO state change is fully determined
by the UTXO transition (the application of which is also part of the conditions).

\begin{figure}[htb]
  \emph{State transitions}
  \begin{equation*}
    \_ \vdash
    \var{\_} \trans{utxow}{\_} \var{\_}
    \subseteq \powerset (\UTxOEnv \times \UTxOState \times \Tx \times \UTxOState)
  \end{equation*}
  %
  \caption{UTxO with witnesses state update types}
  \label{fig:ts-types:utxo-witness}
\end{figure}

\begin{figure}
  \begin{equation}
    \label{eq:utxo-witness-inductive-goguen}
    \inference[UTxO-witG]
    {
      \var{tx}~\leteq~\fun{toGoguenTx}~{tx}_o \\~\\
      \var{txb}\leteq\txbody{tx} &
      \var{txw}\leteq\fun{txwits}~{tx} &
      \var{tx}~\in~\GoguenTx \\
      (utxo, \wcard, \wcard, \wcard) \leteq \var{utxoSt} \\
      \var{witsKeyHashes} \leteq \{\fun{hashKey}~\var{vk} \vert \var{vk} \in
      \dom (\txwitsVKey{txw}) \}\\~\\
      \forall \var{validator} \in \fun{txscripts}~{txw} \cap \ScriptMSig,\\
      \fun{runMSigScript}~\var{validator}~\var{tx}\\~\\
      \{ h \mid (\wcard, h) \in \fun{scriptsNeeded}~\var{utxo}~\var{tx}\} ~=~ \dom (\fun{indexedScripts}~{tx}) \\
      \forall (\var{purp}, h) \in ~\fun{scriptsNeeded}~\var{utxo}~\var{tx}, ~h\mapsto s~\in~\fun{indexedScripts}~{tx},\\
       s \in \ScriptPlutus~\Leftrightarrow \exists r, \var{purp} \mapsto r \in \fun{findRdmr}~{tx}
      \\~\\
      \forall \var{cert}~\in~\fun{txcerts}~{txb}, \fun{regCred}~{cert}\in \PolicyID \Leftrightarrow
      \var{cert} \in~ \DCertDeRegKey \\~\\
      \forall~\var{txin}\in\fun{txinputs}~{txb},
      \var{txin} \mapsto \var{(\wcard,\wcard,h_d)} \in \var{utxo},
      \var{h_d} ~\in \fun{dom}(\fun{indexedDats}~{tx})
      \\
      ~
      \\
      \forall~ix \mapsto (a,v,d_h,\True) ~\in~\fun{txouts}~{txb}, \\
       \var{d_h}\in \fun{dom}~ (\fun{indexedDats}~{tx})
      \\
      ~
      \\
      \fun{ppHash}~{txb}~=~\fun{hashLanguagePP}~\var{pp}~(\fun{cmlangs}~(\fun{txscripts}~\var{txw})) \\~\\
      \fun{txrdmrs}~\var{txw} ~=~ \emptyset \Leftrightarrow \fun{rdmrsHash}~{txb}~=~\Nothing \\
      \fun{txrdmrs}~\var{txw} ~\neq~ \emptyset \Leftrightarrow
      \fun{hash}~(\fun{txrdmrs}~\var{txw})~ =~  \fun{rdmrsHash}~{txb} \\
      \\~\\
      \forall \var{vk} \mapsto \sigma \in \txwitsVKey{txw},
      \mathcal{V}_{\var{vk}}{\serialised{tx_{o}}}_{\sigma} \\
      \fun{witsVKeyNeeded}~{utxo}~{tx}~{genDelegs} \subseteq \var{witsKeyHashes}
      \\~\\
      genSig \leteq
      \left\{
        \fun{hashKey}~gkey \vert gkey \in\dom{genDelegs}
      \right\}
      \cap
      \var{witsKeyHashes}
      \\
      \left\{
        c\in\txcerts{txb}~\cap\DCertMir
      \right\} \neq\emptyset \implies \vert genSig\vert \geq \Quorum \wedge
      \fun{d}~\var{pp} > 0
      \\~\\
      \var{mdh}\leteq\fun{txMDhash}~\var{txb}
      &
      \var{md}\leteq\fun{txMD}~\var{tx}
      \\
      (\var{mdh}=\Nothing \land \var{md}=\Nothing)
      \lor
      (\var{mdh}=\fun{hashMD}~\var{md})
      \\~\\
      {
        \begin{array}{r}
          \var{slot}\\
          \var{pp}\\
          \var{stkCreds}\\
          \var{stpools}\\
          \var{genDelegs}\\
        \end{array}
      }
      \vdash \var{utxoSt} \trans{\hyperref[fig:rules:utxo-shelley]{utxo}}{tx}
      \var{utxoSt'}\\
    }
    {
      \begin{array}{r}
        \var{slot}\\
        \var{pp}\\
        \var{stkCreds}\\
        \var{stpools}\\
        \var{genDelegs}\\
      \end{array}
      \vdash \var{utxoSt} \trans{utxow}{{tx}_o} \varUpdate{\var{utxoSt'}}
    }
  \end{equation}
  \caption{UTxO with witnesses inference rules for GoguenTx}
  \label{fig:rules:utxow-goguen}
\end{figure}

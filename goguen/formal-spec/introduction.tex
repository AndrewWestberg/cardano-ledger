\section{Introduction}

The main changes described in this document are extending the UTxO
model of the Shelley era and the introduction of scripts that are not
evaluated by the ledger.

The transition from a Shelley ledger to a Goguen ledger model includes, most
notably, a change from a basic UTxO model to an extended UTxO model, which
we denote EUTxO in this document.

\begin{note}
  Give a short explanation of EUTxO (incl. validators and redeemers)
  and native vs non-native languages (incl. cost models and exunits).
\end{note}

Native scripts on the other hand, are processed entirely by the ledger rules.
Currently, this includes only the multisignature scripts.
Because of this, their execution cost can easily be assessed before processing them.
In the current version of the Shelley ledger, there is no assessment of cost
of checking multisignatures (abbreviated MSig's). Instead, any additional fees 
incurred as a result of
spending MSig-locked outputs are proportional to the change in transaction
size due to including all the necessary signatures (rather than the
cost of verifying them). Therefore, the transaction size is the only
value needed to calculate the fees of all Shelley transactions, and MSig
scripts do not require any cost model.
\begin{note}
  I don't really buy that argument. The size includes a bunch of other
  stuff, and I don't see any reason why two transactions of the same
  size couldn't have processing costs differing by a factor of 3 or
  so. One could argue that it's cheap in general, and the cost grows
  linearly in the size, so we don't really care.
\end{note}

\section{Introduction}

This document describes the changes to the Shelley-MA ledger required
for the Goguen ledger. The two major components of these changes are
the extension the UTxO model of the Shelley era and the introduction
of non-native scripts, i.e. scripts that are not evaluated by the ledger.

Native scripts are processed entirely by the ledger rules, which means
that their execution cost can easily be assessed before processing
them. In the current version of the Shelley ledger, there is no
assessment of the cost of checking multisignature scripts. Instead,
any additional fees incurred as a result of spending MSig-locked
outputs are proportional to the change in transaction size due to
including all the necessary signatures (rather than the cost of
verifying them).

Non-native scripts on the other hand can perform arbitary
(Turing-complete) computations, which means it is impossible to assess
their execution cost without executing them. Thus, a transaction is
required to carry a budget of $\ExUnits$, which reflect an amount of
resources (such as memory usage and execution steps), and this budget
influences the transaction fees. To change the $\ExUnits$ required to
run the same computation without a hard fork (for example because a
more efficient interpreter was introduced), every scripting language
converts the actual execution cost into $\ExUnits$ using a cost model,
$\CostMod$, which is a protocol parameter.

The changes to the UTxO model required are adapded from
\cite{plutus_eutxo}. Transaction outputs get an additional piece of
data, that gets passed to a script validating the spending of that
output. Additionally, a transaction also needs to supply a redeemer,
which is an additional piece of data, for everything that is validated
by non-native scripts.
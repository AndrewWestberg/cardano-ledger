\section{Blockchain layer}
\label{sec:chain}

\newcommand{\Proof}{\type{Proof}}
\newcommand{\Seedl}{\mathsf{Seed}_\ell}
\newcommand{\Seede}{\mathsf{Seed}_\eta}
\newcommand{\activeSlotCoeff}[1]{\fun{activeSlotCoeff}~ \var{#1}}
\newcommand{\slotToSeed}[1]{\fun{slotToSeed}~ \var{#1}}

\newcommand{\T}{\type{T}}
\newcommand{\vrf}[3]{\fun{vrf}_{#1} ~ #2 ~ #3}
\newcommand{\verifyVrf}[4]{\fun{verifyVrf}_{#1} ~ #2 ~ #3 ~#4}

\newcommand{\HashHeader}{\type{HashHeader}}
\newcommand{\HashBBody}{\type{HashBBody}}
\newcommand{\bhHash}[1]{\fun{bhHash}~ \var{#1}}
\newcommand{\bHeaderSize}[1]{\fun{bHeaderSize}~ \var{#1}}
\newcommand{\bSize}[1]{\fun{bSize}~ \var{#1}}
\newcommand{\bBodySize}[1]{\fun{bBodySize}~ \var{#1}}
\newcommand{\OCert}{\type{OCert}}
\newcommand{\BHeader}{\type{BHeader}}
\newcommand{\BHBody}{\type{BHBody}}

\newcommand{\bheader}[1]{\fun{bheader}~\var{#1}}
\newcommand{\hsig}[1]{\fun{hsig}~\var{#1}}
\newcommand{\bprev}[1]{\fun{bprev}~\var{#1}}
\newcommand{\bhash}[1]{\fun{bhash}~\var{#1}}
\newcommand{\bvkcold}[1]{\fun{bvkcold}~\var{#1}}
\newcommand{\bseedl}[1]{\fun{bseed}_{\ell}~\var{#1}}
\newcommand{\bprfn}[1]{\fun{bprf}_{n}~\var{#1}}
\newcommand{\bseedn}[1]{\fun{bseed}_{n}~\var{#1}}
\newcommand{\bprfl}[1]{\fun{bprf}_{\ell}~\var{#1}}
\newcommand{\bocert}[1]{\fun{bocert}~\var{#1}}
\newcommand{\bnonce}[1]{\fun{bnonce}~\var{#1}}
\newcommand{\bleader}[1]{\fun{bleader}~\var{#1}}
\newcommand{\hBbsize}[1]{\fun{hBbsize}~\var{#1}}
\newcommand{\bbodyhash}[1]{\fun{bbodyhash}~\var{#1}}
\newcommand{\overlaySchedule}[4]{\fun{overlaySchedule}~\var{#1}~\var{#2}~{#3}~\var{#4}}

\newcommand{\PrtclState}{\type{PrtclState}}
\newcommand{\PrtclEnv}{\type{PrtclEnv}}
\newcommand{\OverlayEnv}{\type{OverlayEnv}}
\newcommand{\VRFState}{\type{VRFState}}
\newcommand{\NewEpochEnv}{\type{NewEpochEnv}}
\newcommand{\NewEpochState}{\type{NewEpochState}}
\newcommand{\PoolDistr}{\type{PoolDistr}}
\newcommand{\BBodyEnv}{\type{BBodyEnv}}
\newcommand{\BBodyState}{\type{BBodyState}}
\newcommand{\RUpdEnv}{\type{RUpdEnv}}
\newcommand{\ChainEnv}{\type{ChainEnv}}
\newcommand{\ChainState}{\type{ChainState}}
\newcommand{\ChainSig}{\type{ChainSig}}



\subsection{Block Body Transition}
\label{sec:block-body-trans}


In Figure~\ref{fig:rules:bbody}, we have added the check that the sum total of
script fees all transactions in a block pay do not exceed the maximum total fees per
block (stored as a protocol parameter).

\begin{figure}[ht]
  \begin{equation}\label{eq:bbody}
    \inference[Block-Body]
    {
      \var{txs} \leteq \bbody{block}
      &
      \var{bhb} \leteq \bhbody\bheader{block}
      &
      \var{hk} \leteq \hashKey\bvkcold{bhb}
      \\~\\
      \bBodySize{txs} = \hBbsize{bhb}
      &
      \fun{hash}~{txs} = \bbodyhash{bhb}
      \\~\\
      \sum_{tx\in txs} \fun{txexunits}~(\txbody~{tx}) \leq \fun{maxBlockExUnits}~\var{pp}
      \\~\\
      {
        {\begin{array}{c}
                 \bslot{bhb} \\
                 \var{pp} \\
                 \var{reserves}
        \end{array}}
        \vdash
             \var{ls} \\
        \trans{\hyperref[fig:rules:ledger-sequence]{ledgers}}{\var{txs}}
             \var{ls}' \\
      }
    }
    {
      {\begin{array}{c}
               \var{oslots} \\
               \var{pp} \\
               \var{reserves}
      \end{array}}
      \vdash
      {\left(\begin{array}{c}
            \var{ls} \\
            \var{b} \\
      \end{array}\right)}
      \trans{bbody}{\var{block}}
      {\left(\begin{array}{c}
            \varUpdate{\var{ls}'} \\
            \varUpdate{\fun{incrBlocks}~{(\bslot{bhb}\in\var{oslots})}~{hk}~{b}} \\
      \end{array}\right)}
    }
  \end{equation}
  \caption{BBody rules}
  \label{fig:rules:bbody}
\end{figure}

We have also defined a function that transforms the Shelley ledger state into
the Goguen one, see Figure~\ref{fig:functions:to-shelley}. Note that here we
refer to Shelley-era protocol parameter type as $\ShelleyPParams$, and the Goguen
type as $\PParams$. We use the notation $\var{chainstate}_{x}$ to represent
variable $x$ in the chain state. We do not specify the variables that
remain unchanged during the transition.

\begin{note}
  \textbf{What creation slots should Shelley era UTxOs have?}
  It is not yet clear whether we will follow the approach of making the creation
  slot of every Shelley-era UTxO entry the first slot of Goguen. It is possible
  to look through all the blocks and add the correct one, but not clear if
  necessary. The other option is to allow both types out outputs, some without
  the creation slot.
\end{note}

%%
%% Figure - Shelley to Goguen Transition
%%
\begin{figure}[htb]
  \emph{Types and Constants}
  %
  \begin{align*}
      & s_{last} \\
      & \text{last slot of Shelley era} \\
      & \NewParams ~=~ (\Language \mapsto \CostMod) \times \Prices \times \ExUnits \times \ExUnits \\
      & \text{the type of new parameters to add for Goguen}
      \nextdef
      & \mathsf{ivPP} ~\in~ \NewParams \\
      & \text{the initial values for new Goguen parameters}
  \end{align*}
  \emph{Shelley to Goguen Transition Functions}
  %
  \begin{align*}
      & \fun{mkUTxO} ~\in~ \Slot \to \ShelleyUTxO  \to \UTxO  \\
      & \fun{mkUTxO}~s~\var{utxo} ~=~ \{~ \var{txin} \mapsto ((a,\fun{coinToValue}~c),s) ~\vert~
      \var{txin} \mapsto \var{(a,c)}\in ~\var{utxo}~\} \\
      & \text{make UTxO Goguen}
      \nextdef
      & \fun{toGoguen} \in ~ \ShelleyChainState \to \ChainState \\
      & \fun{toGoguen}~\var{chainstate} =~\var{chainstate'} \\
      &~~\where \\
      &~~~~\var{chainstate'}_{utxo}~=~\fun{mkUTxO}~s_{last}~\var{utxo} \\
      &~~~~\var{chainstate'}_{pparams}~=~\var{pp}\cup \mathsf{ivPP}\\
      & \text{transform Shelley chain state to Goguen state}
  \end{align*}
  \caption{Shelley to Goguen State Transtition}
  \label{fig:functions:to-shelley}
\end{figure}

The transformation we use in the preceeding rules to turn a Shelley
transaction into a Goguen one is given in Figure
\ref{fig:functions:to-shelley}. Recall that it stays the same if the same
if it was already a Goguen one.

\begin{figure}[htb]
  \emph{Functions}
  \begin{align*}
      & \fun{mkIns} ~\in~ \powerset{\ShelleyTxIn} \to \powerset{\TxIn}  \\
      & \fun{mkIns}~\var{ins} ~=~ \{~ (\var{txin}, \Yes) ~\vert~
      \var{txin} \in \var{ins}~\} \\
      & \text{transform Shelley inputs into Goguen inputs}
      \nextdef
      & \fun{toGoguenTx} ~\in~  \Tx \to \GoguenTx \\
      & \text{outputs a Goguen tx given any tx} \\
      & \fun{toGoguenTx} ~=
          \begin{cases}
           \fun{tg}~\var{tx}  & \text{if~} \var{tx} \in \ShelleyTx \\
                \var{tx} & \text{otherwise}
              \end{cases}
      \nextdef
      & \fun{tg} ~\in~  \Tx \to \GoguenTx \\
      & \text{transform a Shelley transaction into a Goguen transaction as follows:} \\
      & ~~\fun{txinputs}~{txb'} ~=~ \fun{mkIns}~(\fun{txins}~{txb}) \\
      & ~~\fun{forge}~{txb'} ~= ~\epsilon \\
      & ~~\fun{txexunits}~{txb'} ~= ~(0,0) \\
      & ~~\fun{txfst}~{txb'} ~= ~0 \\
      & ~~\fun{ppHash}~{txb'} ~= ~\Nothing \\
      & ~~\fun{rdmrsHash}~{txb'} ~= ~\Nothing \\~\\
      & ~~\fun{txwits}~{tx'} ~= ~(\epsilon,\emptyset,\emptyset,\epsilon) \\
      & ~~\fun{txvaltag}~{tx'} ~= ~\Yes \\
      &~~      \where \\
      & ~~~~~~~ \var{txb}~=~\txbody{tx} \\
      & ~~~~~~~ \var{txb'}~=~\txbody{tx'}
  \end{align*}
  \caption{Shelley to Goguen Transaction Interpretation}
  \label{fig:functions:to-shelley}
\end{figure}

\section{Notation}
\label{sec:notation-shelley}

This document uses the following general notation.  Other notation is exactly as defined in the Shelley Ledger specification~\ref{XX}.

\vspace{20pt}
\begin{center}
\begin{tabular}{||l|p{4in}||}\hline
  $\mathbb{N}$ & The (canonical) symbol for the natural numbers.
\\\hline
  $\mathbb{H}$ & The type of byte strings.
\\\hline
Aggregated Addition &
Given a type
  $\type{FM}~\in~ \powerset(\type{X} \times \type{Y})$,
    where addition is defined on terms of type $\type{Y}$, and a term $\var{fm} \in \type{FM}$,
    we overload the $\sum$ notation as follows:
    \[\sum_{(x, y)\in\var{fm}} (x,y) :=
    \{ x\mapsto \sum_{(x,y)\in\var{fm}} y \} \]
%
    In the case where $\type{Y}~=~\type{A}\mapsto\type{B}$ is itself a finite map,
    and addition is defined on terms of type $\type{B}$,
    we interpret
    \[\sum_{(x, (a\mapsto b))\in\var{fm}} (x,a\mapsto b) :=
    \{ x \mapsto (a\mapsto \sum_{(x,a\mapsto b)\in\var{fm}} b) \} \]
%
    We define $\sum$ on a set of the form
    \[\var{fm} \in \type{FM}~~ \subseteq ~~ \{ x \mapsto y \vert x \in \mathsf{X}, y \in \mathsf{Y} \} \]
%
    in a similar way,
    \[\sum_{(x\mapsto y)\in\var{fm}} x \mapsto y :=
    \{ x\mapsto \sum_{x\mapsto y\in\var{fm}} y \} \]
%
    We use $+$ to also denote this form of overloaded addition operation.
    \\\hline
Other Operations &
  Similar to the definition of aggregated addition, we also
  use other scalar operations on finite maps. These include \emph{multiplication} $\times$,
  \emph{floor}, etc.
\\\hline
$\leq$ &
  This symbol is overloaded to represent the conjunction of the
  pairwise $\leq$ comparison of entries with the same index in a list or a vector.
  Every other comparison symbol is overloaded in a similar way.
  \\\hline
\end{tabular}
\end{center}

\todo[inline]{Why ``:=''?}

\todo[inline]{Is the aggregated addition identical to the definition in the multi-asset specification?  If so refer to it, if not explain why not.}
\clearpage

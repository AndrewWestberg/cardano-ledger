\section{Transaction}
\label{sec:transactions}

This section outlines the changes that are needed to the transaction
structure to enable non-native scripts to validate
forging tokens, spending outputs, verifying certificates, and
verifying withdrawals.
%
Figure \ref{fig:defs:utxo-shelley-1} gives the modified transaction types.
We make the following changes and additions over the corresponding types in the
Shelley specification~\cite{XX}:

\begin{itemize}
  \item $\Value$ represents the type of multi-asset tokens. For details,
  see the multi-asset formal specification~\cite{XX}.

  \item $\RdmrsHash$ is the type of a hash of the indexed redeemer structure,
  included within a transaction (this structure is described in detail below).

  \item $\ScriptMSig$ is the type of the existing native multi-signature scripts.

  \item $\ScriptPlutus$ is the type of scripts of the initial version of Plutus.

  \item $\Data$ is the same type as in the Plutus libraries (see the note on ``Data Representation'' below).
    \begin{note}
      It may or may not be, but we don't particularly care. We just use this to pass data to plutus.
    \end{note}

  \item $\Script$ is a sum type that includes all possible scripts, both native and non-native.

  \item $\IsValidating$ is a tag that indicates that a transaction
  expects that all its non-native scripts validate.
  This tag is added by the block creator when
  constructing a block, and its correctness is verified as part of the script execution.

  \item $\IsFee$ is a tag that indicates when an input has been marked
    to be used for paying transaction fees.
    This provides a way to prevent
  the entire value of a transaction's UTxO entries being used as fees if a script fails to validate.

  \item $\TxIn$ is a transaction input. It includes a reference to the UTxO entry that it is spending
  and an $\IsFee$ tag.

  \item $\TxOutND$ is a transaction output with no datum hash
  (this type is for VKey and multi-signature script outputs).

  \item $\TxOutP$ is the type of a transaction output that is locked by
  a non-native script, which thus needs to include a datum hash.

  \item $\HasDV$
  is a tag that is attached to a transaction output if it is \emph{locked} by a Plutus
  script. This tag indicates whether the transaction carrying the output
  contains, in its set of datum objects, the full datum that corresponds
  to the datum hash.

  Note that it is up to the user to decide whether the transaction should have the full datum. The purpose of
  including it is just so that it can be communicated to the user who will be spending
  the output in the future, and who will require the full datum in order to validate
  the Plutus script that is locking the output. However, this tag must be applied
  correctly, otherwise the transaction will not validate.
  \begin{note}
    The purpose of this tag is to ensure that it is evident if such a
    datum has been removed. We may do something else and remove this tag.
  \end{note}

  \item $\TxOut$ is the type of transaction outputs, either
  $\TxOutND$ or a pair of $\TxOutP$ and $\HasDV$.

  \item $\Tag$ lets us differentiate what a script
  can validate, i.e. \\

  \begin{tabular}{l@{~to validate~}l}
  $\mathsf{inputTag}$ & spending a script UTxO entry, \\
  $\mathsf{forgeTag}$ & forging tokens, \\
  $\mathsf{certTag}$  & certificates with script credentials, and  \\
  $\mathsf{wdrlTag}$ & reward withdrawals from script addresses.
  \end{tabular}

  \item $\RdmrPtr$ is a pair of a tag and an index. This type is
  used to index the Plutus redeemers that are included in a transaction. See
  below.

\end{itemize}


\begin{figure*}[htb]
  \emph{Abstract types}
  %
  \begin{equation*}
    \begin{array}{r@{~\in~}l@{\quad\quad\quad\quad}r}
      \var{v} &\Value & \text{Multi-asset value}\\
      \var{hdv} &\RdmrsHash & \text{Indexed redeemer hash}\\
      \var{msig} & \ScriptMSig & \text{Multi-signature scripts} \\
      \var{plc} & \ScriptPlutus & \text{Plutus scripts from the initial Plutus version} \\
      \var{dat} & \Data & \text{The Plutus $\Data$ type} \\
    \end{array}
  \end{equation*}
  %
  \emph{Script types}
  %
  \begin{equation*}
    \begin{array}{r@{~\in~}l@{\qquad=\qquad}lr}
      \var{scr} & \Script & \ScriptPlutus \uniondistinct \ScriptMSig \\
      \var{isv} & \IsValidating & \Bool \\
    \end{array}
  \end{equation*}
%
  \emph{Derived types}
  %
  \begin{equation*}
    \begin{array}{r@{~\in~}l@{\qquad=\qquad}lr}
      \var{isf}
      & \IsFee
      & \Bool
%      & \text {tag for inputs used to pay script fees}
      \\
      \var{txin}
      & \TxIn
      & \TxId \times \Ix \times \IsFee
%      & \text{transaction input}
      \\
      (\var{addr}, v)
      & \type{TxOutND}
      & \Addr \times \Value
%      & \text{vk address output}
      \\
      (\var{addr}, v, \var{hashscr_d})
      & \type{TxOutP}
      & \type{TxOutND} \times \DataHash
%      & \text{script address output}
      \\
      \var{hdv}
      & \HasDV
      & \Bool
      %      & \text {tag for outputs that come with datums}
      \\
      \var{txotx}
      & \TxOut
      & \TxOutND \uniondistinct (\TxOutP \times \HasDV)
%      & \text{transaction outputs in a transaction}
      \\
      \var{tag}
      & \Tag
      & \{\mathsf{inputTag},~\mathsf{forgeTag},~\mathsf{certTag},~\mathsf{wdrlTag}\}
      \\
      \var{dvin}
      & \RdmrPtr
      & \Tag \times \Ix
%      & \text{reverse pointer to thing dv is for}
    \end{array}
  \end{equation*}
  \caption{Definitions for Transactions}
  \label{fig:defs:utxo-shelley-1}
\end{figure*}


\begin{figure*}[htb]
  \emph{Transaction Types}
  %
  \begin{equation*}
    \begin{array}{r@{~~}l@{~~}l@{\qquad}l}
      \var{wits} ~\in~ \TxWitness ~=~
       & (\VKey \mapsto \Sig) & \fun{txwitsVKey} & \text{VKey signatures}\\
       & \times ~\powerset{\Script}  & \fun{txscripts} & \text{All scripts}\\
       & \times~ \powerset{\Data} & \fun{txdats} & \text{All datum objects}\\
       & \times ~(\RdmrPtr \mapsto \Data)& \fun{txrdmrs}& \text{Indexed redeemers}\\
    \end{array}
  \end{equation*}
  %
  \begin{equation*}
    \begin{array}{r@{~~}l@{~~}l@{\qquad}l}
      \var{txbody} ~\in~ \TxBody ~=~
      & \powerset{\TxIn} & \fun{txinputs}& \text{Inputs}\\
      &\times ~(\Ix \mapsto \TxOut) & \fun{txouts}& \text{Outputs}\\
      & \times~ \seqof{\DCert} & \fun{txcerts}& \text{Certificates}\\
       & \times ~\Value  & \fun{forge} &\text{A forged value}\\
       & \times ~\ExUnits  & \fun{txexunits}& \text{Script exec. budget}\\
       & \times ~\Coin & \fun{txfee} &\text{Non-script fee}\\
       & \times ~(\Slot\times\Slot) & \fun{txfst},~\fun{txttl} & \text{Validity interval}\\
       & \times~ \Wdrl  & \fun{txwdrls} &\text{Reward withdrawals}\\
       & \times ~\Update  & \fun{txUpdates} & \text{Update proposals}\\
       & \times ~\PPHash^?  & \fun{ppHash} & \text{Hash of PPs}\\
       & \times ~\RdmrsHash^? & \fun{rdmrsHash} & \text{Hash of indexed redeemers}\\
       & \times ~\MetaDataHash^? & \fun{txMDhash} & \text{Metadata hash}\\
    \end{array}
  \end{equation*}
  %
  \begin{equation*}
    \begin{array}{r@{~~}l@{~~}l@{\qquad}l}
      \var{txg} ~\in~ \GoguenTx ~=~
      & \TxBody & \fun{txbody} & \text{Body}\\
      & \times ~\TxWitness & \fun{txwits} & \text{Witnesses}\\
      & \times ~\IsValidating & \fun{txvaltag}&\text{Validation tag}\\
      & \times ~\MetaData^? & \fun{txMD}&\text{Metadata}\\
    \end{array}
  \end{equation*}
  %
  \begin{equation*}
    \begin{array}{r@{~\in~}l@{\qquad=\qquad}l@{\qquad\qquad}l}
\qquad\qquad      \var{tx} & \Tx & \ShelleyTx \uniondistinct \GoguenTx &
      \text{All valid transactions}\\
    \end{array}
  \end{equation*}
  %
  \emph{Accessor Functions}
  \begin{equation*}
    \begin{array}{r@{~\in~}l@{\qquad}l}
      \fun{getValue} & \TxOut \uniondistinct \UTxOOut \to \Value & \text{Output value} \\
      \fun{getAddr} & \TxOut \uniondistinct \UTxOOut \to \Addr & \text{Output address} \\
      \fun{getDataHash} & \TxOut \uniondistinct \UTxOOut \to \DataHash & \text{Data hash}
    \end{array}
  \end{equation*}
  \caption{Definitions for transactions, cont.}
  \label{fig:defs:utxo-shelley-2}
\end{figure*}


\textbf{Data Representation.}
The type $\Data$ is a Plutus type that represents all valid ground types in Plutus.
\begin{note}
  No, this needs to be fixed later on.
There is a similar type in the
ledger. We do not assume these are the same $\Data$, but we do assume there
is structural equality between them. TODO: {Fix this.  Is one a subset of the other?  Are they actually related in any way; if so, how?}
  Is this supposed to be this similar type? From what I've seen, those
  two types are not actually equal, so this needs an update.
\end{note}

\subsection{Witnessing}
Figure \ref{fig:defs:utxo-shelley-2} defines the witness type, $\TxWitness$.  This contains everything
in a transaction that is needed for witnessing, namely:

\begin{itemize}
  \item VKey signatures;
  \item a set of scripts, containing all the scripts (for any purpose) that are needed to \emph{validate} the transaction,
  \item a set of terms of type $\Data$, which contains all required datum objects; and
  \item a map of $\Data$ values indexed by $\RdmrPtr$, which includes all the
  required redeemers.
\end{itemize}

Note that there is a difference between the way scripts and datum objects are included in
a transaction (as a set) versus how redeemers are included
(as an indexed structure). % This is driven by their different relationships to the item:

\begin{description}
\item
  [Hash reference (script/datum object):]
  Scripts and datum objects are referred to explicitly via their hashes,
  which are included in the $\UTxO$ or the transaction. Thus, they can be
  looked up in the tranasaction without any key in the data structure.

  \item[No hash reference (redeemers):] In the case of redeemers,
  we use a reverse pointer approach and
  index the redeemer by a pointer to the item for which it will be used.
  For details on how a script finds its redeemer, see Section~\ref{sec:scripts-inputs}.
\end{description}

\subsection{Goguen transactions}
We have also made the following changes to
the body of the transaction:

\begin{itemize}
  \item We include a term of type $\ExUnits$. This is a budget of execution units
  that may be used by all non-native scripts in the transaction.
  The budget is intended to be computed by executing the scripts off-chain prior to transaction submission.
  \item We include a hash of the the indexed redeemer structure
    with accessor $\fun{rdmrsHash}$. The reason for including this hash is explained below.
  \item We include a hash of the subset of current protocol parameters that is relevant for script execution.
\end{itemize}

A complete Goguen transaction comprises:

\begin{enumerate}
  \item the transaction body,
  \item all the information that is needed to witness transactions,
  \item the $\IsValidating$ tag, which indicates whether all scripts with all inputs
  that are executed during the script execution phase validate.
  The correctness of the tag is verified as part of the ledger rules, and the block is
  deemed to be invalid if it is applied incorrectly.
  If it is set to \emph{false}, then the block can be re-applied without script re-validation.
  This tag cannot be signed, since it is applied by the block producer.
  \item any transaction metadata.
\end{enumerate}


\subsection{Processing Shelley Transactions in the Goguen Era}

To simplify the transition from the Shelley to the Goguen era, the
ledger rules support both Shelley and Goguen transaction formats
simultaneously: $\Tx ~~=~~ \ShelleyTx \uniondistinct \GoguenTx$, where
Shelley transactions have the type $\ShelleyTx$, and Goguen
transactions have type $\GoguenTx$.

Shelley transactions contain fewer data items than Goguen transactions, so we can interpret
a Shelley transaction as a Goguen one with some missing fields, and process it using the normal Goguen ledger
rules.  Shelley witnesses cannot, however, be transformed into Goguen witnesses.
Section~\ref{XX} specifies how to verify signatures before a
Shelley transaction is transformed to a Goguen one.

\subsection{Additional Role of Signatures on TxBody}
\begin{note}
  Make it more obvious that there is only one kind of transaction, but two validation outcomes.
\end{note}
The transaction body must contain all data
(or at least a hash of the data) that can influence
on-chain transfer of value
(see Figure \ref{fig:defs:utxo-shelley-2}).
This means that, for example,
every input that is spent and every output that is created are present in the body of the transaction.
Since a valid transaction can now be ``fully validated'' or ``only paying
fees'' (explained in detail in Section~\ref{sec:utxo}), anything that
can influence these outcomes must also be included in the body.

There is no need to ever sign anything that is related to validator scripts or datum objects,
since a hash of every validator script that is needed
is included in the body of the transaction, and the hash of every datum that is needed
is recorded in the UTxO.
%
The body of the transaction, however, is signed by every key
whose outputs are being spent.
This provides
protection against tampering with Plutus interpreter arguments, which may cause
script validation failure (so placing the transaction in the category of ``only paying fees'').
For this reason, the hash of the indexed redeemer structure and the protocol parameters that are used by
script interpreters are included in the body of the transaction. In the future, other parts of the ledger
state may also need to be included in this hash, if they are passed as
arguments to new script interpreter. Note also that data from the UTxO
is passed to the interpreter, but does not require this type of hash comparison.
This is because, if the entries
have already been spent, the phase 1 validation check will fail.

The owner of any tokens that are spent as part of a given transaction
must sign the transaction body. The body also includes the for-fee tags that are attached to inputs so,
the owners of tokens that are spent by the transaction have
signed their selection of inputs that will be put in the fee pot in case of script validation failure.
As before, a change in the body of the transaction
will make the transaction completely invalid, rather than cause the fee-paying script validation
failure or change the amount of fees it pays.

\subsection{Partially processing transactions}
\begin{note}
  See note above.
\end{note}
Note that in Goguen, it is possible for a transaction
to either be fully processed in full, or else, in case of script validation failure,
to do nothing apart from paying fees (see Section~\ref{sec:utxo} for details).
We have considered two ways to prevent all the Ada in the transaction outputs from going into
the fee pot if script validation fails:
%
\begin{enumerate}
  \item programmatically select the inputs which will be used to pay fees;
  \item allow the user to decide which inputs will be used to pay fees.
\end{enumerate}
%
To maximise flexibility, we have chosen to use the second option.
This also allows users to write their own programmatic solutions to choosing for-fees inputs.

Only inputs that are locked by VKeys or native scripts can
be used to pay fees. In Shelley, spending VKey or script outputs is
validated as part of a single rule application, so either all
the required signatures will be valid, or a transaction is completely
invalid. We have retained this model, but have chosen a different approach to
charging users for running Plutus scripts, see
Section~\ref{sec:two-phase}.

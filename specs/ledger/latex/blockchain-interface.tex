\section{Blockchain interface}
\label{sec:blockchain-interface}

\newcommand{\DIEnv}{\type{DIEnv}}
\newcommand{\DIState}{\type{DIState}}

\newcommand{\UPIEnv}{\type{UPIEnv}}
\newcommand{\UPIState}{\type{UPIState}}

\subsection{Delegation interface}
\label{sec:delegation-interface}

\begin{figure}[htb]
  \emph{Delegation interface environments}
  \begin{equation*}
    \DIEnv =
    \left(
      \begin{array}{r@{~\in~}lr}
        \mathcal{K} & \powerset{\VKeyGen} & \text{allowed delegators}\\
        \var{e} & \Epoch & \text{epoch}\\
        \var{s} & \Slot & \text{slot}\\
        \var{pps} & \PPMMap & \text{Protocol parameters}
      \end{array}
    \right)
  \end{equation*}

  \emph{Delegation interface states}
  \begin{equation*}
    \DIState
    = \left(
      \begin{array}{r@{~\in~}lr}
        \var{dms} & \VKeyGen \mapsto \VKey & \text{delegation map}\\
        \var{dws} & \VKeyGen \mapsto \Slot & \text{when last delegation occurred}\\
        \var{sds} & \seqof{(\Slot \times (\VKeyGen \times \VKey))} & \text{scheduled delegations}\\
        \var{eks} & \powerset{(\Epoch \times \VKeyGen)} & \text{key-epoch delegations}
      \end{array}
    \right)
  \end{equation*}

  \emph{Delegation transitions}
  \begin{equation*}
    \_ \vdash \_ \trans{deleg}{\_} \_ \in
    \powerset (\DIEnv \times \DIState \times \seqof{\DCert} \times \DIState)
  \end{equation*}
  \caption{Delegation interface transition-system types}
  \label{fig:ts-types:delegation-interface}
\end{figure}

\subsubsection{Delegation interface rules}
\label{sec:delegation-interface-rules}

\begin{figure}[htb]
  \begin{equation}
    \label{eq:rule:delegation-interface}
    \inference
    {
      \var{certificateLiveness} \mapsto d \in \var{pps} \\
      {\begin{array}{l}
         \mathcal{K} \\
         e\\
         s\\
         d
       \end{array}}
      \vdash
      {
        \left(
          \begin{array}{l}
            \var{sds}\\
            \var{eks}
          \end{array}
        \right)
      }
      \trans{sdelegs}{\Gamma}
      {
        \left(
          \begin{array}{l}
            \var{sds'}\\
            \var{eks'}
          \end{array}
        \right)
      }
      &
      {
        \left(
          \begin{array}{l}
            \var{dms}\\
            \var{dws}
          \end{array}
        \right)
      }
      \trans{adelegs}{[.., s] \restrictdom \var{sds'}}
      {
        \left(
          \begin{array}{l}
            \var{dms'}\\
            \var{dws'}
          \end{array}
        \right)
      }
    }
    {
      {\begin{array}{l}
         \mathcal{K} \\
         e\\
         s\\
         pps
      \end{array}}
      \vdash
      {
        \left(
          \begin{array}{l}
            \var{dms}\\
            \var{dws}\\
            \var{sds}\\
            \var{eks}
          \end{array}
        \right)
      }
      \trans{deleg}{\Gamma}
      {
        \left(
          \begin{array}{l}
            \var{dms'}\\
            \var{dws'}\\
            \var{[s-d,~ s+d]} \restrictdom \var{sds'}\\
            \var{[e, ..]} \restrictdom \var{eks'}
          \end{array}
        \right)
      }
    }
  \end{equation}
  \caption{Delegation interface rules}
  \label{fig:rules:delegation-interface}
\end{figure}

\subsubsection{Delegation interface functions}
\label{sec:delegation-interface-functions}

\begin{figure}[htb]
  \begin{align*}
    & \fun{delegates} \in \DIState \to (\VKeyGen \times \VKey) \to \Bool & \text{delegation relationship}\\
    & \fun{delegates}~s~(\var{vk_s}, \var{vk_d}) = \var{vk_s} \mapsto \var{vk_d} \in (\var{dms}~s)\\
    \nextdef
    & \fun{initialDS} \in \VKeyGen \to \DIState & \text{initial delegation state}\\
    & \fun{initialDS}~\var{ks} =
      \left(
      \begin{array}{l}
        \var{ks} \mapsto \var{ks}\\
        \emptyset\\
        \epsilon\\
        \emptyset\\
      \end{array}
      \right)
  \end{align*}
  \caption{Delegation interface functions}
\end{figure}

\clearpage

\subsection{Update-proposals interface}
\label{sec:update-proposals-interface}

Figure~\ref{fig:ts-types:upi} defines the types of the transition systems
related with the update-proposals interface. The acronyms in the transition
labels have the following meaning:
\begin{description}
\item[UPIREG] Update-proposal-interface registration.
\item[UPIVOTE] Update-proposal-interface vote.
\item[UPIEND] Update-proposal-interface endorsement.
\item[UPIEC] Update-proposal-interface epoch-change.
\end{description}

\begin{figure}[htb]
  \emph{Update-proposals interface environments}
  \begin{align*}
    & \UPIEnv
      = \left(
      \begin{array}{r@{~\in~}lr}
        \var{e_c} & \Epoch & \text{current epoch}\\
        \var{s_n} & \Slot & \text{current slot number}\\
        \var{dms} & \VKeyGen \mapsto \VKey & \text{delegation map}\\
      \end{array}\right)
  \end{align*}
  %
  \emph{Update-proposals interface states}
  \begin{align*}
    & \UPIState = \\
    & \left(
      \begin{array}{r@{~\in~}lr}
        \var{e_p} & \Epoch & \text{previously seen epoch}\\
        (\var{pv}, \var{pps}) & \ProtVer \times \PPMMap
        & \text{current protocol information}\\
        \var{fads} & \seqof{(\Slot \times (\ProtVer \times \PPMMap))}
        & \text{future protocol version adoptions}\\
        \var{avs} & \ApName \mapsto (\ApVer \times \Slot)
        & \text{application versions}\\
        \var{rpus} & \UPropId \mapsto (\ProtVer \times \PPMMap)
        & \text{registered protocol update proposals}\\
        \var{raus} & \UPropId \mapsto (\ApName \times \mathbb{N})
        & \text{registered software update proposals}\\
        \var{cps} & \UPropId \mapsto \Slot & \text{confirmed proposals}\\
        \var{vts} & \powerset{(\UPropId \times \VKeyGen)} & \text{proposals votes}\\
        \var{bvs} & \powerset{(\ProtVer \times \VKeyGen)}
                           & \text{endorsement-key pairs}\\
        \var{pws} & \UPropId \mapsto \Slot & \text{proposal timestamps}
      \end{array}\right)\\
  \end{align*}
  %
  \emph{Update-proposals interface transitions}
  \begin{equation*}
    \begin{array}{r@{~\in~}l}
      \_ \vdash \_ \trans{upireg}{\_} \_ &
      \powerset (\UPIEnv \times \UPIState \times \UProp \times \UPIState)\\
      \_ \vdash \_ \trans{upivote}{\_} \_ &
      \powerset (\UPIEnv \times \UPIState \times \Vote \times \UPIState)\\
      \_ \vdash \_ \trans{upiend}{\_} \_ &
      \powerset (\UPIEnv \times \UPIState
      \times (\ProtVer \times \VKey) \times \UPIState)\\
      \_ \vdash \_ \trans{upiec}{\_} \_ &
      \powerset (\UPIEnv \times \UPIState \times \Epoch \times \UPIState)
    \end{array}
  \end{equation*}
  \caption{Update-proposals interface transition-system types}
  \label{fig:ts-types:upi}
\end{figure}

\begin{figure}[htb]
  \begin{equation}
    \label{eq:rule:upi-reg-interface}
    \inference
    {
      {
        \begin{array}{l}
          \var{pv}\\
          \var{pps}\\
          \var{avs}\\
          \var{dms}
        \end{array}
      }
      \vdash
      {
        \left(
          \begin{array}{l}
            \var{rpus}\\
            \var{raus}
          \end{array}
        \right)
      }
      \trans{upreg}{\var{up}}
      {
        \left(
          \begin{array}{l}
            \var{rpus'}\\
            \var{raus'}
          \end{array}
        \right)
      }
      &
      pws' = pws \unionoverrideRight \{ \upId{up} \mapsto s_n\}
    }
    {
      {
        \begin{array}{l}
          e_c\\
          s_n\\
          \var{dms}
        \end{array}
      }
      \vdash
      {
        \left(
          \begin{array}{l}
            \var{e_p}\\
            (\var{pv}, \var{pps})\\
            \var{fads}\\
            \var{avs}\\
            \var{rpus}\\
            \var{raus}\\
            \var{cps}\\
            \var{vts}\\
            \var{bvs}\\
            \var{pws}
          \end{array}
        \right)
      }
      \trans{upireg}{\var{up}}
      {
        \left(
          \begin{array}{l}
            \var{e_p}\\
            (\var{pv}, \var{pps})\\
            \var{fads}\\
            \var{avs}\\
            \var{rpus'}\\
            \var{raus'}\\
            \var{cps}\\
            \var{vts}\\
            \var{bvs}\\
            \var{pws'}
          \end{array}
        \right)
      }
    }
  \end{equation}
  \caption{Update-proposals registration rules}
  \label{fig:rules:upi-reg-interface}
\end{figure}

\clearpage

Rule~\ref{eq:rule:upi-vote} models the effect of voting on an update proposal:
after a vote, a proposal might get confirmed, which means that it will be added
to the set $\var{cps'}$. In such case, the mapping of application names to
their latest version known to the ledger will be updated to include the
information in the confirmed proposal. Note that, unlike protocol updates,
software updates take effect as soon as a proposal is confirmed. In this rule,
we also delete the confirmed proposal id's from the set of registered
application update proposals ($\var{raus}$), since this information is no
longer needed once the application-name to software-version map ($\var{avs}$) is
updated.

\begin{figure}[htb]
  \begin{equation}
    \label{eq:rule:upi-vote}
    \inference
    {
      {
        \begin{array}{l}
          s_n\\
          \var{pps}\\
          \var{\dom~pws}\\
          \var{dms}
        \end{array}
      }
      \vdash
      {
        \left(
          \begin{array}{l}
            \var{cps}\\
            \var{vts}
          \end{array}
        \right)
      }
      \trans{upvote}{\var{v}}
      {
        \left(
          \begin{array}{l}
            \var{cps'}\\
            \var{vts'}
          \end{array}
        \right)
      }\\
      \var{avs_{new}} = \{ \var{an} \mapsto (\var{av}, \var{s_n})
      \mid \var{pid} \mapsto (\var{an}, \var{av}) \in \var{raus}
      ,~ \var{pid} \in \dom~\var{cps'}
      \}
    }
    {
      {
        \begin{array}{l}
          e_c\\
          s_n\\
          \var{dms}
        \end{array}
      }
      \vdash
      {
        \left(
          \begin{array}{l}
            \var{e_p}\\
            (\var{pv}, \var{pps})\\
            \var{fads}\\
            \var{avs}\\
            \var{rpus}\\
            \var{raus}\\
            \var{cps}\\
            \var{vts}\\
            \var{bvs}\\
            \var{pws}
          \end{array}
        \right)
      }
      \trans{upivote}{\var{v}}
      {
        \left(
          \begin{array}{l}
            \var{e_p}\\
            (\var{pv}, \var{pps})\\
            \var{fads}\\
            \var{avs} \unionoverrideRight \var{avs_{new}}\\
            \var{rpus}\\
            \dom~ \var{cps} \subtractdom \var{raus}\\
            \var{cps'}\\
            \var{vts'}\\
            \var{bvs}\\
            \var{pws}
          \end{array}
        \right)
      }
    }
  \end{equation}
  \caption{Voting on update-proposals rules}
  \label{fig:rules:upi-vote}
\end{figure}

Figure~\ref{fig:st-diagram-sw-up} shows the different states in which a
software proposal update might be: if valid, a software update proposal becomes
active whenever it is included in a block. If the update proposal gets enough
votes, then the corresponding software update proposal becomes confirmed, and
the ledger state reflects this. If the voting period ends without an update
proposal being confirmed, then the corresponding software update proposal gets
rejected.
%
Protocol updates on the other hand, involve a slightly different logic, and the
state transition diagram for these kind of updates is shown in
Figure~\ref{fig:st-diagram-pt-up}.

\begin{figure}[ht]
  \centering
  \begin{tikzpicture}[ align = center
                     , node distance = 6em and 12em
                     , text width = 5em
                     , font = \footnotesize
                     , >={Latex[width=0.5em, length=0.5em]}
                     , every node/.style = { rectangle
                                         , rounded corners
                                         , draw = black
                                         , align = center
                                         , minimum height = 4em }
                     ]

  \node (active) [fill = blue!10] {Active};
  \node (rejected) [below = of active, fill = red!10] {Rejected};
  \node (confirmed) [right = of active, fill = green!30] {Confirmed};

  \tikzset{every node/.style={align=center, text width=10em, text=brown}}

  \draw[->] (active)
  edge node [above] {Software Update proposal\\ gets enough votes}
  (confirmed);

  \draw[->] (active)
  edge node [left] {Voting period \\ends}
  (rejected);

  \end{tikzpicture}

  \caption{State-transition diagram for software-updates}
  \label{fig:st-diagram-sw-up}
\end{figure}

\clearpage

The interface rule for protocol-version endorsement makes use of the
$\trans{upend}{}$ transition, passing only the portion of the state relevant
to it. In addition, the unconfirmed proposals that are older than $u$ blocks
are removed from the parts of the state that hold:
\begin{itemize}
\item the registered protocol and software update proposals,
\item the votes associated with the proposals,
\item the set of endorsement-key pairs, and
\item the block number in which proposals where added.
\end{itemize}

In Rule~\ref{eq:rule:upi-pend}, the set of proposal id's $\var{pid_{keep}}$
contains only those proposals that haven't expired yet or that are confirmed.
Once a proposal $\var{up}$ is confirmed, it is removed from the set of
confirmed proposals ($\var{cps}$) either when it is adopted or when a protocol
version higher than that of $\var{up}$ gets adopted (see
Rule~\ref{eq:rule:upi-ec}).
%
The set of endorsement-key pairs is cleaned here as well as in the epoch change
rule (Rule~\ref{eq:rule:upi-ec}). The reason for this is that this set grows at
each block, and it can get considerably large if no proposal gets adopted at
the end on an epoch.

\begin{figure}[htb]
  \begin{equation}
    \label{eq:rule:upi-pend}
    \inference
    {
      {
        \begin{array}{l}
          k\\
          s_n\\
          (\var{pv}, \var{pps})\\
          \var{cps}\\
          \var{rpus}
        \end{array}
      }
      \vdash
      {
        \left(
          \begin{array}{l}
            \var{fads}\\
            \var{bvs}
          \end{array}
        \right)
      }
      \trans{upend}{(\var{bv}, \var{vk})}
      {
        \left(
          \begin{array}{l}
            \var{fads'}\\
            \var{bvs'}
          \end{array}
        \right)
      }\\
      \var{chainStability} \mapsto k \in  \var{pps} &
      \var{upropTTL} \mapsto u \in \var{pps} \\
      {
        \begin{array}{r@{~=~}l}
          \var{pids_{keep}} & \dom~(pws \restrictrange [s_n - u, ..]) \cup \dom~\var{cps}\\
          \var{vs_{keep}} & \dom~(\range~\var{rpus'})\\
          \var{rpus'} & \var{pids_{keep}} \restrictdom \var{rpus}
        \end{array}
      }
    }
    {
      {
        \begin{array}{l}
          e_c\\
          s_n\\
          \var{dms}
        \end{array}
      }
      \vdash
      {
        \left(
          \begin{array}{l}
            \var{e_p}\\
            (\var{pv}, \var{pps})\\
            \var{fads}\\
            \var{avs}\\
            \var{rpus}\\
            \var{raus}\\
            \var{cps}\\
            \var{vts}\\
            \var{bvs}\\
            \var{pws}
          \end{array}
        \right)
      }
      \trans{upiend}{(\var{bv}, \var{vk})}
      {
        \left(
          \begin{array}{l}
            \var{e_p}\\
            (\var{pv}, \var{pps})\\
            \var{fads'}\\
            \var{avs}\\
            \var{rpus'}\\
            \var{pids_{keep}} \restrictdom \var{raus}\\
            \var{cps}\\
            \var{pids_{keep}} \restrictdom \var{vts}\\
            \var{vs_{keep}}  \restrictdom \var{bvs}\\
            \var{pids_{keep}} \restrictdom \var{pws}
          \end{array}
        \right)
      }
    }
  \end{equation}
  \caption{Proposal endorsement rules}
  \label{fig:rules:upi-pend}
\end{figure}

\clearpage

Rule~\ref{eq:rule:upi-ec} models how the epoch, protocol-version and its
parameters are changed depending on the epoch in the signal ($e_n$ in this
case). A change in the epoch only occurs if the new epoch is greater than the
previously seen epoch ($e_p$).
%
On an epoch change, this rule will pick a candidate that gathered enough
endorsements at least $2 \cdot k$ slots ago. If a protocol-version candidate
cannot gather enough endorsements $2 \cdot k$ slots before the end of an
epoch, the proposal can only be adopted in the next epoch.
%
Figure~\ref{fig:up-confirmed-too-late} shows an example of a proposal being
confirmed too late in an epoch, where it is not possible to get enough
endorsements in the remaining window. In this Figure we take $k = 2$, and we
assume $5$ endorsements are needed to consider a proposal as candidate for
adoption.
%
Note that, in the final state, we use union override to define the updated
parameters ($\var{pps} \unionoverrideRight \var{pps'}$). This is because candidate
proposal might only update some parameters of the protocol.

Rule~\ref{eq:rule:upi-ec} performs cleanup of several state variables:
\begin{itemize}
\item We keep only the information of proposals whose proposed
  protocol-versions are greater than the newly adopted protocol version
  ($\var{pv'}$). This means that active proposals will be discarded even if the
  voting period is not over (it makes no sense to vote on a proposal that
  proposes to upgrade to an older version of the protocol).
\item The registered software-update proposals need not be cleaned here, since
  this is done either when a proposal gets confirmed or when it expires.
\item Rule~\ref{eq:rule:pvbump-change} keeps only those candidates that can be
  adopted in future epochs.
\end{itemize}

\begin{figure}[htb]
  \begin{equation}
    \label{eq:rule:pvbump-change-noop}
    \inference
    {
      e_n \leq e_p
    }
    {
      {\begin{array}{l}
         k\\
         s_n
       \end{array}}
      \vdash
      {
        \left(
          \begin{array}{l}
            \var{e_p}\\
            \var{(\var{pv}, \var{pps})}\\
            \var{fads}
          \end{array}
        \right)
      }
      \trans{pvbump}{\var{e_n}}
      {
        \left(
          \begin{array}{l}
            \var{e_p}\\
            \var{(\var{pv}, \var{pps})}\\
            \var{fads}
          \end{array}
        \right)
      }
    }
  \end{equation}
  \nextdef
  \begin{equation}
    \label{eq:rule:pvbump-change-epoch-only}
    \inference
    {
      [.., s_n - 2 \cdot k] \restrictdom \var{fads} = \epsilon &  e_p < e_n
    }
    {
      {\begin{array}{l}
         k\\
         s_n
       \end{array}}
      \vdash
      {
        \left(
          \begin{array}{l}
            \var{e_p}\\
            \var{(\var{pv}, \var{pps})}\\
            \var{fads}
          \end{array}
        \right)
      }
      \trans{pvbump}{\var{e_n}}
      {
        \left(
          \begin{array}{l}
            \var{e_n}\\
            \var{(\var{pv}, \var{pps})}\\
            \var{fads}
          \end{array}
        \right)
      }
    }
  \end{equation}
  \nextdef
  \begin{equation}
    \label{eq:rule:pvbump-change}
    \inference
    {
      \wcard ; (\wcard , (\var{pv_c}, \var{pps_c})) \leteq [.., s_n - 2 \cdot k] \restrictdom \var{fads}
      & e_p < e_n
    }
    {
      {\begin{array}{l}
         k\\
         s_n
       \end{array}}
      \vdash
      {
        \left(
          \begin{array}{l}
            \var{e_p}\\
            \var{(\var{pv}, \var{pps})}\\
            \var{fads}
          \end{array}
        \right)
      }
      \trans{pvbump}{\var{e_n}}
      {
        \left(
          \begin{array}{l}
            \var{e_n}\\
            \var{(\var{pv_c}, \var{pps_c})}\\
            {[s_n - 2 \cdot k + 1, ..]} \restrictdom fads
          \end{array}
        \right)
      }
    }
  \end{equation}
  \nextdef
  \begin{equation}
    \label{eq:rule:upi-ec}
    \inference
    {
      \var{chainStability} \mapsto k \in  \var{pps} &
      {\begin{array}{l}
         k\\
         s_n
       \end{array}}
      \vdash
      {
        \left(
          \begin{array}{l}
            \var{e_p}\\
            \var{(\var{pv}, \var{pps})}\\
            \var{fads}
          \end{array}
        \right)
      }
      \trans{pvbump}{\var{e_n}}
      {
        \left(
          \begin{array}{l}
            \var{e'}\\
            \var{(\var{pv'}, \var{pps'})}\\
            \var{fads'}
          \end{array}
        \right)
      }\\ ~ \\
      \var{pids_{keep}} = \{ \var{pid} \mid
      \var{pid} \mapsto (\var{pv_i}, \wcard) \in \var{rpus}
      ,~ \var{pv'} < \var{pv_i}\}
    }
    {
      {
        \begin{array}{l}
          e_c\\
          s_n\\
          \var{dms}\\
        \end{array}
      }
      \vdash
      {
        \left(
          \begin{array}{l}
            \var{e_p}\\
            \var{(\var{pv}, \var{pps})}\\
            \var{fads}\\
            \var{avs}\\
            \var{rpus}\\
            \var{raus}\\
            \var{cps}\\
            \var{vts}\\
            \var{bvs}\\
            \var{pws}
          \end{array}
        \right)
      }
      \trans{upiec}{\var{e_n}}
      {
        \left(
          \begin{array}{l}
            \var{e'}\\
            \var{(\var{pv'}, \var{pps} \unionoverrideRight \var{pps'})}\\
            \var{fads'}\\
            \var{avs}\\
            \var{pids_{keep}} \restrictdom \var{rpus}\\
            \var{raus}\\
            \var{pids_{keep}} \restrictdom \var{cps}\\
            \var{pids_{keep}} \restrictdom \var{vts}\\
            {[\var{pv'}, ..]} \restrictdom \var{bvs}\\
            \var{pids_{keep}} \restrictdom \var{pws}
          \end{array}
        \right)
      }
    }
  \end{equation}
  \caption{Block version adoption on epoch change rules}
  \label{fig:rules:upi-ec}
\end{figure}

\begin{figure}[htb]
  \centering
  \begin{tikzpicture}
    %%
    %% Macros used in this picture
    %%
    %
    % Number of slots
    \pgfmathsetmacro{\nrSlots}{12}
    % Slot in which the proposal gets confirmed
    \pgfmathsetmacro{\cSlot}{1}
    % Our special K.
    \pgfmathsetmacro{\K}{4}
    % Epoch end.
    \pgfmathsetmacro{\eend}{11}
    % Number of positive votes needed
    \pgfmathsetmacro{\votes}{5}

    % Draw the horizontal line
    \draw[thick, -Triangle] (0,0) -- (\nrSlots,0)
    node[font=\scriptsize,below left=3pt and -8pt]{slots};

    % % draw vertical lines
    \foreach \x in {0,1,...,\nrSlots}
    \draw (\x cm, 3pt) -- (\x cm, -3pt);

    % Add a label to with the block number in which the proposal got confirmed.
    \node at (\cSlot, -.7) {$b_j$};

    % Update in cps
    \node at (\cSlot, -1.5) {$\var{cps'} = \var{cps} \cup \{ \var{pid} \mapsto b_j \}$};

    % The no-endorsements red bar.
    \draw[red, line width=4pt] (\cSlot, .5) -- +(\K, 0);

    % Brace above the no-endorsement window bar.
    \draw[thick, red, decorate, decoration={brace, amplitude=5pt}]
    (\cSlot, .7) -- +(\K, 0)
    node[black!20!red, midway, above=4pt, font=\scriptsize] {No endorsements possible};

    % The endorsements window.
    \coordinate (ewStart) at (\cSlot + \K, .5);
    \coordinate (ewEnd) at ($(\eend - \K, .5)$);
    \draw[green, line width=4pt]
    (ewStart) -- (ewEnd);

    % Brace above the endorsements window
    \coordinate (ewStartB) at ($(ewStart) + (0, 0.2)$);
    \coordinate (ewEndB) at ($(ewEnd) + (0, 0.2)$);
    \draw[thick, green, decorate, decoration={brace, amplitude=5pt}]
    (ewStartB) -- (ewEndB)
    node[black!50!green, midway, above=18pt, font=\scriptsize] {Endorsements window};

    % The no-candidates change window.
    \coordinate (nccStart) at (\eend - \K, .5);
    \coordinate (nccEnd) at ($(\eend, .5)$);
    \draw[gray, line width=4pt]
    (nccStart) -- (nccEnd);

    % Brace above the no-candidates change window.
    \coordinate (nccStartB) at ($(nccStart) + (0, 0.2)$);
    \coordinate (nccEndB) at ($(nccEnd) + (0, 0.2)$);
    \draw[thick, gray, decorate, decoration={brace, amplitude=5pt}]
    (nccStartB) -- (nccEndB)
    node[gray, midway, above=5pt, font=\scriptsize] {No candidates change for current epoch};


    % The 2k before end-of-epoch window.
    \coordinate (beeStart) at (\cSlot + \K, -.5);
    \coordinate (beeEnd) at ($(\cSlot + \K + \votes, -.5)$);
    \draw[blue, line width=4pt]
    (beeStart) -- (beeEnd);

    % Brace on above the 2k before end-of-epoch window.
    \coordinate (beeStartB) at ($(beeStart) - (0, 0.2)$);
    \coordinate (beeEndB) at ($(beeEnd) - (0, 0.2)$);
    \draw[thick, blue, decorate, decoration={brace, amplitude=5pt}]
    (beeEndB) -- (beeStartB)
    node[black!20!blue, midway, below=5pt, font=\scriptsize] {$\votes$ endorsements needed};

    \draw[blue, line width=2pt] (\eend, 3pt) -- (\eend, -3pt);

    \draw[<-] (\cSlot, 4pt) -- +(-1, 1)
    node [above=2pt, black, font=\scriptsize]
    {Proposal with id $\var{pid}$ gets confirmed};

    \draw[->] (\eend, -4pt) -- +(1, -1)
    node[right, blue, font=\scriptsize] {End of epoch};
  \end{tikzpicture}
  \caption{An update proposal confirmed too late}
  \label{fig:up-confirmed-too-late}
\end{figure}

Figure~\ref{fig:st-diagram-pt-up} shows the different states a protocol-update
proposal can be in, and what causes the transitions between them.

\begin{figure}[ht]
  \centering
  \begin{tikzpicture}[ align = center
                     , node distance = 8em and 16em
                     , text width = 5em
                     , font = \footnotesize
                     , >={Latex[width=0.5em, length=0.5em]}
                     , every node/.style = { rectangle
                                         , rounded corners
                                         , draw = black
                                         , align = center
                                         , minimum height = 4em }
                     ]

  \node (active) [fill = blue!10] {Active};
  \node (rejected) [below = of active, fill = red!10] {Rejected};
  \node (confirmed) [right = of active, fill = green!10] {Confirmed};
  \node (adopted) [below = of confirmed, fill = green!30] {Adopted};

  \tikzset{every node/.style={align=center, text width=10em, text=brown}}

  \draw[->] (active)
  edge node [above] {Protocol update proposal\\ gets enough votes}
  (confirmed);

  \draw[->] (active)
  edge node [left] {Voting period \\ends}
  (rejected);

  \draw[->] (confirmed)
  edge node [above left]
  {Proposal with higher version adopted}
  (rejected);

  \draw[->] (confirmed)
  edge node [right, text width=8em]
  {Proposal gets enough endorsements $2 \cdot k$ slots before the end of an epoch}
  (adopted);

  \end{tikzpicture}

  \caption{State-transition diagram for protocol-updates}
  \label{fig:st-diagram-pt-up}
\end{figure}

\section{Protocol Parameters}
\label{sec:protocol-parameters}

\begin{note}
  The content of this section is mostly not about protocol parameters, so this should be refactored.
\end{note}

We require the following types (see Figure~\ref{fig:defs:protocol-parameters})
in addition to those that are already defined in the Shelley specification.

\vspace{12pt}
\begin{tabular}{lp{5in}}
  $\Language$ &
  This represents the language name/tag (including the Plutus
  version number).
  \\
  $\ExUnits$ &
  A term of this type has two integer values,
  $(mem, steps)$.
  These represent abstract notions of the relative memory usage and script execution steps,
  respectively (a ``unit model'').
  \\
  $\CostMod$ &
  A term of this type represents the vector of coefficients that are used to generate
  a term of type $\ExUnits$ given a vector of some resource primitives.  The mapping is defined
  concretely by the specific version of the Plutus interpreter that is associated with $\Language$.
%  We keep this type as
%   abstract in the specification - it is defined concretely in the Plutus interpreter.
%  The
%  conversion to $\ExUnits$ is also done by the interpreter (thus, is opaque to the ledger rules).
  \\
  $\Prices$ &
  A term of this type comprises three integer values,
  $\var{(pr_{init}, pr_{mem}, pr_{steps})}$: $pr_{init}$ represents the initial
  cost of running a script; $pr_{mem}$ is the price (in Ada) per unit of memory, and $pr_{steps}$ is the price (in Ada) per
  reduction step. This is used to calculate the Ada cost for a specific script execution.
\end{tabular}
\vspace{12pt}

We have also need a number of additional protocol parameters and accessor functions: ...\todo{List these.}

\subsection{Language Versions and Backwards Compatibility Requirements}
\label{sec:versions}

Each version of a language is considered to be a different language as identified by the $\Language$ type.
Each such language needs to be interpreted by a language-specific interpreter, called from the ledger implementation,
and provided with the arguments that it requires (which may be version-specific).
It is necessary for the ledger to be cabable of executing scripts for all current language and previous languages.
This implies that it is necessary to maintain all ledger
data that is needed by any past or current language. Introducing a new language will also
involve a hard fork, since the ledger rules must be updated to use the new interpreter.

\subsection{Determinism of Script Evaluation}
\label{sec:determinism}

The data that is passed to the interpreter
includes the validator, redeemer, information about the transaction carrying
the script, and some ledger data or protocol parameters.
It is necessary to maintain a predictable (deterministic) validation outcome over the period between transaction
submission and script processing.
%
In order to achieve this,
any data that is passed to the interpreter must be
identical to the data that was provided when the transaction that carries the script was
constructed.
Because of this requirement, the carrying transation thus includes a hash of any such data.
When the transaction is processed, as part of the UTXOW rule, this hash is compared with a hash of the equivalent data. Thus, scripts are only executed if they will be executed with the correct data.

The $\fun{hashLanguagePP}$ function (Figure~\ref{fig:defs:protocol-parameters}) selects the protocol parameters that are relevant to
a given set of languages and computes their hash.
%
At this time, the only parameter that is passed to the interpreter is the cost model.

\subsection{Script Evaluation Cost Model and Prices}
\label{sec:cost-mod}

A cost model is used to convert resource primitives into the
more abstract $\ExUnits$. This conversion is done by the interpreter executing the script,
which means we can keep the cost model abstract in this specification.
The actual cost models are recorded in the protocol parameter $\var{costmdls}$.
%
By having distinct cost models for each version and changing the conversion coefficients, we can discourage users from
paying into scripts that have been made using old versions of Plutus, by making these more expensive to execute.
\begin{note}
  This seems like a bad idea. What if funds are locked by a script that forces every output of a transaction to be locked by the same script?
\end{note}
%
The calculation of the actual cost, in Ada, of running
a script that takes $\var{exunits} \in \ExUnits$ resources to run,
is done by a formula specified in the ledger rules, which uses the
$\var{prices}$ parameter. This is a parameter that applies to all
scripts and cannot be varied for individual languages. This parameter
reflects the real-world costs of electicity, hardware etc.

\textbf{Limiting Script Execution Costs.}
The $\var{maxTxExUnits}$ and $\var{maxBlockExUnits}$ protocol parameters are
used to limit the total per-transaction and per-block resource use. These only apply to non-native scripts.
Per-block resource use needs to be limited to ensure that the time required to verify a block has an upper bound.

\begin{figure*}[htb]
  \emph{Abstract types}
  %
  \begin{equation*}
    \begin{array}{r@{~\in~}l@{\qquad\qquad\qquad\qquad\qquad\qquad\qquad\qquad\qquad}r}
      \var{cm} & \CostMod & \text{Coefficients for the cost model} \\
    \end{array}
  \end{equation*}
  %
  \emph{Derived types}
  \begin{equation*}
    \begin{array}{r@{~\in~}l@{\quad=\quad}l@{\qquad}r}
      \var{lg}
      & \Language
      & \{\Plutus, \dotsb\}
      & \text{Script Language}
      \\
      \var{(pr_{init}, pr_{mem}, pr_{steps})}
      & \Prices
      & \Coin \times \Coin \times \Coin
      & \text {Coefficients for $\ExUnits$ prices}
      \\
      \var{(mem, steps)}
      & \ExUnits
      & \N \times \N
      & \text{Abstract execution units} \\
    \end{array}
  \end{equation*}
  %
  \emph{Protocol Parameters}
  %
  \begin{equation*}
      \begin{array}{r@{~\in~}l@{\qquad}r}
        \var{costmdls} \mapsto (\Language \mapsto \CostMod) & \PParams & \text{Script exec. cost model}\\
        \var{prices} \mapsto \Prices & \PParams & \text{Coefficients for $\ExUnits$ prices} \\
        \var{maxTxExUnits} \mapsto \ExUnits & \PParams & \text{Max. total tx script exec. resources}\\
        \var{maxBlockExUnits} \mapsto \ExUnits & \PParams & \text{Max. total block script exec. resources}\\
      \end{array}
  \end{equation*}
  %
  \emph{Accessor Functions}
  %
  \begin{center}
  \fun{costmdls},~\fun{maxTxExUnits},~\fun{maxBlockExUnits},~\fun{prices}
  \end{center}
  %
  \emph{Helper Functions}
  %
  \begin{align*}
    & \fun{hashLanguagePP} \in \PParams \to \powerset{(\Language)} \to \PPHash^?   \\
    & \fun{hashLanguagePP}~\var{pp}~\var{lgs} = \begin{cases}
         \fun{hash}~(\var{lgs} \restrictdom \fun{costmdls}~{pp})
                           & \text{if~} (\var{lgs} \restrictdom \fun{costmdls}~{pp}) \neq \{\} \\
              \Nothing & \text{otherwise} \\
      \end{cases} \\[3ex]
    & \text{The $\fun{hashLanguagePP}$ calculation creates a hash for the protocol parameters} \\ & \text{that are relevant to
    a given set of versions.}
  \end{align*}
  %
  \caption{Definitions Used in Protocol Parameters}
  \label{fig:defs:protocol-parameters}
\end{figure*}

\subsubsection{Protocol Parameter Hash Comparison Considerations}

Not all protocol parameters are relevant to all Plutus languages.
We have defined two helper functions that can be used to select the protocol parameters that need to be hashed for a specific script
(see Figure \ref{fig:defs:functions-chain-helper}). The first, $\fun{language}$, is an accessor function that returns the language tag for a given script, of type
$\Script$ (defined in Section~\ref{sec:transactions}).
The second, $\fun{cmlangs}$, returns the \emph{set} of language tags for the given set of scripts whose languages have a corresponding cost model.

\begin{figure*}[htb]
  \emph{Helper Functions}
  %
  \begin{align*}
    \fun{language} ~\in~& \Script \to \Language \\
    &\text{Returns the language tag ($\Plutus$ for the first version of Plutus)}
    \nextdef
    \fun{cmlangs} ~\in~& \powerset{\Script} \to \powerset{(\Language)} \\
    \fun{cmlangs}~ \var{scrts} ~=~ & \{ \fun{language}~\var{scr} ~\vert~
      \var{scr}~\in~ \var{scrts}, \fun{language}~\var{scr} \in \{\Plutus\}  \}\\
    &\text{Returns all languages that have cost models.}
  \end{align*}
  \caption{Helper functions for Languages and Cost Model}
  \label{fig:defs:functions-chain-helper}
\end{figure*}

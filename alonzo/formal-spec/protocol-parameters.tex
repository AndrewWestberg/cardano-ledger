\section{Language Versions and Cost Models}
\label{sec:protocol-parameters}

We require the following types (see Figure~\ref{fig:defs:protocol-parameters})
in addition to those that are already defined in the Shelley specification~\cite{shelley_spec}.

\vspace{12pt}
\begin{tabular}{lp{5in}}
  $\Language$ &
  This represents the language name/tag, including the
  version number.
  \\
  $\ExUnits$ &
  A term of this type contains two integer values,
  $(mem, steps)$.
  These represent abstract notions of the relative memory usage and script execution steps,
  respectively.
  \\
  $\CostMod$ &
  A term of this type represents the vector of coefficients that are used to generate
  a term of type $\ExUnits$ given a vector of some resource primitives. The mapping is defined
  concretely by the specific version of the Plutus interpreter that is associated with $\Language$.
  We keep this type as abstract in the specification.
  \\
  $\Prices$ &
  A term of this type comprises two integer values that correspond to the components of $\ExUnits$,
  $\var{(pr_{mem}, pr_{steps})}$:
  $pr_{mem}$ is the price (in Ada) per unit of memory, and $pr_{steps}$ is the price (in Ada) per
  reduction step. This is used to calculate the Ada cost for a specific script execution.
\end{tabular}

\subsection{Language Versions and Backwards Compatibility Requirements}
\label{sec:versions}

In the $\Language$ type, each \emph{version} of a language is considered to be a different language (so there might be several versions of the Plutus language, each of which would be considered to
be different).
Each such language needs to be interpreted by a language-specific interpreter that is called from the ledger implementation.
The interpreter is provided with the (language- and version-specific) arguments that it requires.
It is necessary for the ledger to be capable of executing scripts in all languages it has ever supported.
This implies that it is necessary to maintain all forms of ledger
data that is needed by any past or current language, which constrains future ledger designs.
Introducing a new language will require a protocol version update, since the datatypes need to support the new language and the ledger rules must be updated to use the new interpreter.

\subsection{Determinism of Script Evaluation}
\label{sec:determinism}

The data that is passed to the interpreter
includes the validator script, the redeemer, possibly a datum from the UTxO, information about the transaction that
embeds the script, any relevant ledger data, and any relevant protocol parameters.
It is necessary for the validation outcome of any scripts to remain the same during the entire
period between transaction
submission and completion of the script processing.
%
In order to achieve this,
any data that is passed to the interpreter must be determined by the transaction body.
In property \ref{prop:fixed-inputs}, we make precise what deterministic script evaluation means.
The transaction body therefore includes a hash of any such data that is not uniquely determined by other parts of the transaction body or the UTXO.
When the transaction is processed, as part of the UTXOW rule, this hash is compared with a hash of the data that is passed to the interpreter. This
ensures that scripts are only executed if they have been provided with the intended data.

The $\fun{getLanguageView}$ function (Figure~\ref{fig:defs:protocol-parameters}) filters the data relevant
to a given language from the protocol parameters. Recall here that $\Tppm$ is the type of the
protocol parameter $\var{ppm}~\in~\Ppm$, and that $\PParams~=~\Ppm~\mapsto~\Tppm$.
The type $\LangDepView$ is a subrecord of $\PParams$.
%
At the time of writing, the only parameter that needs to be passed to the $\PlutusVI$
interpreter is the cost model, specifically only the cost model for the $\PlutusVI$ language.

\subsection{Script Evaluation Cost Model and Prices}
\label{sec:cost-mod}

To convert resource primitives into the
more abstract $\ExUnits$ during script execution a cost model needs to be supplied to the interpreter.
The cost models required for this purpose are recorded in the $\var{costmdls}$ protocol parameter.
%
The calculation of the actual cost, in Ada, of running
a script that takes $\var{exunits} \in \ExUnits$ resources to run,
is done by a formula in the ledger rules, which uses the
$\var{prices}$ parameter. This is a parameter that applies to all
scripts and that cannot be varied for individual languages. This parameter
reflects the real-world costs in terms of energy usage, hardware resources etc.

In Alonzo, the protocol parameter $\var{minUTxOValue}$ is deprecated, and replaced by
$\var{coinsPerUTxOWord}$. This specifies directly the deposit required for storing
bytes of data on the ledger in the form of UTxO entries.

\textbf{Limiting Script Execution Costs.}
The $\var{maxTxExUnits}$ and $\var{maxBlockExUnits}$ protocol parameters are
used to limit the total per-transaction and per-block resource use. These only apply to phase-2 scripts.
The parameters are used to ensure that the time and memory that are required to verify a block are bounded.

\textbf{Value size limit as a protocol parameter.}
The new parameter $\var{maxValSize}$ replaces the constant $\mathsf{maxValSize}$
used (in the ShelleyMA era) to limit the size of the $\Value$ part of an output in a
serialised transaction.

\textbf{Collateral inputs.}
Collateral inputs in a transaction are used to cover the transaction fees in the case
that a phase-2 script fails (see Section~\ref{sec:transctions}).
The term \emph{collateral} refers to the total Ada contained in the UTxOs referenced
by collateral inputs.

The parameter $\var{collateralPercent}$ is used to specify the percentage of
the total transaction fee its collateral must (at minimum) cover. The
collateral inputs must not themselves be locked by a script. That is, they must
be VKey inputs. The parameter $\var{maxCollateralInputs}$ is used to limit, additionally,
the total number of collateral inputs, and thus the total number of additional
signatures that must be checked during validation.

\begin{figure*}[htb]
  \emph{Abstract types}
  %
  \begin{equation*}
    \begin{array}{r@{~\in~}l@{\qquad\qquad\qquad}r}
      \var{cm} & \CostMod & \text{Coefficients for the cost model}
    \end{array}
  \end{equation*}
  %
  \emph{Derived types}
  \begin{equation*}
    \begin{array}{r@{~\in~}l@{\quad=\quad}l@{\qquad}r}
      \var{lg}
      & \Language
      & \{\PlutusVI, \dotsb\}
      & \text{Script Language}
      \\
      \var{(pr_{mem}, pr_{steps})}
      & \Prices
      & \Coin \times \Coin
      & \text {Coefficients for $\ExUnits$ prices}
      \\
      \var{(mem, steps)}
      & \ExUnits
      & \N \times \N
      & \text{Abstract execution units}
      \\
      \var{ldv}
      & \LangDepView
      & \Ppm~\mapsto~\Tppm
      & \text{PParams view}
    \end{array}
  \end{equation*}
  %
  \emph{Deprecated Protocol Parameters}
  %
  \begin{equation*}
      \begin{array}{r@{~\in~}l@{\qquad}r}
        \var{minUTxOValue} \mapsto \Coin & \PParams & \text{Min. amount of Ada each UTxO must contain}
      \end{array}
  \end{equation*}
  %
  \emph{New Protocol Parameters}
  %
  \begin{equation*}
      \begin{array}{r@{~\in~}l@{\qquad}r}
        \var{costmdls} \mapsto (\Language \mapsto \CostMod) & \PParams \\
        % & \text{Script exec. cost model}\\
        \var{prices} \mapsto \Prices & \PParams \\
        % & \text{Coefficients for $\ExUnits$ prices} \\
        \var{maxTxExUnits} \mapsto \ExUnits & \PParams \\
        % & \text{Max. total tx script exec. resources}\\
        \var{maxBlockExUnits} \mapsto \ExUnits & \PParams \\
        % & \text{Max. total block script exec. resources}\\
        \var{maxValSize} \mapsto \N & \PParams \\
        % & \text{Max size a value can be}\\
        \var{coinsPerUTxOWord} \mapsto \Coin & \PParams \\
        % & \text{Min. Lovelace per word (8 bytes) a UTxO entry must contain}
        \var{collateralPercent} \mapsto \N & \PParams \\
        % & \text{Percentage of Tx fee which collateral must cover}
        \var{maxCollateralInputs} \mapsto \N & \PParams \\
        % & \text{Maximum number of VK inputs that can be used to cover collateral}
      \end{array}
  \end{equation*}
  %
  \emph{Accessor Functions}
  %
  \begin{center}
  \fun{costmdls},~\fun{maxTxExUnits},~\fun{maxBlockExUnits},~\fun{prices}
  \end{center}
  %
  \emph{Helper Functions}
  %
  \begin{align*}
    & \fun{getLanguageView} \in \PParams \to \Language \to \LangDepView \\
    & \fun{getLanguageView}~\var{pp}~\PlutusVI = \var{costmdls}~\mapsto~ (\fun{costmdls}~{pp}~\PlutusVI)
  \end{align*}
  %
  \caption{Definitions Used in Protocol Parameters}
  \label{fig:defs:protocol-parameters}
\end{figure*}

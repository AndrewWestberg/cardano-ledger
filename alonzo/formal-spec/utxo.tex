\section{UTxO}
\label{sec:utxo}

\begin{figure*}[htb]
  \emph{Functions}
  %
  \begin{align*}
    & \fun{isNonNativeScriptAddress} \in \Tx \to \Addr \to \Bool \\
    & \fun{isNonNativeScriptAddress}~tx~a = \\
      &\quad\begin{cases}
        s \in \ScriptNonNative & a \in \AddrScr \land \fun{validatorHash}~a \mapsto s \in \fun{txscripts} (\fun{txwits}~tx) \\
        \False & \text{otherwise}
      \end{cases}
                 \nextdef
    & \fun{totExunits} \in \Tx \to \ExUnits \\
    & \fun{totExunits}~\var{tx} = \sum_{\wcard \mapsto (\wcard, eu) \in \fun{txrdmrs}~(\fun{txwits}~tx)} eu
    \nextdef
    & \fun{feesOK} \in \PParams \to \Tx \to \UTxO \to \Bool  \\
    & \fun{feesOK}~\var{pp}~tx~utxo~= \\
    &~~      \minfee{pp}~{tx} \leq \txfee{txb} \wedge (\fun{totExunits}~tx \neq (0, 0) \Rightarrow \\
    &~~~~~~(\forall (a, \wcard, \_) \in \fun{range}~(\fun{txinputs_{fee}}~{txb} \restrictdom \var{utxo}), \neg \fun{isNonNativeScriptAddress}~tx~a) \\
    &~~~~~~\wedge~ \var{balance} \in \Coin \\
    &~~~~~~      \wedge~ \var{balance} \geq \txfee{txb}) \\
    &~~      \where \\
    & ~~~~~~~ \var{txb}~=~\txbody{tx} \\
    & ~~~~~~~ \var{balance}~=~\fun{ubalance}~(\fun{txinputs_{fee}}~{txb} \restrictdom \var{utxo})
    \nextdef
    & \fun{txins} \in \TxBody \to \powerset{\TxId \times \Ix} \\
    & \fun{txins} ~\var{txb} = \fun{txinputs}~txb \cup \fun{txinputs_{fee}}~txb
    \nextdef
    & \fun{txscriptfee} \in \Prices \to \ExUnits \to \Coin \\
    & \fun{txscriptfee}~(pr_{mem}, pr_{steps})~ (\var{mem, steps})
    = \var{pr_{mem}}*\var{mem} + \var{pr_{steps}}*\var{steps}
    \nextdef
    &\fun{minfee} \in \PParams \to \Tx \to \Coin \\
    &\fun{minfee}~\var{pp}~\var{tx} = \\
    &~~(\fun{a}~\var{pp}) \cdot \fun{txSize}~\var{tx} + (\fun{b}~\var{pp}) +
    \hldiff{\fun{txscriptfee}~(\fun{prices}~{pp})~(\fun{totExunits}~(\fun{txbody}~{tx}))}
  \end{align*}
  \caption{Functions related to fees}
  \label{fig:functions:utxo}
\end{figure*}

We have added or changed several functions that deal with fees as shown in Figure \ref{fig:functions:utxo}.

\begin{itemize}
  \item $\fun{isNonNativeScriptAddress}$ is a predicate that checks
  whether an address is used as a script address with a non-native
  script.
  \item $\fun{totExunits}$ calculates the total $\ExUnits$ in a transaction.
  \item The predicate $\fun{feesOK}$ checks whether the transaction is
  paying the necessary fees, and that it does it correctly. That is, it checks that:
  \begin{enumerate}[label=({\roman*})]
    \item the fee amount that the transaction states it is paying suffices to cover
    the minimum fee that the transaction is obligated to pay; and if the transaction uses non-native scripts, that
    \item the fee-marked inputs do not belong to non-native script addresses;
    \item all the fee-marked inputs contain strictly Ada and no other kinds of token; and
    \item the fee-marked inputs are sufficient to cover the fee amount that is stated
    in the transaction.
  \end{enumerate}
  \item The function $\fun{txins}$ returns the UTxO keys of transaction inputs.
  \item $\fun{txscriptfee}$ calculates the fee that a transaction must pay for script
  execution based on the amount of $\ExUnits$ it has budgeted, and the prices in the current protocol parameters
  for each component of $\ExUnits$.
  \item The minimum fee calculation, $\fun{minfee}$, includes the script
  fees that the transaction is obligated to pay in order to run its scripts.
\end{itemize}

Note that when creating a transaction, the wallet is responsible for
determining the fees. Thus, it also has to execute the non-native scripts
and include the fees for their execution.

\subsection{Combining Scripts with Their Inputs}
\label{sec:scripts-inputs}

Figure~\ref{fig:functions:script1} shows the functions that are needed to
retrieve all the data that is relevant to Plutus script validation.
These include:

\begin{itemize}
\item $\Time$ is the system time
\item
  $\ScriptPurpose$ is a sum type of all parts of a transaction that may
  require a script witness to validate. Note that this contains the data
  (eg. a certificate $c \in \fun{txcerts}~{txb}$,
  or a transaction input $tin \in \fun{txcerts}~{txb}$) of the item being validated,
  not just a tag indicating the type.
\item
  $\fun{indexof}$ is a helper function that finds the index of a given certificate, value, input, or
  withdrawal in a list, finite map, or set of such objects.
  It assumes there is some ordering on each of these structures.
\item
  $\fun{indexedRdmrs}$ indexes the redeemers by the purpose they are used for.
\end{itemize}


\begin{figure}[htb]
  \emph{Abstract types}
  %
  \begin{equation*}
    \begin{array}{r@{~\in~}l@{}lr}
      \var{tm}
      & \Time
      & \text{System time}
    \end{array}
  \end{equation*}
  %
  \emph{Derived types}
  %
  \begin{equation*}
    \begin{array}{r@{~\in~}l@{\qquad=\qquad}lr}
      \var{sp}
      & \ScriptPurpose
      & \PolicyID \uniondistinct \UTxOIn \uniondistinct \AddrRWD \uniondistinct \DCert
%      & \text{item the script is validated for}
    \end{array}
  \end{equation*}
  %
  \emph{Abstract functions}
  \begin{align*}
    &\fun{indexof} \in \DCert \to \seqof{\DCert} \to \Ix\\
    &\fun{indexof} \in \AddrRWD \to \Wdrl \to \Ix\\
    &\fun{indexof} \in \UTxOIn \to \powerset{\TxIn} \to \Ix\\
    &\fun{indexof} \in \PolicyID \to \Value \to \Ix
  \end{align*}
  %
  \emph{Indexing functions}
  \begin{align*}
    &\fun{indexedRdmrs} \in \Tx \to \ScriptPurpose \to \Data^?\\
    &\fun{indexedRdmrs}~tx~sp =
      \begin{cases}
        d & rdptr \mapsto d \in \fun{txrdmrs}~(\fun{txwits}~{tx}) \} \\
        \Nothing & \text{otherwise}
      \end{cases} \\
    & ~~\where \\
    & ~~\quad \var{txb} = \txbody{tx} \\
    & ~~\quad \var{rdptr} = \begin{cases}
        (\mathsf{certTag},\fun{indexof}~\var{sp}~(\fun{txcerts}~{txb}))   & \var{sp}~\in~\DCert \\
        (\mathsf{wdrlTag},\fun{indexof}~\var{sp}~(\fun{txwdrls}~{txb}))   & \var{sp}~\in~\AddrRWD \\
        (\mathsf{mintTag},\fun{indexof}~\var{sp}~(\fun{mint}~{txb}))    & \var{sp}~\in~\PolicyID \\
        (\mathsf{inputTag},\fun{indexof}~\var{sp}~(\fun{txinputs}~{txb})) & \var{sp}~\in~\UTxOIn
      \end{cases}
  \end{align*}
  \caption{Indexing script and data objects}
  \label{fig:functions:script1}
\end{figure}


\subsection{Plutus Script Validation}
Figure~\ref{fig:defs:functions-valid} shows the abstract functions that are used for script validation.

\begin{itemize}
\item $\fun{slotToTime}$ translates a slot number to system time if possible.
If it is not possible to do this translation, $\Nothing$ is returned.
The reason it may not be possible to translate is that the slot number
is too far in the future for the system to accurately
predict the exact time to which it refers.

\item
  $\fun{valContext}$ constructs the validation context.
  This includes all the necessary transaction and chain state data that needs to be passed to the script interpreter.
    It has a $\UTxO$ as its argument to recover the full information of the inputs of the transaction,
    but only the inputs of the transaction are provided to scripts.
\item
  $\fun{runPLCScript}$ validates Plutus scripts. It takes the following
  arguments:
  \begin{itemize}
  \item A cost model, that is used to calculate the $\ExUnits$ that are needed for script execution;
  \item A script to execute;
  \item A list of terms of type $\Data$ that will be passed to the script; and
  \item the execution unit budget.
  \end{itemize}
  It outputs the validation result (as a Boolean).
  Note that script execution stops if the full budget has been spent before validation is complete.
\end{itemize}


\textbf{Slot to time translation.}
One of the inputs to scripts is the transaction validity interval (recall here that
they do not actually see the current slot number). The length of a
slot may change in a future era. In this case, for a script written in a previous
era, if we were to pass the transaction validity interval expressed as slot numbers,
we get that

\begin{itemize}
  \item the script logic is expressed in terms of slot numbers
  which likely assume the slot length of the era in which it was created
  \item the slot numbers in the validity interval of the transaction are used
  assuming slot length of the current era
\end{itemize}

Therefore, the slot numbers inside the contract and the slot numbers in the transaction
map differently onto points in time. The ledger does not have access to data (or conversion functions)
that is needed to convert transaction validity interval slots numbers to slots numbers
that correspond to the contracts world view. To address this, we decided to pass
non-native contracts (in all languages and all eras) the system time instead of slot numbers.
The conversion function is implemented by consensus (see~\cite{shelley_consensus}),
which has the information to do this correctly for

\begin{itemize}
  \item all slots prior to the current slot
  \item a number of slots after the current slot (this number is determined by
  the consensus's forecast window)
\end{itemize}

\textbf{Validation context construction.}
  As additional non-native scripting languages become supported in the future, scripts of different
  languages may expect different (or differently structured) validation context, ie. different
  representation of transaction and chain state data.
  The $\fun{valContext}$ function will be implemented differently
  for each new language, with the construction of the validation context
  dependent on both the language of the script being validated and the ledger/transaction structure
  of the current era.

  In order to ensure that running scripts of all script languages is supported indefinitely across
  future eras, the $\fun{valContext}$ function must be total (ie. always produce
  a validation context). Recall here that while \emph{running} all scripts must be supported across
  all future ledger changes,
  it is not a requirement that every script must \emph{validate} within the context of some transaction. \newline

\textbf{Know your contract arguments.}
  A Plutus validator script may receive either a list of three terms of type $\Data$, in case it validates the spending of script outputs
  or two terms (redeemer and context, with no datum), for all other uses.
  Script authors must keep this in mind when writing scripts, since the ledger call to the interpreter is oblivious to what
  arguments are required.

\begin{figure*}[htb]
  \emph{Abstract Script Validation Functions}
  %
  \begin{align*}
     &\fun{slotToTime} \in \Slot \to \Time^? \\
     &\text{Translate slot number to system time or fail} \\~\\
     &\fun{valContext} \in \Language \to \UTxO \to \Tx \to \ScriptPurpose \to \Data \\
     &\text{Build Validation Data} \\~\\
     &\fun{runPLCScript} \in \CostMod \to\ScriptPlutus \to
    \seqof{\Data} \to \ExUnits \to \IsValidating \\
     &\text{Validate a Plutus script, taking resource limits into account}
  \end{align*}
  %
  \emph{Notation}
  %
  \begin{align*}
    \llbracket \var{script_v} \rrbracket_{\var{cm},\var{exunits}}~\var{d}
    &=& \fun{runPLCScript} ~{cm}~\var{script_v}~\var{d}~\var{exunits}
  \end{align*}
  \caption{Script Validation, cont.}
  \label{fig:defs:functions-valid}
\end{figure*}

Figure \ref{fig:functions:script2} contains the functions used to
match scripts with their corresponding inputs and pass them to the
evaluator.

Note that no ``checks'' are performed within these functions.
Missing validators, missing inputs, incorrect hashes, the wrong type of script etc,
are caught during the application of the UTXOW rule (before these functions are ever applied).
%
Various items of data are involved in building the inputs for script validation:

\begin{itemize}
\item The hash of the validator script;

\item The hash of the required datum, if any;

\item The corresponding full validator and datum object, which are looked up in transaction witnesses;

\item The redeemer, which is contained in the transaction's indexed redeemer structure
and which is located using the $\fun{indexedRdmrs}$ function; and

\item the validation data, built using the UTxO, the transaction itself,
and the current item being validated.
\end{itemize}


\begin{figure}[htb]
  \begin{align*}
    & \fun{getData} \in \Tx \to \UTxO \to \ScriptPurpose \to \seqof{\Data} \\
    & \fun{getData}~{tx}~{utxo}~{sp}~=~
      \begin{cases}
        [\var{d}] & \var{sp} \mapsto (\_, \_, h_d) \in \var{utxo}, \var{h_d}\mapsto \var{d} \in \fun{txdats}~(\fun{txwits}~tx) \\
        \epsilon  & \text{otherwise}
      \end{cases}
    \nextdef
    & \fun{collectNNScriptInputs} \in \PParams \to \Tx \to \UTxO \to \seqof{(\ScriptNonNative \times \seqof{\Data} \times \ExUnits \times \CostMod)} \\
    & \fun{collectNNScriptInputs} ~\var{pp}~\var{tx}~ \var{utxo} ~=~ \\
    & ~~\fun{toList} \{ (\var{script}, (d; \fun{valContext}~(\fun{language}~{script})~\var{utxo}~\var{tx}~\var{sp}; \fun{getData}~tx~utxo~sp), \var{eu}, \var{cm}) \mid \\
    & ~~~~(\var{sp}, \var{scriptHash}) \in \fun{scriptsNeeded}~{utxo}~{tx}, \\
    & ~~~~(\var{d}, \var{eu}) := \fun{indexedRdmrs}~tx~sp, \\
    & ~~~~\var{scriptHash}\mapsto \var{script}\in \fun{txscripts}~(\fun{txwits}~tx), \\
    & ~~~~\fun{language}~{script} \mapsto \var{cm} \in \fun{costmdls}~{pp} \}
    \nextdef
    & \fun{evalScripts} \in \seqof{(\ScriptNonNative \times \seqof{\Data} \times \ExUnits \times \CostMod)} \to \Bool \\
    & \fun{evalScripts}~\epsilon = \True \\
    & \fun{evalScripts}~((\var{sc}, \var{d}, \var{eu}, \var{cm});\Gamma) =
      \llbracket sc \rrbracket_{cm,\var{eu}} d \land \fun{evalScripts}~\Gamma
    \nextdef
    &\fun{runNativeScript} \in\ScriptNative \to \Tx \to \IsValidating
  \end{align*}
  \caption{Scripts and their Arguments}
  \label{fig:functions:script2}
\end{figure}

\subsection{Two-Phase Transaction Validation for Non-Native Scripts}
\label{sec:two-phase}

Transactions are validated in two phases:
the first phase consists of every aspect of transaction validation apart from executing the non-native scripts; and
the second phase involves actually executing those scripts.
This ensures that users pay for the computational resources that are needed to validate non-native scripts, even
if script validation fails. %
In order to handle script execution, an additional transition system is used, called UTXOS.
It performs the appropriate UTxO state changes, based on the
value of the $\IsValidating$ tag, which it checks using the $\fun{evalScripts}$ function.

In general, there is no way to check \emph{a-priori} that the budget that has been supplied is sufficient for the transaction.
This can only be done by actually running the scripts. From the perspective of the ledger, there is no difference
between a script that exhausts the $\ExUnits$ budget during validation, and one that fails to validate.
If a transaction contains a failing script, the only change to the ledger that is made
is to collect all inputs that have been marked for fees.

It is always in the interest of the slot leader to have the new block validate,
and for it to contain only valid transactions. This motivates the
slot leader to:

\begin{enumerate}
  \item Correctly apply the $\IsValidating$ tag;
  \item Include all transactions that validate within the block,
  \textit{even when there is a 2nd step script validation failure};
  \item Exclude any transactions that are invalid in some way \textit{other than 2nd step script validation failure}.
\end{enumerate}

One important reason for adding the validation tag
to a transaction is that re-applying blocks will not require repeat
execution of scripts in the transactions inside a block, which would increase execution costs.
In fact, when replaying
blocks, all the witnessing information can be thrown away.

\subsection{The UTXOS transition system}
\label{sec:utxo-state-trans}

We have defined a separate transition system, UTXOS, to represent the two distinct
UTxO state changes: i) when all the scripts in a transaction validating; and
ii) when at least one fails to validate. Its transition types
are identical to the UTXO transition (Figure
\ref{fig:ts-types:utxo-scripts}).

\begin{figure}[htb]
  \emph{State transitions}
  \begin{equation*}
    \_ \vdash
    \var{\_} \trans{utxos}{\_} \var{\_}
    \subseteq \powerset (\UTxOEnv \times \UTxOState \times \Tx \times \UTxOState)
  \end{equation*}
  %
  \caption{UTxO script state update types}
  \label{fig:ts-types:utxo-scripts}
\end{figure}

There are two rules, corresponding to the two possible state changes of the
UTxO state in the UTXOS transition system (Figure~\ref{fig:rules:utxo-state-upd}).
%
In both cases, the $\fun{evalScripts}$ function is called upon to verify that the $\IsValidating$
tag has been applied correctly. The function $\fun{collectNNScriptInputs}$ is used to build
the inputs $\var{sLst}$ for the $\fun{evalScripts}$ function.
%
The first rule
applies when the validation tag is $\True$.
In this case, the states of the UTxO, fee
  and deposit pots, and updates are updated exactly as in the current Shelley
  ledger spec.
%
  The second rule
  applies when the validation tag is $\False$.
  In this case, the UTxO state changes as follows:

  \begin{enumerate}
    \item All the
    UTxO entries corresponding to the transaction inputs selected for covering
    fees are removed;

    \item The sum total of the value of the marked UTxO entries
    is added to the fee pot.
  \end{enumerate}


\begin{figure}[htb]
  \begin{equation}
    \inference[Scripts-Yes]
    {
    \var{txb}\leteq\txbody{tx} &
    \var{sLst} := \fun{collectNNScriptInputs}~\var{pp}~\var{tx}~\var{utxo}
    \\
    ~
    \\
    \fun{txvaltag}~\var{tx} = \fun{evalScripts}~\var{sLst} = \True
    \\~\\
    {
      \begin{array}{r}
        \var{slot} \\
        \var{pp} \\
        \var{genDelegs} \\
      \end{array}
    }
    \vdash \var{pup} \trans{\hyperref[fig:rules:update]{ppup}}{\fun{txup}~\var{tx}} \var{pup'}
    \\~\\
    \var{refunded} \leteq \keyRefunds{pp}{txb}
    \\
    \var{depositChange} \leteq
      (\deposits{pp}~{poolParams}~{(\txcerts{txb})}) - \var{refunded}
    }
    {
    \begin{array}{l}
      \var{slot}\\
      \var{pp}\\
      \var{poolParams}\\
      \var{genDelegs}\\
    \end{array}
      \vdash
      \left(
      \begin{array}{r}
        \var{utxo} \\
        \var{deposits} \\
        \var{fees} \\
        \var{pup} \\
      \end{array}
      \right)
      \trans{utxos}{tx}
      \left(
      \begin{array}{r}
        \varUpdate{\var{(\txins{txb} \subtractdom \var{utxo}) \cup \outs{slot}~{txb}}}  \\
        \varUpdate{\var{deposits} + \var{depositChange}} \\
        \varUpdate{\var{fees} + \txfee{txb}} \\
        \varUpdate{\var{pup'}} \\
      \end{array}
      \right) \\
    }
  \end{equation}
  \begin{equation}
    \inference[Scripts-No]
    {
    \var{txb}\leteq\txbody{tx} &
    \var{sLst} := \fun{collectNNScriptInputs}~\var{pp}~\var{tx}~\var{utxo}
    \\
    ~
    \\
    \fun{txvaltag}~\var{tx} = \fun{evalScripts}~\var{sLst} = \False
    }
    {
    \begin{array}{l}
      \var{slot}\\
      \var{pp}\\
      \var{poolParams}\\
      \var{genDelegs}\\
    \end{array}
      \vdash
      \left(
      \begin{array}{r}
        \var{utxo} \\
        \var{deposits} \\
        \var{fees} \\
        \var{pup} \\
      \end{array}
      \right)
      \trans{utxos}{tx}
      \left(
      \begin{array}{r}
        \varUpdate{\var{\fun{txinputs_{fee}}~{txb} \subtractdom \var{utxo}}}  \\
        \var{deposits} \\
        \varUpdate{\var{fees} + \fun{ubalance}~(\fun{txinputs_{fee}}~{txb}\restrictdom \var{utxo})} \\
        \var{pup} \\
      \end{array}
      \right)
    }
  \end{equation}
  \caption{State update rules}
  \label{fig:rules:utxo-state-upd}
\end{figure}

Figure \ref{fig:rules:utxo-shelley} shows the $\type{UTxO-inductive}$
transition rule for the UTXO transition type.
This rule has the following changes:

\begin{enumerate}
  \item The transaction pays fees correctly, as defined above;

  \item The end of the transaction validity interval is translatable into
  system time (ie. within the consensus's forecast window). This is checked
  by $\fun{slotToTime}$, which returns $\Nothing$ if the end slot is outside.
  Note that we do not need to check that the start slot can be converted to
  time, because all pasts slots can be converted into time correctly.

  \item $\fun{adaPerUTxOWord}$ is now a protocol parameter explicitly, the
  $\fun{utxoEntrySize}$ calculation is defined differently than for ShelleyMA
  (see Section~\ref{sec:value-size})

  \item $\fun{maxValPerc}$ is now also a protocol parameter (not a constant).
  It represents a percentage (not a size in bytes) of the total transaction
  size that the size of a $\Value$ in an output can be.

 \item The execution unit budget for a transaction is within the maximum
  permitted number of units for a transaction;

  \item The UTXOS state transition is valid.
\end{enumerate}

The resulting state transition is defined entirely by the application of the
UTXOS rule.

\begin{figure}[htb]
  \begin{equation}\label{eq:utxo-inductive-shelley}
    \inference[UTxO-inductive]
    {
      \var{txb}\leteq\txbody{tx} &
      \fun{ininterval}~\var{slot}~(\fun{txvldt}~{tx}) \\
      \hldiff{\var{(\wcard, i_f)}\leteq\fun{txvldt}~{tx}} & \hldiff{\fun{slotToTime}~i_f \neq \Nothing} \\
      \txins{txb} \neq \emptyset
      & \hldiff{\fun{feesOK}~pp~tx~utxo}
      & \txins{txb} \subseteq \dom \var{utxo}
      \\
      \consumed{pp}{utxo}{txb} = ~\produced{pp}{poolParams}~{txb}
      \\~\\
      \mathsf{adaID}\notin \supp {\fun{mint}~tx} \\~\\
      \forall txout \in \txouts{txb}, \\
      \fun{getValue}~txout \geq \fun{coinToValue}(\hldiff{\fun{utxoEntrySize}~{txout} * \fun{adaPerUTxOWord}~pp}) \\~
      \\
      \forall txout \in \txouts{txb},\\
      \hldiff{\fun{serSize}~(\fun{getValue}~txout) ~\leq ~ \fun{quot}~((\fun{maxTxSize}~pp) * (\fun{maxValPerc}~pp))~100} \\~
      \\
      \forall (\wcard\mapsto (a,~\wcard)) \in \txouts{txb}, \fun{netId}~a = \NetworkId
      \\
      \forall (a\mapsto\wcard) \in \txwdrls{txb}, \fun{netId}~a = \NetworkId
      \\
      \fun{txsize}~{tx}\leq\fun{maxTxSize}~\var{pp} \\
      \hldiff{\fun{totExunits}~{tx} \leq \fun{maxTxExUnits}~{pp}}
      \\
      ~
      \\
      \hldiff{{
        \begin{array}{c}
          \var{slot}\\
          \var{pp}\\
          \var{poolParams}\\
          \var{genDelegs}\\
        \end{array}
      }
      \vdash
      {
        \left(
          \begin{array}{r}
            \var{utxo} \\
            \var{deposits} \\
            \var{fees} \\
            \var{pup}\\
          \end{array}
        \right)
      }
      \trans{utxos}{\var{tx}}
      {
        \left(
          \begin{array}{r}
            \var{utxo'} \\
            \var{deposits'} \\
            \var{fees'} \\
            \var{pup'}\\
          \end{array}
        \right)
      }
    }}
    {
      \begin{array}{l}
        \var{slot}\\
        \var{pp}\\
        \var{poolParams}\\
        \var{genDelegs}\\
      \end{array}
      \vdash
      \left(
      \begin{array}{r}
        \var{utxo} \\
        \var{deposits} \\
        \var{fees} \\
        \var{pup}\\
      \end{array}
      \right)
      \trans{utxo}{tx}
      \left(
      \begin{array}{r}
        \varUpdate{\var{utxo'}}  \\
        \varUpdate{\var{deposits'}} \\
        \varUpdate{\var{fees'}} \\
        \varUpdate{\var{pup'}}\\
      \end{array}
      \right)
    }
  \end{equation}
  \caption{UTxO inference rules}
  \label{fig:rules:utxo-shelley}
\end{figure}

\subsection{Witnessing}
\label{sec:wits}

Because of two-phase transaction validation, non-native script validation is not part of transaction witnessing.
However, native script validation does remain part of transaction witnessing.
When witnessing a transaction, we therefore need to validate only the native scripts.

\begin{figure}[htb]
  \begin{align*}
    & \hspace{-1cm}\fun{scriptsNeeded} \in \UTxO \to \Tx \to \powerset (\ScriptPurpose \times \ScriptHash) \\
    & \hspace{-1cm}\fun{scriptsNeeded}~\var{utxo}~\var{tx} = \\
    & ~~\{ (\var{i}, \fun{validatorHash}~a) \mid i \mapsto (a, \wcard, \wcard) \in \var{utxo},
      i\in\fun{txinsScript}~{(\fun{txins~\var{txb}})}~{utxo}\} \\
    \cup & ~~\{ (\var{a}, \fun{stakeCred_{r}}~\var{a}) \mid a \in \dom (\AddrRWDScr
           \restrictdom \fun{txwdrls}~\var{txb}) \} \\
    \cup & ~~\{ (\var{cert}, \var{c}) \mid \var{cert} \in (\DCertDeleg \cup \DCertDeRegKey)\cap\fun{txcerts}~(\txbody{tx}), \\
    & ~~~~~~\var{c} \in \cwitness{cert} \cap \AddrScr\} \\
      \cup & ~~\{ (\var{pid}, \var{pid}) \mid \var{pid} \in \supp~(\fun{mint}~\var{txb}) \} \\
    & \where \\
    & ~~~~~~~ \var{txb}~=~\txbody{tx}
    \nextdef
    & \hspace{-1cm}\fun{checkRedeemers} \in \Tx \to \UTxO \to (\ScriptPurpose \times \ScriptHash) \to \Bool \\
    & \hspace{-1cm}\fun{checkRedeemers}~\var{tx}~\var{utxo}~(\var{sp},\var{h})~=~ \exists s, (h\mapsto s) \in \fun{txscripts}~(\fun{txwits}~tx)\\
    \land & ~ (s \in \ScriptNonNative~\Rightarrow \fun{indexedRdmrs}~tx~sp \neq \Nothing) \\
    \nextdef
    & \hspace{-1cm}\fun{languages} \in \TxWitness \to \powerset{\Language} \\
    & \hspace{-1cm}\fun{languages}~\var{txw}~=~
      \{\fun{language}~s \mid s \in \range (\fun{txscripts}~{txw}) \cap \ScriptNonNative\}
  \end{align*}
  \caption{UTXOW helper functions}
  \label{fig:functions-witnesses}
\end{figure}

We have made the following changes and additions to the UTXOW preconditions:

\begin{itemize}

\item All the native scripts in the transaction validate;

  \item For every item that needs to be validated by a non-native script, the transaction contains
    a redeemer;

    \item The datums included in the witnesses are exactly those that are required for validating;

    \item The transaction contains exactly those scripts that are required for witnessing and no
    additional ones;

    \item
    The hash of the subset of protocol parameter values that have been included in the transaction body is the same as
    the hash of the same subset of protocol parameters that are currently contained in the ledger;

\end{itemize}

If these conditions are all satisfied, then the resulting UTxO state change is fully determined
by the UTXO transition (the application of which is also part of the conditions).

\begin{figure}[htb]
  \emph{State transitions}
  \begin{equation*}
    \_ \vdash
    \var{\_} \trans{utxow}{\_} \var{\_}
    \subseteq \powerset (\UTxOEnv \times \UTxOState \times \Tx \times \UTxOState)
  \end{equation*}
  %
  \caption{UTxO with witnesses state update types}
  \label{fig:ts-types:utxo-witness}
\end{figure}

\begin{figure}
  \begin{equation}
    \label{eq:utxo-witness-inductive-alonzo}
    \inference[UTxO-witG]
    {
      \var{txb}\leteq\txbody{tx} &
      \var{txw}\leteq\fun{txwits}~{tx} \\
      (utxo, \wcard, \wcard, \wcard) \leteq \var{utxoSt} \\
      \var{witsKeyHashes} \leteq \{\fun{hashKey}~\var{vk} \vert \var{vk} \in
      \dom (\txwitsVKey{txw}) \}\\~\\
      \hldiff{\forall \var{s} \in \range (\fun{txscripts}~{txw}) \cap \ScriptNative,
      \fun{runNativeScript}~\var{s}~\var{tx}}\\~\\
      \hldiff{\forall sph \in \fun{scriptsNeeded}~\var{utxo}~\var{tx}, \fun{checkRedeemers}~tx~utxo~sph} \\
      \hldiff{\{ s \mid (\_, s) \in \fun{scriptsNeeded}~\var{utxo}~\var{tx}\} = \dom (\fun{txscripts}~{txw})} \\
      \hldiff{\{ h \mid (\_ \mapsto (a, \_, h)) \in \txins{txb} \restrictdom \var{utxo}, \fun{isNonNativeScriptAddress}~{tx}~{a}\} =} \\
      \hldiff{\dom (\fun{txdats}~{txw})}
      \\~\\
      \forall \var{vk} \mapsto \sigma \in \txwitsVKey{txw},
      \mathcal{V}_{\var{vk}}{\serialised{txbodyHash}}_{\sigma} \\
      \hldiff{\fun{witsVKeyNeeded}~{utxo}~{tx}~{genDelegs} \cup \fun{reqSignerHashes}~txb \subseteq \var{witsKeyHashes}}
      \\~\\
      genSig \leteq
      \left\{
        \fun{hashKey}~gkey \vert gkey \in\dom{genDelegs}
      \right\}
      \cap
      \var{witsKeyHashes}
      \\
      \left\{
        c\in\txcerts{txb}~\cap\DCertMir
      \right\} \neq\emptyset \implies \vert genSig\vert \geq \Quorum \wedge
      \fun{d}~\var{pp} > 0
      \\~\\
      \var{adh}\leteq\fun{txADhash}~\var{txb}
      &
      \var{ad}\leteq\fun{txAD}~\var{tx}
      \\
      (\var{adh}=\Nothing \land \var{ad}=\Nothing)
      \lor
      (\var{adh}=\fun{hashAD}~\var{ad})
      \\~\\
      \hldiff{\fun{sdHash}~{txb}~=~\fun{hashWitnessPPData}~\var{pp}~(\fun{languages}~{txw})~(\fun{txrdmrs}~{txw})}
      \\~\\
      {
        \begin{array}{r}
          \var{slot}\\
          \var{pp}\\
          \var{poolParams}\\
          \var{genDelegs}\\
        \end{array}
      }
      \vdash \var{utxoSt} \trans{\hyperref[fig:rules:utxo-shelley]{utxo}}{tx}
      \var{utxoSt'}\\
    }
    {
      \begin{array}{r}
        \var{slot}\\
        \var{pp}\\
        \var{poolParams}\\
        \var{genDelegs}\\
      \end{array}
      \vdash \var{utxoSt} \trans{utxow}{tx} \varUpdate{\var{utxoSt'}}
    }
  \end{equation}
  \caption{UTxO with witnesses inference rules for Tx}
  \label{fig:rules:utxow-alonzo}
\end{figure}

\section{Transactions}
\label{sec:transactions}

This section outlines the changes that are needed to the transaction
structure to enable non-native scripts to validate
the minting of tokens, spending outputs, verifying certificates, and
verifying withdrawals.
%
Figure \ref{fig:defs:utxo-shelley-1} gives the modified transaction types and helper functions.
We make the following changes and additions:

\begin{itemize}
  \item $\WitnessPPDataHash$ is the type of a hash of script execution
  related data. It is the hash of $\fun{txrdmrs}$ and relevant protocol parameters.

  \item $\ScriptNative$ is the type of native scripts.

  \item $\ScriptNonNative$ is the type of non-native scripts.

  \item $\Data$ is a type for communicating data to script.  It is kept abstract here.

  \item $\Script$ is the type of all scripts, both native and non-native.

  \item $\IsValidating$ is a tag that indicates that a transaction
  expects that all its non-native scripts will be validated.
  This tag is added by the block creator when
  constructing a block, and its correctness is verified as part of the script execution.

  \item $\Datum$ is a type alias for the $\Data$ type, used to signify that
  terms of this type are intended to be used strictly as datum objects.

  \item $\Redeemer$ is also a type alias for the $\Data$ type, used to signify that
  terms of this type are intended to be used strictly as redeemers.

  \item $\TxOut$ is the type of transaction outputs. These are extended to include
  an optional hash of a datum. Note that \emph{any} output can
  optionally include such a hash, even though only non-native scripts
  can actually use the hash value. Since it is in general impossible for the ledger implementation to
  know at the time of transaction creation whether or not an output belongs to a non-native
  script, we simply allow any output to contain a hash of a datum.

  \item $\Tag$ lets us differentiate what a script
  can validate, i.e. \\
  \begin{tabular}{l@{~to validate~}l}
  $\mathsf{Spend}$ & spending a script UTxO entry; \\
  $\mathsf{Mint}$ & minting tokens; \\
  $\mathsf{Cert}$  & certificates with script credentials; and  \\
  $\mathsf{Rewrd}$ & reward withdrawals from script addresses.
  \end{tabular}

  \item $\RdmrPtr$ is a pair of a tag and an index. This type is
  used to index the Plutus redeemers that are included in a transaction, as discussed
  below.

\end{itemize}

We add the following helper functions:

\begin{itemize}
  \item $\fun{language}$ selects the language of a non-native script, and $\Nothing$ in case
  the script is native.
  \item $\fun{hashData}$ hashes a value of type $\Data$.
  \item $\fun{hashWitnessPPData}$ hashes the protocol parameters and witnesses relevant to script execution.
\end{itemize}

\begin{figure*}[htb]
  \emph{Abstract types}
  %
  \begin{equation*}
    \begin{array}{r@{~\in~}l@{\quad\quad\quad\quad}r}
      \var{wph} &\WitnessPPDataHash & \text{Script data hash}\\
      \var{plc} & \ScriptPlutus & \text{Plutus core scripts} \\
      \var{d} & \Data & \text{Generic script data}
    \end{array}
  \end{equation*}
  %
  \emph{Script types}
  %
  \begin{equation*}
    \begin{array}{r@{~\in~}l@{\qquad=\qquad}lr}
      \var{sc} & \ScriptNonNative & \ScriptPlutus \\
      \var{sc} & \Script & \ScriptNative \uniondistinct \ScriptNonNative \\
      \var{isv} & \IsValidating & \Bool \\
      \var{d} & \Datum & \Data \\
      \var{r} & \Redeemer & \Data
    \end{array}
  \end{equation*}
%
  \emph{Derived types}
  %
  \begin{equation*}
    \begin{array}{r@{~\in~}l@{\qquad=\qquad}lr}
      \var{vi}
      & \ValidityInterval
      & \Slot^? \times \Slot^?
%      & \text{validity interval}
      \\
      \var{txout}
      & \TxOut
      & \Addr \times \Value \times \hldiff{\DataHash^?}
%      & \text{transaction outputs in a transaction}
      \\
      \var{tag}
      & \Tag
      & \{\mathsf{Spend},~\mathsf{Mint},~\mathsf{Cert},~\mathsf{Rewrd}\}
      \\
      \var{rdptr}
      & \RdmrPtr
      & \Tag \times \Ix
%      & \text{reverse pointer to thing dv is for}
    \end{array}
  \end{equation*}
  %
  \emph{Helper Functions}
  %
  \begin{align*}
    \fun{language}& \in \Script \to \Language^?\\
    \fun{hashData}& \in \Data \to \DataHash
    \nextdef
    \fun{hashWitnessPPData}&\in\PParams \to \powerset{\Language} \to (\RdmrPtr \mapsto \Redeemer \times \ExUnits)\\
    & ~~~~ \to \WitnessPPDataHash^? \\
    \fun{hashWitnessPPData}&~{pp}~{langs}~{rdmrs}~=~ \\
                          &\begin{cases}
                            \Nothing & rdmrs~=~\emptyset \land langs~=~\emptyset \\
                            \fun{hash} (rdmrs, \{ \fun{getLanguageView}~pp~l \mid l \in langs \}) & \text{otherwise}
                          \end{cases}
    \nextdef
    \fun{getCoin}& \in \TxOut \to \Coin \\
    \fun{getCoin}&~\hldiff{{(\_, \var{v}, \_)}} ~=~\fun{valueToCoin}~{v}
  \end{align*}
  \caption{Definitions for Transactions}
  \label{fig:defs:utxo-shelley-1}
\end{figure*}


\begin{figure*}[htb]
  \emph{Transaction Types}
  %
  \begin{equation*}
    \begin{array}{r@{~~}l@{~~}l@{\qquad}l}
      \var{wits} ~\in~ \TxWitness ~=~
       & (\VKey \mapsto \Sig) & \fun{txwitsVKey} & \text{VKey signatures}\\
       & \times ~(\ScriptHash \mapsto \Script)  & \fun{txscripts} & \text{All scripts}\\
       & \times~ \hldiff{(\DataHash \mapsto \Datum)} & \hldiff{\fun{txdats}} & \text{All datum objects}\\
       & \times ~\hldiff{(\RdmrPtr \mapsto \Redeemer \times \ExUnits)}& \hldiff{\fun{txrdmrs}}& \text{Redeemers/budget}\\
    \end{array}
  \end{equation*}
  %
  \begin{equation*}
    \begin{array}{r@{~~}l@{~~}l@{\qquad}l}
      \var{txbody} ~\in~ \TxBody ~=~
      & \powerset{\TxIn} & \fun{txinputs}& \text{Inputs}\\
      & \times ~\hldiff{\powerset{\TxIn}} & \hldiff{\fun{collateral}} & \text{Collateral inputs}\\
      & \times ~(\Ix \mapsto \TxOut) & \fun{txouts}& \text{Outputs}\\
      & \times~ \seqof{\DCert} & \fun{txcerts}& \text{Certificates}\\
       & \times ~\Value  & \fun{mint} &\text{A minted value}\\
       & \times ~\Coin & \fun{txfee} &\text{Total fee}\\
       & \times ~\ValidityInterval & \fun{txvldt} & \text{Validity interval}\\
       & \times~ \Wdrl  & \fun{txwdrls} &\text{Reward withdrawals}\\
       & \times ~\Update^?  & \fun{txUpdates} & \text{Update proposals}\\
       & \times ~\hldiff{\powerset{\KeyHash}} & \hldiff{\fun{reqSignerHashes}} & \text{Hashes of VKeys passed to scripts}\\
       & \times ~\hldiff{\WitnessPPDataHash^?} & \hldiff{\fun{wppHash}} & \text{Hash of script-related data}\\
       & \times ~\AuxiliaryDataHash^? & \fun{txADhash} & \text{Auxiliary data hash}\\
       & \times ~\hldiff{\Network^?} & \hldiff{\fun{txnetworkid}} & \text{Tx network ID}\\
    \end{array}
  \end{equation*}
  %
  \begin{equation*}
    \begin{array}{r@{~~}l@{~~}l@{\qquad}l}
      \var{ad} ~\in~ \AuxiliaryData ~=~
      & \seqof{\Script} & \fun{scripts}& \text{Optional scripts}\\
      & \times \hldiff{\powerset{\Datum}} & \hldiff{\fun{dats}}& \text{Optional datum objects}\\
      & \times ~\Metadata^? & \fun{txMD}&\text{Metadata}
    \end{array}
  \end{equation*}
  %
  \begin{equation*}
    \begin{array}{r@{~~}l@{~~}l@{\qquad}l}
      \var{tx} ~\in~ \Tx ~=~
      & \TxBody & \fun{txbody} & \text{Body}\\
      & \times ~\TxWitness & \fun{txwits} & \text{Witnesses}\\
      & \times ~\hldiff{\IsValidating} & \hldiff{\fun{isValidating}}&\text{Validation tag}\\
      & \times ~\AuxiliaryData^? & \fun{auxiliaryData}&\text{Auxiliary data}\\
    \end{array}
  \end{equation*}
  \caption{Definitions for transactions, cont.}
  \label{fig:defs:utxo-shelley-2}
\end{figure*}

\textbf{Auxiliary Data. }
Figure \ref{fig:defs:utxo-shelley-2} gives the Alonzo era auxiliary data found in a transaction.
In addition to a list of scripts and a metadata object, transaction auxiliary data now
contains a set of datum objects.

\subsection{Witnessing}
Figure \ref{fig:defs:utxo-shelley-2} defines the witness type, $\TxWitness$.  This contains everything
in a transaction that is needed for witnessing, namely:

\begin{itemize}
  \item VKey signatures;
  \item a map of scripts indexed by their hashes, including non-native scripts;
  \item a map of terms of type $\Datum$ indexed by their hashes, containing all required datum objects; and
  \item a map of a pair of a $\Redeemer$ object and an $\ExUnits$ value indexed by $\RdmrPtr$,
  containing the redeemers and execution units budgets.
\end{itemize}

Note that there is a difference between the way scripts and datum objects are included in
a transaction (as a set) versus how redeemers are included
(as an indexed structure).

\begin{description}
\item
  [Hash reference (script/datum object):]
  Scripts and datum objects are referred to explicitly via their hashes,
  which are included in the $\UTxO$ or the transaction. Thus, they can be
  looked up in the tranasaction without any key in the data structure.

  \item[No hash reference (redeemers):] For redeemers,
  we use a reverse pointer approach and
  index the redeemer by a pointer to the item for which it will be used.
  For details on how a script finds its redeemer, see Section~\ref{sec:scripts-inputs}.
\end{description}

\subsection{Transactions}
We have also made the following changes to
the body of the transaction:

\begin{itemize}
  \item There is a set of \emph{collateral inputs} that is used to cover a percentage of
    transaction fees (usually $100$ percent or more)
    \emph{in case the transaction contains failing non-native scripts}. They are not collected otherwise.
    The purpose of these is to cover the resource use costs incurred by block producers running not only
    scripts that validate, but the failing one also. Only the inputs that are locked by VKeys can
    be used as collateral, and they must only contain ada.

    It is permitted to use the same inputs as both a collateral and a regular inputs, as exactly
    one set of inputs is ever collected : collateral ones in the case of script failure, and regular inputs
    in the case when all scripts validate.

  \item There is a new field called $\fun{reqSignerHashes}$ that is used to specify a set of signers (via the
    hashes of their VKeys) that the scripts a transaction is carrying expect to have signed the transaction.
    These signatures are required in addition to the ones required by VKey addresses, and
    these keys are passed to the Plutus scipts during validation.

    Unlike signatures required by MSig scripts, the signatures of the keys
    (whose hashes are) included
    in this field can never be stripped from the transaction. This is because
    the hashes of the signing keys are listed inside the signed body.
  \item We include a hash of some script-related data, specifically the redeemers and protocol parameters,
    with accessor $\fun{wppHash}$.

  \item We include the ID of the network on which the transaction lives, $\fun{txnetworkid}$.
  This is in addition to the network ID's already included in the reward and output addresses.
\end{itemize}

A complete transaction is comprised of:

\begin{enumerate}
  \item the transaction body,
  \item all the information that is needed to witness transactions,
  \item the $\IsValidating$ tag, which indicates whether all scripts with all inputs
  that are executed during the script execution phase validate.
  The correctness of the tag is verified as part of the ledger rules, and the block is
  deemed to be invalid if it is applied incorrectly.
  If it is set to $\False$, then the block can be re-applied without script re-validation.
  This tag cannot be signed, since it is applied by the block producer.
  \item any auxiliary data.
\end{enumerate}

Note that unlike in Shelley or ShelleyMA eras, transactions inside blocks and transactions
a user submits are now of different types. A transaction inside a block is composed of
the user-submitted transaction paired with the $\IsValidating$ tag, which is
added by the block producer. The user cannot
add their own tag because of how this tag is used in transaction validation, see
Section~\ref{sec:two-phase}.

\subsection{Additional Role of Signatures on TxBody}

The transaction body and the UTxO must uniquely determine all the data
that can influence on-chain transfer of value.
This is so that the signatures can ensure that this data is not tampered with.
As before, a transaction that does not include the correct signatures is completely invalid.
Thus, all data that influences script validation outcomes must also be determined by the body.

Scripts and datum objects need not be included in the body,
since a hash of every script that is needed
is included in the body of the transaction or the UTxO (depending on what the script is validating),
and the hash of every datum that is needed is recorded in the UTxO.
%
The hash of the indexed redeemer structure and the protocol parameters that are used by
script interpreters are included in the body of the transaction. In the future, other parts of the ledger
state may also need to be included in this hash, if they are passed as
arguments to new script interpreter.

The owner of any tokens that are spent as part of a given transaction
must sign the transaction body. This means that
the owners of tokens that are spent by the transaction will have
signed their selection of the inputs that will be put in the fee pot in case of script validation failure, and
thus have had the opportunity to check the outcome of those scripts before signing the transaction body.

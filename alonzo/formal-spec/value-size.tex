\section{Output Size}
\label{sec:value-size}

Figure \ref{fig:test} gives the formula for calculating the size of a UTxO entry
in the Alonzo era. In addition to the data found in the UTxO in the ShelleyMA
era, the hash of a datum (or $\Nothing$) is added to the output type, which
is accounted for in the size calculation.

\begin{figure*}[h]
  \emph{Constants}
  \begin{align*}
  & \mathsf{JustDatHashSize} \in \MemoryEstimate \\
  & \text{The size of a datum hash wrapped in $\DataHash^?$} \\~
  \\
  & \mathsf{NothingSize} \in \MemoryEstimate \\
  & \text{The size of $\Nothing$ wrapped in $\DataHash^?$}
  \end{align*}
  %
  \emph{Helper Functions}
  \begin{align*}
    & \fun{datHashSize} \in \DataHash^? \to \MemoryEstimate \\
    & \fun{datHashSize}~ \Nothing = \mathsf{NothingSize} \\
    & \fun{datHashSize}~ \wcard = \mathsf{JustDatHashSize} \\
    & \text{Return the size of $\DataHash^?$} \\~\\
    & \fun{utxoEntrySize} \in \TxOut \to \MemoryEstimate \\
    & \fun{utxoEntrySize}~\var{(tout, d)} = \mathsf{utxoEntrySizeWithoutVal} + (\fun{size} (\fun{getValue}~tout)) + \mathsf{datHashSize}~d \\
    & \text{Calculate the size of a UTxO entry}
  \end{align*}
  \caption{Value Size}
  \label{fig:test}
\end{figure*}

\begin{note}
  Get datHashSize from heapwords on the code
\end{note}

How much memory is used exactly is an implementation detail, with the current constants, in Haskell words (8 bytes):

\begin{itemize}
  \item $\mathsf{JustDatHashSize} = $
  \item $\mathsf{NothingSize} = $
\end{itemize}

\section{Formal Properties}
\label{sec:properties}

This appendix collects the main formal properties that the new ledger rules are expected to satisfy.

\begin{enumerate}[label=P{\arabic*}:\ ]
\item
  \emph{Consistency with Shelley.}
  \begin{itemize}
    \item properties 15.6 - 15.16 (except 15.8 and 15.13) hold specifically in the $\fun{txValTag}~tx~=~\True$ case, because
    the calculations refer to the UTxO as if it was updated by a scripts-validating transaction
    \item other properties hold as-is
  \end{itemize}

\item
  \emph{Consistency with Multi-Asset.}
  \begin{itemize}
    \item properties 8.1 and 8.2 hold specifically in the $\fun{txValTag}~tx~=~\True$ case, because
    the calculations refer to the UTxO as if it was updated by a scripts-validating transaction
    \item other properties hold as-is
  \end{itemize}
\end{enumerate}


\begin{definition}
  For a state $s$ that is used in any subtransaction of
  $\mathsf{CHAIN}$, we define $\Utxo(s) \in \UTxO$ to be the $\UTxO$
  contained in $s$, or an empty map if it does not exist. This is
  similar to the definition of $\Val$ in the Shelley specification.

  Similarly, we also define $\field_{v}~(s)$ for the field $v$ that is part of
  the ledger state $s$, referenced by the typical variable name (eg. $\var{fees}$
  for the fee pot on the ledger).
\end{definition}

We also define a helper function $\varphi$ as follows:
\[\varphi(x, tx) :=
  \begin{cases}
    x & \fun{txValTag}~tx = \True \\
    0 & \fun{txValTag}~tx = \False
  \end{cases}\]
This function is useful to distinguish between the two
cases where a transaction can mint tokens or not, depending on whether
its scripts validate.

\begin{property}[General Accounting]
  \label{prop:pov}
  The \emph{general accounting} property in Alonzo encompasses two parts
  \begin{itemize}
    \item preservation of value property expressed via the $\fun{produced}$ and $\fun{consumed}$
    calculations, applicable for transactions with $\fun{txValTag}~tx~=~\True$, which
    is specified as in the ShelleMA POV.
    \item the preservation of value for $\fun{txValTag}~tx~=~\False$, when
    only the collateral fees are collected into the fee pot.
  \end{itemize}

Both of these are expressed in the following lemma.

\begin{lemma}
  For all environments $e$, transactions $tx$ and states $s, s'$, if
  \begin{equation*}
    e\vdash s\trans{utxos}{tx}s'
  \end{equation*}
  then
  \begin{equation*}
    \Val(s) + \varphi(\fun{mint}~(\fun{txbody}~tx), tx) = \Val(s')
  \end{equation*}
\end{lemma}

\begin{proof}
  In the case that $\fun{txValTag}~tx = \True$ holds, the proof is
  identical to the proof of Lemma 9.1 of the multi-asset
  specification. Otherwise, the transition must have been done via the
  $\mathsf{Scripts-No}$ rule, which removes
  $\fun{collateral}~(\fun{txbody}~tx)$ from the UTxO, and increases the fee pot by the amount of the sum of Ada in the
  collateral UTxO entries. The value contained in $s$ is not changed.

  Note that in the $\fun{feesOK}$ function, there is a check that verifies
  that, in the case that there are non-native scripts, there are no non-Ada tokens in the UTxOs
  which the collateral inputs reference, so non-Ada tokens do not get minted, burned, or transfered
  in this case.
\end{proof}
\end{property}

\begin{property}[No Changes Except Fees]
  \label{prop:fees-only}
  If a transaction is accepted and marked as paying collateral only
  (i.e. $\fun{txValTag}~tx = \False$), then the only change to the ledger
  when processing the transaction is that the collateral inputs
  are moved to the fee pot.

  \begin{lemma}
    For all environments $e$, transactions $tx$ and states $s, s'$, if $\fun{txvaltag}~tx = \False$ and
    \begin{equation*}
      e\vdash s\trans{ledger}{tx}s'
    \end{equation*}
    then
    \begin{itemize}
      \item $\Utxo(s') = \fun{collateral}~{tx} \subtractdom \Utxo(s)$
      \item $\field_{fees}~(s') = \field~\var{fees}~(s) + \fun{coin}~(\Val~(\fun{collateral}~{tx} \subtractdom \Utxo(s)))$
      \item $\field_{v}~(s') = \field_{v}~(s)$ for all $v~\neq~{fees}, ~v~\neq~{utxo}$
    \end{itemize}
  \end{lemma}

  \begin{proof}
  The top-level single-transaction processing transition is $\mathsf{LEDGER}$.
   In the case that $\fun{txValTag}~tx = \False$, this transition calls the rule
   that does not update $\DPState$ at all, only the $\UTxOState$. This state update is specified
   in the $\mathsf{UTXOS}$ transition (and applied via the $\mathsf{UTXO}$ and $\mathsf{UTXOW}$ transitions).

   The only parts of the state that are updated are the UTxO, where the collateral entries
   are removed, and the fee pot, which is increased by the amount of the sum of Ada in the
   collateral UTxO entries.
  \end{proof}
\end {property}

\begin{property}[Validating when No NN Scripts]
  \label{prop:is-valid}

Whenever a valid transaction does not have any non-native scripts, its
$\fun{txValTag} = \True$.

\begin{lemma}
  For all environments $e$, transactions $tx$ and states $s, s'$ such that
  \begin{equation*}
    e\vdash s\trans{ledger}{tx}s',
  \end{equation*}
  If $\range (\fun{txscripts}~tx) \cap \ScriptNonNative = \emptyset$
  then $\fun{txValTag} = \True$.
\end{lemma}
\begin{proof}
  With the same argument as previously, we only need to discuss the
  equivalent claim for the $\mathsf{UTXOS}$ transition. Under these
  assumptions, $\var{sLst} := \fun{collectTwoPhaseScriptInputs}$ is an empty
  list. Thus $\fun{evalScripts}~sLst = \True$, and the transition rule
  had to be $\mathsf{Scripts{-}Yes}$.
\end{proof}
\end{property}

\begin{property}[Paying fees]
  \label{prop:pay-fees}

A transaction always pays into the fee pot.

\begin{lemma}
  For all environments $e$, transactions $tx$ and states $s, s'$ such that
  \begin{equation*}
    e\vdash s\trans{ledger}{tx}s',
  \end{equation*}
  The following holds : $\field_{fees}~(s)~<~\field_{fees}~(s')$.
\end{lemma}
\begin{proof}
  The fee pot is updated by the $\mathsf{Scripts{-}Yes}$ and the $\mathsf{Scripts{-}No}$
  rules, one of which is necessarily called by a valid transaction.
\end{proof}
\end{property}

\begin{property}[Replay protection]
  \label{prop:replay}

A transaction always removes at least one UTxO entry from the ledger, which provides
replay protection.

\begin{lemma}
  For all environments $e$, transactions $tx$ and states $s, s'$ such that
  \begin{equation*}
    e\vdash s\trans{ledger}{tx}s',
  \end{equation*}
  The following holds : $\Utxo~(s)~\nsubseteq~\Utxo~(s')$.
\end{lemma}
\begin{proof}
  The UTxO is updated by the $\mathsf{Scripts{-}Yes}$ and the $\mathsf{Scripts{-}No}$
  rules, on of which is necessarily called by a valid transaction. Both of these
  rules remove UTxOs corresponding to a set of inputs.

  In both cases, there is a check that the removed inputs set is non-empty.
  The $\mathsf{Scripts{-}Yes}$ rule of the $\mathsf{UTXOS}$ transition
  removes the UTxOs associated with the input set $\fun{txinputs}~{tx}$ from the ledger.
  The $\fun{txinputs}~{tx}$ set must be non-empty because there is a check in the
  $\mathsf{UTXO}$ rule (which calls $\mathsf{UTXOS}$) that ensure this is true.

  For the $\mathsf{Scripts{-}No}$ rule of the $\mathsf{UTXOS}$ transition
  removes the UTxOs associated with the input set $\fun{collateral}~{tx}$ from the ledger.
  The $\fun{collateral}~{tx}$ set must be non-empty because
  the $\fun{feesOK}$ function (called by the same rule that calls $\mathsf{UTXO},
  \mathsf{UTXOS}$) ensures that in the case that the $tx$ contains non-native scripts,
  $\fun{collateral}~{tx}$ must be non-empty.

  Note that by property~\ref{prop:is-valid}, non-native scripts must always be present
  if $\fun{txValTag}~tx = \False$ (that is, whenever $\mathsf{Scripts{-}No}$ rule is used).
\end{proof}
\end{property}

\begin{property}[UTxO-changing transitions]
  \label{prop:utxo-change}

Only the $\mathsf{UTXOS}$ transition affects the ledger UTxO in Alonzo.

\begin{lemma}
  For all environments $e$, transitions $\mathsf{TRNS}$, transactions $tx$ and states $s, s'$ such that
  \begin{equation*}
    e\vdash s\trans{trns}{tx}s',
  \end{equation*}
  The following holds for some states $u, u'$ and environment $e'$ :
  \begin{equation*}
    \Utxo(s) \neq \Utxo (s')~ \Rightarrow~e'\vdash u\trans{utxos}{tx}u'~\wedge~\Utxo(s)=\Utxo (u)~\wedge~\Utxo(s')=\Utxo (u')
  \end{equation*}
\end{lemma}
\begin{proof}
  The $\mathsf{UTXOS}$ transition is the only one that updates the UTxO, but not via
  the result of another transition, so any update from the UTxO of $s$ to that of $s'$
  must be done by $\mathsf{UTXOS}$.
\end{proof}
\end{property}

\begin{enumerate}
\item
  \emph{Transaction Validation.}
  If a Shelley transaction is accepted, it is fully processed.

  What does this mean? Is this: Transaction consistency - any transaction that is successfully validated can be executed successfully by the interpreter.
\item
  \emph{Extended UTxO Validation.}
  If a transaction extends the UTxO, all its scripts validate, and
  if it has a script that does not validate, it cannot extend the
  UTxO.
\item
  \emph{Non-Token Minting.}
  A valid transaction that does not mint tokens satisfies the
  accounting property of the Shelley ledger where the type $\Coin$ is
  replaced by $\Value$.
\item
  \emph{Script consistency.} Scripts are only processed by valid version of the interpreter.
\item
  \emph{Script backwards compatibility.} All scripts that could be processed by any previous version of the interpreter
    remain valid indefinitely.
\item
  \emph{Cost consistency.} - no transaction exceeds the specified resource bounds.
\item
  \emph{Cost Increase.} if a transaction is valid, it will remain valid if you increase the execution units
\item
  \emph{Cost Lower Bound.} if a transaction contains at least one valid script, it must have at least one execution unit
\item
  \emph{Tx backwards Compatibility.} Any transaction that was accepted in a previous version of the ledger rules
    has exactly the same cost and effect, except that the transaction output is extended.
\item \emph{Run all scripts.} All scripts attached to a transaction are run
\item \emph{Scripts are run on correct inputs.}
  A script is run for everything witnessed with a non-native script, and the script will
get all the correct inputs, which must be present.
\item \emph{Determinism of script validation.} In an otherwise valid transaction which remains valid
  despite ledger state changes, the outcome
  of script validation is not affected by these ledger state changes.
\item \emph{Script inputs are fixed.} All data that a non-native script gets as input
is fixed either by being signed in a transaction, or fixed by its hash living on the ledger
\item
  ... \todo{Anything else?}
\end{enumerate}

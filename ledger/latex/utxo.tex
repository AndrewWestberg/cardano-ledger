\section{UTxO}
\label{sec:state-trans-utxo-1}

The transition rules for unspent outputs are presented in
Figure~\ref{fig:rules:utxo}. The states of the UTxO transition system,
along with their types are defined in Figure~\ref{fig:defs:utxo}.
Functions on these types are defined in Figure~\ref{fig:derived-defs:utxo}.

\begin{figure*}[htb]
  \emph{Primitive types}
  %
  \begin{equation*}
    \begin{array}{r@{~\in~}lr}
      \var{txid} & \TxId & \text{transaction id}\\
      %
      ix & \Ix & \text{index}\\
      %
      \var{addr} & \Addr & \text{address}\\
      %
      c & \Lovelace & \text{currency value}\\
      %
      pc & \PrtclConsts & \text{protocol constants}
    \end{array}
  \end{equation*}
  \emph{Derived types}
  %
  \begin{equation*}
    \begin{array}{r@{~\in~}l@{\qquad=\qquad}r@{~\in~}lr}
      \var{txin}
      & \TxIn
      & (\var{txid}, \var{ix})
      & \TxId \times \Ix
      & \text{transaction input}
      \\
      \var{txout}
      & \type{TxOut}
      & (\var{addr}, c)
      & \Addr \times \Lovelace
      & \text{transaction output}
      \\
      \var{utxo}
      & \UTxO
      & \var{txin} \mapsto \var{txout}
      & \TxIn \mapsto \TxOut
      & \text{unspent tx outputs}
    \end{array}
  \end{equation*}
  %
  \emph{Abstract types}
  \begin{equation*}
    \begin{array}{r@{~\in~}lr}
      \var{tx} & \Tx & \text{transaction}\\
    \end{array}
  \end{equation*}
  %
  \emph{Abstract Functions}
  \begin{equation*}
    \begin{array}{r@{~\in~}lr}
      \txid{} & \Tx \to \TxId & \text{compute transaction id}\\
      %
      \fun{txbody} & \Tx \to \powerset{\TxIn} \times (\Ix \mapsto \TxOut)
                                  & \text{transaction body}\\
      %
      \fun{txfee} & \Tx \to \Lovelace & \text{transaction fee}\\
      %
      \fun{minfee} & \PrtclConsts \to \Tx \to \Lovelace & \text{minimum fee}
    \end{array}
  \end{equation*}
  \caption{Definitions used in the UTxO transition system}
  \label{fig:defs:utxo}
\end{figure*}

\begin{figure}
  \begin{align*}
    & \fun{txins} \in \Tx \to \powerset{\TxIn}
    & \text{transaction inputs} \\
    & \txins{tx} = \var{inputs} \where \txbody{tx} = (\var{inputs}, ~\wcard)
    \nextdef
    & \fun{txouts} \in \Tx \to \UTxO
    & \text{transaction outputs as UTxO} \\
    & \fun{txouts} ~ \var{tx} =
      \left\{ (\fun{txid} ~ \var{tx}, \var{ix}) \mapsto \var{txout} ~
      \middle| \begin{array}{l@{~}c@{~}l}
                 (\_, \var{outputs}) & = & \txbody{tx} \\
                 \var{ix} \mapsto \var{txout} & \in & \var{outputs}
               \end{array}
      \right\}
    \nextdef
    & \fun{balance} \in \UTxO \to \Lovelace
    & \text{UTxO balance} \\
    & \fun{balance} ~ utxo = \sum_{(~\wcard ~ \mapsto (\wcard, ~c)) \in \var{utxo}} c
  \end{align*}
  \caption{Functions used in UTxO rules}
  \label{fig:derived-defs:utxo}
\end{figure}

\begin{figure}
  \emph{UTxO transitions}
  \begin{equation*}
    \_ \vdash
    \var{\_} \trans{utxo}{\_} \var{\_}
    \subseteq \powerset (\PrtclConsts \times \UTxO \times \Tx \times \UTxO)
  \end{equation*}
  \caption{UTxO transition-system types}
  \label{fig:ts-types:utxo}
\end{figure}

\begin{figure}
  \begin{equation}\label{eq:utxo-inductive}
    \inference
    { \txins{tx} \subseteq \dom \var{utxo} & \minfee{pc}{tx} \leq \txfee{tx}\\
      \balance{(\txouts{tx})}  + \txfee{tx} =
        \balance{(\txins{tx} \restrictdom \var{utxo})}
    }
    {\var{pc} \vdash \var{utxo} \trans{utxo}{tx}
      (\txins{tx} \subtractdom \var{utxo}) \cup \txouts{tx}
    }
  \end{equation}
  \caption{UTxO inference rules}
  \label{fig:rules:utxo}
\end{figure}

Rule~\ref{eq:utxo-inductive} specifies under which conditions a transaction can
be applied to a set of unspent outputs, and how the set of unspent output
changes with a transaction:
\begin{itemize}
\item Each input spent in the transaction must be in the set of unspent
  outputs.
\item The balance of the unspent outputs in a transaction (i.e. the total
  amount paid in a transaction) must be equal or less than the amount of spent
  inputs.
\item If the above conditions hold, then the new state will not have the inputs
  spent in transaction $\var{tx}$ and it will have the new outputs in
  $\var{tx}$.
\end{itemize}

\begin{note}
  $\Lovelace$ is defined as a primitive type, but there is a difference
  between implementing it with $\mathbb{N}$ versus $\mathbb{Z}$.
  Since this is a pure UTxO ledger, $\mathbb{N}$ suffices.
  If, however, $\mathbb{Z}$ is used, then extra validation is required
  to ensure that all $\TxOut$ are non-negative.
  This extra condition would be added to \cref{eq:utxo-inductive}.
\end{note}

\subsection{Properties}
\label{sec:utxo-properties}

\begin{todo}
  Can we prove properties of the transition system of this section? For
  instance we might like to formalize ``double spending'' and prove that these
  rules prevent it. Do we want it?
\end{todo}

\subsection{Witnesses}
\label{sec:witnesses}

The rules for witnesses are presented in Figure~\ref{fig:rules:utxow}.
The definitions used in Rule~\ref{eq:utxo-witness-inductive} are given in
Figure~\ref{fig:defs:utxow}. Note that
Rule~\ref{eq:utxo-witness-inductive} uses the transition relation defined in
Figure~\ref{fig:rules:utxo}. The main reason for doing this is to define
the rules incrementally, modeling different aspects in isolation to keep the
rules as simple as possible. Also note that the $\trans{utxo}{}$ relation could
have been defined in terms of $\trans{utxow}{}$ (thus composing the rules in a
different order). The choice here is arbitrary.

\begin{figure}
  \emph{Abstract functions}
  %
  \begin{equation*}
    \begin{array}{r@{~\in~}lr}
      \fun{wits} & \Tx \to \powerset{(\VKey \times \Sig)}
      & \text{witnesses of a transaction}\\
      \fun{hash_{spend}} & \Addr \mapsto \Hash
      & \text{hash of a spending key in an address}\\
    \end{array}
  \end{equation*}
  \caption{Definitions used in the UTxO transition system with witnesses}
  \label{fig:defs:utxow}
\end{figure}

\begin{figure}
  \begin{align*}
    & \addr{}{} \in \UTxO \to \TxIn \mapsto \Addr & \text{address of an input}\\
    & \addr{utxo} = \{ i \mapsto a \mid i \mapsto (a, \wcard) \in \var{utxo} \} \\
    \nextdef
    & \fun{addr_h} \in \UTxO \to \TxIn \mapsto \Hash & \text{hash of an input address}\\
    & \fun{addr_h}~utxo = \{ i \mapsto h \mid i \mapsto (a, \wcard) \in \var{utxo}
      \wedge a \mapsto h \in \fun{hash_{spend}} \}
  \end{align*}
  \caption{Functions used in rules witnesses}
  \label{fig:derived-defs:utxow}
\end{figure}

\begin{figure}
  \emph{UTxO with witness transitions}
  \begin{equation*}
    \var{\_} \vdash
    \var{\_} \trans{utxow}{\_} \var{\_}
    \subseteq \powerset (\PrtclConsts \times \UTxO \times \Tx \times \UTxO)
  \end{equation*}
  \caption{UTxO with witness transition-system types}
  \label{fig:ts-types:utxow}
\end{figure}

\begin{figure}
  \begin{equation}
    \label{eq:utxo-witness-inductive}
    \inference
    { \var{pc} \vdash \var{utxo} \trans{utxo}{tx} \var{utxo'}\\ ~ \\
      & \forall i \in \txins{tx} \cdot \exists (\var{vk}, \sigma) \in \wits{\var{tx}}
      \cdot
      \mathcal{V}_{\var{vk}}{\serialised{\txbody{tx}}}_{\sigma}
      \wedge  \fun{addr_h}~{utxo}~i = \hash{vk}\\
    }
    {\var{pc} \vdash \var{utxo} \trans{utxow}{tx} \var{utxo'}}
  \end{equation}
  \caption{UTxO with witnesses inference rules}
  \label{fig:rules:utxow}
\end{figure}

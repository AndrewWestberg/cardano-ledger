\section{Update mechanism}
\label{sec:update}

\newcommand{\UpdProp}{\ensuremath{\type{UpdProp}}}
\newcommand{\UpdPropBody}{\ensuremath{\type{UpdPropBody}}}
\newcommand{\ProtVer}{\ensuremath{\type{ProtVer}}}
\newcommand{\ProtAtt}{\ensuremath{\type{ProtAtt}}}
\newcommand{\ProtParams}{\ensuremath{\type{ProtParams}}}
\newcommand{\UPVEnv}{\ensuremath{\type{UPVEnv}}}
\newcommand{\UPVState}{\ensuremath{\type{UPVState}}}

\newcommand{\upSize}[1]{\ensuremath{\fun{upSize}~\var{#1}}}
\newcommand{\upPV}[1]{\ensuremath{\fun{upPV}~\var{#1}}}

\lstset{ frame=tb,
       , language=Haskell
       , basicstyle=\footnotesize\ttfamily,
       , keywordstyle=\color{blue!80},
       , commentstyle=\itshape\color{purple!40!black},
       , identifierstyle=\bfseries\color{green!40!black},
       , stringstyle=\color{orange},
       }

\lstMakeShortInline[columns=fixed]`

\subsection{Aspects that we need to model}
\label{sec:aspects-to-model}

\begin{description}
\item[Authentication] Update proposals and votes are authenticated (properly
  signed).
\item[Authorization] Only genesis keys (via deleg certs) can post update
  proposals.
  \begin{itemize}
  \item Only then they can vote on them.
  \end{itemize}
\item[Voting deadlines] voting ends when a majority of the voters (4/7 if we
assume no stake) agree on the proposal.
\item[Block versions] (= protocol versions)
\item[Soft-forks] a protocol version changes according to the fork-resolution
  rule (75\% of stake create blocks with new-version).
\item[Hard-forks] ??? Do we need to model anything here?.
\end{description}

In particular the rules need to capture:

\begin{enumerate}
\item Each genesis key can post (either directly or via its delegate), one update proposal per-epoch.
\item When does an update proposal becomes \textbf{active}.
\item We cannot add a proposal if there exists one with the same id\footnote{id
    = hash of the update proposal} that is already active.
\item If there are two proposals submitted for the same block version, then
  these proposals need to agree on the values of the updated parameters.
\end{enumerate}

Remarks:
\begin{itemize}
\item We do not distinguish between protocol or software updates. The ledger
  only cares about the mechanisms by which the protocol parameters are changed.
\end{itemize}

\subsection{Update proposals}
\label{sec:update-proposals}

\begin{figure}[htb]
  \emph{Abstract types}
  %
  \begin{equation*}
    \begin{array}{r@{~\in~}lr}
      \var{up} & \UpdProp & \text{(protocol) update proposal}\\
      \var{pa} & \ProtAtt & \text{protocol attribute}\\
    \end{array}
  \end{equation*}
  %
  \emph{Derived types}
  \begin{equation*}
    \begin{array}{r@{~\in~}l@{\qquad=\qquad}r@{~\in~}lr}
      \var{pv} & \ProtVer & (\var{maj}, \var{min}, \var{alt})
      & (\mathbb{N}, \mathbb{N}, \mathbb{N}) & \text{protocol version}\\
      \var{pps} & \ProtParams & \var{pps} & \ProtAtt \mapsto \Value & \text{protocol parameters}
    \end{array}
  \end{equation*}
  \emph{Abstract functions}
  %
  \begin{align*}
    & \fun{upIssuer} \in \UpdProp \to \VKey & \text{issuer of the update proposal}\\
    & \fun{upSize} \in \UpdProp \to \mathbb{N} & \text{update proposal size}\\
    & \fun{upPV} \in \UpdProp \to \mathbb{\ProtVer} & \text{update proposal protocol version}
  \end{align*}
  \caption{Update proposals definitions}
  \label{fig:defs:update-proposals}
\end{figure}

\subsection{Update proposals activation}
\label{sec:update-proposals-activation}

First we model the validity of a proposal.

\begin{figure}[htb]
  \emph{Update proposals validity environments}
  \begin{equation*}
    \UPVEnv =
    \left(
      \begin{array}{r@{~\in~}lr}
        \var{pv} & \ProtVer & \text{adopted (current) protocol version}\\
        \var{pps} & \ProtParams & \text{adopted protocol parameters}\\
      \end{array}
    \right)
  \end{equation*}
  %
  \emph{Update proposals validity states}
  \begin{equation*}
    \UPVState
    = \left(
      \begin{array}{r@{~\in~}lr}
        \var{aps} & \powerset{\UpdProp} & \text{active proposals}\\
      \end{array}
    \right)
  \end{equation*}
  %
  \emph{Update proposals validity transitions}
    \begin{equation*}
    \var{\_} \vdash
    \var{\_} \trans{UPV}{\_} \var{\_}
    \subseteq \powerset (\UPVEnv \times \UPVState \times \UpdProp \times \UPVState)
  \end{equation*}
  \caption{Update proposals validity transition-system types}
  \label{fig:ts-types:up-validity}
\end{figure}

Terse explanation of Rule~\ref{eq:rule:up-validity}: a new proposal:
\begin{itemize}
\item must not exceed the maximum side, as specified by the current
  protocol parameters.
\item must not exist in the set of active proposals.
\item must have a unique version among the current active proposals. This
  implies that a proposal is uniquely determined by its block version.
\item must increase one of the components (major, minor, or alternative) of the
  current version.
\end{itemize}

\begin{figure}[htb]
  \begin{equation}
    \label{eq:rule:up-validity}
    \inference
    {
      \var{maxUpSize} \mapsto \var{mus} \in \var{pps} & \upSize{up} \leq \var{mus}\\
      \var{up} \notin \var{aps}\\
      \langle \forall \var{up'} \in \var{aps} \cdot \upPV{up'} \neq \upPV{up} \rangle\\
      \var{pv} = (\var{mj_a}, \var{mi_a}, \var{a_a})
      & \upPV{up} = (\var{mj_p}, \var{mi_p}, \var{a_p})\\
      & |\var{mj_a} - \var{mj_p}|
      + |\var{mi_a} - \var{mi_p}|
      + |\var{a_a} - \var{a_p}| \leq 1
      & (\var{mj_p}, \var{mi_p}, \var{a_p}) < (\var{mj_a}, \var{mi_a}, \var{a_a})\\
    }
    {
      {
        \begin{array}{l}
          \var{pv}\\
          \var{pps}
        \end{array}
      }
      \vdash
      {
        \left(
          \begin{array}{l}
            \var{aps}
          \end{array}
        \right)
      }
      \trans{UPV}{\var{up}}
      {
        \left(
          \begin{array}{l}
            \var{aps \cup \{\var{up}\}}
          \end{array}
        \right)
      }
    }
  \end{equation}
  \caption{Update proposals validity rules}
  \label{fig:rules:up-validity}
\end{figure}

\clearpage

\begin{figure}[htb]
  \emph{Update proposals limits  environments}
    \begin{equation*}
    \UPVEnv =
    \left(
      \begin{array}{r@{~\in~}lr}
        \var{e} & \Epoch & \text{current epoch}
      \end{array}
    \right)
  \end{equation*}
  \caption{Update proposals limits transition-system types}
  \label{fig:ts-types:up-limits}
\end{figure}

\subsection{Voting on update proposals}
\label{sec:voting-on-update-proposals}


\subsection{Update proposals confirmation}
\label{sec:update-proposals-confirmation}


\subsection{Update proposals adoption}
\label{sec:update-proposals-adoption}


\subsection{Information in the ledger state}
\label{sec:information-in-ledger-state}

The ledger state has to expose some parameters of the protocol version to its
clients. In the current implementation these parameters are kept in the
`BlockVersionData` structure.

\begin{lstlisting}
  data BlockVersionData = BlockVersionData { ... }
\end{lstlisting}

The following parameters are likely to be needed by the consumers of the ledger
layer:

\begin{itemize}
\item Slot duration (`bvdSlotDuration`)
\item Maximum block size (`bvdMaxBlockSize`)
\item Maximum header size (`bvdMaxHeaderSize`)
\item Maximum transaction size (`bvdMaxTxSize`)
\item Maximum proposal size (`bvdMaxProposalSize`)
\item Transaction fee policy (`bvdTxFeePolicy`)
\end{itemize}

The following parameters will be assumed to be constant:
\begin{itemize}
\item MPC threshold (`bvdMpcThd`)
\item Heavy delegation threshold (`bvdHeavyDelThd`)
\item Update vote threshold (`bvdUpdateVoteThd`)
\item Update proposal threshold (`bvdUpdateProposalThd`)
\item Soft fork rule parameters (`bvdSoftforkRule`)
\end{itemize}

At the moment we don't know whether we need these:

\begin{itemize}
\item Script Version (`bvdScriptVersion`)
\item Update implicit (`bvdUpdateImplicit`)
\end{itemize}

Finally, the `bvdUnlockStakeEpoch` field of `BlockVersionData` does not need to
be modeled.

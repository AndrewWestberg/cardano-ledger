\documentclass[11pt,a4paper]{article}
\usepackage[margin=2.5cm]{geometry}
\usepackage{iohk}
\usepackage{microtype}
\usepackage{mathpazo} % nice fonts
\usepackage{amsmath}
\usepackage{amssymb}
\usepackage{latexsym}
\usepackage{mathtools}
\usepackage{stmaryrd}
\usepackage{extarrows}
\usepackage{slashed}
\usepackage[colon]{natbib}
\usepackage[unicode=true,pdftex,pdfa]{hyperref}
\usepackage{xcolor}
\usepackage[capitalise,noabbrev,nameinlink]{cleveref}
\usepackage{float}
\floatstyle{boxed}
\restylefloat{figure}
\usepackage{listings} % for code blocks.
%%
%% Package `semantic` can be used for writing inference rules.
%%
\usepackage{semantic}
%% Setup for the semantic package
\setpremisesspace{20pt}

%%
%% Types
%%
\newcommand{\Tx}{\type{Tx}}
\newcommand{\Ix}{\type{Ix}}
\newcommand{\TxId}{\type{TxId}}
\newcommand{\Addr}{\type{Addr}}
\newcommand{\UTxO}{\type{UTxO}}
\newcommand{\Value}{\type{Value}}
\newcommand{\Coin}{\type{Coin}}
\newcommand{\PrtclConsts}{\type{PrtclConsts}}
%% Adding witnesses
\newcommand{\TxIn}{\type{TxIn}}
\newcommand{\TxOut}{\type{TxOut}}
\newcommand{\VKey}{\type{VKey}}
\newcommand{\SKey}{\type{SKey}}
\newcommand{\Hash}{\type{Hash}}
\newcommand{\SkVk}{\type{SkVk}}
\newcommand{\Sig}{\type{Sig}}
\newcommand{\Data}{\type{Data}}
%% Adding delegation
\newcommand{\Epoch}{\type{Epoch}}
\newcommand{\VKeyGen}{\type{VKeyGen}}
%% Blockchain
\newcommand{\Gkeys}{\var{G_{keys}}}
\newcommand{\Block}{\type{Block}}
\newcommand{\SlotId}{\type{SlotId}}
\newcommand{\CEEnv}{\type{CEEnv}}
\newcommand{\CEState}{\type{CEState}}
\newcommand{\BDEnv}{\type{BDEnv}}
\newcommand{\BDState}{\type{BDState}}

%%
%% Functions
%%
\newcommand{\txins}[1]{\fun{txins}~ \var{#1}}
\newcommand{\txid}[1]{\fun{txid}~ \var{#1}}
\newcommand{\txouts}[1]{\fun{txouts}~ \var{#1}}
\newcommand{\values}[1]{\fun{values}~ #1}
\newcommand{\balance}[1]{\fun{balance}~ \var{#1}}
%% UTxO witnesses
\newcommand{\inputs}[1]{\fun{inputs}~ \var{#1}}
\newcommand{\wits}[1]{\fun{wits}~ \var{#1}}
\newcommand{\verify}[3]{\fun{verify} ~ #1 ~ #2 ~ #3}
\newcommand{\sign}[2]{\fun{sign} ~ #1 ~ #2}
\newcommand{\serialised}[1]{\llbracket \var{#1} \rrbracket}
\newcommand{\addr}[1]{\fun{addr}~ \var{#1}}
\newcommand{\hash}[1]{\fun{hash}~ \var{#1}}
\newcommand{\txbody}[1]{\fun{txbody}~ \var{#1}}
\newcommand{\txfee}[1]{\fun{txfee}~ \var{#1}}
\newcommand{\minfee}[2]{\fun{minfee}~ \var{#1}~ \var{#2}}
% wildcard parameter
\newcommand{\wcard}[0]{\underline{\phantom{a}}}
%% Adding ledgers...
\newcommand{\utxo}[1]{\fun{utxo}~ #1}
%% Delegation
\newcommand{\delegatesName}{\fun{delegates}}
\newcommand{\delegates}[3]{\delegatesName~#1~#2~#3}
\newcommand{\dwho}[1]{\fun{dwho}~\var{#1}}
\newcommand{\depoch}[1]{\fun{depoch}~\var{#1}}
%% Delegation witnesses
\newcommand{\dbody}[1]{\fun{dbody}~\var{#1}}
\newcommand{\dwit}[1]{\fun{dwit}~\var{#1}}
%% Blockchain
\newcommand{\bwit}[1]{\fun{bwit}~\var{#1}}
\newcommand{\bslot}[1]{\fun{bslot}~\var{#1}}
\newcommand{\bbody}[1]{\fun{bbody}~\var{#1}}
\newcommand{\bdlgs}[1]{\fun{bdlgs}~\var{#1}}

\begin{document}

\hypersetup{
  pdftitle={A Formal Specification of the Cardano Ledger},
  breaklinks=true,
  bookmarks=true,
  colorlinks=false,
  linkcolor={blue},
  citecolor={blue},
  urlcolor={blue},
  linkbordercolor={white},
  citebordercolor={white},
  urlbordercolor={white}
}

\title{A Formal Specification of the Cardano Ledger}

\author{Jared Corduan  \\ {\small \texttt{jared.corduan@iohk.io}} \\
   \and Polina Vinogradova \\ {\small \texttt{polina.vinogradova@iohk.io}} \\
   \and Matthias G\"udemann  \\ {\small \texttt{matthias.gudemann@iohk.io}}}

%\date{}

\maketitle

\begin{abstract}
This documents defines the rules for extending a ledger with transactions.
The transactions will affect both UTxO and stake delegation.
It is intended to serve as the specification for random generators of transactions
which adhere to the rules presented here.
\end{abstract}

\section*{List of Contributors}
\label{acknowledgements}

Nicol\'as Arqueros,
Nicholas Clarke,
Duncan Coutts,
Ruslan Dudin,
Sebastien Guillemot,
Vincent Hanquez,
Ru Horlick,
Michael Hueschen,
Philipp Kant,
Jean-Christophe Mincke,
Damian Nadales,
Nicolas Di Prima.


\tableofcontents
\listoffigures

\section{Introduction}
\label{sec:introduction}
This document is a formal specification of the functionality of the ledger
on the blockchain. The blockchain layer of the
protocol and the interaction between the ledger and the blockchain
layer is presented in a separate document, see~\cite{shelley_consensus}. The details of the
background and the larger context
for the design decisions formalized in this document are presented
in~\cite{delegation_design}

In this work,
we present important properties any implementation of the ledger must have.
Specifically, we model the following aspects
of the functionality of the ledger on the blockchain:

\begin{description}
\item[Preservation of value] relationship between the total value of input and
  outputs in a new transaction, and the unspent outputs.
\item[Witnesses] authentication of parts of the transaction data by means of
  cryptographic entities (such as signatures and private keys) contained in
  these transactions.
\item[Delegation] validity of delegation certificates, which delegate
  block-signing rights.
\item[Stake] staking rights associated to an address.
\end{description}

While the blockchain protocol is a reactive system driven by the arrival
of blocks causing updates to the ledger, the formal description is a collection
of rules which is a
static description of what a \textit{valid ledger} is. The specifics of the
semantics we use to define and apply
the rules we present in this document are explained in detail in
\cite{small_step_semantics}. A valid ledger state can only
reached by applying a sequence of inference rules, and any valid ledger state
is reachable by applying some sequence of these rules.

The structure of the rules we give here is such that their application is
deterministic. That is, given a specific initial state and relevant environmental
constants, there is no ambiguity
about which rule should be applied at any given time (i.e. which state
transition is allowed take place). This is an important property which reflects
the reality of the implementation - the blockchain evolves in a particular way
given some user activity and the passage of time, and its behaviour is
never unexpected.


\section{Notation}\label{sec:notation}

\begin{description}
\item[Powerset] Given a set $\type{X}$, $\powerset{\type{X}}$ is the set of all
  the subsets of $X$.
\item[Sequences] Given a set $\type{X}$, $\seqof{\type{X}}$ is the set of
  sequences having elements taken from $\type{X}$. The empty sequence is
  denoted by $\epsilon$, and given a sequence $\Lambda$, $\Lambda; \type{x}$ is
  the sequence that results from appending $\type{x} \in \type{X}$ to
  $\Lambda$.
\item[Functions] $A \to B$ denotes a \textbf{total function} from $A$ to $B$.
  Given a function $f$ we write $f~a$ for the application of $f$ to argument
  $a$.
\item[Fibre] Given a function $f: A \to B$ and $b\in B$, we write
  $f^{-1}~b$ for the \textbf{fibre} of $f$ at $b$, which is defined by
  $\{a \mid\ f a =  b\}$.
\item[Maps and partial functions] $A \mapsto B$ denotes a \textbf{partial
    function} from $A$ to $B$, which can be seen as a map (dictionary) with
  keys in $A$ and values in $B$. Given a map $m \in A \mapsto B$, notation
  $a \mapsto b \in m$ is equivalent to $m~ a = b$.
\end{description}

\section{Cryptographic primitives}
\label{sec:crypto-primitives}

Figure~\ref{fig:crypto-defs} introduces the cryptographic abstractions used in
this document.

\begin{figure}
  \emph{Abstract types}
  %
  \begin{equation*}
    \begin{array}{r@{~\in~}lr}
      \var{vk} & \SKey & \text{signing key}\\
      \var{vk} & \VKey & \text{verifying key}\\
      \var{hk} & \Hash & \text{hash of a key}\\
      \sigma & \Sig  & \text{signature}\\
      \var{d} & \Data  & \text{data}\\
    \end{array}
  \end{equation*}
  \emph{Derived types}
  \begin{equation*}
    \begin{array}{r@{~\in~}lr}
      (sk, vk) & \SkVk & \text{signing-verifying key pairs}
    \end{array}
  \end{equation*}
  \emph{Abstract functions}
  %
  \begin{equation*}
    \begin{array}{r@{~\in~}lr}
      \hash{} & \VKey \to \Hash
      & \text{hash function} \\
      %
      \fun{verify} & \VKey \times \Data \times \Sig
      & \text{verification relation}\\
    \end{array}
  \end{equation*}
  \emph{Constraints}
  \begin{align*}
    & \forall (sk, vk) \in \SkVk,~ m \in \Data,~ \sigma \in \Sig \cdot
      \verify{vk}{m}{\sigma} \iff \sign{sk}{m} = \sigma
  \end{align*}
  \emph{Notation for serialized and verified data}
  \begin{align*}
    & \serialised{x} & \text{serialised representation of } x\\
    & \mathcal{V}_{\var{vk}}{\serialised{m}}_{\sigma} = \verify{vk}{m}{\sigma}
      & \text{shorthand notation for } \fun{verify}
  \end{align*}
  \caption{Cryptographic definitions}
  \label{fig:crypto-defs}
\end{figure}

\section{Serialization}
\label{sec:serialization}

\begin{todo}
  Discuss here serialization and
  \href{https://iohk.myjetbrains.com/youtrack/issue/CDEC-628}{composable
    serialization}
\end{todo}

\section{UTxO}
\label{sec:state-trans-utxo-1}

The transition rules for unspent outputs are presented in
Figure~\ref{fig:rules:utxo}. The states of the UTxO transition system,
along with their types are defined in Figure~\ref{fig:defs:utxo}.
Functions on these types are defined in Figure~\ref{fig:derived-defs:utxo}.

\begin{figure*}
  \emph{Primitive types}
  %
  \begin{equation*}
    \begin{array}{r@{~\in~}lr}
      \var{txid} & \TxId & \text{transaction id}\\
      %
      ix & \Ix & \text{index}\\
      %
      \var{addr} & \Addr & \text{address}\\
      %
      c & \Coin & \text{currency value}\\
      %
      pc & \PrtclConsts & \text{protocol constants}
    \end{array}
  \end{equation*}
  \emph{Derived types}
  %
  \begin{equation*}
    \begin{array}{r@{~\in~}l@{\qquad=\qquad}r@{~\in~}lr}
      \var{txin}
      & \TxIn
      & (\var{txid}, \var{ix})
      & \TxId \times \Ix
      & \text{transaction input}
      \\
      \var{txout}
      & \type{TxOut}
      & (\var{addr}, c)
      & \Addr \times \Coin
      & \text{transaction output}
      \\
      \var{utxo}
      & \UTxO
      & \var{txin} \mapsto \var{txout}
      & \TxIn \mapsto \TxOut
      & \text{unspent tx outputs}
    \end{array}
  \end{equation*}
  %
  \emph{Abstract types}
  \begin{equation*}
    \begin{array}{r@{~\in~}lr}
      \var{tx} & \Tx & \text{transaction}\\
    \end{array}
  \end{equation*}
  %
  \emph{Abstract Functions}
  \begin{equation*}
    \begin{array}{r@{~\in~}lr}
      \txid{} & \Tx \to \TxId & \text{compute transaction id}\\
      %
      \fun{txbody} & \Tx \to \powerset{\TxIn} \times (\Ix \mapsto \TxOut)
                                  & \text{transaction body}\\
      %
      \fun{txfee} & \Tx \to \Coin & \text{transaction fee}\\
      %
      \fun{minfee} & \PrtclConsts \to \Tx \to \Coin & \text{minimum fee}
    \end{array}
  \end{equation*}
  \caption{Definitions used in the UTxO transition system}
  \label{fig:defs:utxo}
\end{figure*}

\begin{figure}
  \begin{align*}
    & \fun{txins} \in \Tx \to \powerset{\TxIn}
    & \text{transaction inputs} \\
    & \txins{tx} = \var{inputs} \where \txbody{tx} = (\var{inputs}, ~\wcard)
    \nextdef
    & \fun{txouts} \in \Tx \to \UTxO
    & \text{transaction outputs as UTxO} \\
    & \fun{txouts} ~ \var{tx} =
      \left\{ (\fun{txid} ~ \var{tx}, \var{ix}) \mapsto \var{txout} ~
      \middle| \begin{array}{l@{~}c@{~}l}
                 (\_, \var{outputs}) & = & \txbody{tx} \\
                 \var{ix} \mapsto \var{txout} & \in & \var{outputs}
               \end{array}
      \right\}
    \nextdef
    & \fun{balance} \in \UTxO \to \Coin
    & \text{UTxO balance} \\
    & \fun{balance} ~ utxo = \sum_{(~\wcard ~ \mapsto (\wcard, ~c)) \in \var{utxo}} c
  \end{align*}

  \begin{align*}
    \var{ins} \restrictdom \var{utxo}
    & = \{ i \mapsto o \mid i \mapsto o \in \var{utxo}, ~ i \in \var{ins} \}
    & \text{domain restriction}
    \\
    \var{ins} \subtractdom \var{utxo}
    & = \{ i \mapsto o \mid i \mapsto o \in \var{utxo}, ~ i \notin \var{ins} \}
    & \text{domain exclusion}
    \\
    \var{utxo} \restrictrange \var{outs}
    & = \{ i \mapsto o \mid i \mapsto o \in \var{utxo}, ~ o \in \var{outs} \}
    & \text{range restriction}
  \end{align*}
  \caption{Functions used in UTxO rules}
  \label{fig:derived-defs:utxo}
\end{figure}

\begin{figure}
  \emph{UTxO transitions}
  \begin{equation*}
    \_ \vdash
    \var{\_} \trans{utxo}{\_} \var{\_}
    \subseteq \powerset (\PrtclConsts \times \UTxO \times \Tx \times \UTxO)
  \end{equation*}
  \caption{UTxO transition-system types}
  \label{fig:ts-types:utxo}
\end{figure}

\begin{figure}
  \begin{equation}\label{eq:utxo-inductive}
    \inference[UTxO-inductive]
    { \txins{tx} \subseteq \dom \var{utxo} & \minfee{pc}{tx} \leq \txfee{tx}\\
      \balance{(\txouts{tx})}  + \txfee{tx} =
        \balance{(\txins{tx} \restrictdom \var{utxo})}
    }
    {\var{pc} \vdash \var{utxo} \trans{utxo}{tx}
      (\txins{tx} \subtractdom \var{utxo}) \cup \txouts{tx}
    }
  \end{equation}
  \caption{UTxO inference rules}
  \label{fig:rules:utxo}
\end{figure}

Rule~\ref{eq:utxo-inductive} specifies under which conditions a transaction can
be applied to a set of unspent outputs, and how the set of unspent output
changes with a transaction:
\begin{itemize}
\item Each input spent in the transaction must be in the set of unspent
  outputs.
\item The balance of the unspent outputs in a transaction (i.e. the total
  amount paid in a transaction) must be equal or less than the amount of spent
  inputs.
\item If the above conditions hold, then the new state will not have the inputs
  spent in transaction $\var{tx}$ and it will have the new outputs in
  $\var{tx}$.
\end{itemize}

\begin{note}
  $\Coin$ is defined as a primitive type, but there is a difference
  between implementing it with $\mathbb{N}$ versus $\mathbb{Z}$.
  Since this is a pure UTxO ledger, $\mathbb{N}$ suffices.
  If, however, $\mathbb{Z}$ is used, then extra validation is required
  to ensure that all $\TxOut$ are non-negative.
  This extra condition would be added to \cref{eq:utxo-inductive}.
\end{note}

\subsection{Properties}
\label{sec:utxo-properties}

\begin{todo}
  Can we prove properties of the transition system of this section? For
  instance we might like to formalize ``double spending'' and prove that these
  rules prevent it. Do we want it?
\end{todo}

\subsection{Witnesses}
\label{sec:witnesses}

The rules for witnesses are presented in Figure~\ref{fig:rules:utxow}.
The definitions used in Rule~\ref{eq:utxo-witness-inductive} are given in
Figure~\ref{fig:defs:utxow}. Note that
Rule~\ref{eq:utxo-witness-inductive} uses the transition relation defined in
Figure~\ref{fig:rules:utxo}. The main reason for doing this is to define
the rules incrementally, modeling different aspects in isolation to keep the
rules as simple as possible. Also note that the $\trans{utxo}{}$ relation could
have been defined in terms of $\trans{utxow}{}$ (thus composing the rules in a
different order). The choice here is arbitrary.

\begin{figure}
  \emph{Abstract functions}
  %
  \begin{equation*}
    \begin{array}{r@{~\in~}lr}
      \fun{wits} & \Tx \to \powerset{(\VKey \times \Sig)}
      & \text{witnesses of a transaction}\\
      \fun{hash_{spend}} & \Addr \mapsto \Hash
      & \text{hash of a spending key in an address}\\
    \end{array}
  \end{equation*}
  \caption{Definitions used in the UTxO transition system with witnesses}
  \label{fig:defs:utxow}
\end{figure}

\begin{figure}
  \begin{align*}
    & \addr{}{} \in \UTxO \to \TxIn \mapsto \Addr & \text{address of an input}\\
    & \addr{utxo} = \{ i \mapsto a \mid i \mapsto (a, \wcard) \in \var{utxo} \} \\
    \nextdef
    & \fun{addr_h} \in \UTxO \to \TxIn \mapsto \Hash & \text{hash of an input address}\\
    & \fun{addr_h}~utxo = \{ i \mapsto h \mid i \mapsto (a, \wcard) \in \var{utxo}
      \wedge a \mapsto h \in \fun{hash_{spend}} \}
  \end{align*}
  \caption{Functions used in rules witnesses}
  \label{fig:derived-defs:utxow}
\end{figure}

\begin{figure}
  \emph{UTxO with witness transitions}
  \begin{equation*}
    \var{\_} \vdash
    \var{\_} \trans{utxow}{\_} \var{\_}
    \subseteq \powerset (\PrtclConsts \times \UTxO \times \Tx \times \UTxO)
  \end{equation*}
  \caption{UTxO with witness transition-system types}
  \label{fig:ts-types:utxow}
\end{figure}

\begin{figure}
  \begin{equation}
    \label{eq:utxo-witness-inductive}
    \inference[UTxO-wit]
    { \var{pc} \vdash \var{utxo} \trans{utxo}{tx} \var{utxo'}\\ ~ \\
      & \forall i \in \txins{tx} \cdot \exists (\var{vk}, \sigma) \in \wits{\var{tx}}
      \cdot
      \mathcal{V}_{\var{vk}}{\serialised{\txbody{tx}}}_{\sigma}
      \wedge  \fun{addr_h}~{utxo}~i = \hash{vk}\\
    }
    {\var{pc} \vdash \var{utxo} \trans{utxow}{tx} \var{utxo'}}
  \end{equation}
  \caption{UTxO with witnesses inference rules}
  \label{fig:rules:utxow}
\end{figure}

\section{Delegation}
\label{sec:delegation}
%%
%% Types
%%
\newcommand{\AddrRWD}{\type{Addr_{rwd}}}
\newcommand{\DState}{\type{DState}}
\newcommand{\DWState}{\type{DWState}}
\newcommand{\DWEnv}{\type{DWEnv}}
\newcommand{\PState}{\type{PState}}
\newcommand{\DCertBody}{\type{DCertBody}}

%%
%% Functions
%%
\newcommand{\RegKey}[1]{\textsc{RegKey}(#1)}
\newcommand{\DeregKey}[1]{\textsc{DeregKey}(#1)}
\newcommand{\Delegate}[1]{\textsc{Delegate}(#1)}
\newcommand{\RegPool}[1]{\textsc{RegPool}(#1)}
\newcommand{\RetirePool}[1]{\textsc{RetirePool}(#1)}
\newcommand{\cauthor}[1]{\fun{author}~ \var{#1}}
\newcommand{\pool}[1]{\fun{pool}~ \var{#1}}
\newcommand{\retire}[1]{\fun{retire}~ \var{#1}}
\newcommand{\addrRw}[1]{\fun{addr_{rwd}}~ \var{#1}}

%%
%% Constants
%%
\newcommand{\emax}{\mathsf{E_{max}}}

We briefly describe the motivation and context for delegation.
The full context is contained in \cite{delegation_design}.

For stake to be active in the blockchain protocol,
(i.e. eligible for participation in the leader election)
the associated verification stake key must be registered
and its staking rights must be delegated to an active stake pool.
\footnote{Individuals who wish to participate in the protocol can
register themselves as a stake pool.}

Stake keys are registered (deregistered) through the use of
registration (deregistration) certificates.
Registered stake keys are delegated through the use of delegation certificates.
Finally, stake pools are registered (retired) through the use of
registration (retirement) certificates.

Stake pool retirement is handled a bit differently than stake key deregistration.
Stake keys are considered inactive as soon as a deregistration certificate
is applied to the ledger state.
Stake pool retirement certificates, however, specify the epoch in
which it will retire.

Delegation requires the following to be tracked by the ledger state:
the registered stake keys, the delegation map from registered stake keys to stake
pools, the registered stake pools, and upcoming stake pool retirements.
Additionally, the blockchain protocol rewards eligible stake, and so we must
also include a mapping from active stake keys to rewards.

In \cref{fig:delegation-definitons} we give the delegation primitives,
and in \cref{fig:delegation-transitions} we give the delegation transition rule.

The rules for registering and delegating stake keys are given in \cref{fig:delegation-rules}.
The rules for registering stake pools are given in \cref{fig:pool-rules}.

\begin{note}
  The current rules allow for delegation to a non-existent pool.
  Such stake is not eligible for leader election.
  This allows for a clean separation between the rules in
  \cref{fig:delegation-rules} and \cref{fig:pool-rules}.
  We may, however, later choose to enforce that delegation certificates
  target a registered pool. It would then make sense to remove
  mappings in $\var{delegations}$ when stake pools retire.
\end{note}

%%
%% Figure - Delegation Definitions
%%
\begin{figure}
  \emph{Abstract types}
  %
  \begin{equation*}
    \begin{array}{r@{~\in~}lr}
      a & \AddrRWD & \text{reward address} \\
      epoch & \Epoch & \text{epoch}
    \end{array}
  \end{equation*}
  %
  \emph{Constants}
  \begin{equation*}
    \begin{array}{r@{~\in~}lr}
      \emax & \Epoch & \text{epoch bound on pool retirement}
    \end{array}
  \end{equation*}
  %
  \emph{Delegation Certificate types}
  %
  \begin{equation*}
  \begin{array}{r@{}c@{}l}
    \DCert &=& \DCertRegKey \uniondistinct \DCertDeRegKey \uniondistinct \DCertDeleg \\
                &\hfill\uniondistinct\;& \DCertRegPool \uniondistinct \DCertRetirePool \\
    \RegKey{c} \in \DCert &\iff& c \in \DCertRegKey \\
    \DeregKey{c} \in \DCert &\iff& c \in \DCertDeRegKey \\
    \Delegate{c} \in \DCert &\iff& c \in \DCertDeleg \\
    \RegPool{c} \in \DCert &\iff& c \in \DCertRegPool\\
    \RetirePool{c} \in \DCert &\iff& c \in \DCertRetirePool \\
  \end{array}
  \end{equation*}
  %
  \emph{Abstract functions}
  %
  \begin{equation*}
  \begin{array}{r@{~\in~}lr}
  \fun{hash} & \VKey \to \Hash
  & \text{hashing a key}
  \\
  \fun{addr_{rwd}} & \Hash \to \AddrRWD
  & \text{address of a hashkey}
  \\
  \fun{author} & \DCert \to \Hash
  & \text{certificate author}
  \\
  \fun{pool} & \DCertDeleg \to \Hash
  & \text{pool being delegated to}
  \\
  \fun{retire} & \DCertRetirePool \to \Epoch
  & \text{epoch of pool retirement}
  \end{array}
  \end{equation*}
  %

  \caption{Delegation Definitions}
  \label{fig:delegation-definitons}
\end{figure}

%%
%% Figure - Delegation Transitions
%%
\begin{figure}
  \emph{Delegation States}
  %
  \begin{equation*}
    \begin{array}{l}
    \DState =
    \left(\begin{array}{r@{~\in~}lr}
      \var{stkeys} & \powerset (\Hash) & \text{registered stake keys}\\
      \var{rewards} & \AddrRWD \mapsto \Coin & \text{rewards}\\
      \var{delegations} & \Hash \mapsto \Hash & \text{delegations}\\
    \end{array}\right)
    \\
    \\
    \PState =
    \left(\begin{array}{r@{~\in~}lr}
      \var{stpools} & \Hash \mapsto \DCertRegPool & \text{registered stake pools}\\
      \var{retiring} & \Hash \mapsto \Epoch & \text{retiring stake pools}\\
    \end{array}\right)
    \end{array}
  \end{equation*}
  %
  \emph{Delegation Transitions}
  \begin{equation*}
    \_ \trans{deleg}{\_} \_ \in
      \powerset (\DState \times \DCert \times \DState)
  \end{equation*}
  %
  \begin{equation*}
    \_ \vdash \_ \trans{pool}{\_} \_ \in
      \powerset (\Epoch \times \DState \times \DCert \times \DState)
  \end{equation*}
  %
  \caption{Delegation Transitions}
  \label{fig:delegation-transitions}
\end{figure}

%%
%% Figure - Delegation Rules
%%
\begin{figure}
  \centering
  \begin{equation}\label{eq:deleg-reg}
    \inference[Deleg-Reg]
    {
      \RegKey{c} & \cauthor{c} = hk & hk \notin \var{stkeys}
    }
    {
      \left(
      \begin{array}{r}
        \var{stkeys} \\
        \var{rewards} \\
        \var{delegations}
      \end{array}
      \right)
      \trans{deleg}{\var{c}}
      \left(
      \begin{array}{rcl}
        \var{stkeys} & \union & \{\var{hk}\}\\
        \var{rewards} & \unionoverride & \{\addrRw \var{hk} \mapsto 0\}\\
        \var{delegations}
      \end{array}
      \right)
    }
  \end{equation}

  \begin{equation}\label{eq:deleg-dereg}
    \inference[Deleg-Dereg]
    {
      \DeregKey{c} & \cauthor{c} = hk & hk \in \var{stkeys}
    }
    {
      \left(
      \begin{array}{r}
        \var{stkeys} \\
        \var{rewards} \\
        \var{delegations}
      \end{array}
      \right)
      \trans{deleg}{\var{c}}
      \left(
      \begin{array}{rcl}
        \var{stkeys} & \setminus & \{\var{hk}\}\\
        \{\addrRw \var{hk}\} & \subtractdom & \var{rewards} \\
        \{\var{hk}\} & \subtractdom & \var{delegations}
      \end{array}
      \right)
    }
  \end{equation}

  \begin{equation}\label{eq:deleg-deleg}
    \inference[Deleg-Deleg]
    {
      \Delegate{c} & \cauthor{c} = hk & hk \in \var{stkeys}
    }
    {
      \left(
      \begin{array}{r}
        \var{stkeys} \\
        \var{rewards} \\
        \var{delegations}
      \end{array}
      \right)
      \trans{deleg}{c}
      \left(
      \begin{array}{rcl}
        \var{stkeys} \\
        \var{rewards} \\
        \var{delegations} & \unionoverride & \{\var{hk} \mapsto \pool c\}
      \end{array}
      \right)
    }
  \end{equation}
  \caption{Delegation Inference Rules}
  \label{fig:delegation-rules}
\end{figure}

%%
%% Figure - Pool Rules
%%
\begin{figure}
  \begin{equation}\label{eq:pool-reg}
    \inference[Pool-Reg]
    {
      \RegPool{c} & \cauthor{c} = hk
    }
    {
      \var{cepoch} \vdash
      \left(
      \begin{array}{r}
        \var{stpools} \\
        \var{retiring}
      \end{array}
      \right)
      \trans{pool}{c}
      \left(
      \begin{array}{rcl}
        \var{stpools} & \unionoverride & \{\var{hk} \mapsto c\} \\
        \{\var{hk}\} & \subtractdom & \var{retiring} \\
      \end{array}
      \right)
    }
  \end{equation}


  \begin{equation}\label{eq:pool-ret}
    \inference[Pool-Retire]
    {
    \RetirePool{c} & \cauthor{c} = hk & \var{hk} \in \dom \var{stpools} \\
    \var{e} = \retire{c} & \var{cepoch} < \var{e} < \var{cepoch} + \emax
  }
  {
    \var{cepoch} \vdash
    \left(
      \begin{array}{r}
        \var{stpools} \\
        \var{retiring}
      \end{array}
    \right)
    \trans{pool}{c}
    \left(
      \begin{array}{rcl}
        \var{stpools} \\
        \var{retiring} & \unionoverride & \{\var{hk} \mapsto \var{e}\} \\
      \end{array}
    \right)
  }
  \end{equation}

  \begin{equation}\label{eq:pool-reap}
    \inference[Pool-Reap]
    {
      \var{retired} = \var{retiring}^{-1}~\var{cepoch}
      & \var{retired} \neq \emptyset
    }
    {
      \var{cepoch} \vdash
      \left(
      \begin{array}{r}
        \var{stpools} \\
        \var{retiring}
      \end{array}
      \right)
      \trans{pool}{c}
      \left(
      \begin{array}{rcl}
        \var{retired} & \subtractdom & \var{stpools} \\
        \var{retired} & \subtractdom & \var{retiring} \\
      \end{array}
      \right)
    }
  \end{equation}
  \caption{Pool Inference Rule}
  \label{fig:pool-rules}

\end{figure}

\subsection{Witnesses}
\label{sec:delegation-witnesses}

The rule for certificate witnesses is given in
Figure~\ref{fig:rules:delegationw}. The new definitions introduced in this rule
are given in Figure~\ref{fig:defs:delegationw}.

\begin{figure}
  \emph{Abstract types}
  \begin{equation*}
    \begin{array}{r@{~\in~}lr}
      tx & \Tx & \text{transaction}\\
      cb & \DCertBody & \text{certificate body}\\
    \end{array}
  \end{equation*}
  %
  \emph{Abstract functions}
  \begin{equation*}
    \begin{array}{r@{~\in~}lr}
      \fun{dbody} & \DCert \mapsto \DCertBody
      & \text{body of the delegation certificate}\\
      \fun{dwit} & \DCert \mapsto (\VKey \times \Sig)
      & \text{witness for the delegation certificate}
    \end{array}
  \end{equation*}
  %
  \emph{Delegation Witnesses environment}
  \begin{equation*}
    \DWEnv =
    \left(
      \begin{array}{r@{~\in~}lr}
        \var{tx} & \Tx & \text{transaction}\\
        \var{epoch} & \Epoch & \text{epoch}\\
      \end{array}
    \right)
  \end{equation*}
  %
  \emph{Delegation Witnesses state}
  \begin{equation*}
    \DWState =
    \left(
      \begin{array}{r@{~\in~}lr}
        \var{dstate} & \DState & \text{delegation state}\\
        \var{pstate} & \PState & \text{pool state}\\
      \end{array}
    \right)
  \end{equation*}
  %
  \emph{Delegation Witnesses Transition}
  \begin{equation*}
    \_ \vdash \_ \trans{delegw}{\_} \_ \in
      \powerset (
        \DWEnv \times \DWState \times \DCert \times \DWState)
  \end{equation*}
  \caption{Delegation witnesses definitions}
  \label{fig:defs:delegationw}
\end{figure}

\begin{figure}
  \emph{Delegation with witness rule}
  \begin{equation}
    \label{eq:deleg-witnesses}
    \inference[Deleg-wit]
    { \dwit{c} = (\var{vk_s}, \sigma)
      & \var{dstate} \trans{deleg}{c} \var{dstate'}
      & \var{cepoch} \vdash \var{pstate}
      \trans{pool}{c} \var{pstate'}
      \\ ~ \\
      \verify{vk_s}{\serialised{(\dbody{c},~\txbody \var{tx})}}{\sigma}
    }
    { \left(
      \begin{array}{r}
        \var{tx} \\
        \var{cepoch}
      \end{array}
      \right)
      \vdash
      \left(
      \begin{array}{r}
        \var{dstate} \\
        \var{pstate}
      \end{array}
      \right)
      \trans{delegw}{c}
      \left(
      \begin{array}{rcl}
        \var{dstate'} \\
        \var{pstate'}
      \end{array}
      \right)
    }
  \end{equation}
  \caption{Delegation witnesses inference rules}
  \label{fig:rules:delegationw}
\end{figure}



\section{Update mechanism}
\label{sec:update}

\newcommand{\UProp}{\ensuremath{\type{UProp}}}
\newcommand{\UPropId}{\ensuremath{\type{UPropId}}}
\newcommand{\UPropBody}{\ensuremath{\type{UPropBody}}}
\newcommand{\ProtVer}{\ensuremath{\type{ProtVer}}}
\newcommand{\ProtAtt}{\ensuremath{\type{ProtAtt}}}
\newcommand{\ProtParams}{\ensuremath{\type{ProtParams}}}
\newcommand{\UPVEnv}{\ensuremath{\type{UPVEnv}}}
\newcommand{\UPVState}{\ensuremath{\type{UPVState}}}
\newcommand{\UPLEnv}{\ensuremath{\type{UPLEnv}}}
\newcommand{\UPLState}{\ensuremath{\type{UPLState}}}
\newcommand{\Vote}{\ensuremath{\type{Vote}}}
\newcommand{\VEnv}{\ensuremath{\type{VEnv}}}
\newcommand{\VState}{\ensuremath{\type{VState}}}

\newcommand{\upSize}[1]{\ensuremath{\fun{upSize}~\var{#1}}}
\newcommand{\upPV}[1]{\ensuremath{\fun{upPV}~\var{#1}}}
\newcommand{\upId}[1]{\ensuremath{\fun{upId}~\var{#1}}}
\newcommand{\upIssuer}[1]{\ensuremath{\fun{upIssuer}~\var{#1}}}
\newcommand{\upParams}[1]{\ensuremath{\fun{upParams}~\var{#1}}}
\newcommand{\vCaster}[1]{\ensuremath{\fun{vCaster}~\var{#1}}}
\newcommand{\vPropId}[1]{\ensuremath{\fun{vPropId}~\var{#1}}}
\newcommand{\vSig}[1]{\ensuremath{\fun{vSig}~\var{#1}}}

\lstset{ frame=tb,
       , language=Haskell
       , basicstyle=\footnotesize\ttfamily,
       , keywordstyle=\color{blue!80},
       , commentstyle=\itshape\color{purple!40!black},
       , identifierstyle=\bfseries\color{green!40!black},
       , stringstyle=\color{orange},
       }

\lstMakeShortInline[columns=fixed]`

\subsection{Aspects that we need to model}
\label{sec:aspects-to-model}

\begin{description}
\item[Authentication] Update proposals and votes are authenticated (properly
  signed).
\item[Authorization] Only genesis keys (via deleg certs) can post update
  proposals.
  \begin{itemize}
  \item Only then they can vote on them.
  \end{itemize}
\item[Voting deadlines] voting ends when a majority of the voters (4/7 if we
assume no stake) agree on the proposal.
\item[Block versions] (= protocol versions)
\item[Soft-forks] a protocol version changes according to the fork-resolution
  rule (75\% of stake create blocks with new-version).
\item[Hard-forks] ??? Do we need to model anything here?.
\end{description}

In particular the rules need to capture:

\begin{enumerate}
\item Each genesis key can post (either directly or via its delegate), one update proposal per-epoch.
\item When does an update proposal becomes \textbf{active}.
\item We cannot add a proposal if there exists one with the same id\footnote{id
    = hash of the update proposal} that is already active.
\item If there are two proposals submitted for the same block version, then
  these proposals need to agree on the values of the updated parameters.
\end{enumerate}

Remarks:
\begin{itemize}
\item We do not distinguish between protocol or software updates. The ledger
  only cares about the mechanisms by which the protocol parameters are changed.
\end{itemize}

\subsection{Update proposals}
\label{sec:update-proposals}

\begin{figure}[htb]
  \emph{Abstract types}
  %
  \begin{equation*}
    \begin{array}{r@{~\in~}lr}
      \var{up} & \UProp & \text{(protocol) update proposal}\\
      \var{pa} & \ProtAtt & \text{protocol attribute}\\
    \end{array}
  \end{equation*}
  %
  \emph{Derived types}
  \begin{equation*}
    \begin{array}{r@{~\in~}l@{\qquad=\qquad}r@{~\in~}lr}
      \var{pv} & \ProtVer & (\var{maj}, \var{min}, \var{alt})
      & (\mathbb{N}, \mathbb{N}, \mathbb{N}) & \text{protocol version}\\
      \var{pps} & \ProtParams & \var{pps} & \ProtAtt \mapsto \Value & \text{protocol parameters}
    \end{array}
  \end{equation*}
  \emph{Abstract functions}
  %
  \begin{align*}
    & \fun{upIssuer} \in \UProp \to \VKeyGen & \text{issuer of the update proposal}\\
    & \fun{upSize} \in \UProp \to \mathbb{N} & \text{update proposal size}\\
    & \fun{upPV} \in \UProp \to \mathbb{\ProtVer} & \text{update proposal protocol version}\\
    & \fun{upID} \in \UProp \to \mathbb{\UPropId} & \text{update proposal id}
  \end{align*}
  \caption{Update proposals definitions}
  \label{fig:defs:update-proposals}
\end{figure}

\subsection{Update proposals activation}
\label{sec:update-proposals-activation}

First we model the validity of a proposal.

\begin{figure}[htb]
  \emph{Update proposals validity environments}
  \begin{equation*}
    \UPVEnv =
    \left(
      \begin{array}{r@{~\in~}lr}
        \var{pv} & \ProtVer & \text{adopted (current) protocol version}\\
        \var{pps} & \ProtParams & \text{adopted protocol parameters}\\
      \end{array}
    \right)
  \end{equation*}
  %
  \emph{Update proposals validity states}
  \begin{equation*}
    \UPVState
    = \left(
      \begin{array}{r@{~\in~}lr}
        \var{rups} & \powerset{\UPropId \mapsto (\ProtVer \times \ProtParams)}
        & \text{registered update proposals}\\
      \end{array}
    \right)
  \end{equation*}
  %
  \emph{Update proposals validity transitions}
    \begin{equation*}
    \var{\_} \vdash
    \var{\_} \trans{upv}{\_} \var{\_}
    \subseteq \powerset (\UPVEnv \times \UPVState \times \UProp \times \UPVState)
  \end{equation*}
  \caption{Update proposals validity transition-system types}
  \label{fig:ts-types:up-validity}
\end{figure}

Terse explanation of Rule~\ref{eq:rule:up-validity}: a new proposal:
\begin{itemize}
\item must not exceed the maximum side, as specified by the current
  protocol parameters.
\item must not exist in the set of active proposals.
\item must have a unique version among the current active proposals. This
  implies that a proposal is uniquely determined by the protocol version it
  proposes.
\item must increase one of the (major, minor, or alternative) of the
  current version in a consistent manner:
  \begin{itemize}
  \item The proposed version must be lexicographically bigger than the current
    version.
  \item The major versions of the proposed and current version must differ in
    at most one.
  \item If the proposed major version is equal to the current major
    version, then the proposed minor version must be incremented by one.
  \item If the proposed major version is larger than the current major, then
    the proposed minor version must be zero.
  \end{itemize}
\end{itemize}

\begin{figure}[htb]
  \begin{equation}
    \label{eq:func:can-follow}
    \begin{array}{r c l}
      \fun{canFollow}~(\var{mj_n}, \var{mi_n}, \var{a_n})~(\var{mj_p}, \var{mi_p}, \var{a_p})
      & = & (\var{mj_p}, \var{mi_p}, \var{a_p}) < (\var{mj_n}, \var{mi_n}, \var{a_n})\\
      & \wedge & (|\var{mj_n} - \var{mj_p}| \leq 1\\
      & \wedge & (\var{mj_p} = \var{mj_n} \Rightarrow \var{mi_p} + 1 = \var{mi_n}))\\
      & \wedge & (\var{mj_p} + 1 = \var{mj_n} \Rightarrow \var{mi_n} = 0)
    \end{array}
  \end{equation}
  \caption{Update validity functions}
\end{figure}

\setpremisesspace{15pt}
\begin{figure}[htb]
  \begin{equation}
    \label{eq:rule:up-validity}
    \inference
    {
      {\begin{array}{l l}
        \var{maxUpSize} \mapsto \var{mus} \in \var{pps}
        & \upSize{up} \leq \var{mus}\\
        \upId{up} = \var{pid}
        & \var{pid} \notin \dom \var{rups} \\
        \upPV{up} = \var{nv}
        & \var{nv} \notin \dom (\range \var{rups})\\
       \end{array}}\\
     \fun{canFollow}~\var{nv}~\var{pv}
    }
    {
      {
        \begin{array}{l}
          \var{pv}\\
          \var{pps}
        \end{array}
      }
      \vdash
      {
        \left(
          \begin{array}{l}
            \var{rups}
          \end{array}
        \right)
      }
      \trans{upv}{\var{up}}
      {
        \left(
          \begin{array}{l}
            \var{rups} \unionoverride \{ \var{pid} \mapsto (\var{nv}, \upPV{up}) \}
          \end{array}
        \right)
      }
    }
  \end{equation}
  \caption{Update proposals validity rules}
  \label{fig:rules:up-validity}
\end{figure}

\clearpage

\begin{figure}[htb]
  \emph{Update proposals limits  environments}
    \begin{equation*}
    \UPLEnv =
    \left(
      \begin{array}{r@{~\in~}lr}
        \var{e} & \Epoch & \text{current epoch}\\
        \mathcal{K} & \powerset{\VKeyGen} & \text{genesis keys}\\
      \end{array}
    \right)
  \end{equation*}
  %
  \emph{Update proposals limits states}
  \begin{equation*}
    \UPLState
    = \left(
      \begin{array}{r@{~\in~}lr}
        \var{eps} & \powerset{(\Epoch \times \VKeyGen)} & \text{proposals per-key per-epoch}\\
      \end{array}
    \right)
  \end{equation*}
  %
  \emph{Update proposals limits transitions}
  \begin{equation*}
    \var{\_} \vdash
    \var{\_} \trans{upl}{\_} \var{\_}
    \subseteq \powerset (\UPLEnv \times \UPLState \times \UProp \times \UPLState)
  \end{equation*}
  \caption{Update proposals limits transition-system types}
  \label{fig:ts-types:up-limits}
\end{figure}

\begin{figure}[htb]
  \begin{equation}
    \label{eq:rule:up-limits}
    \inference
    {\upIssuer{up} = \var{vk}
      & \var{vk} \in \mathcal{K}
      & (e_c, \var{vk}) \notin \var{eps}
    }
    {
      {\begin{array}{l}
         \var{e_c}\\
         \mathcal{K}
       \end{array}
      }
      \vdash
      {
        \left(
          \begin{array}{l}
            \var{eps}
          \end{array}
        \right)
      }
      \trans{upl}{\var{up}}
      {
        \left(
          \begin{array}{l}
            \var{eps \cup \{(e_c, \var{vk})\}}
          \end{array}
        \right)
      }
    }
  \end{equation}
  \caption{Update proposals limits rules}
  \label{fig:rules:up-limits}
\end{figure}

\begin{todo}
TODO: add witnesses as well.
\end{todo}

\begin{figure}[htb]
  \begin{equation}
    \label{eq:rule:up-activation}
    \inference
    {
      {
        \begin{array}{l}
          \var{pv}\\
          \var{pps}
        \end{array}
      }
      \vdash
      {
        \left(
          \begin{array}{l}
            \var{rups}\\
          \end{array}
        \right)
      }
      \trans{upv}{\var{up}}
      {
        \left(
          \begin{array}{l}
            \var{rups'}\\
          \end{array}
        \right)
      }
      &
      {\begin{array}{l}
          \var{e_c}\\
          \var{dms}
        \end{array}
      }
      \vdash
      {
        \left(
          \begin{array}{l}
            \var{eps}
          \end{array}
        \right)
      }
      \trans{upl}{\var{up}}
      {
        \left(
          \begin{array}{l}
            \var{eps'}
          \end{array}
        \right)
      }
    }
    {
      {
        \begin{array}{l}
          \var{pv}\\
          \var{pps}\\
          \var{e_c}\\
          \var{dms}
        \end{array}
      }
      \vdash
      {
        \left(
          \begin{array}{l}
            \var{rups}\\
            \var{eps}
          \end{array}
        \right)
      }
      \trans{upa}{\var{up}}
      {
        \left(
          \begin{array}{l}
            \var{rups'}\\
            \var{eps'}
          \end{array}
        \right)
      }
    }
  \end{equation}
  \caption{Update activation rules}
  \label{fig:rules:up-activation}
\end{figure}

\clearpage

\subsection{Voting on update proposals}
\label{sec:voting-on-update-proposals}

\begin{figure}[htb]
  \emph{Abstract types}
  %
  \begin{equation*}
    \begin{array}{r@{~\in~}lr}
      \var{v} & \Vote & \text{vote on an update proposal}
    \end{array}
  \end{equation*}
  %
  \emph{Abstract functions}
  \begin{align*}
    & \fun{vCaster} \in \Vote \to \VKey & \text{caster of a vote}\\
    & \fun{vPropId} \in \Vote \to \UPropId & \text{proposal id that is being voted}\\
    & \fun{vSig} \in \Vote \to \Sig & \text{vote signature}
  \end{align*}
  \caption{Voting definitions}
  \label{fig:defs:voting}
\end{figure}

\begin{figure}[htb]
  \emph{Voting environments}
  \begin{align*}
    & \VEnv
      = \left(
      \begin{array}{r@{~\in~}lr}
        \var{rups} & \powerset{\UPropId \mapsto (\ProtVer \times \ProtParams)}
        & \text{registered update proposals}\\
        \var{dms} & \VKeyGen \mapsto \VKey & \text{delegation map}
      \end{array}\right)
  \end{align*}
  %
  \emph{Voting states}
  \begin{align*}
    & \VState
      = \left(
      \begin{array}{r@{~\in~}lr}
        \var{vts} & \powerset{(\UPropId \times \VKeyGen)} & \text{votes}
      \end{array}\right)
  \end{align*}
  %
  \emph{Voting transitions}
    \begin{equation*}
    \_ \vdash \_ \trans{vote}{\_} \_ \in
    \powerset (\VEnv \times \VState \times \Vote \times \VState)
    \end{equation*}
  \caption{Voting transition-system types}
  \label{fig:ts-types:voting}
\end{figure}

Terse explanation of Rule~\ref{eq:rule:voting}:
\begin{itemize}
\item Only genesis keys can vote on an update proposal, although votes can be
  cast by delegates of these genesis keys.
\item We count one vote per genesis key that delegated to the key that is
  casting the vote.
\item The vote must refer to an active update proposal.
\item The proposal id must be signed by the key that is casting the vote.
\item It might be possible for the same genesis key to vote multiple times for
  the same proposal, however this vote will be counted once (note that we're
  taking the union of the key-proposal-id pairs).
\end{itemize}

\begin{figure}[htb]
  \begin{equation}
    \label{eq:rule:voting}
    \inference
    {
      \vPropId{v} = \var{pid} &  \vCaster{v} = \var{vk} &
      \var{vts}_{\var{pid}} =
      \{ (\var{pid}, \var{vk_s}) \mid \var{vk_s} \mapsto \var{vk} \in \var{dms} \}\\
      & \var{pid} \in \dom \var{rups} &
      \mathcal{V}_{\var{vk}}\serialised{\var{pid}}_{(\vSig{v})}\\
    }
    {
      {
        \begin{array}{l}
          \var{rups}\\
          \var{dms}
        \end{array}
      }
      \vdash
      {
        \left(
          \begin{array}{l}
            \var{vts}
          \end{array}
        \right)
      }
      \trans{vote}{\var{v}}
      {
        \left(
          \begin{array}{l}
            \var{vts} \cup \var{vts}_{\var{pid}}\\
          \end{array}
        \right)
      }
    }
  \end{equation}
  \caption{Update voting rules}
  \label{fig:rules:voting}
\end{figure}

\clearpage

\subsection{Update proposals confirmation}
\label{sec:update-proposals-confirmation}

This section models when update proposals get confirmed.

\begin{figure}[htb]
  \begin{equation}
    \label{eq:rule:up-no-confirmation}
    \inference
    {
      {
        \begin{array}{l}
          \var{rups}\\
          \var{dms}
        \end{array}
      }
      \vdash
      {
        \left(
          \begin{array}{l}
            \var{vts}
          \end{array}
        \right)
      }
      \trans{vote}{\var{v}}
      {
        \left(
          \begin{array}{l}
            \var{vts'}
          \end{array}
        \right)
      }\\
      \vPropId{v} = \var{pid}
      & \var{pcThr} \mapsto t \in \var{pps}
      & \size{\{\var{pid}\} \restrictdom \var{vts'}} < t
    }
    {
      {
        \begin{array}{l}
          b_n\\
          \var{pps}\\
          \var{dms}
        \end{array}
      }
      \vdash
      {
        \left(
          \begin{array}{l}
            \var{rups}\\
            \var{cps}\\
            \var{vts}
          \end{array}
        \right)
      }
      \trans{upc}{\var{v}}
      {
        \left(
          \begin{array}{l}
            \var{rups}\\
            \var{cps}\\
            \var{vts'}
          \end{array}
        \right)
      }
    }
  \end{equation}
  \nextdef
  \begin{equation}
    \label{eq:rule:up-confirmation}
    \inference
    {
      {
        \begin{array}{l}
          \var{rups}\\
          \var{dms}
        \end{array}
      }
      \vdash
      {
        \left(
          \begin{array}{l}
            \var{vts}
          \end{array}
        \right)
      }
      \trans{vote}{\var{v}}
      {
        \left(
          \begin{array}{l}
            \var{vts'}
          \end{array}
        \right)
      }\\
      \vPropId{v} = \var{pid}
      & \var{pcThr} \mapsto t \in \var{pps}
      & t \leq \size{\{\var{pid}\} \restrictdom \var{vts'}}
    }
    {
      {
        \begin{array}{l}
          \var{b_n}\\
          \var{pps}\\
          \var{dms}
        \end{array}
      }
      \vdash
      {
        \left(
          \begin{array}{l}
            \var{rups}\\
            \var{cps}\\
            \var{vts}
          \end{array}
        \right)
      }
      \trans{upc}{\var{v}}
      {
        \left(
          \begin{array}{l}
            \{\var{pid}\} \subtractdom \var{rups} \\
            \var{cps} \unionoverride  \{\var{pid} \mapsto b_n\} \\
            \{\var{pid}\} \subtractdom \var{vts'} 
          \end{array}
        \right)
      }
    }
  \end{equation}
  \caption{Update-proposals confirmation rules}
  \label{fig:rules:up-confirmation}
\end{figure}

\clearpage

\subsection{Update proposals adoption}
\label{sec:update-proposals-adoption}

\begin{equation}
  \label{eq:predicate:adopt}
  \begin{array}{r c l}
    \fun{canAdopt}~\var{pps}~\var{bvs}~\var{bv}
    & =
    & \var{upAdptThr} \mapsto t \in pps\\
    & \wedge
    & t \leq \dfrac{\size{\var{bvs} \restrictrange \{\var{bv}\}}}{\size{\var{bvs}}}\\
  \end{array}
\end{equation}

\begin{figure}[htb]
  \begin{equation}
    \label{eq:rule:snocbv}
    \inference
    {
      \var{bvs'} = \var{bvs};\var{bv} & m = \size{bvs'}
      & \var{bvsWinSize} \mapsto w \in \var{pps}
    }
    {
      {
        \begin{array}{l}
          \var{pps}
        \end{array}
      }
      \vdash
      {
        \left(
          \begin{array}{l}
            bvs
          \end{array}
        \right)
      }
      \trans{snocbv}{bv}
      {
        \left(
          \begin{array}{l}
            {[m - w, ..]} \restrictdom \var{bvs'}
          \end{array}
        \right)
      }
    }
  \end{equation}
  %
  \nextdef
  %
  \begin{equation}
    \label{eq:rule:up-adopted}
    \inference
    {
      \var{bv} = \var{pv}
      &
      {
        \begin{array}{l}
          \var{pps}
        \end{array}
      }
      \vdash
      {
        \left(
          \begin{array}{l}
            bvs
          \end{array}
        \right)
      }
      \trans{snocbv}{bv}
      {
        \left(
          \begin{array}{l}
            bvs'
          \end{array}
        \right)
      }
    }
    {
      {
        \begin{array}{l}
          k\\
          b_n
        \end{array}
      }
      \vdash
      {
        \left(
          \begin{array}{l}
            \var{pv}\\
            \var{pps}\\
            \var{cps}\\
            \var{rups}\\
            \var{bvs}
          \end{array}
        \right)
      }
      \trans{upadopt}{\var{bv}}
      {
        \left(
          \begin{array}{l}
            \var{pv}\\
            \var{pps}\\
            \var{cps}\\
            \var{rups}\\
            \var{bvs'}
          \end{array}
        \right)
      }
    }
  \end{equation}
  %
  \nextdef
  %
  \begin{equation}
    \label{eq:rule:up-no-adoption}
    \inference
    {
      \var{bv} \neq \var{pv}
      &
      {
        \begin{array}{l}
          \var{pps}
        \end{array}
      }
      \vdash
      {
        \left(
          \begin{array}{l}
            bvs
          \end{array}
        \right)
      }
      \trans{snocbv}{bv}
      {
        \left(
          \begin{array}{l}
            bvs'
          \end{array}
        \right)
      }
      & \neg (\fun{canAdopt}~\var{pps}~\var{bvs'}~\var{bv})\\
      \var{pid} \mapsto (\var{bv}, \wcard) \in \var{rups}
      & \var{pid} \in \dom~(\var{cps} \restrictrange [.., b_n - k])
    }
    {
      {
        \begin{array}{l}
          k\\
          b_n
        \end{array}
      }
      \vdash
      {
        \left(
          \begin{array}{l}
            \var{pv}\\
            \var{pps}\\
            \var{cps}\\
            \var{rups}\\
            \var{bvs}
          \end{array}
        \right)
      }
      \trans{upadopt}{\var{bv}}
      {
        \left(
          \begin{array}{l}
            \var{pv}\\
            \var{pps}\\
            \var{cps}\\
            \var{rups}\\
            \var{bvs'}
          \end{array}
        \right)
      }
    }
  \end{equation}
  %
  \nextdef
  %
  \begin{equation}
    \label{eq:rule:up-adoption}
    \inference
    {
      \var{bv} \neq \var{pv}
      &
            {
        \begin{array}{l}
          \var{pps}
        \end{array}
      }
      \vdash
      {
        \left(
          \begin{array}{l}
            bvs
          \end{array}
        \right)
      }
      \trans{snocbv}{bv}
      {
        \left(
          \begin{array}{l}
            bvs'
          \end{array}
        \right)
      }
      &
      \fun{canAdopt}~\var{pps}~\var{bvs'}~\var{bv}\\
      \var{pid} \mapsto (\var{bv}, \var{pps'}) \in \var{rups}
      & \var{pid} \in \dom~(\var{cps} \restrictrange [.., b_n - k])\\
    }
    {
      {
        \begin{array}{l}
          k\\
          b_n
        \end{array}
      }
      \vdash
      {
        \left(
          \begin{array}{l}
            \var{pv}\\
            \var{pps}\\
            \var{cps}\\
            \var{rups}\\
            \var{bvs}
          \end{array}
        \right)
      }
      \trans{upadopt}{\var{bv}}
      {
        \left(
          \begin{array}{l}
            \var{bv}\\
            \var{pps'}\\
            \{ \var{pid} \} \subtractdom \var{cps} \\
            \{ \var{pid} \} \subtractdom \var{rups}\\
            \var{bvs'}
          \end{array}
        \right)
      }
    }
  \end{equation}
  \caption{Update-proposals adoption rules}
  \label{fig:rules:up-adoption}
\end{figure}

\subsection{Deviations from the actual implementation}
\label{sec:deviation-actual-impl}

The current specification of the voting mechanism deviates from the actual
implementation, although it should be backwards compatible the latter. These
deviations are required to simplify the voting and update mechanism removing
unnecessary features and reducing accidental complexity. The following
subsections highlight the differences between the this specification and the
current implementation.

\subsubsection{Positive votes}
\label{sec:only-positive-votes}

Votes are only positive. Genesis keys can only vote positively for an update
proposal. In the current implementation stakeholders can vote for or against a
proposal, which makes the voting logic more complex:
\begin{itemize}
\item there are more cases to consider
\item the current voting validation rules allow voters to change their minds
  (by flipping their vote) at most once, which requires to keep track how a
  stake holder voted and how many times. Contrast this with
  Rule~\ref{eq:rule:voting} where we only need to keep track of the set of
  key-proposal-id's pairs.
\end{itemize}

\subsubsection{Alternative version numbers}
\label{sec:alt-version-numbers-constraints}

Alternative version numbers are only lexicographically constrained. The current
implementation seems to be dependent on the order in which the update proposals
arrive: given a new update proposal $\var{up}$, if a set $X$ of update
proposals with the same minor and major versions than $\var{up}$ exist, then
the alternative version of $\var{up}$ has to be one more than the maximum
alternative number of $X$. Not only this logic seems to be brittle since it
depends on the order of arrival of the update proposals, but it requires a more
complex check (which depends on state) to determine if a proposed version is
consistent. By being more lenient on the alternative versions of update
proposals we can simplify the version checking logic considerably.

\subsubsection{Update proposal adoption}
\label{sec:up-adoption}

An update proposal can become immediately adopted if the number of blocks
signed with the version of the proposal exceeds a certain threshold. In the
current implementation this occurs only at the epoch boundary block.

\subsubsection{Cleanup}
\label{sec:up-cleanup}

Update proposals that are older than $u$ blocks w.r.t. the current block are
discarded from the state, along with its information. The current
implementation makes use of the implicit agreement rule, and the epoch boundary
checks: this leads to plenty of different states for a proposal: active,
adopted, confirmed, competing, never-to-become-adopted, rejected, discarded. If
the cleanup of proposals can be done in the way specified here we will avoid a
great deal of cognitive complexity when reasoning about the update system.

\subsubsection{Adoption threshold}
\label{sec:adoption-threshold}

The current implementation adopts a proposal with version $\var{pv}$ if the
portion of block issuers' stakes, which issued blocks with this version, is
greater than the threshold given by:

\begin{lstlisting}
max spMinThd (spInitThd - (t - s) * spThdDecrement)  
\end{lstlisting}

where:
\begin{itemize}
\item \lstinline{spMinThd} is a minimum threshold required for adoption.
\item \lstinline{spInitThd} is an initial threshold.
\item \lstinline{spThdDecrement} is the decrement constant of the initial
  threshold.
\end{itemize}

In this specification we consider a fixed threshold, called $\var{upAdptThr}$
until it becomes clear why a dynamic alternative is needed.

\subsection{Information in the ledger state}
\label{sec:information-in-ledger-state}

The ledger state has to expose some parameters of the protocol version to its
clients. In the current implementation these parameters are kept in the
`BlockVersionData` structure.

\begin{lstlisting}
  data BlockVersionData = BlockVersionData { ... }
\end{lstlisting}

The following parameters are likely to be needed by the consumers of the ledger
layer:

\begin{itemize}
\item Slot duration (`bvdSlotDuration`)
\item Maximum block size (`bvdMaxBlockSize`)
\item Maximum header size (`bvdMaxHeaderSize`)
\item Maximum transaction size (`bvdMaxTxSize`)
\item Maximum proposal size (`bvdMaxProposalSize`)
\item Transaction fee policy (`bvdTxFeePolicy`)
\end{itemize}

The following parameters will be assumed to be constant:
\begin{itemize}
\item MPC threshold (`bvdMpcThd`)
\item Heavy delegation threshold (`bvdHeavyDelThd`)
\item Update vote threshold (`bvdUpdateVoteThd`)
\item Update proposal threshold (`bvdUpdateProposalThd`)
\item Soft fork rule parameters (`bvdSoftforkRule`)
\end{itemize}

At the moment we don't know whether we need these:

\begin{itemize}
\item Script Version (`bvdScriptVersion`)
\item Update implicit (`bvdUpdateImplicit`)
\end{itemize}

Finally, the `bvdUnlockStakeEpoch` field of `BlockVersionData` does not need to
be modeled.

\section{Blockchain layer}
\label{sec:blockchain-layer}
\begin{note}
  This section provides a \textbf{proposal} on how the ledger rules can be used
  to build the blockchain ones. It was mainly developed to help me
  understanding what the blockchain layer requires from the ledger layer, and
  the aspects that need to be modeled in the former. In addition, my
  expectation with this section is that we can discuss which tasks should be
  completed in order to finish a first draft of the blockchain and ledger
  specifications, so that we can move forward with the generators. This section
  was not intended as a replacement of the blockchain spec, which can be found
  in a different document.-- Damian Nadales
\end{note}

\subsection{Chain extension}
\label{sec:chain-extension}

The chain extension rule is given in Figure~\ref{fig:rules:chain-extension},
and the definitions used in this rule are presented in
Figure~\ref{fig:defs:chain-extension}. Rule~\ref{eq:rule:chain-extension}
relies on transitions $\trans{bdeleg}{}$, which specify the delegation
behavior, and $\trans{butxo}{}$ which models the evolution of unspent outputs
after applying the transitions in a block. Rules for delegation and unspent
outputs in the context of a block are given in
Sections~\ref{sec:block-delegation} and \ref{sec:block-utxo} respectively.

\begin{note}
  The protocol constants are currently modeled as a part of the
  chain extension environment. Once we adding voting
  (the mechanism which controls the protocol constants)
  to the model, it will probably make sense to move them to the
  chain extension states. The protocol constants will also be
  added to the ledger state, along with some kind of rule which
  combines voting with the UTxO rule:
  \begin{equation*}
    \inference[Voting-combine]
    {
      \var{pc} \trans{voting}{} \var{pc'} &
      \var{pc'} \vdash \var{utxo} \trans{utxo}{} \var{utxo'}\\ ~ \\
    }
    {\ldots}
  \end{equation*}
\end{note}

\begin{figure}
  \emph{Abstract types}
  \begin{equation*}
    \begin{array}{r@{~\in~}lr}
      \var{b} & \Block & \text{block}\\
      \var{s} & \SlotId & \text{slot id}\\
    \end{array}
  \end{equation*}
  \emph{Abstract functions}
  \begin{equation*}
    \begin{array}{r@{~\in~}lr}
    \fun{bwit} & \Block \to (\VKey \times \Sig) & \text{block witness}\\
      \fun{bepoch} & \Block \to \Epoch & \text{block epoch}\\
      \fun{bslot} & \Block \to \SlotId & \text{block slot id}\\
    \fun{s_0} & \SlotId  & \text{slot zero (smallest slot id)}\\
    \end{array}
  \end{equation*}
  \caption{Blockchain extension definitions}
  \label{fig:defs:chain-extension}
\end{figure}

\begin{figure}
  \emph{Chain extension environment}
  \begin{equation*}
    \CEEnv =
    \left(
      \begin{array}{r@{~\in~}lr}
        \var{\Gkeys} & \powerset{\VKeyGen} & \text{genesis keys}\\
        \var{K} & \mathbb{N} & \text{number of nodes}\\
        \var{t} & \mathbb{Q} & \text{byzantine nodes ratio}\\
        \var{d} & \Epoch & \text{delegation liveness parameter}\\
        \var{pc} & \PrtclConsts & \text{protocol constants}\\
      \end{array}
    \right)
  \end{equation*}
  \emph{Chain extension states}
  \begin{equation*}
    \CEState =
    \left(
      \begin{array}{r@{~\in~}lr}
        \beta & \seqof{\Block} & \text{blockchain}\\
        \var{utxo} & \UTxO & \text{blockchain unspent outputs}\\
        \var{dmap} & \VKeyGen \mapsto \VKey & \text{blockchain delegation map}\\
        \var{signers} & \seqof{\VKeyGen} & \text{last $K$ blockchain signers}\\
        \var{sid_c} & \SlotId & \text{current slot}\\
        \var{pdlgs} & \SlotId \mapsto \seqof{\DCert} & \text{pending delegations}\\
        \var{ekeys} & \Epoch \mapsto \powerset{\VKeyGen} & \text{keys delegated per epoch}
      \end{array}
    \right)
  \end{equation*}
  \emph{Chain extension transitions}
  \begin{equation*}
    \_ \vdash \_ \trans{chain}{\_} \_ \in
      \powerset (\CEEnv \times \CEState \times \Block \times \CEState)
  \end{equation*}
  \caption{Chain extension transition-system types}
  \label{fig:ts-types:chain-extension}
\end{figure}

\begin{figure}
  \begin{equation}
    \label{eq:rule:chain-base}
    \inference[Chain-base]
    {}
    {\left(
        \begin{array}{l}
          \epsilon\\
          \var{utxo}\\
          \var{\{ (\var{vk}, \var{vk}) \mid \var{vk} \in \Gkeys\}}\\
          \emptyset\\
          \var{s_0}\\
          \emptyset\\
          \emptyset
        \end{array}
      \right)
    }
  \end{equation}

  \begin{equation}
    \label{eq:rule:chain-extension}
    \inference[Chain-ext]
    {\var{dmap}~\var{vk_g} = vk_d & \bwit{b} = (\var{vk_d}, \sigma)
      & \bslot{b} = \var{sid_n} & \var{vk_g} \in \Gkeys\\
      \var{sid_c} < \var{sid_n} & \size{\{vk_g\} \restrictdom signers} \leq K * t &
      \verify{vk_d}{\serialised{\bbody{b}}}{\sigma} \\
      \var{signers'} =
         \{ (\var{sid}, \var{vk})
          \mid  (\var{sid}, \var{vk}) \in \var{signers} \cup \{(\var{sid_n}, vk_g)\}
          , \var{sid_n} - K \leq \var{sid} \}\\
      \var{cepoch} = \fun{bepoch}~b &
      {\begin{array}{l}
         \var{cepoch}\\
         \var{sid_c}\\
         d
       \end{array}}
      \vdash
      {
        \left(
          \begin{array}{l}
            \var{dmap}\\
            \var{pdlgs}\\
            \var{ekeys}
          \end{array}
        \right)
      }
      \trans{bdeleg}{b}
      {
        \left(
          \begin{array}{r}
            \var{dmap'}\\
            \var{pdlgs'}\\
            \var{ekeys'}
          \end{array}
        \right)
      }
      \\ ~ \\
      {
        \var{pc} \vdash
        \left(
          \begin{array}{l}
            \var{utxo}\\
          \end{array}
        \right)
      }
      \trans{butxo}{b}
      {
        \left(
          \begin{array}{r}
            \var{utxo'}\\
          \end{array}
        \right)
      }
    }
    {
      {\begin{array}{l}
         \Gkeys\\
         K\\
         t\\
         d\\
         pc
      \end{array}}
      \vdash
      {
        \left(
          \begin{array}{l}
            \beta\\
            \var{utxo}\\
            \var{dmap}\\
            \var{signers}\\
            \var{sid_c}\\
            \var{pdlgs}\\
            \var{ekeys}
          \end{array}
        \right)
      }
      \trans{chain}{b}
      {
        \left(
          \begin{array}{l}
            \beta; b\\
            \var{utxo}\\
            \var{dmap'}\\
            \var{signers'}\\
            \var{sid_n}\\
            \var{pdlgs'}\\
            \var{ekeys'}
          \end{array}
        \right)
      }
    }
  \end{equation}
  \caption{Chain extension rules}
  \label{fig:rules:chain-extension}
\end{figure}

\subsection{Block delegation}
\label{sec:block-delegation}

The rule for delegation of certificates in a block is shown in
Figure~\ref{fig:rules:block-delegation}, and the new definitions used in this
rule are presented in Figure~\ref{fig:defs:block-delegation}.
Rule~\ref{eq:rule:block-delegation} relies on an inference rule that models the
state changes after applying a sequence of delegation certificates. Such rule
is shown in Figure~\ref{fig:rules:delegation-sequence}.

\begin{figure}
  \emph{Abstract functions}
  \begin{equation*}
    \begin{array}{r@{~\in~}lr}
      \fun{bdlgs} & \Block \mapsto \seqof{\DCert} & \text{delegation certificates in the block}\\
    \end{array}
  \end{equation*}
  \caption{Block delegation definitions}
  \label{fig:defs:block-delegation}
\end{figure}

\begin{figure}
  \emph{Block delegation environments}
  \begin{equation*}
    \BDEnv =
    \left(
      \begin{array}{r@{~\in~}lr}
        \var{cepoch} & \Epoch & \text{current epoch}\\
        \var{sid_c} & \SlotId & \text{current slot id}\\
        \var{d} & \Epoch & \text{delegation liveness parameter}\\
      \end{array}
    \right)
  \end{equation*}
  \emph{Block delegation states}
  \begin{equation*}
    \BDState =
    \left(
      \begin{array}{r@{~\in~}lr}
        \var{dmap} & \VKeyGen \mapsto \VKey & \text{blockchain delegation map}\\
        \var{pdlgs} & \SlotId \mapsto \seqof{\DCert} & \text{pending delegations}\\
        \var{ekeys} & \Epoch \mapsto \powerset{\VKeyGen} & \text{keys delegated per epoch}
      \end{array}
    \right)
  \end{equation*}
  \emph{Block delegation transitions}
  \begin{equation*}
    \_ \vdash \_ \trans{bdeleg}{\_} \_ \in
      \powerset (\BDEnv \times \BDState \times \Block \times \CEState)
  \end{equation*}
  \caption{Block delegation transition-system types}
  \label{fig:ts-types:block-delegation}
\end{figure}

\begin{figure}
  \begin{equation}
    \label{eq:rule:block-delegation}
    \inference[Block-dlg]
    {
      \var{pdlgs'} = \var{pdlgs} \unionoverride \{(\bslot{b} + d) \mapsto \bdlgs{b} \}\\
      \var{spast} = \{ \var{sid} \mid \var{sid} \in \var \dom ~ \var{pdlgs'}
                                    ,~ \var{sid} \leq \var{sid_c}\}\\
      \Gamma = \fun{concat}~ [ \var{dlgs} \mid (\var{sid}, \var{dlgs}) \in \var{pdlgs}
                                          ,~ \var{sid} \leq \var{sid_c}]\\
      \var{cepoch} \vdash
      {
        \left(
          \begin{array}{l}
            \var{dmap}\\
            \var{ekeys}
          \end{array}
        \right)
      }
      \trans{delegs}{\Gamma}
      {
        \left(
          \begin{array}{r}
            \var{dmap'}\\
            \var{ekeys'}
          \end{array}
        \right)
      }
    }
    {
      \begin{array}{l}
        \var{cepoch}\\
        \var{sid_c}\\
        d
      \end{array}
      \vdash
      {
        \left(
          \begin{array}{l}
            \var{dmap}\\
            \var{pdlgs}\\
            \var{ekeys}
          \end{array}
        \right)
      }
      \trans{bdeleg}{b}
      {
        \left(
          \begin{array}{r}
            \var{dmap'}\\
            \var{\var{spast} \subtractdom pdlgs'}\\
            \var{ekeys'}
          \end{array}
        \right)
      }
    }
  \end{equation}
  \caption{Block delegation rules}
  \label{fig:rules:block-delegation}
\end{figure}

\begin{figure}
  \begin{equation}
    \inference[Seq-delg-base]
    {}
    { \var{cepoch} \vdash \left(
        \begin{array}{r}
          \var{dmap}\\
          \var{ekeys}
        \end{array}
      \right)
        \trans{delegs}{\epsilon}
      \left(
        \begin{array}{r}
          \var{dmap}\\
          \var{ekeys}
        \end{array}
      \right)
    }
    \label{eq:rule:sequence-delegation-base}
  \end{equation}

  \begin{equation}
    \inference[Seq-delg-ind]
    { \var{cepoch} \vdash
      {\left(
        \begin{array}{r}
          \var{dmap}\\
          \var{ekeys}
        \end{array}
      \right)}
      \trans{delegs}{\Gamma}
      {\left(
        \begin{array}{r}
          \var{dmap'}\\
          \var{ekeys'}
        \end{array}
      \right)}
    \\ ~ \\
    \var{cepoch} \vdash
    {\left(
        \begin{array}{r}
          \var{dmap'}\\
          \var{ekeys'}
        \end{array}
      \right)}
      \trans{delegw}{c}
      {\left(
        \begin{array}{r}
          \var{dmap''}\\
          \var{ekeys''}
        \end{array}
      \right)}
    }
    { \left(
        \begin{array}{r}
          \var{dmap}\\
          \var{ekeys}
        \end{array}
      \right)
      \trans{delegs}{\Gamma; c}
      \left(
        \begin{array}{r}
          \var{dmap''}\\
          \var{ekeys''}
        \end{array}
      \right)
    }
    \label{eq:rule:sequence-delegation-inductive}
  \end{equation}
  \caption{Delegation sequence rules}
  \label{fig:rules:delegation-sequence}
\end{figure}

\subsection{Block UTxO}
\label{sec:block-utxo}

\begin{todo}
  Block-UTxO rules will have the same structure as the rules presented in
  Section~\ref{sec:block-delegation}.
\end{todo}


\addcontentsline{toc}{section}{References}
\bibliographystyle{plainnat}
\bibliography{references}

\end{document}

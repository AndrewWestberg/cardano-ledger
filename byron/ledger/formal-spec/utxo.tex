\newcommand{\PPMMap}{\ensuremath{\type{PParams}}}
\newcommand{\Lmax}{\ensuremath{\mathbb{L}_{\var{max}}}}
\section{UTxO}
\label{sec:state-trans-utxo-1}

The transition rules for unspent transaction outputs are presented in
Figure~\ref{fig:rules:utxo}. The states of the UTxO transition system, along
with their types are defined in Figure~\ref{fig:defs:utxo}:
\begin{itemize}
\item we define the protocol parameters as an abstract type, this type is made
  concrete in Section~\ref{sec:update}, where the update mechanism is
  discussed.
\item The lovelace supply cap ($\fun{lovelaceCap}$) is treated as an abstract
  function in this specification. In the actual system this value equals
  $$
  45 \times 10^{15}
  $$
\end{itemize}


Functions on the types introduced in \cref{fig:rules:utxo} are defined in
\cref{fig:derived-defs:utxo}. In particular, note that in function
$\fun{minfee}$ we make use of the fact that the $\Lovelace$ type is an alias
for the set of the integers numbers ($\mathbb{Z}$).

Rule~\ref{eq:utxo-bootstrap}, models the fact that the \textbf{reserves} of
the system are set to:
$$ \fun{lovelaceCap} - \balance{utxo_0} $$
The Lovelace amount $45 \times 10^{15}$ is the initial money supply in the
system.

\begin{figure*}[htb]
  \emph{Abstract types}
  %
  \begin{equation*}
    \begin{array}{r@{~\in~}lr}
      \var{tx} & \Tx & \text{transaction}\\
      %
      \var{txid} & \TxId & \text{transaction id}\\
      %
      ix & \Ix & \text{transaction index}\\
      %
      \var{addr} & \Addr & \text{address}\\
      %
      \var{pps} & \PPMMap & \text{protocol parameters}
    \end{array}
  \end{equation*}
  %
  \emph{Derived types}
  %
  \begin{equation*}
    \begin{array}{r@{~\in~}l@{\qquad=\qquad}r@{~\in~}lr}
      \ell & \Lovelace
      & n  & \mathbb{Z}
      & \text{currency value}
      \\
      \var{txin}
      & \TxIn
      & (\var{txid}, \var{ix})
      & \TxId \times \Ix
      & \text{transaction input}
      \\
      \var{txout}
      & \type{TxOut}
      & (\var{addr}, c)
      & \Addr \times \Lovelace
      & \text{transaction output}
      \\
      \var{utxo}
      & \UTxO
      & m
      & \TxIn \mapsto \TxOut
      & \text{unspent tx outputs}
    \end{array}
  \end{equation*}
  %
  \emph{Abstract Functions}
  \begin{equation*}
    \begin{array}{r@{~\in~}lr}
      \txid{} & \Tx \to \TxId & \text{compute transaction id}\\
      %
      \fun{txbody} & \Tx \to \powerset{\TxIn} \times (\Ix \mapsto \TxOut)
                                  & \text{transaction body}\\
      %
      \fun{a} & \PPMMap \to \mathbb{Z} & \text{minumum fee factor}\\
      %
      \fun{b} & \PPMMap \to \mathbb{Z} & \text{minumum fee constant}\\
      %
      \fun{txSize} & \Tx \to \mathbb{Z} & \text{abstract size of a transaction}\\
      %
      \fun{lovelaceCap} & \Lovelace & \text{lovelace supply cap}
    \end{array}
  \end{equation*}
  %
  \emph{Constraints}
  \begin{equation}
    \label{eq:txid-injective}
    \forall \var{tx_i}, \var{tx_j} \cdot
    \txid{\var{tx_i}} = \txid{\var{tx_j}} \Rightarrow \var{tx_i} = \var{tx_j}
  \end{equation}
  \caption{Definitions used in the UTxO transition system}
  \label{fig:defs:utxo}
\end{figure*}

\begin{figure}
  \begin{align*}
    & \fun{txins} \in \Tx \to \powerset{\TxIn}
    & \text{transaction inputs} \\
    & \txins{tx} = \var{inputs} \where \txbody{tx} = (\var{inputs}, ~\wcard)
    \nextdef
    & \fun{txouts} \in \Tx \to \UTxO
    & \text{transaction outputs as UTxO} \\
    & \fun{txouts} ~ \var{tx} =
      \left\{ (\fun{txid} ~ \var{tx}, \var{ix}) \mapsto \var{txout} ~
      \middle| \begin{array}{l@{~}c@{~}l}
                 (\_, \var{outputs}) & = & \txbody{tx} \\
                 \var{ix} \mapsto \var{txout} & \in & \var{outputs}
               \end{array}
      \right\}
    \nextdef
    & \fun{balance} \in \UTxO \to \mathbb{Z}
    & \text{UTxO balance} \\
    & \fun{balance} ~ utxo = \sum_{(~\wcard ~ \mapsto (\wcard, ~c)) \in \var{utxo}} c\\
   \nextdef
   %
    & \fun{minfee} \in \PPMMap \to \Tx \to \mathbb{Z} & \text{minimum fee}\\
    & \fun{minfee}~\var{pps}~\var{tx} =
      \fun{a}~\var{pps} * \fun{txSize}~\var{tx} + \fun{b}~\var{pps}
  \end{align*}
  \caption{Functions used in UTxO rules}
  \label{fig:derived-defs:utxo}
\end{figure}

\begin{figure}
  \emph{UTxO environments}
  \begin{equation*}
    \UTxOEnv =
    \left(
      \begin{array}{r@{~\in~}lr}
        \var{utxo_0} & \UTxO & \text{genesis UTxO}\\
        \var{pps} & \PPMMap & \text{protocol parameters map}
      \end{array}
    \right)
  \end{equation*}

  \emph{UTxO states}
  \begin{equation*}
    \UTxOState =
    \left(
      \begin{array}{r@{~\in~}lr}
        \var{utxo} & \UTxO & \text{unspent transaction outputs}\\
        \var{reserves} & \Lovelace & \text{system's reserves}
      \end{array}
    \right)
  \end{equation*}

  \emph{UTxO transitions}
  \begin{equation*}
    \_ \vdash
    \var{\_} \trans{utxo}{\_} \var{\_}
    \subseteq \powerset (\UTxOEnv \times \UTxOState \times \Tx \times \UTxOState)
  \end{equation*}
  \caption{UTxO transition-system types}
  \label{fig:ts-types:utxo}
\end{figure}

\begin{figure}
  \begin{equation}\label{eq:utxo-bootstrap}
    \inference
    {
    }
    {
      {\begin{array}{l}
        utxo_0\\
        pps
      \end{array}}
      \vdash
      \trans{utxo}{}
      \left(
        \begin{array}{l}
          \var{utxo_0}\\
          \fun{lovelaceCap} - \balance{utxo_0}
        \end{array}
      \right)
    }
  \end{equation}
  \nextdef
  \begin{equation}\label{eq:utxo-inductive}
    \inference
    { \txins{tx} \subseteq \dom \var{utxo}\\
      \var{fee} \leteq \balance{(\txins{tx} \restrictdom \var{utxo})} - \balance{(\txouts{tx})}
      & \minfee{pps}{tx} \leq \var{fee}\\
      \txins{tx} \neq \emptyset & \forall \wcard \mapsto (\wcard, c) \in \txouts{tx} \cdot 0 < c
    }
    {
      {\begin{array}{l}
        utxo_0\\
        pps
      \end{array}}
      \vdash
      \left(
          \begin{array}{l}
            \var{utxo}\\
            \var{reserves}
          \end{array}
      \right)
      \trans{utxo}{tx}
      \left(
        \begin{array}{l}
          (\txins{tx} \subtractdom \var{utxo}) \cup \txouts{tx}\\
          \var{reserves + \var{fee}}
        \end{array}
      \right)
    }
  \end{equation}
  \caption{UTxO inference rules}
  \label{fig:rules:utxo}
\end{figure}

Rule~\ref{eq:utxo-inductive} specifies under which conditions a transaction can
be applied to a set of unspent outputs, and how the set of unspent output
changes with a transaction:
\begin{itemize}
\item Each input spent in the transaction must be in the set of unspent
  outputs.
\item The minimum fee, which depends on the transaction and the protocol
  parameters, must be less or equal than the difference between the balance of
  the unspent outputs in a transaction (i.e. the total amount paid in a
  transaction) and the amount of spent inputs.
\item The set of inputs must not be empty.
\item All the transaction outputs must be positive. We do not allow $0$ value
  outputs to be consistent with the current implementation.
\item If the above conditions hold, then the new state will not have the inputs
  spent in transaction $\var{tx}$ and it will have the new outputs in
  $\var{tx}$.
\end{itemize}

\clearpage

\subsection{Witnesses}
\label{sec:witnesses}

The rules for witnesses are presented in Figure~\ref{fig:rules:utxow}. In the
initial rules note that $\var{utxoEnv}$ and $\var{utxoSt}$ are tuples, where
$\var{utxoEnv} \in \UTxOEnv$ and $\var{utxoSt} \in \UTxOState$. The definitions
used in Rule~\ref{eq:utxo-witness-inductive} are given in
Figure~\ref{fig:defs:utxow}. Note that Rule~\ref{eq:utxo-witness-inductive}
uses the transition relation defined in Figure~\ref{fig:rules:utxo}. The main
reason for doing this is to define the rules incrementally, modeling different
aspects in isolation to keep the rules as simple as possible. Also note that
the $\trans{utxo}{}$ relation could have been defined in terms of
$\trans{utxow}{}$ (thus composing the rules in a different order). The choice
here is arbitrary.

\begin{figure}[htb]
  \emph{Abstract functions}
  %
  \begin{equation*}
    \begin{array}{r@{~\in~}lr}
      \fun{wits} & \Tx \to \powerset{(\VKey \times \Sig)}
      & \text{witnesses of a transaction}\\
      \fun{hash_{vkey}} & \Addr \mapsto \HashKey
      & \text{hash of a verifying key in an address}\\
    \end{array}
  \end{equation*}
  \caption{Definitions used in the UTxO transition system with witnesses}
  \label{fig:defs:utxow}
\end{figure}
\begin{figure}[htb]
  \begin{align*}
    & \addr{}{} \in \UTxO \to \TxIn \mapsto \Addr & \text{addresses of inputs}\\
    & \addr{utxo} = \{ i \mapsto a \mid i \mapsto (a, \wcard) \in \var{utxo} \} \\
    \nextdef
    & \fun{addr_h} \in \UTxO \to \TxIn \mapsto \HashKey & \text{verifying key hashes}\\
    & \fun{addr_h}~utxo = \{ i \mapsto h \mid i \mapsto (a, \wcard) \in \var{utxo}
      \wedge a \mapsto h \in \fun{hash_{vkey}} \}
  \end{align*}
  \caption{Functions used in rules witnesses}
  \label{fig:derived-defs:utxow}
\end{figure}

\begin{figure}
  \emph{UTxO with witness transitions}
  \begin{equation*}
    \var{\_} \vdash
    \var{\_} \trans{utxow}{\_} \var{\_}
    \subseteq \powerset
    (\UTxOEnv \times \UTxOState \times (\Tx \times \powerset{(\VKey \times \Sig)}) \times \UTxOState)
  \end{equation*}
  \caption{UTxO with witness transition-system types}
  \label{fig:ts-types:utxow}
\end{figure}

\begin{figure}
  \begin{equation}\label{eq:utxow-bootstrap}
    \inference
    {
      {\begin{array}{l}
         utxoEnv
      \end{array}}
      \vdash
      \trans{\hyperref[fig:rules:utxo]{utxo}}{}
      \left(
        \var{utxoSt}
      \right)
    }
    {
      {\begin{array}{l}
         utxoEnv
      \end{array}}
      \vdash
      \trans{utxow}{}
      \left(
        \var{utxoSt}
      \right)
    }
  \end{equation}
  \nextdef
  \begin{equation}
    \label{eq:utxo-witness-inductive}
    \inference
    { \var{utxoEnv}
      \vdash
      {\left(
        \begin{array}{l}
          \var{utxo}\\
          \var{reserves}
        \end{array}
      \right)}
      \trans{\hyperref[fig:rules:utxo]{utxo}}{tx}
      {\left(
        \begin{array}{l}
          \var{utxo'}\\
          \var{reserves'}
        \end{array}
      \right)}
      & \var{maxTxSize} \mapsto \var{t} \in \var{pps} & \fun{txSize}~\var{tx} \leq t \\ ~ \\
      & \forall i \in \txins{tx} \cdot \exists (\var{vk}, \sigma) \in \wits{\var{tx}}
      \cdot
      \mathcal{V}^\sigma_{\var{vk}}~{\serialised{\txbody{tx}}}
      \wedge  \fun{addr_h}~{utxo}~i = \hash{vk}\\
    }
    {\var{utxoEnv} \vdash
      \left(
        \begin{array}{l}
          \var{utxo}\\
          \var{reserves}
        \end{array}
      \right)
      \trans{utxow}{tx}
      \left(
        \begin{array}{l}
          \var{utxo'}\\
          \var{reserves'}
        \end{array}
      \right)
    }
  \end{equation}
  \caption{UTxO with witnesses inference rules}
  \label{fig:rules:utxow}
\end{figure}

\clearpage

\subsection{Transaction sequences}
\label{sec:transaction-seqs}

\cref{fig:rules:utxow-seq} models the application of a sequence of
transactions. For the sake of concison we omit the types of this transition
system, since they are the same as the ones of $\trans{utxow}{}$.

\begin{figure}[htb]
  \begin{equation}\label{eq:utxows-bootstrap}
    \inference
    {
      {\begin{array}{l}
         utxoEnv
      \end{array}}
      \vdash
      \trans{\hyperref[fig:rules:utxo]{utxow}}{}
      \left(
        \var{utxoSt}
      \right)
    }
    {
      {\begin{array}{l}
         utxoEnv
      \end{array}}
      \vdash
      \trans{utxows}{}
      \left(
        \var{utxoSt}
      \right)
    }
  \end{equation}
  \nextdef
  \begin{equation}
    \label{eq:rule:utxow-seq-base}
    \inference
    {}
    {\var{utxoEnv} \vdash \var{utxoSt} \trans{utxows}{\epsilon} \var{utxoSt}}
  \end{equation}
  %
  \nextdef
  %
  \begin{equation}
    \label{eq:rule:utxow-seq-ind}
    \inference
    {
      \var{utxoEnv} \vdash \var{utxoSt} \trans{utxows}{\Gamma} \var{utxoSt'}
      &
      \var{utxoEnv} \vdash \var{utxoSt'} \trans{\hyperref[fig:rules:utxow]{utxow}}{\var{tx}} \var{utxoSt''}
    }
    {\var{utxoEnv} \vdash \var{utxoSt} \trans{utxows}{\Gamma;\var{tx}} \var{utxoSt''}}
  \end{equation}
  \caption{UTxO sequence rules}
  \label{fig:rules:utxow-seq}
\end{figure}

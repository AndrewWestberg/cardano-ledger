\section{TxInfo Construction}
\label{sec:txinfo}

This section specifies exactly what parts of the transaction and ledger
state are used by the $\fun{txInfo}$ function to construct the
$\TxInfo$ element that, together with the script purpose for which the script
is being run, gets passed as a $\Data$ argument to the Plutus interpreter.

\subsection{Type Translations}

We give the Plutus types corresponding to the ledger counterparts in Figures \ref{fig:txinfo-types}
and \ref{fig:txinfo-types-two}. Details for types that have non-identity translation
functions between ledger and Plutus types are in Figure \ref{fig:txinfo-translations}.

\textbf{Notation.}
In this Section we use the notation $\type{P.PlutusType}$ for the Plutus type
$\type{PlutusType}$ (specified in the Plutus library) to distinguish it from ledger types.

If $\type{LT}$ is a ledger type, and $\type{P.LT}$ is the corresponding Plutus type to
which elements of $\type{LT}$ must be translated, we denote the
translation function by:

\[ \fun{toPlutusType_{LT}} : \type{LT} \to \type{P.LT} \]

As a shorthand, we always write $t_P$ for $\fun{toPlutusType}_{LT}~t$, if $t$ is of type $\type{LT}$.

In many cases this function is simply the identity function, and all
the cases in which it is not the identity will be described below.

\textbf{Sets and maps.}
Since the $\Data$ type is effectively a serialization, we need to make some arbitrary choices
for how to encode certain types that aren't inherently ordered. We encode sets as
lexicographically ordered lists, and maps in the same way by treating them first as
sets of key-value pairs.

\textbf{Untranslatable Types.}
Certain types, such as the bootstrap address type, cannot be passed to scripts, and
are therefore translated as $\Nothing$. Any type that cannot be fully translated
as a Plutus-library type is also translated to $\Nothing$.
For example, since a bootstrap address $a \in \AddrBS$
is not translatable, neither is $a\in\Addr$, and $(a,\wcard,\wcard) \in \TxOut$
also translates to $\Nothing$.

\textbf{Certificates. }
Translating certain kinds of certificates drops the data in the
certificates, in particular, the $\DCertGen$ and $\DCertMir$ ones.
The $\DCertRegPool$ pool registration certificate has a non-identity translation function,
$\fun{transPoolCert}$.

\textbf{Time Translation.}
Functions needed to implement conversion of a slot number to POSIX time are
given in Figure \ref{fig:time-funcs}.

\textbf{Time Range Translation.}
Performed by $\fun{transVITime}$ given in Figure \ref{fig:time-funcs}.
In case when both upper and lower bounds are set to specific slots,
resulting \type{P.POSIXTimeRange} will have non-strict lower bound
and strict upper bound.

\textbf{Value Translation.}
\textbf{CAUTION}: Details provided here expose internals of the \type{Value} abstraction. 
It is \textbf{strongly} recommended to use abstractions provided by the Plutus library, 
where these details are handled under the hood for the user by library functions. 
Providing these details aims to help developers with very specific needs, 
who understand the risk of bypassing abstractions provided by the library.

During $\fun{toPlutusType}_{\Value} \in \Value \to \type{P.Value}$ translation (Figure \ref{fig:time-funcs}),
if there is no Ada in the \type{Value}, then translated \type{P.Value} will contain 0 Ada value,
i.e. there will be \type{P.Value} with 0 Ada in it. Consequently, $\TxInfo$ mint field 
will contain 0 Ada \type{P.Value} among other assets.

\textbf{Pointer Address Resolution. }
Note that the $\Ptr$ addresses translated and passed to Plutus scripts are
not resolvable to their corresponding key or script staking credencials. This
is because doing a lookup in the resolution map, which part of the ledger state,
could break determinism.

\textbf{Well-Formed Scripts and Data. }
During the transaction encoding and decoding process, a transaction is discarded if it is not
encoded correctly. This includes, in particular, a check that $\Script$ and $\Data$
elements contained in the transaction are well-formed. Functions that perfom these checks are
in Figure \ref{fig:data-script-check}. The $\fun{P.validateScript}$ function
is a Plutus-library function which checks whether a bytestring represents a
Plutus script.

\begin{figure*}[htb]
  \emph{Hash Types} \\
  \begin{equation*}
    \begin{array}{l@{~~~}l}
      \ScriptHash & \type{P.ValidatorHash} \\
      \KeyHash & \type{P.PubKeyHash} \\
      \DataHash & \type{P.DatumHash} \\
      \TxId & \type{P.TxId} \\
    \end{array}
  \end{equation*}
  \emph{Transaction Input, ie. Output Reference}  \\
  \begin{equation*}
    \begin{array}{l@{~~~}l}
      \TxIn & \type{P.TxOutRef} \\
    \end{array}
  \end{equation*}
  \emph{Credential Types} \\
  \begin{equation*}
    \begin{array}{l@{~~~}l}
      \Ptr & \type{P.StakingPtr} \\
      \Credential & \type{P.Credential} \\
      \Credential & \type{P.StakingHash} \\
      \StakeCredential & \type{P.StakingCredential} \\
    \end{array}
  \end{equation*}
  \emph{Value Types} \\
  \begin{equation*}
    \begin{array}{l@{~~~}l}
      \Wdrl & \type{P.StakingCredential} \times \Integer \\
      \PolicyID & \type{P.CurrencySymbol} \\
      \AssetName & \type{P.TokenName} \\
      \Coin & \Integer \\
      \Quantity & \Integer \\
    \end{array}
  \end{equation*}
  \emph{Certificate Types} \\
  \begin{equation*}
    \begin{array}{l@{~~~}l}
      \DCert & \type{P.DCert} \\
      \DCertRegKey & \type{P.DCertDelegRegKey} \\
      \DCertDeRegKey & \type{P.DCertDelegDeRegKey} \\
      \DCertDeleg & \type{P.DCertDelegDelegate} \\
      \DCertRetirePool & \type{P.DCertPoolRetire} \\
    \end{array}
  \end{equation*}
  \emph{Data-dropping Types} \\
  \begin{equation*}
    \begin{array}{l@{~~~}l}
      \RewardAcnt & \type{P.StakingHash} \\
      \DCertGen & \type{P.DCertGenesis} \\
      \DCertMir & \type{P.DCertMir} \\
    \end{array}
  \end{equation*}
  \caption{TxInfo and Constituent Types}
  \label{fig:txinfo-types}
\end{figure*}

\begin{figure*}[htb]
  \emph{Script Purpose Types} \\
  \begin{equation*}
    \begin{array}{l@{~~~}l}
      \mathsf{Cert} & \type{P.Certifying} \\
      \mathsf{Rwrd} & \type{P.Rewarding} \\
      \mathsf{Mint} & \type{P.Minting} \\
      \mathsf{Spend} & \type{P.Spending} \\
    \end{array}
  \end{equation*}
  \emph{Execution Budget Types} \\
  \begin{equation*}
    \begin{array}{l@{~~~}l}
      \Integer & \type{P.ExCPU} \\
      \Integer  & \type{P.ExMemory}\\
      \ExUnits & \type{P.ExBudget}  \\
    \end{array}
  \end{equation*}
  \emph{TxInfo} \\
    \begin{align*}
    & \TxInfo = \\
      & ~~~~ \seqof{\type{P.TxInInfo^?}}  % & \text{inputs} \\ %\type{P.txInfoInputs} &
      & \text{realized inputs} \\
      & ~~~~ \times \seqof{\type{P.TxOut^?}}
      & \text{outputs} \\ %\type{P.txInfoOutputs}
      & ~~~~ \times \type{P.Value}
      & \text{fee} \\ %= transValue (inject @(Mary.Value (Crypto era)) fee), \type{P.txInfoFee}
      & ~~~~ \times \type{P.Value}
      & \text{mint field} \\ % = transValue forge, \type{P.txInfoMint}
      & ~~~~ \times \seqof{\type{P.DCert}}
      & \text{list of certificates} \\ %= foldr (\c ans \to transDCert c : ans) [] (certs' tbody), \type{P.txInfoDCert}
      & ~~~~ \times (\type{P.StakingCredential} \times \Integer)
      & \text{reward withdrawal} \\ % = Map.toList (transWdrl (wdrls' tbody)), \type{P.txInfoWdrl}
      & ~~~~ \times \type{P.POSIXTimeRange}
      & \text{validity interval in POSIX time} \\ % = timeRange, \type{P.txInfoValidRange}
      & ~~~~ \times \seqof{\type{P.PubKeyHash}}
      & \text{key hashes of required signatures} \\ % = map transKeyHash (Set.toList (reqSignerHashes' tbody)), \type{P.txInfoSignatories}
      & ~~~~ \times \seqof{(\DataHash \times \Data)}
      & \text{list of datums and their hashes} \\ % = map transDataPair datpairs, \type{P.txInfoData}
      & ~~~~ \times \type{P.TxId}  % =  (transSafeHash (hashAnnotated @(Crypto era) tbody)) \type{P.txInfoId}
      & \text{transaction ID}
  \end{align*}
  \caption{TxInfo and Constituent Types}
  \label{fig:txinfo-types-two}
\end{figure*}

\begin{figure*}[htb]
  \emph{Abstract Types}
  \begin{equation*}
    \begin{array}{r@{~\in~}l@{\quad\quad\quad\quad}r}
      \var{scs} & \type{Seconds} & \text{time in seconds}
    \end{array}
  \end{equation*}
  %
  \emph{Abstract Conversion Functions}
  \begin{align*}
    & \fun{utcTimeToPOSIXSeconds} \UTCTime \to \type{Seconds} \\
    & \text{Convert UTC Time to seconds}
  \end{align*}
  %
  \emph{Rounding Functions}
  \begin{align*}
    & \fun{truncate} \type{Fraction} \to \Integer \\
    & \text{Round a fractional value to the nearest integer}
  \end{align*}
  %
  \emph{Slot to Time Conversion}
  \begin{align*}
    & \fun{slotToPOSIXTime} \in \PParams \to \EpochInfo \to \SystemStart \to \Slot \to \type{P.POSIXTime} \\
    & \fun{slotToPOSIXTime}~pp~ei~sysS~vs~=~\\
    & ~~~~\fun{truncate}~ ((\fun{utcTimeToPOSIXSeconds}~(\fun{epochInfoSlotToUTCTime}~pp~ei~sysS~vs))~*~1000) \\
    & \text{Translate validity interval to a POSIX time range}
  \end{align*}
  \caption{Types and Functions Used in Time Conversion}
  \label{fig:time-funcs}
\end{figure*}

\begin{figure*}[htb]
  \emph{Conversion Functions}
  \begin{align*}
    & \fun{toPlutusType}_{\Addr} \in \Addr  \to  \type{P.Address} \\
    & \fun{toPlutusType}_{\Addr} ~a = \begin{cases}
      \Nothing & \text{if}~a \in \AddrBS \\
      (ob,~st) & \text{if}~a = (\wcard, ob_P,~st_P)
    \end{cases} \\
    & \text{Address translation}
    \nextdef
    & \fun{toPlutusType}_{\Value} \in \Value \to \type{P.Value} \\
    & \fun{toPlutusType}_{\Value}~ (c,~ mp) = \{~ pid_P\mapsto (aid_P \mapsto q_P) \mid pid \mapsto (aid \mapsto q) \in mp ~\} \\
    & ~~~~\cup \{~\type{P.adaSymbol}\mapsto (\type{P.adaToken} \mapsto c_P) ~\} \\
    & \text{Value translation}
    \nextdef
    & \fun{toPlutusType}_{\TxOut} \in \TxOut \to \type{P.TxOut} \\
    & \fun{toPlutusType}_{\TxOut}~ (a, v, h) = (a_P, v_P, h_P) \\
    & \text{Output translation}
    \nextdef
    & \fun{toPlutusType}_{\DCert} \in \DCert \to \type{P.DCert} \\
    & \fun{toPlutusType}_{\DCert}~\var{c} = \begin{cases}
      (\fun{poolId}~\var{c},~\fun{poolVrf}~\var{c}) & \text{if } \var{c}\in \DCertRegPool \\
      c_P & \text{otherwise}
      \end{cases} \\
    & \text{Certificate translation}
    \nextdef
    & \fun{transVITime} \in \PParams \to \EpochInfo \to \SystemStart \to \ValidityInterval \to \type{P.POSIXTimeRange} \\
    & \fun{transVITime} ~pp ~ei~ sysS~ (vs,~ vf) = \\
    &  ~~~~ \begin{cases}
        \type{P.always} & \text{if } vs = \Nothing ~\wedge~ vf = \Nothing \\
        \type{P.to}~(\fun{slotToPOSIXTime}~pp~ei~sysS~vf) & \text{if } vs = \Nothing ~\wedge~ vf \neq \Nothing \\
        \type{P.from}~(\fun{slotToPOSIXTime}~pp~ei~sysS~vs) & \text{if } vs \neq \Nothing ~\wedge~ vf = \Nothing \\
        (\fun{slotToPOSIXTime}~pp~ei~sysS~vs,~\fun{slotToPOSIXTime}~pp~ei~sysS~vf) & \text{if } vs \neq \Nothing ~\wedge~ vf \neq \Nothing \\
      \end{cases} \\
  \end{align*}
  \caption{TxInfo Constituent Type Translation Functions}
  \label{fig:txinfo-translations}
\end{figure*}

\begin{figure*}[htb]
  \begin{align*}
    & \fun{validScript} \in \Script \to \Bool \\
    & \fun{validScript} ~sc = \begin{cases}
      \True & \text{if $sc \in \ScriptPhOne$} \\
      \type{P.validateScript} ~sc & \text{if $sc \in \ScriptPhTwo$}
    \end{cases} \\
    & \text{Checks if a script is constructed correctly}
  \end{align*}
  \caption{Script and Data construction correctness checks}
  \label{fig:data-script-check}
\end{figure*}

\subsection{Building Transaction Summary}

The functions in Figure \ref{fig:txinfo-funcs} are needed to build a $\Data$ summary of a transaction.
The function $\type{P.toData}$ converts a Plutus-library element
into a $\Data$ element.

\textbf{The $\fun{txInfo}$ Function.}
$\fun{txInfo}$ summarizes all the necessary transaction and chain state information
    that needs to be passed to the script interpreter. Below, we specify how to
    represent relevant transaction data in terms of a Plutus
    library-defined type. The $\fun{txInfo}$ function builds this representation.

    The $\Language$ argument
    is required because different languages have different expectations of the
    format and contents of the $\TxInfo$ summary. The $\EpochInfo$ and $\SystemStart$
    arguments are needed for translating slot numbers to POSIX time.

    Note that $\fun{txInfo}$ has a $\UTxO$ argument. Even though the full ledger UTxO
    is passed to it, we define this function in such a way that the only
    entries in the ledger UTxO map that a Plutus script
    actually sees via the argument $\fun{txInfo}$ builds are the ones corresponding to the transaction
    inputs (ie. its realized inputs). This is done in order to maintain the locality of
    evaluation. For details, see the deterministic script evaluation property~\ref{prop:fixed-inputs}.

\textbf{The $\fun{valContext}$ Function.}
    $\fun{valContext}$ constructs the \emph{validation context}, also referred to as the
    script context. A validation context is
    a $\Data$ element which encodes both the summary of the transaction and ledger information
    (this is supplied by the $\fun{txInfo}$ summarization function), and the script purpose.

\begin{figure}
  \emph{Plutus Library Functions}
  %
  \begin{align*}
      &\fun{P.toData} \in \type{P.T} \to \Data \\
      &\text{Constructs a Data element from a Plutus-library-type element of type P.T}
  \end{align*}
    \emph{Ledger Functions}
    %
    \begin{align*}
        &\fun{txInfo} \in \Language \to \PParams \to \EpochInfo \to \SystemStart \to \UTxO \to \Tx \to \TxInfo \\
        &\fun{txInfo}~ pp~ ei~ sysS~ utxo~tx = \\
        & ~~~~ (\{~(\var{txin}_P, \var{txout}_P) \mid \var{txin}\in \fun{txinputs}~tx,~(\var{txin}\mapsto\var{txout})\in\var{utxo}~\}, \\
        & ~~~~ \{~\var{tout}_P\mid\var{tout}\in\fun{txouts}~{tx}~\} , \\
        & ~~~~ (\fun{inject}~(\fun{txfee}~{tx}))_P, \\
        & ~~~~ (\fun{mint}~{tx})_P , \\
        & ~~~~ [~ c_P \mid c \in \fun{txcerts}~{tx} ~] , \\
        & ~~~~ \{~(s_P,~c_P)\mid s\mapsto c \in \fun{txwdrls}~{tx}~\} , \\
        & ~~~~ \fun{transVITime} ~pp ~ei~ sysS~ (\fun{txvldt}~tx)  , \\
        & ~~~~ \{~k_P\mid k \in \dom \fun{txwitsVKey}~{tx}~\} , \\
        & ~~~~ \{~(h_P,~d_P)\mid h\mapsto d \in \fun{txdats}~{tx}~\} , \\
        & ~~~~ (\fun{txid}~{tx})_P) \\
        &\text{Summarizes transaction data}
        \nextdef
        &\fun{valContext} \in \type{P.TxInfo} \to \ScriptPurpose \to \Data \\
        &\fun{valContext}~\var{txinfo}~\var{sp} = \fun{P.toData}~(txinfo,~sp_P) \\
        &\text{Pairs transaction data with a script purpose}
    \end{align*}
  \caption{Transaction Summarization Functions}
  \label{fig:txinfo-funcs}
\end{figure}

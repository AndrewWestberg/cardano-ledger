\section{Formal Properties}
\label{sec:properties}

This appendix collects the main formal properties that the new ledger rules are expected to satisfy.
Note here that in every property in this section we consider a only phase-1 valid
transactions, ie. ones that can indeed get processed.

\begin{enumerate}[label=P{\arabic*}:\ ]
\item
  \emph{Consistency with Shelley.}
  \begin{itemize}
    \item properties 15.6 - 15.16 (except 15.8 and 15.13) hold specifically in the $\fun{txValTag}~tx~=~\True$ case, because
    the calculations refer to the UTxO as if it was updated by a scripts-validating transaction
    \item other properties hold as-is
  \end{itemize}

\item
  \emph{Consistency with Multi-Asset.}
  \begin{itemize}
    \item properties 8.1 and 8.2 hold specifically in the $\fun{txValTag}~tx~=~\True$ case, because
    the calculations refer to the UTxO as if it was updated by a scripts-validating transaction
    \item other properties hold as-is
  \end{itemize}
\end{enumerate}


\begin{definition}
  For a state $s$ that is used in any subtransaction of
  $\mathsf{CHAIN}$, we define $\Utxo(s) \in \UTxO$ to be the $\UTxO$
  contained in $s$, or an empty map if it does not exist. This is
  similar to the definition of $\Val$ in the Shelley specification.

  Similarly, we also define $\field_{v}~(s)$ for the field $v$ that is part of
  the ledger state $s$, referenced by the typical variable name (eg. $\var{fees}$
  for the fee pot on the ledger).
\end{definition}

We also define a helper function $\varphi$ as follows:
\[\varphi(x, tx) :=
  \begin{cases}
    x & \fun{txValTag}~tx = \True \\
    0 & \fun{txValTag}~tx = \False
  \end{cases}\]
This function is useful to distinguish between the two
cases where a transaction can mint tokens or not, depending on whether
its scripts validate.

\begin{property}[General Accounting]
  \label{prop:pov}
  The \emph{general accounting} property in Alonzo encompasses two parts
  \begin{itemize}
    \item preservation of value property expressed via the $\fun{produced}$ and $\fun{consumed}$
    calculations, applicable for transactions with $\fun{txValTag}~tx~=~\True$, which
    is specified as in the ShelleyMA POV.
    \item the preservation of value for $\fun{txValTag}~tx~=~\False$, when
    only the collateral fees are collected into the fee pot.
  \end{itemize}

Both of these are expressed in the following lemma.

\begin{lemma}
  For all environments $e$, transactions $tx$ and states $s, s'$, if
  \begin{equation*}
    e\vdash s\trans{utxos}{tx}s'
  \end{equation*}
  then
  \begin{equation*}
    \Val(s) + \varphi(\fun{mint}~(\fun{txbody}~tx), tx) = \Val(s')
  \end{equation*}
\end{lemma}

\begin{proof}
  In the case that $\fun{txValTag}~tx = \True$ holds, the proof is
  identical to the proof of Lemma 9.1 of the multi-asset
  specification. Otherwise, the transition must have been done via the
  $\mathsf{Scripts-No}$ rule, which removes
  $\fun{collateral}~(\fun{txbody}~tx)$ from the UTxO, and increases the fee pot by the amount of the sum of Ada in the
  collateral UTxO entries. The value contained in $s$ is not changed.

  Note that in the $\fun{feesOK}$ function, there is a check that verifies
  that, in the case that there are phase-2 scripts, there are no non-Ada tokens in the UTxOs
  which the collateral inputs reference, so non-Ada tokens do not get minted, burned, or transferred
  in this case.
\end{proof}
\end{property}

\begin{property}[Extended UTxO Validation]
  \label{prop:fees-only}

  If a phase-1 valid transaction extends the UTxO, all its scripts validate
  (i.e. $\fun{txValTag}~tx = \True$).

  If a transaction is accepted and marked as paying collateral only
  (i.e. $\fun{txValTag}~tx = \False$), then the only change to the ledger
  when processing the transaction is that the collateral inputs
  are moved to the fee pot. In particular, it does not extend the UTxO.

  \begin{lemma}
    For all environments $e$, transactions $tx$ and states $s, s'$, if  and
    \begin{equation*}
      e\vdash s\trans{ledger}{tx}s'
    \end{equation*}
    then
    \[\Utxo(s') \nsubseteq \Utxo(s)~\Rightarrow~\fun{txValTag}~tx = \True \]

    Also,
    \[\fun{txvaltag}~tx = \False~ \Rightarrow~\]
    \begin{itemize}
      \item $ \field_{v}~(s') = \field_{v}~(s) \text{ for all } v~\neq~{fees}, ~v~\neq~{utxo}$
      \item $\field_{fees}~(s') = \field~\var{fees}~(s) + \fun{coin}~(\Val~(\fun{collateral}~{tx} \subtractdom \Utxo(s)))$
      \item $\Utxo(s') = \fun{collateral}~{tx} \subtractdom \Utxo(s)$
    \end{itemize}
  \end{lemma}

  \begin{proof}
    A ledger UTxO update is specified explicitly only by the $\mathsf{UTXOS}$ transition,
    and propagated (ie. applied as-specified) to a chain-state update via the
    $\mathsf{UTXO}$ and $\mathsf{UTXOW}$ transitions.

    The only case in which the $\mathsf{UTXOS}$ transition adds entries to the UTxO
    map is in the $\mathsf{Scripts-No}$ rule, when $\fun{txValTag}~tx = \True$.
    Adding entries to the UTxO map implies exactly that the updated map is not a
    submap of the original, so we get that

    \[\Utxo(s') \nsubseteq \Utxo(s)~\Rightarrow~\fun{txValTag}~tx = \True \]

   In the case that $\fun{txValTag}~tx = \False$, $\mathsf{LEDGER}$ calls the rule
   that does not update $\DPState$ at all, only the $\UTxOState$. This state update is specified
   in the $\mathsf{UTXOS}$ transition (and applied via the $\mathsf{UTXO}$ and $\mathsf{UTXOW}$ transitions).

   The only parts of the state that are updated are
   \begin{itemize}
     \item the fee pot, which is increased by the amount of the sum of Ada in the
     collateral UTxO entries, and
     \item the UTxO, where the collateral entries
     are removed.
   \end{itemize}
  \end{proof}
\end {property}

\begin{property}[Validating when No NN Scripts]
  \label{prop:is-valid}

Whenever a valid transaction does not have any phase-2 scripts, its
$\fun{txValTag} = \True$.

\begin{lemma}
  For all environments $e$, transactions $tx$ and states $s, s'$ such that
  \begin{equation*}
    e\vdash s\trans{ledger}{tx}s',
  \end{equation*}
  If $\range (\fun{txscripts}~tx) \cap \ScriptPhTwo = \emptyset$
  then $\fun{txValTag} = \True$.
\end{lemma}
\begin{proof}
  With the same argument as previously, we only need to discuss the
  equivalent claim for the $\mathsf{UTXOS}$ transition. Under these
  assumptions, $\var{sLst} := \fun{collectTwoPhaseScriptInputs}$ is an empty
  list. Thus $\fun{evalScripts}~sLst = \True$, and the transition rule
  had to be $\mathsf{Scripts{-}Yes}$.
\end{proof}
\end{property}

\begin{property}[Paying fees]
  \label{prop:pay-fees}

In the case that all phase-2 scripts in a transaction validate, the transaction pays
at least the minimum transaction fee amount into the fee pot. In the case that
some do not validate, it must pay at least the percentage of the minimum fee
the collateral is required to cover.

\begin{lemma}
  For all environments $e$, transactions $tx$ and states $s, s'$ such that
  \begin{equation*}
    e\vdash s\trans{ledger}{tx}s'
  \end{equation*}
  The following holds : \\
  $\field_{fees}~(s')~\geq~\field_{fees}~(s)~+
  \begin{cases}
    \fun{minfee}~{tx} & \fun{txValTag} = \True \\
    \fun{quot}~(\fun{collateralPercent}~(\field_{pp}~(s)))~*~(\fun{minfee}~{tx})~100 & \fun{txValTag} = \False\\
  \end{cases}$.
\end{lemma}
\begin{proof}
  The fee pot is updated by the $\mathsf{Scripts{-}Yes}$ and the $\mathsf{Scripts{-}No}$
  rules, one of which is necessarily called by a valid transaction via the sequence
  of transitions called in the order $\mathsf{LEDGER}, \mathsf{UTXOW}, \mathsf{UTXO}, \mathsf{UTXOS}$.

  The $\mathsf{Scripts{-}Yes}$ rule (i.e. $\fun{txValTag} = \True$)
  transfers the amount of Lovelace in the $\fun{txfee}$ field of the
  transaction to the fee pot, which is checked to be at least the $\fun{minfee}$
  by $\fun{feesOK}$ in the $\mathsf{UTXO}$ transition. So, we get

  \[\field_{fees}~(s')~=~\field_{fees}~(s)~+~\fun{txfee}~{tx}~\geq~\field_{fees}~(s)~+~\fun{minfee}~{tx}\]

  The $\mathsf{Scripts{-}No}$ rule (i.e. $\fun{txValTag} = \False$) removes the
  $\fun{collateral}$ inputs from the
  UTxO, and adds the balance of these realized inputs to the fee pot.

  \[\field_{fees}~(s')~=~field_{fees}~(s)~+~\fun{ubalance}~(\fun{collateral}~{txb} \restrictdom \var{utxo})\]

  We can conclude that $\fun{txrdmrs}$ is non-empty whenever $\fun{txValTag} = \False$
  from the following observations :
  \begin{itemize}
    \item $\fun{txValTag} = \False$ whenever $\fun{evalScripts}$ fails.
    \item $\fun{evalScripts}$ is a $\wedge$-fold over a list of scripts and
    their inputs, containing all scripts that need to be run to validate this transaction
    \item All phase-1 scripts must succeed, as this is checked in phase-1 validation (UTXOW
    rule check). Therefore,
    $\fun{evalScripts}$ encounters a failing script which is phase-2.
    \item All phase-2 scripts necessarily have an associated redeemer attached to the
    transaction (UTXOW rule check)
  \end{itemize}

  See Property~\ref{prop:fixed-inputs} for more details on what inputs $\fun{evalScripts}$ has,
  and what we can say about the outcome of its computation with respect to its inputs.

  The $\fun{feesOK}$ check enforces that if a transaction has a non-empty
  $\fun{txrdmrs}$ field, the balance of the realized collateral inputs
  (which $\field_{fees}~(s)$ is increased by as part of processing $tx$) is

  \[\fun{ubalance}~(\fun{collateral}~{txb} \restrictdom \var{utxo})~\geq~\fun{quot}~(\txfee~{txb} * (\fun{collateralPercent}~pp))~100\]

  which, since $\txfee~{txb}~\geq~\fun{minfee}~{txb}$, gives

  \[ \geq~\fun{quot}~(\fun{minfee}~{txb} * (\fun{collateralPercent}~pp))~100)) \]

  where $\fun{txbody}~{tx}~=~{txb}$. This shows that the total collateral that gets
  moved into the fee pot from the UTxO by the $\mathsf{Scripts{-}No}$ rule is at
  least the minimum transaction fee scaled by the collateral percent parameter.

  \[\field_{fees}~(s')~\geq~\field_{fees}~(s)~+~\fun{quot}~(\fun{minfee}~{txb} * (\fun{collateralPercent}~pp))~100))\]
\end{proof}

An immediate corollary to this property is that when the $\fun{collateralPercent}$ is set to 100 or more,
the transaction always pays at least the minimum fee. This, in turn, implies that it
pays at least the fee that it has stated in the $\fun{txfee}$ field.

\begin{corollary}
  For all environments $e$, transactions $tx$ and states $s, s'$ such that
  \begin{equation*}
    e\vdash s\trans{ledger}{tx}s'~~\wedge~~\fun{collateralPercent}~(\field_{pp}~(s)) \geq 100
  \end{equation*}
  The following holds :
  \[\field_{fees}~(s')~\geq~\field_{fees}~(s)~+~\fun{minfee}~{tx} \]
\end{corollary}

\begin{proof}
  In both two cases in the lemma above, the $\field_{fees}$ field in the ledger state
  is increased by at least $\fun{minfee}~{tx}$ :
  \begin{itemize}
    \item when $\fun{txValTag} = \True$, the lemma states already that
    \[\field_{fees}~(s')~\geq~\field_{fees}~(s)~+~\fun{minfee}~{tx}\]
    \item when $\fun{txValTag} = \False$, we have
    $ \field_{fees}~(s')~\geq~\field_{fees}~(s)~+~\fun{quot}~(\fun{minfee}~{txb} * (\fun{collateralPercent}~pp))~100$ \\
    $~~~\geq~\field_{fees}~(s)~+~\fun{quot}~(\fun{minfee}~{tx}~*~100)~100$ \\
    $~~~=~\field_{fees}~(s)~+~\fun{minfee}~{tx}$
  \end{itemize}
\end{proof}

\end{property}

\begin{property}[Correct tag]
  \label{prop:correct-tag}

The $\fun{txValTag}~tx$ tag of a phase-1 valid transaction must match the result of the $\fun{evalScripts}$
function for that transaction.

\begin{lemma}
  For all environments $e$, transactions $tx$ and states $s, s'$ such that
  \begin{equation*}
    e\vdash s\trans{utxos}{tx}s',
  \end{equation*}
  The following holds :
    \[\fun{txValTag}~tx ~=~ \fun{evalScripts}~{tx}~{sLst} \]
  where
  \[ \var{sLst} \leteq \fun{collectTwoPhaseScriptInputs} ~\field_{pp}(s)~\var{tx}~ \Utxo(s) \]
\end{lemma}
\begin{proof}
  Inspecting the $\mathsf{Scripts{-}Yes}$ and the $\mathsf{Scripts{-}No}$ rules of the $\mathsf{UTXOS}$ transition,
  we see that both the $\fun{txValTag}$ and the result of $\fun{evalScripts}$
  are $\True$ in the former, and $\False$ in the latter.
\end{proof}
\end{property}

\begin{property}[Replay protection]
  \label{prop:replay}

A transaction always removes at least one UTxO entry from the ledger, which provides
replay protection.

\begin{lemma}
  For all environments $e$, transactions $tx$ and states $s, s'$ such that
  \begin{equation*}
    e\vdash s\trans{ledger}{tx}s',
  \end{equation*}
  The following holds : $\Utxo~(s)~\nsubseteq~\Utxo~(s')$.
\end{lemma}
\begin{proof}
  The UTxO is updated by the $\mathsf{Scripts{-}Yes}$ and the $\mathsf{Scripts{-}No}$
  rules, on of which is necessarily called by a valid transaction. Both of these
  rules remove UTxOs corresponding to a set of inputs.

  In both cases, there is a check that the removed inputs set is non-empty.
  The $\mathsf{Scripts{-}Yes}$ rule of the $\mathsf{UTXOS}$ transition
  removes the UTxOs associated with the input set $\fun{txinputs}~{tx}$ from the ledger.
  The $\fun{txinputs}~{tx}$ set must be non-empty because there is a check in the
  $\mathsf{UTXO}$ rule (which calls $\mathsf{UTXOS}$) that ensure this is true.

  For the $\mathsf{Scripts{-}No}$ rule of the $\mathsf{UTXOS}$ transition
  removes the UTxOs associated with the input set $\fun{collateral}~{tx}$ from the ledger.
  The $\fun{collateral}~{tx}$ set must be non-empty because
  the $\fun{feesOK}$ function (called by the same rule that calls $\mathsf{UTXO},
  \mathsf{UTXOS}$) ensures that in the case that the $tx$ contains phase-2 scripts,
  $\fun{collateral}~{tx}$ must be non-empty.

  Note that by property~\ref{prop:is-valid}, phase-2 scripts must always be present
  if $\fun{txValTag}~tx = \False$ (that is, whenever $\mathsf{Scripts{-}No}$ rule is used).
\end{proof}
\end{property}

\begin{property}[UTxO-changing transitions]
  \label{prop:utxo-change}

The $\mathsf{UTXOS}$ transition fully specifies the change in the ledger $\UTxO$
for each transaction.

\begin{lemma}
  For all environments $e$, transitions $\mathsf{TRNS}$, transactions $tx$ and states $s, s'$ such that
  \begin{equation*}
    e\vdash s\trans{trns}{tx}s'
  \end{equation*}
  The following holds :
  \begin{equation*}
    \Utxo(s) \neq \Utxo(s')~\Rightarrow~e'\vdash u\trans{utxos}{tx}u'
  \end{equation*}
  where $\Utxo(s) = \Utxo(u)~\wedge~\Utxo(s') = \Utxo(u')$ for some environment $e'$,
  and states $u, u'$.
\end{lemma}
\begin{proof}
  By inspecting each transition in this specification, as well as those inherited from the
    Shelley one, we see that a non-trivial update from the UTxO of $s$ to that of $s'$
    must be done by $\mathsf{UTXOS}$. Every other transition that changes the UTxO and
    has a signal of type $\Tx$ only applies the $\mathsf{UTXOS}$.
\end{proof}
\end{property}

\begin{property}[Deterministic script evaluation]
  \label{prop:fixed-inputs}

The deterministic script evaluation property, also stated as
"script interpreter arguments are fixed", is a property of the Alonzo ledger
that guarantees that the outcome of phase-2 script validation in a (phase-1 valid) transaction
depends only on the contents of that transaction, and not on when, by who, or in what
block that transaction was submitted.

For this property, we make some assumptions about things that are outside the scope of this specification :

\begin{itemize}
  \item The Plutus script
  interpreter is a pure function that receives only the arguments provided by the ledger when it is
  invoked by calling the $\fun{runPLCScript}$ function. We assume that this is
  also true in an implementation. In particular, the interpreter does not
  obtain any system randomness, etc.

  \item We assume that constructing the validation context from $\TxInfo$ and
  the $\ScriptPurpose$ by $\fun{valContext}$ is deterministic.

  \item We do not need to check here that the hashes that are the keys of the
  $\fun{txscripts}$ and $\fun{txdats}$ fields match the computed hashes of the scripts and
  datum objects they index. This is because these hashes must be computed as part of
  the deserialization of a transaction (see the CDDL specification),
  instead of being transmitted as part of the transaction and then checked. We
  assume these hashes are correct.

  \item We assume that the consensus
  function $\fun{epochInfoSlotToUTCTime}$ for converting a slot interval into a
  system time interval is deterministic.

  \item We assume that the two global
  constants, $\EpochInfo$ and $\SystemStart$, which it also takes as parameters,
  cannot change within the span of any interval for which $\fun{epochInfoSlotToUTCTime}$
  is able to convert both endpoints to system time.
\end{itemize}

The assumptions above are implicit in the functional style of definitions in this
specification, but they are worth pointing out for implementation.
With these assumptions in mind, we can say that
script evaluation is deterministic.

We split this result into a lemma and its corollary :

\begin{itemize}
  \item First, we demonstrate that all the scripts and their arguments that are
  collected for phase-2 validation are the same for two phase-1 valid transactions with the same body,
  independent of the (necessarily valid) ledger state to which they are being applied

  \item Second, we derive the corollary that under the same validity assumptions,
  phase-2 validation results in the same outcome for both transactions
\end{itemize}

\begin{lemma}
  \label{lem:inputs}
  For all environments $e, e'$, transactions $tx, tx'$ and states $s, s', u, u'$ such that
  $e$ and $s$ are subsets of fields of some valid chain state $c$, and
  $e'$ and $s'$ are subsets of fields of some valid chain state $c'$,

  \begin{equation*}
    e\vdash s\trans{ledger}{tx}s', \\
    e'\vdash u\trans{ledger}{tx'}u', \\
    \txbody{tx} = \txbody{tx'}
  \end{equation*}

  The following holds :

  \[\fun{toSet}~(\fun{collectTwoPhaseScriptInputs} ~\field_{pp}(s)~\var{tx}~ \Utxo(s))\]
  \[ = \]
  \[\fun{toSet}~(\fun{collectTwoPhaseScriptInputs} ~\field_{pp}(u)~\var{tx'}~ \Utxo(u))\]

\end{lemma}
\begin{proof}

    The $\fun{collectTwoPhaseScriptInputs}$
    function (see \ref{fig:functions:script2})
    constructs a list where each entry contains a Plutus script
    and the arguments passed to the interpreter for its evaluation.

    To show that $\fun{collectTwoPhaseScriptInputs}$ returns a list containing
    the same elements for both $tx$ and $tx'$, we observe that
    the set of elements from which this list is produced via $\fun{toList}$
    is generated using the following functions, as well as set comprehension
    and list operations.
    We demonstrate that the functions used in this definition
    produce the same output for $tx$ and $tx'$, as all the data they
    inspect is fixed by the transaction body :

    \begin{itemize}
      \item $\fun{scriptsNeeded}$ : The $\PolicyID$, $\AddrRWD$, and $\DCert$ data
      output by this function as the second term of the hash-purpose pair
      is obtained directly from the $\fun{mint}$, $\fun{txwdrls}$,
      $\fun{txcerts}$ fields of the transaction, which are all fixed by the
      body of the transaction. These types of script purposes all include
      the hash of the validating script.

      The only other data this function inspects is the UTxO entries associated
      with the $\fun{txinputs}$ (passed via the $\UTxO$ argument) to get the realized inputs.
      We know that the UTxO is a field in a valid chain state for both the phase-1 valid
      transactions $tx$ and
      the $tx'$ ($\Utxo(s)$ and $\Utxo(u)$, respectively). This means that the $\TxId$
      in each input present in either UTxO is a hash of the body of the transaction
      containing the $\TxOut$ part of the UTxO entry indexed by that input. The
      order of the outputs is also fixed by the body, which fixes the $\Ix$ of
      the entry.

      A different value in output part of the UTxO entry (or a different order of
      outputs) would necessarily imply
      that the hash of the body containing that output must be different.
      Therefore, for all Plutus script-locked realized inputs of either transaction,
      the script hash in the payment credential of the address
      (and, by the same argument, the datum hash) are fixed by the inputs in body of the
      transaction, despite not being explicitly contained in it. We then get that

      \[ \fun{scriptsNeeded}~\Utxo(s)~(\txbody{tx}) = \fun{scriptsNeeded}~\Utxo(u)~(\txbody{tx'}) \]

      \item $\fun{getDatum}$ : In the case the script purpose
      is of the input type, the datum this function returns is one that
      is associated with the corresponding realized input. More precisely, it is the datum whose
      hash is specified in the realized input, and is looked up by hash in the $\fun{txdats}$
      transaction field. If there is no datum hash in the realized input, or the script purpose
      is of a non-input type, the empty list is returned.

      The $\mathsf{UTXOW}$ rule checks that the transaction is carrying all datums corresponding
      to its realized inputs. Since the inputs (and realized inputs) are the same
      for $tx$ and $tx'$ (fixed by the body), this guarantees that
      of the datum hashes in the realized inputs (and therefore, their preimages)
      are the same as well.

      \item $\fun{txscripts}$ : in the $\mathsf{UTXOW}$ rule, there is a check that all the
      script preimages of the script hashes returned by the $\fun{scriptsNeeded}$
      function must be present in this field.

      \item $\fun{indexedRdmrs}$ : like $\fun{scriptsNeeded}$, this function
      examines four fields fixed by the transaction body ($\fun{mint}$, $\fun{txwdrls}$,
      $\fun{txcerts}$, and $\fun{txinputs}$).
      It also looks at data in the $\fun{txrdmrs}$ field, which is fixed
      by the transaction body via the $\fun{scriptIntegrityHash}$
      hash. This is done as follows: the $\mathsf{UTXOW}$ rule verifies that this
      hash-containing field matches the computed hash
      of the preimage constructed from several fields of the transaction,
      including $\fun{txrdmrs}$ (this calculation
      is done by the $\fun{hashScriptIntegrity}$ function).

      \item $\fun{language}$ : this is directly conveyed by the type of a script.

      \item $\fun{costmdls}$ : The hash calculated by the $\fun{hashScriptIntegrity}$
      function and compared to the hash value in the $\fun{scriptIntegrityHash}$ field
      must include in the preimage the current cost models of
      all script languages of scripts carried by the transaction. Recall that
      if a cost model changed between when a transaction was submitted and the
      time at which it was processed, the field and the calculated hash values
      will not match.

      \item $\fun{valContext}$ and $\fun{txInfo}$ :
      The $\fun{valContext}$ function is assumed deterministic.
      All fields of the $\TxInfo$
      structure, with the exceptions listed below,
      are straightforward translations of the corresponding transaction body fields (see \ref{sec:txinfo}) that
      are given no additional arguments,
      and therefore completely determined by $tx$ and $tx'$. The fields not directly
      appearing in the body are :

      \begin{itemize}
        \item $\fun{txInfoInputs}$ : this field contains the realized inputs of
        the transaction which are fixed by the transaction and the unique
        UTxO entries to which the inputs correspond. As explained above,
        accessing information in realized inputs for script evaluation
        does not break determinism.

        \item $\fun{txInfoValidRange}$ : this field contains the transaction
        validity interval as system time (converted from the slot numbers, which are
        used to specify the interval in the transaction body). This conversion is
        done by a function defined in the consensus layer, and takes two global
        constants in addition to the slot interval itself. Since the slot interval
        conversion function $\fun{epochInfoSlotToUTCTime}$ necessarily
        succeeds if both $tx$ and $tx'$ pass phase-1 validation, the additional
        global constant arguments must be the same (by assumption). The determinism of this conversion
        is one of the assumptions of this property, and thus gives the same output
        for both transactions.

        \item $\fun{txInfoData}$ : this field is populated with the datums (and their
        hashes) in the transaction field $\fun{txdats}$, which are fixed by the body
        via the $\fun{scriptIntegrityHash}$ field.

        \item $\fun{txInfoId}$ : this field contains the hash of the transaction body,
        which is clearly the same for transactions with the same body.
      \end{itemize}

      Therefore, each field of the $\TxInfo$ structure is
      the same for two transactions with the same body, ie.

      \[ \fun{txInfo}~\PlutusVI~\Utxo(s)~\var{tx} = \fun{txInfo}~\PlutusVI~\Utxo(u)~\var{tx'}\]

    \end{itemize}
\end{proof}

We can now make the general statement about evaluation of all scripts done in
phase-2 validation : for any phase-1 valid transactions with the same body,
the outcome of phase-2 script evaluation is the same. We make the same assumptions
as in Lemma \ref{lem:inputs}.

\begin{corollary}
  For all environments $e, e'$, transactions $tx, tx'$ and states $s, s', u, u'$ such that
  \begin{equation*}
    e\vdash s\trans{ledger}{tx}s', \\
    e'\vdash u\trans{ledger}{tx'}u', \\
    \txbody{tx} = \txbody{tx'}
  \end{equation*}
  The following holds :

  \[\fun{evalScripts}~{tx}~ (\fun{collectTwoPhaseScriptInputs} ~\field_{pp}(s)~\var{tx}~ \Utxo(s))\]
  \[ = \]
  \[\fun{evalScripts}~{tx'}~ (\fun{collectTwoPhaseScriptInputs} ~\field_{pp}(u)~\var{tx'}~ \Utxo(u))\]

\end{corollary}

\begin{proof}
  Let us consider the use of arguments of $\fun{evalScripts}$ (see Figure \ref{fig:functions:script2}).
  The first argument (of type $\Tx$) is only inspected in the case that the script $sc$ (the first element
    in the pair at the head of the list of script-arguments pairs is a phase-1 script. Since all phase-1 scripts
    are checked in phase one of validation (see \ref{fig:rules:utxow-alonzo}) by calling $\fun{validateScript}$
    on all scripts attached to the transaction. For this to apply to $sc$, we must also show
    that $sc$ is a script attached to the transaction (see the second argument explanation).
    Note also that the $tx$ argument passed to $\fun{evalScripts}$ at the use site (in the $\mathsf{UTXOS}$ transition)
    is the same, unmodified $tx$ as is the signal for the LEDGER transition. We verify this by inspecting
    the sequence of transitions through which it is propagated
    ($\mathsf{UTXOS}$, $\mathsf{UTXO}$, $\mathsf{UTXOW}$, and $\mathsf{LEDGER}$), and their signals.

    This allows us to conclude that at every step of the iteration over the script-arguments pairs list,
    the first argument to $\fun{evalScripts}$,

    \begin{itemize}
      \item has no impact on the outcome of script evaluation in the case the script
      being validated at this step is phase-2, as it is completely ignored, and,

      \item because $\fun{collectTwoPhaseScriptInputs}$ filters out all phase-1 scripts,
      is, in fact, ignored always.
    \end{itemize}

    The second argument to $\fun{evalScripts}$, ie. the list of scripts and their arguments,
    has already been shown to contain the same tuples for both transactions in the lemma above.
    The order of the list does not affect the validation outcome, since the interpreter is run
    on each tuple of a script and its arguments independently of all other tuples in the list.
    The function $\fun{evalScripts}$ is a $\wedge$-fold over the list. Thus, we may ignore the order
    of the elements in the generated list as it does not affect the evaluation outcome.

    From this we may conclude that the outcome of both phase-1 and phase-2 script evaluations
    at each step of $\fun{evalScripts}$ must be the same for $tx$ and $tx'$. Therefore,
    the $\wedge$-fold of them done by $\fun{evalScripts}$ also produces the same outcome
    for both transactions.

\end{proof}
\end{property}


\begin{enumerate}
\item
  \emph{Commutativity of Translation.} Translate, then apply to Alonzo ledger is
  the same as apply to MA ledger, then translate the ledger.
\item
  \emph{Zero ExUnits.} If a script is run (there’s a redeemer/exunits) , it will fail with 0 units
\item
  \emph{Cost Increase.} if a transaction is valid, it will remain valid if you increase the execution units
\item
  \emph{Cost Lower Bound.} if a transaction contains at least one valid script, it must have at least one execution unit
\item
  \emph{Tx backwards Compatibility.} Any transaction that was accepted in a previous version of the ledger rules
    has exactly the same cost and effect, except that the transaction output is extended.
\item \emph{Run all scripts.} All scripts attached to a transaction are run
\item
  ... \todo{Anything else?}
\end{enumerate}

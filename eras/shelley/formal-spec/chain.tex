\section{Blockchain layer}
\label{sec:chain}

This chapter introduces the view of the blockchain layer as required for the
ledger. This includes in particular the information required for the epoch
boundary and its rewards calculation as described in Section~\ref{sec:epoch}. It
also covers the transitions that keep track of produced blocks in order to
calculate rewards and penalties for stake pools.

The main transition rule is $\mathsf{CHAIN}$ which calls the subrules
$\mathsf{NEWEPOCH}$ and $\mathsf{UPDN}$, $\mathsf{VRF}$ and $\mathsf{BBODY}$.

\subsection{Block Definitions}
\label{sec:defs-blocks}

\begin{figure*}[htb]
  \emph{Abstract types}
  %
  \begin{equation*}
    \begin{array}{r@{~\in~}lr}
      \var{h} & \HashHeader & \text{hash of a block header}\\
      \var{hb} & \HashBBody & \text{hash of a block body}\\
      \var{bn} & \BlockNo & \text{block number}\\
    \end{array}
  \end{equation*}
  %
  \emph{Operational Certificate}
  %
  \begin{equation*}
    \OCert =
    \left(
      \begin{array}{r@{~\in~}lr}
        \var{vk_{hot}} & \VKeyEv & \text{operational (hot) key}\\
        \var{n} & \N & \text{certificate issue number}\\
        c_0 & \KESPeriod & \text{start KES period}\\
        \sigma & \Sig & \text{cold key signature}\\
      \end{array}
    \right)
  \end{equation*}
  %
  \emph{Block Header Body}
  %
  \begin{equation*}
    \BHBody =
    \left(
      \begin{array}{r@{~\in~}lr}
        \var{prev} & \HashHeader^? & \text{hash of previous block header}\\
        \var{vk} & \VKey & \text{block issuer}\\
        \var{vrfVk} & \VKey & \text{VRF verification key}\\
        \var{blockno} & \BlockNo & \text{block number}\\
        \var{slot} & \Slot & \text{block slot}\\
        \eta & \Seed & \text{nonce}\\
        \var{prf}_{\eta} & \Proof & \text{nonce proof}\\
        \ell & \unitInterval & \text{leader election value}\\
        \var{prf_{\ell}} & \Proof & \text{leader election proof}\\
        \var{bsize} & \N & \text{size of the block body}\\
        \var{bhash} & \HashBBody & \text{block body hash}\\
        \var{oc} & \OCert & \text{operational certificate}\\
        \var{pv} & \ProtVer & \text{protocol version}\\
      \end{array}
    \right)
  \end{equation*}
  %
  \emph{Block Types}
  %
  \begin{equation*}
    \begin{array}{r@{~\in~}l@{\qquad=\qquad}l}
      \var{bh}
      & \BHeader
      & \BHBody \times \Sig
      \\
      \var{b}
      & \Block
      & \BHeader \times \seqof{\Tx}
    \end{array}
  \end{equation*}
  \emph{Abstract functions}
  %
  \begin{equation*}
    \begin{array}{r@{~\in~}lr}
      \bhHash{} & \BHeader \to \HashHeader
                   & \text{hash of a block header} \\
      \bHeaderSize{} & \BHeader \to \N
                   & \text{size of a block header} \\
      \bBodySize{} & \seqof{\Tx} \to \N
                   & \text{size of a block body} \\
      \slotToSeed{} & \Slot \to \Seed
                    & \text{convert a slot to a seed} \\
      \prevHashToNonce{} & \HashHeader^? \to \Seed
                    & \text{convert an optional header hash to a seed} \\
      \fun{bbodyhash} & \seqof{\Tx} \to \HashBBody \\
    \end{array}
  \end{equation*}
  %
  \emph{Accessor Functions}
  \begin{equation*}
    \begin{array}{r@{~\in~}l@{~~~~~~~~~~}r@{~\in~}lr}
      \fun{bheader} & \Block \to \BHeader &
      \fun{bhbody} & \BHeader \to \BHBody \\
      \fun{hsig} & \BHeader \to \Sig &
      \fun{bbody} & \Block \to \seqof{\Tx} \\
      \fun{bvkcold} & \BHBody \to \VKey &
      \fun{bvkvrf} & \BHBody \to \VKey \\
      \fun{bprev} & \BHBody \to \HashHeader^? &
      \fun{bslot} & \BHBody \to \Slot \\
      \fun{bblockno} & \BHBody \to \BlockNo &
      \fun{bnonce} & \BHBody \to \Seed \\
      \fun{\bprfn{}} & \BHBody \to \Proof &
      \fun{bleader} & \BHBody \to \N \\
      \fun{\bprfl{}} & \BHBody \to \Proof &
      \fun{hBbsize} & \BHBody \to \N \\
      \fun{bhash} & \BHBody \to \HashBBody &
      \fun{bocert} & \BHBody \to \OCert \\
    \end{array}
  \end{equation*}
  %
  \caption{Block Definitions}
  \label{fig:defs:blocks}
\end{figure*}

\clearpage

\subsection{MIR Transition}
\label{sec:mir-trans}

The transition which moves the instantaneous rewards is $\mathsf{MIR}$.
Figure~\ref{fig:ts-types:mir} defines the types for the transition.
It has no environment or signal, and the state is $\EpochState$.

\begin{figure}
  \emph{MIR Transitions}
  \begin{equation*}
    \vdash \var{\_} \trans{mir}{} \var{\_} \subseteq
    \powerset (\EpochState \times \EpochState)
  \end{equation*}
  \caption{MIR transition-system types}
  \label{fig:ts-types:mir}
\end{figure}

Figure~\ref{fig:rules:mir} defines the MIR state transition.

If the reserve and treasury pots are large enough to cover the sum
of the corresponding instantaneous rewards,
the reward accounts are increased by the appropriate amount
and the two pots are decreased appropriately.
In either case, if the pots are large enough or not,
we reset both of the instantaneous reward mappings back to the empty mapping.

\begin{figure}[ht]
  \begin{equation}\label{eq:mir}
    \inference[MIR]
    {
      (\var{rewards},~\var{delegations},~
      \var{ptrs},~\var{fGenDelegs},~\var{genDelegs},~\var{i_{rwd}})
        \leteq \var{ds}
      \\
      (\var{treasury},~\var{reserves})\leteq\var{acnt}
      &
      (\var{irReserves},~\var{irTreasury})\leteq\var{i_{rwd}}
      \\~\\
      \var{irwdR}\leteq
        \left\{
        \fun{addr_{rwd}}~\var{hk}\mapsto\var{val}
        ~\vert~\var{hk}\mapsto\var{val}\in(\dom{rewards})\restrictdom\var{irReserves}
        \right\}
      \\
      \var{irwdT}\leteq
        \left\{
        \fun{addr_{rwd}}~\var{hk}\mapsto\var{val}
        ~\vert~\var{hk}\mapsto\var{val}\in(\dom{rewards})\restrictdom\var{irTreasury}
        \right\}
      \\~\\
      \var{totR}\leteq\sum\limits_{\wcard\mapsto v\in\var{irwdR}}v
      &
      \var{totT}\leteq\sum\limits_{\wcard\mapsto v\in\var{irwdT}}v
      \\
      \var{enough}\leteq
          \var{totR}\leq\var{reserves}
          \land\var{totT}\leq\var{treasury}
      \\
      \var{acnt'}\leteq
      {\begin{cases}
          (\var{treasury}-\var{totT},~\var{reserves}-\var{totR})
          & \var{enough}
          \\
          \var{acnt}
          &
          \text{otherwise}
       \end{cases}}
      \\~\\
      \var{rewards'}\leteq
      {\begin{cases}
          \var{rewards}\unionoverridePlus\var{irwdR}\unionoverridePlus\var{irwdT}
          & \var{enough}
          \\
          \var{rewards}
          &
          \text{otherwise}
       \end{cases}}
      \\
      \var{ds'} \leteq
      (\varUpdate{\var{rewards}'},~\var{delegations},~
      \var{ptrs},~\var{fGenDelegs},~\var{genDelegs},
      ~(\varUpdate{\emptyset},~\varUpdate{\emptyset}))
    }
    {
      \vdash
      {\left(\begin{array}{c}
            \var{acnt} \\
            \var{ss} \\
            (\var{us},~(\var{ds},~\var{ps})) \\
            \var{prevPP} \\
            \var{pp} \\
      \end{array}\right)}
      \trans{mir}{}
      {\left(\begin{array}{c}
            \varUpdate{\var{acnt'}} \\
            \var{ss} \\
            (\var{us},~(\varUpdate{\var{ds'}},~\var{ps})) \\
            \var{prevPP} \\
            \var{pp} \\
      \end{array}\right)}
    }
  \end{equation}

  \caption{MIR rules}
  \label{fig:rules:mir}
\end{figure}

\subsection{New Epoch Transition}
\label{sec:new-epoch-trans}

For the transition to a new epoch ($\mathsf{NEWEPOCH}$), the
environment is empty and state is given in
Figure~\ref{fig:ts-types:newepoch}, it consists of

\begin{itemize}
\item The number of the last epoch.
\item The information about produced blocks for each stake pool during the previous epoch.
\item The information about produced blocks for each stake pool during the current epoch.
\item The old epoch state.
\item An optional rewards update.
\item The stake pool distribution of the epoch.
\end{itemize}

\begin{figure}
  \emph{New Epoch states}
  \begin{equation*}
    \NewEpochState =
    \left(
      \begin{array}{r@{~\in~}lr}
        \var{e_\ell} & \Epoch & \text{last epoch} \\
        \var{b_{prev}} & \BlocksMade & \text{blocks made last epoch} \\
        \var{b_{cur}} & \BlocksMade & \text{blocks made this epoch} \\
        \var{es} & \EpochState & \text{epoch state} \\
        \var{ru} & \RewardUpdate^? & \text{reward update} \\
        \var{pd} & \PoolDistr & \text{pool stake distribution} \\
      \end{array}
    \right)
  \end{equation*}
  %
  \emph{New Epoch Transitions}
  \begin{equation*}
    \vdash \var{\_} \trans{newepoch}{\_} \var{\_} \subseteq
    \powerset (\NewEpochState \times \Epoch \times \NewEpochState)
  \end{equation*}
  %
  \emph{Helper function}
  \begin{align*}
      & \fun{calculatePoolDistr} \in \Snapshot \to \PoolDistr \\
      & \fun{calculatePoolDistr}~(\var{stake},~\var{delegs},~\var{poolParams}) = \\
      & ~~~\left\{\var{hk_p}\mapsto(\sigma,~\fun{poolVRF}~\var{p})
            ~\Big\vert~
            {
              \begin{array}{r@{~\in~}l}
                \var{hk_p}\mapsto\sigma & \var{sd} \\
                \var{hk_p}\mapsto\var{p} & \var{poolParams}
              \end{array}
            }
            \right\}\\
      & ~~~~\where \\
      & ~~~~~~~~~\var{total} = \sum_{\_ \mapsto c\in\var{stake}} c \\
      & ~~~~~~~~~\var{sd} = \fun{aggregate_{+}}~\left(\var{delegs}^{-1}\circ
                     \left\{\left(
                       \var{hk}, \frac{\var{c}}{\var{total}}
                     \right) \vert (\var{hk},
                     \var{c}) \in \var{stake}
                     \right\}\right) \\
  \end{align*}

  \caption{NewEpoch transition-system types}
  \label{fig:ts-types:newepoch}
\end{figure}

Figure~\ref{fig:rules:new-epoch} defines the new epoch state transition. It has three rules.
The first rule describes the change in the case of $e$ being equal to the next epoch $e_\ell+ 1$.
It also calls the $\mathsf{MIR}$ and $\mathsf{EPOCH}$ rules and checks that the reward update
is net neutral with respect to the Ada in the system.
This should always hold (by the definition of the $\fun{createRUpd}$ function)
and is present only for extra assurance and for help in proving
that Ada is preserved by this transition.
The second rule deals with the case when the epoch signal $e$ is not one greater than the
current epoch \var{e_\ell}. This rule does not change the state.
The third rule is nearly the same as the first rule, only there is no reward update to apply.
This third rule is defined for completeness, but in practice we hope that it is never used,
since it only applies in the case that epoch $e$ processed no blocks after the first
$\StabilityWindow$-many slots.

In the first case, the new epoch state is updated as follows:

\begin{itemize}
\item The epoch is set to the new epoch $e$.
\item The mapping for the blocks produced by each stake pool for the previous epoch
  is set to the current such mapping.
\item The mapping for the blocks produced by each stake pool for the current epoch
  is set to the empty map.
\item The epoch state is updated with: first applying the rewards update \var{ru},
  then calling the $\mathsf{MIR}$ transition, and finally by calling the
  $\mathsf{EPOCH}$ transition.
\item The rewards update is set to \Nothing.
\item The new pool distribution \var{pd}' is calculated from the delegation map and
  stake allocation of the previous epoch.
\end{itemize}

\begin{figure}[ht]
  \begin{equation}\label{eq:new-epoch}
    \inference[New-Epoch]
    {
      e = e_\ell + 1
      &
      \var{ru} \neq \Nothing
      &
      (\Delta t,~\Delta r,~\var{rs},~\Delta f)\leteq\var{ru}
      \\
      \Delta t+~\Delta r+\left(\sum\limits_{\wcard\mapsto v\in\var{rs}} v\right)+\Delta f=0
      \\
      \var{es'}\leteq\fun{applyRUpd}~\var{ru}~\var{es}
      &
      {
        \vdash
        \var{es'}
          \trans{\hyperref[fig:rules:mir]{mir}}{}\var{es''}
      }
      &
      {
        \vdash
        \var{es''}
          \trans{\hyperref[fig:rules:epoch]{epoch}}{\var{e}}\var{es'''}
      }
      \\~\\
      {\begin{array}{r@{~\leteq~}l}
         (\var{acnt},~\var{ss},~\wcard,~\wcard,~\wcard) & \var{es'''} \\
         (\wcard,~\var{pstake_{set}},~\wcard,~\wcard) & \var{ss} \\
         \var{pd'} & \fun{calculatePoolDistr}~\var{pstake_{set}} \\
       \end{array}}
    }
    {
      \vdash
      {\left(\begin{array}{c}
            \var{e_\ell} \\
            \var{b_{prev}} \\
            \var{b_{cur}} \\
            \var{es} \\
            \var{ru} \\
            \var{pd} \\
      \end{array}\right)}
      \trans{newepoch}{\var{e}}
      {\left(\begin{array}{c}
            \varUpdate{\var{e}} \\
            \varUpdate{\var{b_{cur}}} \\
            \varUpdate{\emptyset} \\
            \varUpdate{\var{es'''}} \\
            \varUpdate{\Nothing} \\
            \varUpdate{\var{pd}'} \\
      \end{array}\right)}
    }
  \end{equation}

  \nextdef

  \begin{equation}\label{eq:not-new-epoch}
    \inference[Not-New-Epoch]
    {
      (e_\ell,~\wcard,~\wcard,~\wcard,~\wcard,~\wcard,~\wcard)\leteq\var{nes}
      &
      e \neq e_\ell + 1
    }
    {
      \vdash\var{nes}\trans{newepoch}{\var{e}} \var{nes}
    }
  \end{equation}

  \nextdef

  \begin{equation}\label{eq:no-reward-update}
    \inference[No-Reward-Update]
    {
      (e_\ell,~\wcard,~\wcard,~\wcard,~\var{ru},~\wcard,~\wcard)\leteq\var{nes}
      &
      e = e_\ell + 1
      &
      \var{ru} = \Nothing
      \\
      {
        \vdash
        \var{es}
          \trans{\hyperref[fig:rules:mir]{mir}}{}\var{es''}
      }
      &
      {
        \vdash
        \var{es''}
          \trans{\hyperref[fig:rules:epoch]{epoch}}{\var{e}}\var{es'''}
      }
      \\~\\
      {\begin{array}{r@{~\leteq~}l}
         (\var{acnt},~\var{ss},~\wcard,~\wcard,~\wcard) & \var{es'''} \\
         (\wcard,~\var{pstake_{set}},~\wcard,~\wcard) & \var{ss} \\
         \var{pd'} & \fun{calculatePoolDistr}~\var{pstake_{set}} \\
       \end{array}}
    }
    {
      \vdash
      {\left(\begin{array}{c}
            \var{e_\ell} \\
            \var{b_{prev}} \\
            \var{b_{cur}} \\
            \var{es} \\
            \var{ru} \\
            \var{pd} \\
      \end{array}\right)}
      \trans{newepoch}{\var{e}}
      {\left(\begin{array}{c}
            \varUpdate{\var{e}} \\
            \varUpdate{\var{b_{cur}}} \\
            \varUpdate{\emptyset} \\
            \varUpdate{\var{es'''}} \\
            \var{ru} \\
            \varUpdate{\var{pd}'} \\
      \end{array}\right)}
    }
  \end{equation}
  \caption{New Epoch rules}
  \label{fig:rules:new-epoch}
\end{figure}

\clearpage

\subsection{Tick Nonce Transition}
\label{sec:tick-nonce-trans}

The Tick Nonce Transition is responsible for updating the epoch nonce and the
previous epoch's hash nonce at the start of an epoch. Its environment is shown in
Figure~\ref{fig:ts-types:ticknonce} and consists of the protocol parameters
$\var{pp}$, the candidate nonce $\eta_c$ and the previous epoch's last block
header hash as a nonce. Its state consists of the epoch nonce $\eta_0$ and
the previous epoch's last block header hash nonce.

\begin{figure}
  \emph{Tick Nonce environments}
  \begin{equation*}
    \TickNonceEnv =
    \left(
      \begin{array}{r@{~\in~}lr}
        \var{pp} & \PParams & \text{protocol parameters} \\
        \eta_c & \Seed & \text{candidate nonce} \\
        \eta_\var{ph} & \Seed & \text{previous header hash as nonce} \\
      \end{array}
    \right)
  \end{equation*}
  %
  \emph{Tick Nonce states}
  \begin{equation*}
    \TickNonceState =
    \left(
      \begin{array}{r@{~\in~}lr}
        \eta_0 & \Seed & \text{epoch nonce} \\
        \eta_h & \Seed & \text{seed generated from hash of previous epoch's last block header} \\
      \end{array}
    \right)
  \end{equation*}
  \label{fig:ts-types:ticknonce}
\end{figure}

The signal to the transition rule $\mathsf{TICKN}$ is a marker indicating
whether we are in a new epoch. If we are in a new epoch, we update the epoch
nonce and the previous hash. Otherwise, we do nothing.

\begin{figure}[ht]
  \begin{equation}\label{eq:tick-nonce-notnewepoch}
   \inference[Not-New-Epoch]
   { }
   {
     {\begin{array}{c}
        \var{pp} \\
        \eta_c \\
        \eta_\var{ph} \\
      \end{array}}
     \vdash
     {\left(\begin{array}{c}
           \eta_0 \\
           \eta_h \\
     \end{array}\right)}
     \trans{tickn}{\mathsf{False}}
     {\left(\begin{array}{c}
           \eta_0 \\
           \eta_h \\
     \end{array}\right)}
   }
 \end{equation}

 \nextdef

 \begin{equation}\label{eq:tick-nonce-newepoch}
   \inference[New-Epoch]
   {
     \eta_e \leteq \fun{extraEntropy}~\var{pp}
   }
   {
     {\begin{array}{c}
        \var{pp} \\
        \eta_c \\
        \eta_\var{ph} \\
      \end{array}}
     \vdash
     {\left(\begin{array}{c}
           \eta_0 \\
           \eta_h \\
     \end{array}\right)}
     \trans{tickn}{\mathsf{True}}
     {\left(\begin{array}{c}
           \varUpdate{\eta_c \seedOp \eta_h \seedOp \eta_e} \\
           \varUpdate{\eta_\var{ph}} \\
     \end{array}\right)}
   }
 \end{equation}

 \caption{Tick Nonce rules}
 \label{fig:rules:tick-nonce}
\end{figure}

\subsection{Update Nonce Transition}
\label{sec:update-nonces-trans}

The Update Nonce Transition updates the nonces until the randomness gets fixed.
The environment is shown in Figure~\ref{fig:ts-types:updnonce} and consists of
the block nonce $\eta$.
The update nonce state is shown in Figure~\ref{fig:ts-types:updnonce} and consists of
the candidate nonce $\eta_c$ and the evolving nonce $\eta_v$.

\begin{figure}
  \emph{Update Nonce environments}
  \begin{equation*}
    \UpdateNonceEnv =
    \left(
      \begin{array}{r@{~\in~}lr}
        \eta & \Seed & \text{new nonce} \\
      \end{array}
    \right)
  \end{equation*}
  %
  \emph{Update Nonce states}
  \begin{equation*}
    \UpdateNonceState =
    \left(
      \begin{array}{r@{~\in~}lr}
        \eta_v & \Seed & \text{evolving nonce} \\
        \eta_c & \Seed & \text{candidate nonce} \\
      \end{array}
    \right)
  \end{equation*}
  %
  \emph{Update Nonce Transitions}
  \begin{equation*}
    \_ \vdash \var{\_} \trans{updn}{\_} \var{\_} \subseteq
    \powerset (\UpdateNonceEnv
               \times \UpdateNonceState
               \times \Slot
               \times \UpdateNonceState
              )
  \end{equation*}
  \caption{UpdNonce transition-system types}
  \label{fig:ts-types:updnonce}
\end{figure}

The transition rule $\mathsf{UPDN}$ takes the slot \var{s} as signal. There are
two different cases for $\mathsf{UPDN}$: one where \var{s} is not yet
\StabilityWindow{} slots from the beginning of the next epoch and one where
\var{s} is less than \StabilityWindow{} slots until the start of the next epoch.

Note that in \ref{eq:update-both}, the nonce candidate $\eta_c$ transitions to
$\eta_v\seedOp\eta$, not $\eta_c\seedOp\eta$. The reason for this is that even
though the nonce candidate is frozen sometime during the epoch, we want the two
nonces to again be equal at the start of a new epoch (so that the entropy added
near the end of the epoch is not discarded).

\begin{figure}[ht]

  \begin{equation}\label{eq:update-both}
    \inference[Update-Both]
    {
      s < \fun{firstSlot}~((\epoch{s}) + 1) - \StabilityWindow
    }
    {
      {\begin{array}{c}
         \eta \\
       \end{array}}
      \vdash
      {\left(\begin{array}{c}
            \eta_v \\
            \eta_c \\
      \end{array}\right)}
      \trans{updn}{\var{s}}
      {\left(\begin{array}{c}
            \varUpdate{\eta_v\seedOp\eta} \\
            \varUpdate{\eta_v\seedOp\eta} \\
      \end{array}\right)}
    }
  \end{equation}

  \nextdef

  \begin{equation}\label{eq:only-evolve}
    \inference[Only-Evolve]
    {
      s \geq \fun{firstSlot}~((\epoch{s}) + 1) - \StabilityWindow
    }
    {
      {\begin{array}{c}
         \eta \\
       \end{array}}
      \vdash
      {\left(\begin{array}{c}
            \eta_v \\
            \eta_c \\
      \end{array}\right)}
      \trans{updn}{\var{s}}
      {\left(\begin{array}{c}
            \varUpdate{\eta_v\seedOp\eta} \\
            \eta_c \\
      \end{array}\right)}
    }
  \end{equation}
  \caption{Update Nonce rules}
  \label{fig:rules:update-nonce}
\end{figure}

\subsection{Reward Update Transition}
\label{sec:reward-update-trans}

The Reward Update Transition calculates a new $\RewardUpdate$ to apply in a
$\mathsf{NEWEPOCH}$ transition. The environment is shown in
Figure~\ref{fig:ts-types:reward-update}, it consists of the produced blocks
mapping \var{b} and the epoch state \var{es}. Its state is an optional reward
update.

\begin{figure}
  \emph{Reward Update environments}
  \begin{equation*}
    \RUpdEnv =
    \left(
      \begin{array}{r@{~\in~}lr}
        \var{b} & \BlocksMade & \text{blocks made} \\
        \var{es} & \EpochState & \text{epoch state} \\
      \end{array}
    \right)
  \end{equation*}
  %
  \emph{Reward Update Transitions}
  \begin{equation*}
    \_ \vdash \var{\_} \trans{rupd}{\_} \var{\_} \subseteq
    \powerset (\RUpdEnv \times \RewardUpdate^? \times \Slot \times \RewardUpdate^?)
  \end{equation*}
  \caption{Reward Update transition-system types}
  \label{fig:ts-types:reward-update}
\end{figure}

The transition rules are shown in Figure~\ref{fig:rules:reward-update}. There
are three cases, one which computes a new reward update, one which leaves the
rewards update unchanged as it has not yet been applied and finally one that
leaves the reward update unchanged as the transition was started too early.

The signal of the transition rule $\mathsf{RUPD}$ is the slot \var{s}. The
execution of the transition role is as follows:

\begin{itemize}
\item If the current reward update \var{ru} is empty and \var{s} is greater than
  the sum of the first slot of its epoch and the duration \RandomnessStabilisationWindow, then a
  new rewards update is calculated and the state is updated.
  (Note the errata in Section~\ref{sec:errata:stability-windows}.)
\item If the current reward update \var{ru} is not \Nothing, i.e., a reward
  update has already been calculated but not yet applied, then the state is not updated.
\item If the current reward update \var{ru} is empty and \var{s} is less than or equal to the sum
  of the first slot of its epoch and the duration to start rewards \RandomnessStabilisationWindow,
  then the state is not updated.
\end{itemize}

\begin{figure}[ht]
  \begin{equation}\label{eq:reward-update}
    \inference[Create-Reward-Update]
    {
      s > \fun{firstSlot}~(\epoch{s}) + \RandomnessStabilisationWindow
      &
      ru = \Nothing
      \\~\\
      ru' \leteq \createRUpd{\SlotsPerEpoch}{b}{es}{\MaxLovelaceSupply}
    }
    {
      {\begin{array}{c}
         \var{b} \\
         \var{es} \\
       \end{array}}
      \vdash
      \var{ru}\trans{rupd}{\var{s}}\varUpdate{\var{ru}'}
    }
  \end{equation}

  \nextdef

  \begin{equation}\label{eq:reward-update-exists}
    \inference[Reward-Update-Exists]
    {
      ru \neq \Nothing
    }
    {
      {\begin{array}{c}
         \var{b} \\
         \var{es} \\
       \end{array}}
      \vdash
      \var{ru}\trans{rupd}{\var{s}}\var{ru}
    }
  \end{equation}

  \nextdef

  \begin{equation}\label{eq:reward-too-early}
    \inference[Reward-Too-Early]
    {
      ru = \Nothing
      \\
      s \leq \fun{firstSlot}~(\epoch{s}) + \RandomnessStabilisationWindow
    }
    {
      {\begin{array}{c}
         \var{b} \\
         \var{es} \\
       \end{array}}
      \vdash
      \var{ru}\trans{rupd}{\var{s}}\var{ru}
    }
  \end{equation}

  \caption{Reward Update rules}
  \label{fig:rules:reward-update}
\end{figure}

\subsection{Chain Tick Transition}
\label{sec:tick-trans}

The Chain Tick Transition performs some chain level
upkeep. The environment consists of a set of genesis keys, and the state is the
epoch specific state necessary for the $\mathsf{NEWEPOCH}$ transition.

Part of the upkeep is updating the genesis key delegation mapping
according to the future delegation mapping.
For each genesis key, we adopt the most recent delegation in $\var{fGenDelegs}$
that is past the current slot, and any future genesis key delegations past the current
slot is removed. The helper function $\fun{adoptGenesisDelegs}$ accomplishes the update.

\begin{figure}
  \emph{Chain Tick Transitions}
  \begin{equation*}
    \vdash \var{\_} \trans{tick}{\_} \var{\_} \subseteq
    \powerset (\NewEpochState \times \Slot \times \NewEpochState)
  \end{equation*}
  \caption{Tick transition-system types}
  \label{fig:ts-types:tick}
  %
  \emph{helper function}
  \begin{align*}
      & \fun{adoptGenesisDelegs} \in \EpochState \to \Slot \to EpochState
      \\
      & \fun{adoptGenesisDelegs}~\var{es}~\var{slot} = \var{es'}
      \\
      & ~~~~\where
      \\
      & ~~~~~~~~~~
      (\var{acnt},~\var{ss},(\var{us},(\var{ds},\var{ps})),~\var{prevPp},~\var{pp})
      \leteq\var{es}
      \\
      & ~~~~~~~~~~
      (~\var{rewards},~\var{delegations},~\var{ptrs},
      ~\var{fGenDelegs},~\var{genDelegs},~\var{i_{rwd}})\leteq\var{ds}
      \\
      & ~~~~~~~~~~\var{curr}\leteq
        \{
          (\var{s},~\var{gkh})\mapsto(\var{vkh},~\var{vrf})\in\var{fGenDelegs}
          ~\mid~
          \var{s}\leq\var{slot}
        \}
      \\
      & ~~~~~~~~~~\var{fGenDelegs'}\leteq
          \var{fGenDelegs}\setminus\var{curr}
      \\
      & ~~~~~~~~~~\var{genDelegs'}\leteq
          \left\{
            \var{gkh}\mapsto(\var{vkh},~\var{vrf})
            ~\mathrel{\Bigg|}~
            {
              \begin{array}{l}
                (\var{s},~\var{gkh})\mapsto(\var{vkh},~\var{vrf})\in\var{curr}\\
                \var{s}=\max\{s'~\mid~(s',~\var{gkh})\in\dom{\var{curr}}\}
              \end{array}
            }
          \right\}
      \\
      & ~~~~~~~~~~\var{ds'}\leteq
          (\var{rewards},~\var{delegations},~\var{ptrs},
          ~\var{fGenDelegs'},~\var{genDelegs}\unionoverrideRight\var{genDelegs'},~\var{i_{rwd}})
      \\
      & ~~~~~~~~~~\var{es'}\leteq
      (\var{acnt},~\var{ss},(\var{us},(\var{ds'},\var{ps})),~\var{prevPp},~\var{pp})
  \end{align*}
\end{figure}

The $\mathsf{TICK}$ transition rule is shown in Figure~\ref{fig:rules:tick}.
The signal is a slot \var{s}.

Three transitions are done:

\begin{itemize}
  \item The $\mathsf{NEWEPOCH}$ transition performs any state change needed if it is the first
    block of a new epoch.
  \item The $\mathsf{RUPD}$ creates the reward update if it is late enough in the epoch.
    \textbf{Note} that for every block header, either $\mathsf{NEWEPOCH}$ or $\mathsf{RUPD}$
    will be the identity transition, and so, for instance, it does not matter if $\mathsf{RUPD}$
    uses $\var{nes}$ or $\var{nes}'$ to obtain the needed state.
\end{itemize}

\begin{figure}[ht]
  \begin{equation}\label{eq:tick}
    \inference[Tick]
    {
      {
        \vdash
        \var{nes}
        \trans{\hyperref[fig:rules:new-epoch]{newepoch}}{\epoch{slot}}
        \var{nes}'
      }
      \\~\\
      (\wcard,~\var{b_{prev}},~\wcard,~\var{es},~\wcard,~\wcard)\leteq\var{nes} \\
      \\~\\
      {
        {\begin{array}{c}
           \var{b_{prev}} \\
           \var{es} \\
         \end{array}}
        \vdash \var{ru'}\trans{\hyperref[fig:rules:reward-update]{rupd}}{\var{slot}} \var{ru''}
      }
      \\~\\
      (\var{e_\ell'},~\var{b_{prev}'},~\var{b_{cur}'},~\var{es'},~\var{ru'},~\var{pd'})
      \leteq\var{nes'}
      \\
      \var{es''}\leteq\fun{adoptGenesisDelegs}~\var{es'}~\var{slot}
      \\
      \var{nes''}\leteq
      (\var{e_\ell'},~\var{b_{prev}'},~\var{b_{cur}'},~\var{es''},~\var{ru''},~\var{pd'})
      \\~\\
    }
    {
      \vdash\var{nes}\trans{tick}{\var{slot}}\varUpdate{\var{nes''}}
    }
  \end{equation}
  \caption{Tick rules}
  \label{fig:rules:tick}
\end{figure}

\clearpage

\subsection{Operational Certificate Transition}
\label{sec:oper-cert-trans}

The Operational Certificate Transition environment consists of the genesis key
delegation map $\var{genDelegs}$ and the set of stake pools $\var{stpools}$. Its state
is the mapping of operation certificate issue numbers.  Its signal is a block
header.

\begin{figure}
  \emph{Operational Certificate environments}
  \begin{equation*}
    \OCertEnv =
    \left(
      \begin{array}{r@{~\in~}lr}
        \var{stpools} & \powerset{\type{KeyHash}} & \text{stake pools} \\
        \var{genDelegs} & \powerset{\type{KeyHash}} & \text{genesis key delegates}\\
      \end{array}
    \right)
  \end{equation*}
  %
  \emph{Operational Certificate Transitions}
  \begin{equation*}
    \var{\_} \vdash \var{\_} \trans{ocert}{\_} \var{\_} \subseteq
    \powerset (\OCertEnv \times \KeyHash_{pool} \mapsto \N \times \BHeader \times \KeyHash_{pool} \mapsto \N)
  \end{equation*}
  %
  \emph{Operational Certificate helper function}
  \begin{align*}
      & \fun{currentIssueNo} \in \OCertEnv \to (\KeyHash_{pool} \mapsto \N)
                                           \to \KeyHash_{pool}
                                           \to \N^? \\
      & \fun{currentIssueNo}~(\var{stpools}, \var{genDelegs})~ \var{cs} ~\var{hk} =
      \begin{cases}
        \var{hk}\mapsto \var{n} \in \var{cs} & n \\
        \var{hk} \in \var{stpools} & 0 \\
        \var{hk} \in \var{genDelegs} & 0 \\
        \text{otherwise} & \Nothing
      \end{cases}
  \end{align*}
  \caption{OCert transition-system types}
  \label{fig:ts-types:ocert}
\end{figure}

The transition rule is shown in Figure~\ref{fig:rules:ocert}. From the block
header body \var{bhb} we first extract the following:

\begin{itemize}
  \item The operational certificate, consisting of the hot key \var{vk_{hot}},
    the certificate issue number \var{n}, the KES period start \var{c_0} and the cold key
  signature.
\item The cold key \var{vk_{cold}}.
\item The slot \var{s} for the block.
\item The number of KES periods that have elapsed since the start period on the certificate.
\end{itemize}

Using this we verify the preconditions of the operational certificate state
transition which are the following:

\begin{itemize}
\item The KES period of the slot in the block header body must be greater than or equal to
  the start value \var{c_0} listed in the operational certificate,
  and less than $\MaxKESEvo$-many KES periods after \var{c_0}.
  The value of $\MaxKESEvo$ is the agreed-upon lifetime of an operational certificate,
  see \cite{delegation_design}.
\item \var{hk} exists as key in the mapping of certificate issues numbers to a KES
  period \var{m} and that period is less than or equal to \var{n}.
\item The signature $\tau$ can be verified with the cold verification key
  \var{vk_{cold}}.
\item The KES signature $\sigma$ can be verified with the hot verification key
  \var{vk_{hot}}.
\end{itemize}

After this, the transition system updates the operational certificate state by
updating the mapping of operational certificates where it overwrites the entry
of the key \var{hk} with the KES period \var{n}.

\begin{figure}[ht]
  \begin{equation}\label{eq:ocert}
    \inference[OCert]
    {
      (\var{bhb},~\sigma)\leteq\var{bh}
      &
      (\var{vk_{hot}},~n,~c_{0},~\tau) \leteq \bocert{bhb}
      &
      \var{vk_{cold}} \leteq \bvkcold{bhb}
      \\
      \var{hk} \leteq \hashKey{vk_{cold}}
      &
      \var{s}\leteq\bslot{bhb}
      &
      t \leteq \kesPeriod{s} - c_0
      \\~\\
      c_0 \leq \kesPeriod{s} < c_0 + \MaxKESEvo
      \\
      \fun{currentIssueNo} ~ \var{oce} ~ \var{cs} ~ \var{hk} = m
      &
      m \leq n
      \\~\\
      \mathcal{V}_{\var{vk_{cold}}}{\serialised{(\var{vk_{hot}},~n,~c_0)}}_{\tau}
      &
      \mathcal{V}^{\mathsf{KES}}_{vk_{hot}}{\serialised{bhb}}_{\sigma}^{t}
      \\
    }
    {
      \var{oce}\vdash\var{cs}
      \trans{ocert}{\var{bh}}\varUpdate{\var{cs}\unionoverrideRight\{\var{hk}\mapsto n\}}
    }
  \end{equation}
  \caption{OCert rules}
  \label{fig:rules:ocert}
\end{figure}

The OCERT rule has six predicate failures:
\begin{itemize}
\item If the KES period is less than the KES period start in the certificate,
  there is a \emph{KESBeforeStart} failure.
\item If the KES period is greater than or equal to the KES period end (start +
  $\MaxKESEvo$) in the certificate, there is a \emph{KESAfterEnd} failure.
\item If the period counter in the original key hash counter mapping is larger
  than the period number in the certificate, there is a \emph{CounterTooSmall}
  failure.
\item If the signature of the hot key, KES period number and period start is
  incorrect, there is an \emph{InvalidSignature} failure.
\item If the KES signature using the hot key of the block header body is
  incorrect, there is an \emph{InvalideKesSignature} failure.
\item If there is no entry in the key hash to counter mapping for the cold key,
  there is a \emph{NoCounterForKeyHash} failure.
\end{itemize}

\subsection{Verifiable Random Function}
\label{sec:verif-rand-funct}

In this section we define a function $\fun{vrfChecks}$ which performs all the VRF related checks
on a given block header body.
In addition to the block header body, the function requires the epoch nonce,
the stake distribution (aggregated by pool), and the active slots coefficient from the protocol
parameters. The function checks:

\begin{itemize}
\item The validity of the proofs for the leader value and the new nonce.
\item The verification key is associated with relative stake $\sigma$ in the stake distribution.
\item The $\fun{bleader}$ value of \var{bhb} indicates a possible leader for
  this slot. The function $\fun{checkLeaderVal}$ is defined in \ref{sec:leader-value-calc}.
\end{itemize}

\begin{figure}
  \emph{VRF helper function}
  \begin{align*}
      & \fun{issuerIDfromBHBody} \in \BHBody \to \KeyHash_{pool} \\
      & \fun{issuerIDfromBHBody} = \hashKey{} \circ \bvkcold{} \\
  \end{align*}
  %
  \begin{align*}
      & \fun{mkSeed} \in \Seed \to \Slot \to \Seed \to \Seed \\
      & \fun{mkSeed}~\var{ucNonce}~\var{slot}~\eta_0 =
        \var{ucNonce}~\mathsf{XOR}~(\slotToSeed{slot}\seedOp\eta_0)\\
  \end{align*}
  %
  \begin{align*}
      & \fun{vrfChecks} \in \Seed \to \BHBody \to \Bool \\
      & \fun{vrfChecks}~\eta_0~\var{bhb} = \\
      & \begin{array}{cl}
        ~~~~ &
        \verifyVrf{\Seed}{\var{vrfK}}{(\fun{mkSeed}~\Seede~\var{slot}~\eta_0)}
               {(\bprfn{bhb},~\bnonce{bhb}}) \\
        ~~~~ \land &
               \verifyVrf{\unitInterval}{\var{vrfK}}{(\fun{mkSeed}~\Seedl~\var{slot}~\eta_0)}
               {(\bprfl{bhb},~\bleader{bhb}}) \\
      \end{array} \\
      & ~~~~\where \\
      & ~~~~~~~~~~\var{slot} \leteq \bslot{bhb} \\
      & ~~~~~~~~~~\var{vrfK} \leteq \fun{bvkvrf}~\var{bhb} \\
  \end{align*}
  %
  \begin{align*}
      & \fun{praosVrfChecks} \in \Seed \to \PoolDistr \to \unitIntervalNonNull \to \BHBody \to \Bool \\
      & \fun{praosVrfChecks}~\eta_0~\var{pd}~\var{f}~\var{bhb} = \\
      & \begin{array}{cl}
        ~~~~ & \var{hk}\mapsto (\sigma,~\var{vrfHK})\in\var{pd} \\
        ~~~~ \land & \var{vrfHK} = \hashKey{vrfK} \\
        ~~~~ \land & \fun{vrfChecks}~\eta_0~\var{bhb} \\
        ~~~~ \land & \fun{checkLeaderVal}~(\fun{bleader}~\var{bhb})~\sigma~\var{f} \\
      \end{array} \\
      & ~~~~\where \\
      & ~~~~~~~~~~\var{hk} \leteq \fun{issuerIDfromBHBody}~\var{bhb} \\
      & ~~~~~~~~~~\var{vrfK} \leteq \fun{bvkvrf}~\var{bhb} \\
  \end{align*}
  %
  \begin{align*}
      & \fun{pbftVrfChecks} \in \KeyHash_{vrf} \to \Seed \to \BHBody \to \Bool \\
      & \fun{pbftVrfChecks}~\var{vrfHK}~\eta_0~~\var{bhb} = \\
      & \begin{array}{cl}
        ~~~~ & \var{vrfHK} = \hashKey{(\fun{bvkvrf}~\var{bhb})} \\
        ~~~~ \land & \fun{vrfChecks}~\eta_0~\var{bhb} \\
      \end{array} \\
  \end{align*}
  \label{fig:vrf-checks}
\end{figure}

\clearpage

\subsection{Overlay Schedule}
\label{sec:overlay-schedule}

The transition from the bootstrap era to a fully decentralized network is explained in
section 3.9.2 of~\cite{delegation_design}.
Key to this transition is a protocol parameter $d$ which controls how many slots are governed by
the genesis nodes via OBFT, and which slots are open to any registered stake pool.
The transition system introduced in this section, $\type{OVERLAY}$, covers this mechanism.

This transition is responsible for validating the protocol for both the OBFT blocks
and the Praos blocks, depending on the overlay schedule.

The actual overlay schedule itself it determined by two functions,
$\fun{isOverlaySlot}$ and $\fun{classifyOverlaySlot}$,
which are defined in Figure~\ref{fig:rules:overlay}.
The function $\fun{isOverlaySlot}$ determines if the current slot
is reserved for the OBFT nodes.
In particular, it looks at the current relative slot to
see if the next multiple of $d$ (the decentralization protocol parameter)
raises the ceiling.
If a slot is indeed in the overlay schedule,
the function $\fun{classifyOverlaySlot}$ determines if the current slot
should be a silent slot (no block is allowed) or which core node
is responsible for the block.
The non-silent blocks are the multiples of $1/\ActiveSlotCoeff$,
and the responsible core node are determined by
taking turns according the lexicographic ordering of the core node keyhashes.
\textbf{Note} that $1/\ActiveSlotCoeff$ needs to be natural number,
otherwise the multilples of $\lfloor 1/\ActiveSlotCoeff \rfloor$
will yield a different proportion of active slots than the Praos blocks.

The environments for the overlay schedule transition are:
\begin{itemize}
  \item The decentralization parameter $d$.
  \item The epoch nonce $\eta_0$.
  \item The stake pool stake distribution $\var{pd}$.
  \item The mapping $\var{genDelegs}$ of genesis keys to their cold keys and vrf keys.
\end{itemize}

The states for this transition consist only of the mapping of certificate issue numbers.

This transition establishes that a block producer is in fact authorized.
Since there are three key pairs involved (cold keys, VRF keys, and hot KES keys)
it is worth examining the interaction closely.
First we look at the regular Praos/decentralized setting,
which is given by Equation~\ref{eq:decentralized}.

\begin{itemize}
  \item First we check the operational certificate with $\mathsf{OCERT}$.
  This uses the cold verification key given in the block header.
  We do not yet trust that this key is a registered pool key.
  If this transition is successful, we know that the cold key in the block header has authorized
  the block.
\item  Next, in the $\fun{vrfChecks}$ predicate, we check that the hash of this cold key is in the
  mapping $\var{pd}$, and that it maps to $(\sigma,~\var{hk_{vrf}})$, where
  $(\sigma,~\var{hk_{vrf}})$ is the hash of the VRF key in the header.
  If $\fun{praosVrfChecks}$ returns true, then we know that the cold key in the block header was a
  registered stake pool at the beginning of the previous epoch, and that it is indeed registered
  with the VRF key listed in the header.
\item Finally, we use the VRF verification key in the header, along with the VRF proofs in the
  header, to check that the operator is allowed to produce the block.
\end{itemize}
The situation for the overlay schedule, given by Equation~\ref{eq:active-pbft}, is similar.
The difference is that we check the overlay schedule to see what core node is
supposed to make a block, and then use the genesis delegation mapping to
check the correct cold key hash and vrf key hash.

\begin{figure}
  \emph{Overlay environments}
  \begin{equation*}
    \OverlayEnv =
    \left(
      \begin{array}{r@{~\in~}lr}
        \var{d} & \{0,~1/100,~2/100,~\ldots,~1\} & \text{decentralization parameter} \\
        \var{pd} & \PoolDistr & \text{pool stake distribution} \\
        \var{genDelegs} & \GenesisDelegation & \text{genesis key delegations} \\
        \eta_0 & \Seed & \text{epoch nonce} \\
      \end{array}
    \right)
  \end{equation*}
  %
  \emph{Overlay Transitions}
  \begin{equation*}
    \_ \vdash \var{\_} \trans{overlay}{\_} \var{\_} \subseteq
    \powerset (\OverlayEnv \times (\KeyHash_{pool} \mapsto \N) \times \BHeader \times
    (\KeyHash_{pool} \mapsto \N))
  \end{equation*}
  %
  \emph{Overlay Schedule helper functions}
  \begin{align*}
      & \fun{isOverlaySlot} \in \Slot \to \unitInterval \to \Slot \to \Bool \\
      & \isOverlaySlot{fSlot}{dval}{slot} = \lceil s\cdot dval \rceil < \lceil (s+1)\cdot dval \rceil \\
      & ~~~~ \where s\leteq\slotminus{slot}{fslot}
  \end{align*}
  %
  \begin{align*}
      & \fun{classifyOverlaySlot} \in \Slot \to \powerset{\KeyHashGen} \to \unitInterval
        \to \unitIntervalNonNull \to \Slot \to \Bool \\
      & \classifyOverlaySlot{fSlot}{gkeys}{dval}{asc}{slot} = \\
      & ~~~\begin{cases}
        \fun{elemAt}~\left( \frac{\var{position}}{\var{ascInv}} \bmod |\var{gkeys}| \right)~\var{gkeys}
        &
        \text{if } \var{position} \equiv 0 \pmod{\var{ascInv}} \\
        \Nothing
        &
        \text{otherwise}
        \end{cases} \\
      & ~~~~ \where \\
      & ~~~~~~~ \var{ascInv}\leteq\lfloor 1/\var{asc}\rfloor \\
      & ~~~~~~~ \var{position}\leteq \lceil (\slotminus{slot}{fSlot})\cdot dval \rceil \\
      & ~~~~~~~ \fun{elemAt}~i~\var{set}\leteq i'\text{th lexicographic element of }\var{set}
  \end{align*}
  %
  \caption{Overlay transition-system types}
  \label{fig:ts-types:overlay}
\end{figure}

\begin{figure}[ht]
  \begin{equation}\label{eq:active-pbft}
    \inference[Active-OBFT]
    {
      \var{bhb}\leteq\bheader{bh}
      &
      \var{vk}\leteq\bvkcold{bhb}
      &
      \var{vkh}\leteq\hashKey{vk}
      \\
      \var{slot} \leteq \bslot{bhb}
      &
      \var{fSlot} \leteq \fun{firstSlot}~(\epoch{slot})
      \\
      \var{gkh}\mapsto(\var{vkh},~\var{vrfh})\in\var{genDelegs}
      \\
      \isOverlaySlot{fSlot}{(\dom{genDelegs})}{slot}\\
      \classifyOverlaySlot{fSlot}{(\dom{genDelegs})}{d}{\ActiveSlotCoeff}{slot} = \var{ghk}
      \\~\\
      \fun{pbftVrfChecks}~\var{vrfh}~\eta_0~\var{bhb}
      \\~\\
      {
        {\begin{array}{c}
         \dom{\var{pd}} \\
         \range{\var{genDelegs}} \\
         \end{array}
        }
        \vdash\var{cs}\trans{\hyperref[fig:rules:ocert]{ocert}}{\var{bh}}\var{cs'}
      }
    }
    {
      {\begin{array}{c}
         \var{d} \\
         \eta_0 \\
         \var{pd} \\
         \var{genDelegs} \\
       \end{array}}
      \vdash
      \var{cs}
      \trans{overlay}{\var{bh}}
      \varUpdate{\var{cs}'}
    }
  \end{equation}

  \nextdef

  \begin{equation}\label{eq:decentralized}
    \inference[Decentralized]
    {
      \var{bhb}\leteq\bheader{bh}
      &
      \var{slot} \leteq \bslot{bhb}
      &
      \var{fSlot} \leteq \fun{firstSlot}~(\epoch{slot})
      \\
      \neg\isOverlaySlot{fSlot}{(\dom{genDelegs})}{slot}\\
      \\~\\
      {
        \vdash\var{cs}\trans{\hyperref[fig:rules:ocert]{ocert}}{\var{bh}}\var{cs'}
      }
      \\~\\
      \fun{praosVrfChecks}~\eta_0~\var{pd}~\ActiveSlotCoeff~\var{bhb}
    }
    {
      {\begin{array}{c}
         \var{d} \\
         \eta_0 \\
         \var{pd} \\
         \var{genDelegs} \\
       \end{array}}
      \vdash
      \var{cs}
      \trans{overlay}{\var{bh}}
      \varUpdate{\var{cs}'}
    }
  \end{equation}

  \caption{Overlay rules}
  \label{fig:rules:overlay}
\end{figure}

The OVERLAY rule has nine predicate failures:
\begin{itemize}
\item If in the decentralized case the VRF key is not in the pool distribution,
  there is a \emph{VRFKeyUnknown} failure.
\item If in the decentralized case the VRF key hash does not match the one
  listed in the block header, there is a \emph{VRFKeyWrongVRFKey} failure.
\item If the VRF generated nonce in the block header does not validate
  against the VRF certificate, there is a \emph{VRFKeyBadNonce} failure.
\item If the VRF generated leader value in the block header does not validate
  against the VRF certificate, there is a \emph{VRFKeyBadLeaderValue} failure.
\item If the VRF generated leader value in the block header is too large
  compared to the relative stake of the pool, there is a \emph{VRFLeaderValueTooBig} failure.
\item In the case of the slot being in the OBFT schedule, but without genesis
  key (i.e., $Nothing$), there is a \emph{NotActiveSlot} failure.
\item In the case of the slot being in the OBFT schedule, if there is a
  specified genesis key which is not the same key as in the bock header body,
  there is a \emph{WrongGenesisColdKey} failure.
\item In the case of the slot being in the OBFT schedule, if the hash of the
  VRF key in block header does not match the hash in the genesis delegation mapping,
  there is a \emph{WrongGenesisVRFKey} failure.
\item In the case of the slot being in the OBFT schedule, if the genesis delegate
  keyhash is not in the genesis delegation mapping,
  there is a \emph{UnknownGenesisKey} failure.
  This case should never happen, and represents a logic error.

\end{itemize}

\clearpage

\subsection{Protocol Transition}
\label{sec:protocol-trans}

The protocol transition covers the common predicates of OBFT and Praos,
and then calls $\mathsf{OVERLAY}$ for the particular transitions,
followed by the transition to update the evolving and candidate nonces.

\begin{figure}
  \emph{Protocol environments}
  \begin{equation*}
    \PrtclEnv =
    \left(
      \begin{array}{r@{~\in~}lr}
        \var{d} & \{0,~1/100,~2/100,~\ldots,~1\} & \text{decentralization parameter} \\
        \var{pd} & \PoolDistr & \text{pool stake distribution} \\
        \var{dms} & \KeyHashGen\mapsto\KeyHash & \text{genesis key delegations} \\
        \eta_0 & \Seed & \text{epoch nonce} \\
      \end{array}
    \right)
  \end{equation*}
  %
  \emph{Protocol states}
  \begin{equation*}
    \PrtclState =
    \left(
      \begin{array}{r@{~\in~}lr}
        \var{cs} & \KeyHash_{pool} \mapsto \N & \text{operational certificate issues numbers} \\
        \eta_v & \Seed & \text{evolving nonce} \\
        \eta_c & \Seed & \text{candidate nonce} \\
      \end{array}
    \right)
  \end{equation*}
  %
  \emph{Protocol Transitions}
  \begin{equation*}
    \_ \vdash \var{\_} \trans{prtcl}{\_} \var{\_} \subseteq
    \powerset (\powerset{\PrtclEnv} \times \PrtclState \times \BHeader \times \PrtclState)
  \end{equation*}
  \caption{Protocol transition-system types}
  \label{fig:ts-types:prtcl}
\end{figure}

The environments for this transition are:
\begin{itemize}
  \item The decentralization parameter $d$.
  \item The stake pool stake distribution $\var{pd}$.
  \item The mapping $\var{dms}$ of genesis keys to their cold keys.
  \item The epoch nonce $\eta_0$.
\end{itemize}

The states for this transition consists of:
\begin{itemize}
  \item The operational certificate issue number mapping.
  \item The evolving nonce.
  \item The canditate nonce for the next epoch.
\end{itemize}

\begin{figure}[ht]
  \begin{equation}\label{eq:prtcl}
    \inference[PRTCL]
    {
      \eta\leteq\fun{bnonce}~(\bhbody{bhb})
      \\~\\
      {
        \eta
        \vdash
        {\left(\begin{array}{c}
        \eta_v \\
        \eta_c \\
        \end{array}\right)}
        \trans{\hyperref[fig:rules:update-nonce]{updn}}{\var{slot}}
        {\left(\begin{array}{c}
        \eta_v' \\
        \eta_c' \\
        \end{array}\right)}
      }\\~\\
      {
        {\begin{array}{c}
          \var{d} \\
          \var{pd} \\
          \var{dms} \\
          \eta_0 \\
        \end{array}
        }
        \vdash \var{cs}\trans{\hyperref[fig:rules:overlay]{overlay}}{\var{bh}} \var{cs}'
      }
    }
    {
      {\begin{array}{c}
         \var{d} \\
         \var{pd} \\
         \var{dms} \\
         \eta_0 \\
       \end{array}}
      \vdash
      {\left(\begin{array}{c}
            \var{cs} \\
            \eta_v \\
            \eta_c \\
      \end{array}\right)}
      \trans{prtcl}{\var{bh}}
      {\left(\begin{array}{c}
            \varUpdate{cs'} \\
            \varUpdate{\eta_v'} \\
            \varUpdate{\eta_c'} \\
      \end{array}\right)}
    }
  \end{equation}
  \caption{Protocol rules}
  \label{fig:rules:prtcl}
\end{figure}

The PRTCL rule has no predicate failures, besides those of the two sub-transitons.

\clearpage

\subsection{Block Body Transition}
\label{sec:block-body-trans}

The Block Body Transition updates the block body state which comprises the ledger state and the
map describing the produced blocks.
The environment of the $\mathsf{BBODY}$ transition are
the protocol parameters and the accounting state.
The environments and states are defined in Figure~\ref{fig:ts-types:bbody}, along with
a helper function $\fun{incrBlocks}$, which counts the number of non-overlay blocks
produced by each stake pool.

\begin{figure}
  \emph{BBody environments}
  \begin{equation*}
    \BBodyEnv =
    \left(
      \begin{array}{r@{~\in~}lr}
        \var{pp} & \PParams & \text{protocol parameters} \\
        \var{acnt} & \Acnt & \text{accounting state}
      \end{array}
    \right)
  \end{equation*}
  %
  \emph{BBody states}
  \begin{equation*}
    \BBodyState =
    \left(
      \begin{array}{r@{~\in~}lr}
        \var{ls} & \LState & \text{ledger state} \\
        \var{b} & \BlocksMade & \text{blocks made} \\
      \end{array}
    \right)
  \end{equation*}
  %
  \emph{BBody Transitions}
  \begin{equation*}
    \_ \vdash \var{\_} \trans{bbody}{\_} \var{\_} \subseteq
    \powerset (\BBodyEnv \times \BBodyState \times \Block \times \BBodyState)
  \end{equation*}
  \caption{BBody transition-system types}
  \label{fig:ts-types:bbody}
  %
  \emph{BBody helper function}
  \begin{align*}
      & \fun{incrBlocks} \in \Bool \to \KeyHash_{pool} \to
          \BlocksMade \to \BlocksMade \\
      & \fun{incrBlocks}~\var{isOverlay}~\var{hk}~\var{b} =
        \begin{cases}
          b & \text{if }\var{isOverlay} \\
          b\cup\{\var{hk}\mapsto 1\} & \text{if }\var{hk}\notin\dom{b} \\
          b\unionoverrideRight\{\var{hk}\mapsto n+1\} & \text{if }\var{hk}\mapsto n\in b \\
        \end{cases}
  \end{align*}

\end{figure}

The $\mathsf{BBODY}$ transition rule is shown in Figure~\ref{fig:rules:bbody},
its sub-rule is $\mathsf{LEDGERS}$ which does the update of the ledger
state. The signal is a block from which we extract:

\begin{itemize}
\item The sequence of transactions \var{txs} of the block.
\item The block header body \var{bhb}.
\item The verification key \var{vk} of the issuer of the \var{block} and its
  hash \var{hk}.
\end{itemize}

The transition is executed if the following preconditions are met:

\begin{itemize}
\item The size of the block body matches the value given in the block header body.
\item The hash of the block body matches the value given in the block header body.
\item The $\mathsf{LEDGERS}$ transition succeeds.
\end{itemize}

After this, the transition system updates the mapping of the hashed stake pool
keys to the incremented value of produced blocks (\var{n} + 1), provided the
current slot is not an overlay slot.

\begin{figure}[ht]
  \begin{equation}\label{eq:bbody}
    \inference[Block-Body]
    {
      \var{txs} \leteq \bbody{block}
      &
      \var{bhb} \leteq \bhbody{(\bheader{block})}
      &
      \var{hk} \leteq \hashKey{(\bvkcold{bhb})}
      \\~\\
      \bBodySize{txs} = \hBbsize{bhb}
      &
      \fun{bbodyhash}~{txs} = \bhash{bhb}
      \\~\\
      \\
      \var{slot} \leteq \bslot{bhb}
      &
      \var{fSlot} \leteq \fun{firstSlot}~(\epoch{slot})
      \\~\\
      {
        {\begin{array}{c}
                 \bslot{bhb} \\
                 \var{pp} \\
                 \var{acnt}
        \end{array}}
        \vdash
             \var{ls} \\
        \trans{\hyperref[fig:rules:ledger-sequence]{ledgers}}{\var{txs}}
             \var{ls}' \\
      }
    }
    {
      {\begin{array}{c}
               \var{pp} \\
               \var{acnt}
      \end{array}}
      \vdash
      {\left(\begin{array}{c}
            \var{ls} \\
            \var{b} \\
      \end{array}\right)}
      \trans{bbody}{\var{block}}
      {\left(\begin{array}{c}
            \varUpdate{\var{ls}'} \\
            \varUpdate{\fun{incrBlocks}~{\left(\isOverlaySlot{fSlot}{(\fun{d}~pp)}{slot}\right)}~{hk}~{b}} \\
      \end{array}\right)}
    }
  \end{equation}
  \caption{BBody rules}
  \label{fig:rules:bbody}
\end{figure}

The BBODY rule has two predicate failures:
\begin{itemize}
\item if the size of the block body in the header is not equal to the real size
  of the block body, there is a \emph{WrongBlockBodySize} failure.
\item if the hash of the block body is not also the hash of transactions, there is an \emph{InvalidBodyHash} failure.
\end{itemize}

\clearpage

\subsection{Chain Transition}
\label{sec:chain-trans}

The $\mathsf{CHAIN}$ transition rule is the main rule of the blockchain layer
part of the STS. It calls $\mathsf{BHEAD}$, $\mathsf{PRTCL}$, and $\mathsf{BBODY}$ as sub-rules.

The chain rule has no environment.

The transition checks six things
(via $\fun{chainChecks}$ and $\fun{prtlSeqChecks}$ from Figure~\ref{fig:funcs:chain-helper}):
\begin{itemize}
\item The slot in the block header body is larger than the last slot recorded.
\item The block number increases by exactly one.
\item The previous hash listed in the block header matches the previous
  block header hash which was recorded.
\item The size of \var{bh} is less than or equal to the maximal size that the
  protocol parameters allow for block headers.
\item The size of the block body, as claimed by the block header, is less than or equal to the
  maximal size that the protocol parameters allow for block bodies.
  It will later be verified that the size of the block body matches the size claimed
  in the header (see Figure~\ref{fig:rules:bbody}).
\item The node is not obsolete, meaning that the major component of the
  protocol version in the protocol parameters
  is not bigger than the constant $\MaxMajorPV$.
\end{itemize}


The chain state is shown in Figure~\ref{fig:ts-types:chain}, it consists of the
following:

\begin{itemize}
  \item The epoch specific state $\var{nes}$.
  \item The operational certificate issue number map $\var{cs}$.
  \item The epoch nonce $\eta_0$.
  \item The evolving nonce $\eta_v$.
  \item The candidate nonce $\eta_c$.
  \item The previous epoch hash nonce $\eta_h$.
  \item The last header hash \var{h}.
  \item The last slot \var{s_\ell}.
  \item The last block number \var{b_\ell}.
\end{itemize}

\begin{figure}
  \emph{Chain states}
  \begin{equation*}
    \LastAppliedBlock =
    \left(
      \begin{array}{r@{~\in~}lr}
        \var{b_\ell} & \Slot & \text{last block number} \\
        \var{s_\ell} & \Slot & \text{last slot} \\
        \var{h} & \HashHeader & \text{latest header hash} \\
      \end{array}
    \right)
  \end{equation*}
  %
  \begin{equation*}
    \ChainState =
    \left(
      \begin{array}{r@{~\in~}lr}
        \var{nes} & \NewEpochState & \text{epoch specific state} \\
        \var{cs} & \KeyHash_{pool} \mapsto \N & \text{operational certificate issue numbers} \\
        ~\eta_0 & \Seed & \text{epoch nonce} \\
        ~\eta_v & \Seed & \text{evolving nonce} \\
        ~\eta_c & \Seed & \text{candidate nonce} \\
        ~\eta_h & \Seed & \text{seed generated from hash of previous epoch's last block header} \\
        \var{lab} & \LastAppliedBlock^? & \text{latest applied block} \\
      \end{array}
    \right)
  \end{equation*}
  %
  \emph{Chain Transitions}
  \begin{equation*}
    \vdash \var{\_} \trans{chain}{\_} \var{\_} \subseteq
    \powerset (\ChainState \times \Block \times \ChainState)
  \end{equation*}
  \caption{Chain transition-system types}
  \label{fig:ts-types:chain}
\end{figure}

The $\mathsf{CHAIN}$ transition rule is shown in Figure~\ref{fig:rules:chain}. Its signal is a \var{block}.
The transition uses a few helper functions defined in Figure~\ref{fig:funcs:chain-helper}.

%%
%% Figure - Helper Functions for Chain Rules
%%
\begin{figure}[htb]
  \emph{Chain Transition Helper Functions}
  \begin{align*}
      & \fun{getGKeys} \in \NewEpochState \to \powerset{\KeyHashGen} \\
      & \fun{getGKeys}~\var{nes} = \dom{genDelegs} \\
      &
      \begin{array}{lr@{~=~}l}
        \where
          & (\wcard,~\wcard,~\wcard,~\var{es},~\wcard,~\wcard)
          & \var{nes}
          \\
          & (\wcard,~\wcard,~\var{ls},~\wcard,~\wcard)
          & \var{es}
          \\
          & (\wcard,~((\wcard,~\wcard,~\wcard,~\wcard,~\var{genDelegs},~\wcard),~\wcard))
          & \var{ls}
      \end{array}
  \end{align*}
  %
  \begin{align*}
      & \fun{updateNES} \in \NewEpochState \to \BlocksMade \to \LState \to \NewEpochState \\
      & \fun{updateNES}~
      (\var{e_\ell},~\var{b_{prev}},~\wcard,~(\var{acnt},~\var{ss},~\wcard,~\var{prevPp},~\var{pp}),
       ~\var{ru},~\var{pd})
          ~\var{b_{cur}}~\var{ls} = \\
      & ~~~~
      (\var{e_\ell},~\var{b_{prev}},~\var{b_{cur}},
       ~(\var{acnt},~\var{ss},~\var{ls},~\var{prevPp},~\var{pp}),~\var{ru},~\var{pd})
  \end{align*}
  %
  \begin{align*}
      & \fun{chainChecks} \in \N \to (\N,\N, \ProtVer) \to \BHeader \to \Bool \\
      & \fun{chainChecks}~\var{maxpv}~(\var{maxBHSize},~\var{maxBBSize},~\var{protocolVersion})~\var{bh} = \\
      & ~~~~ m \leq \var{maxpv} \\
      & ~~~~ \land~\bHeaderSize{bh} \leq \var{maxBHSize} \\
      & ~~~~ \land~\hBbsize{(\bhbody{bh})} \leq \var{maxBBSize} \\
      & ~~~~ \where (m,~\wcard)\leteq\var{protocolVersion}
  \end{align*}
  %
  \begin{align*}
      & \fun{lastAppliedHash} \in \LastAppliedBlock^? \to \HashHeader^? \\
      & \fun{lastAppliedHash}~\var{lab} =
        \begin{cases}
          \Nothing & lab = \Nothing \\
          h & lab = (\wcard,~\wcard,~h) \\
        \end{cases}
  \end{align*}
  %
  \begin{align*}
      & \fun{prtlSeqChecks} \to \LastAppliedBlock^? \to \BHeader \to \Bool \\
      & \fun{prtlSeqChecks}~\var{lab}~\var{bh} =
        \begin{cases}
          \mathsf{True}
          &
          lab = \Nothing
          \\
          \var{s_\ell} < \var{slot}
          \land \var{b_\ell} + 1 = \var{bn}
          \land \var{ph} = \bprev{bhb}
          &
          lab = (b_\ell,~s_\ell,~\wcard) \\
        \end{cases} \\
      & ~~~~\where \\
      & ~~~~~~~~~~\var{bhb} \leteq \bhbody{bh} \\
      & ~~~~~~~~~~\var{bn} \leteq \bblockno{bhb} \\
      & ~~~~~~~~~~\var{slot} \leteq \bslot{bhb} \\
      & ~~~~~~~~~~\var{ph} \leteq \lastAppliedHash{lab} \\
  \end{align*}

  \caption{Helper Functions used in the CHAIN transition}
  \label{fig:funcs:chain-helper}
\end{figure}

\begin{figure}[ht]
  \begin{equation}\label{eq:chain}
    \inference[Chain]
    {
      \var{bh} \leteq \bheader{block}
      &
      \var{bhb} \leteq \bhbody{bh}
      &
      \var{s} \leteq \bslot{bhb}
      \\~\\
      \fun{prtlSeqChecks}~\var{lab}~\var{bh}
      \\~\\
      {
        \vdash\var{nes}\trans{\hyperref[fig:rules:tick]{tick}}{\var{s}}\var{nes'}
      } \\~\\
      (\var{e_1},~\wcard,~\wcard,~\wcard,~\wcard,~\wcard)
        \leteq\var{nes} \\
      (\var{e_2},~\wcard,~\var{b_{cur}},~\var{es},~\wcard,~\wcard,~\var{pd})
        \leteq\var{nes'} \\
        (\var{acnt},~\wcard,\var{ls},~\wcard,~\var{pp})\leteq\var{es}\\
        ( \wcard,
          ( (\wcard,~\wcard,~\wcard,~\wcard,~\var{genDelegs},~\wcard),~
          (\wcard,~\wcard,~\wcard)))\leteq\var{ls}\\
          \var{ne} \leteq  \var{e_1} \neq \var{e_2}\\
          \eta_{ph} \leteq \prevHashToNonce{(\lastAppliedHash{lab})}
          \\~\\
          \fun{chainChecks}~
            \MaxMajorPV~(\fun{maxHeaderSize}~\var{pp},~\fun{maxBlockSize}~\var{pp},~\fun{pv}~\var{pp})~
            \var{bh}
          \\~\\
      {
        {\begin{array}{c}
        \var{pp} \\
        \eta_c \\
        \eta_\var{ph} \\
        \end{array}}
        \vdash
        {\left(\begin{array}{c}
        \eta_0 \\
        \eta_h \\
        \end{array}\right)}
        \trans{\hyperref[fig:rules:tick-nonce]{tickn}}{\var{ne}}
        {\left(\begin{array}{c}
        \eta_0' \\
        \eta_h' \\
        \end{array}\right)}
      }\\~\\~\\
      {
        {\begin{array}{c}
            (\fun{d}~\var{pp}) \\
            \var{pd} \\
            \var{genDelegs} \\
            \eta_0' \\
         \end{array}}
        \vdash
        {\left(\begin{array}{c}
              \var{cs} \\
              \eta_v \\
              \eta_c \\
        \end{array}\right)}
        \trans{\hyperref[fig:rules:prtcl]{prtcl}}{\var{bh}}
        {\left(\begin{array}{c}
              \var{cs'} \\
              \eta_v' \\
              \eta_c' \\
        \end{array}\right)}
      } \\~\\~\\
      {
        {\begin{array}{c}
                 \var{pp} \\
                 \var{acnt}
        \end{array}}
        \vdash
        {\left(\begin{array}{c}
              \var{ls} \\
              \var{b_{cur}} \\
        \end{array}\right)}
        \trans{\hyperref[fig:rules:bbody]{bbody}}{\var{block}}
        {\left(\begin{array}{c}
              \var{ls}' \\
              \var{b_{cur}'} \\
        \end{array}\right)}
      }\\~\\
      \var{nes''}\leteq\fun{updateNES}~\var{nes'}~\var{b_{cur}'},~\var{ls'} \\
      \var{lab'}\leteq (\bblockno{bhb},~\var{s},~\bhash{bh} ) \\
    }
    {
      \vdash
      {\left(\begin{array}{c}
            \var{nes} \\
            \var{cs} \\
            \eta_0 \\
            \eta_v \\
            \eta_c \\
            \eta_h \\
            \var{lab} \\
      \end{array}\right)}
      \trans{chain}{\var{block}}
      {\left(\begin{array}{c}
            \varUpdate{\var{nes}''} \\
            \varUpdate{\var{cs}'} \\
            \varUpdate{\eta_0'} \\
            \varUpdate{\eta_v'} \\
            \varUpdate{\eta_c'} \\
            \varUpdate{\eta_h'} \\
            \varUpdate{\var{lab}'} \\
      \end{array}\right)}
    }
  \end{equation}
  \caption{Chain rules}
  \label{fig:rules:chain}
\end{figure}

The CHAIN rule has six predicate failures:
\begin{itemize}
\item If the slot of the block header body is not larger than the last slot or
  greater than the current slot, there is a \emph{WrongSlotInterval} failure.
\item If the block number does not increase by exactly one,
  there is a \emph{WrongBlockNo} failure.
\item If the hash of the previous header of the block header body is not equal
  to the hash given in the environment, there is a \emph{WrongBlockSequence}
  failure.
\item If the size of the block header is larger than the maximally allowed size,
  there is a \emph{HeaderSizeTooLarge} failure.
\item If the size of the block body is larger than the maximally allowed size,
  there is a \emph{BlockSizeTooLarge} failure.
\item If the major component of the protocol version is larger than $\MaxMajorPV$,
  there is a \emph{ObsoleteNode} failure.
\end{itemize}

\clearpage

\subsection{Byron to Shelley Transition}
\label{sec:byron-to-shelley}

This section defines the valid initial Shelley ledger states and
describes how to transition the state held by the Byron ledger to Shelley.
The Byron ledger state $\CEState$ is defined in \cite{byron_chain_spec}.
The valid initial Shelley ledger states are exactly the range of
the function $\fun{initialShelleyState}$ defined in Figure~\ref{fig:functions:initial-shelley-states}.
Figure~\ref{fig:functions:to-shelley} defines the transition function from Byron.
Note that we use the hash of the final Byron header as the first evolving and
candidate nonces for Shelley.

%%
%% Figure - Shelley Initial States
%%
\begin{figure}[htb]
  \emph{Shelley Initial States}
  %
  \begin{align*}
      & \fun{initialShelleyState} \in \LastAppliedBlock^? \to \Epoch \to \UTxO
        \to \Coin \to \GenesisDelegation \\
      & ~~~ \to (\Slot\mapsto\KeyHashGen^?)
        \to \PParams \to \Seed \to \ChainState \\
      & \fun{initialShelleyState}~
      \left(
        \begin{array}{c}
          \var{lab} \\
          \var{e} \\
          \var{utxo} \\
          \var{reserves} \\
          \var{genDelegs} \\
          \var{pp} \\
          \var{initNonce} \\
        \end{array}
      \right)
      =
      \left(
        \begin{array}{c}
          \left(
            \begin{array}{c}
              \var{e} \\
              \emptyset \\
              \emptyset \\
              \left(
                \begin{array}{c}
                  \left(
                    \begin{array}{c}
                      0 \\
                      \var{reserves} \\
                    \end{array}
                  \right) \\
                  \left(
                    \begin{array}{c}
                      (\emptyset, \emptyset) \\
                      (\emptyset, \emptyset) \\
                      (\emptyset, \emptyset) \\
                      \emptyset \\
                      \emptyset \\
                      0
                    \end{array}
                  \right) \\
                  \left(
                    \begin{array}{c}
                      \left(
                        \begin{array}{c}
                          \var{utxo} \\
                          0 \\
                          0 \\
                          (\emptyset,~\emptyset) \\
                        \end{array}
                      \right) \\
                      \left(
                        \begin{array}{c}
                        \left(
                          \begin{array}{c}
                            \emptyset \\
                            \emptyset \\
                            \emptyset \\
                            \emptyset \\
                            \var{genDelegs} \\
                            \emptyset \\
                          \end{array}
                        \right) \\
                        \left(
                          \begin{array}{c}
                            \emptyset \\
                            \emptyset \\
                            \emptyset \\
                          \end{array}
                        \right) \\
                        \end{array}
                      \right) \\
                    \end{array}
                  \right) \\
                  \var{pp} \\
                  \var{pp} \\
                \end{array}
              \right) \\
              \\
              \Nothing \\
              \emptyset \\
            \end{array}
          \right) \\
          \var{cs} \\
          \var{initNonce} \\
          \var{initNonce} \\
          \var{initNonce} \\
          0_{seed} \\
          \var{lab} \\
        \end{array}
      \right) \\
      & ~~~~\where cs = \{\var{hk}\mapsto 0~\mid~(\var{hk},~\wcard)\in\range{genDelegs}\} \\
  \end{align*}

  \caption{Initial Shelley States}
  \label{fig:functions:initial-shelley-states}
\end{figure}

%%
%% Figure - Byron to Shelley State Transition
%%
\begin{figure}[htb]
  \emph{Byron to Shelley Transition}
  %
  \begin{align*}
      & \fun{toShelley} \in \CEState \to \GenesisDelegation \to \BlockNo \to \ChainState \\
      & \fun{toShelley}~
      \left(
        \begin{array}{c}
          \var{s_{last}} \\
          \wcard \\
          \var{h} \\
          (\var{utxo},~\var{reserves}) \\
          \wcard \\
          \var{us}
        \end{array}
      \right)~\var{gd}~\var{bn}
      =
      \fun{initialShelleyState}~
      \left(
        \begin{array}{c}
          (\var{s_{last}}~\var{bn},~\prevHashToNonce{h}) \\
          e \\
          \fun{hash}~{h} \\
          \var{utxo} \\
          \var{reserves} \\
          \var{gd} \\
          \overlaySchedule{e}{(\dom{gd})}{pp} \\
          \var{pp} \\
          \prevHashToNonce{h} \\
        \end{array}
      \right) \\
      & ~~~~\where \\
      & ~~~~~~~~~e = \epoch{s_{last}} \\
      & ~~~~~~~~~pp = \fun{pps}~{us} \\   % this pps function is defined in the Byron chain spec
  \end{align*}

  \caption{Byron to Shelley State Transtition}
  \label{fig:functions:to-shelley}
\end{figure}

%%% Local Variables:
%%% mode: latex
%%% TeX-master: "ledger-spec"
%%% End:

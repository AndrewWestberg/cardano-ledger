\newcommand{\createRUpd}[4]{\fun{createRUpd}~\var{#1}~\var{#2}~\var{#3}~\var{#4}}
\newcommand{\mkApparentPerformance}[3]{\fun{mkApparentPerformance}~{#1}~\var{#2}~\var{#3}}
\newcommand{\Q}{\ensuremath{\mathbb{Q}}}
\newcommand{\ActiveSlotCoeff}{\mathsf{ActiveSlotCoeff}}
\newcommand{\EpochState}{\type{EpochState}}
\newcommand{\BlocksMade}{\type{BlocksMade}}
\newcommand{\RewardUpdate}{\type{RewardUpdate}}
\newcommand{\PrtclState}{\type{PrtclState}}
\newcommand{\PrtclEnv}{\type{PrtclEnv}}
\newcommand{\PoolDistr}{\type{PoolDistr}}
\newcommand{\BHBody}{\type{BHBody}}
\newcommand{\HashHeader}{\type{HashHeader}}
\newcommand{\HashBBody}{\type{HashBBody}}
\newcommand{\BlockNo}{\type{BlockNo}}
\newcommand{\Proof}{\type{Proof}}
\newcommand{\OCert}{\type{OCert}}
\newcommand{\bheader}[1]{\fun{bheader}~\var{#1}}
\newcommand{\verifyVrf}[4]{\fun{verifyVrf}_{#1} ~ #2 ~ #3 ~#4}
\newcommand{\slotToSeed}[1]{\fun{slotToSeed}~ \var{#1}}
\newcommand{\XOR}{\mathsf{XOR}}

\section{Removal of the Overlay Schedule}

The overlay schedule was only used during the early days of the Shelley ledger, and can be safely removed. First, the protocol parameter $\var{d}$ is removed, and any functions that use it are reduced to the case $\var{d} = 0$. The function $\fun{mkApparentPerformance}$ is reduced to one of its branches, and its first argument is dropped. It is only used in the definition of $\fun{rewardOnePool}$, which needs to be adjusted accordingly.

Additionally, the block header body now contains a single VRF value to be used for both the leader check and the block nonce.

\begin{figure}[htb]
    \begin{align*}
      & \fun{mkApparentPerformance} \in \unitInterval \to \N \to \N \to \Q \\
      & \mkApparentPerformance{\sigma}{n}{\overline{N}} = \frac{\beta}{\sigma} \\
      & ~~~\where \\
      & ~~~~~~~\beta = \frac{n}{\max(1, \overline{N})} \\
  \end{align*}
  \caption{Function used in the Reward Calculation}
  \label{fig:functions:rewards}
\end{figure}

The function $\fun{createRUpd}$ is adjusted by simplifying $\eta$.

\begin{figure}[htb]
  \emph{Calculation to create a reward update}
  %
  \begin{align*}
    & \fun{createRUpd} \in \N \to \BlocksMade \to \EpochState \to \Coin \to \RewardUpdate \\
    & \createRUpd{slotsPerEpoch}{b}{es}{total} = \left(
      \Delta t_1,-~\Delta r_1+\Delta r_2,~\var{rs},~-\var{feeSS}\right) \\
    & ~~~\where \\
    & ~~~~~~~\dotsb \\
    & ~~~~~~~\eta =
        \frac{blocksMade}{\floor{{slotsPerEpoch} \cdot \ActiveSlotCoeff}} \\
    & ~~~~~~~\dotsb
  \end{align*}

  \caption{Reward Update Creation}
  \label{fig:functions:reward-update-creation}
\end{figure}

$\fun{incrBlocks}$ gets the same treatment as $\fun{mkApparentPerformance}$. Its invocation in $\mathsf{BBODY}$ needs to be adjusted as well.

\begin{figure}
  \begin{align*}
      & \fun{incrBlocks} \in \KeyHash_{pool} \to
          \BlocksMade \to \BlocksMade \\
      & \fun{incrBlocks}~\var{hk}~\var{b} =
        \begin{cases}
          b\cup\{\var{hk}\mapsto 1\} & \text{if }\var{hk}\notin\dom{b} \\
          b\unionoverrideRight\{\var{hk}\mapsto n+1\} & \text{if }\var{hk}\mapsto n\in b \\
        \end{cases}
  \end{align*}
\end{figure}

\newpage
Finally, the $\mathsf{PRTCL}$ STS needs to be adjusted.
To retire the $\mathsf{OVERLAY}$ STS, we inline the definition of its
'decentralized' case and drop all the unnecessary variables from its environment.
It is invoked in $\mathsf{CHAIN}$, which needs to be adjusted accordingly.

As there is now only a single VRF check, slight modifications are needed for the
definition of the block header body \text{BHBody} type and the function \text{vrfChecks}.
The Shelley era accessor functions $\fun{bleader}$ and $\fun{bnonce}$ are replaced with new functions
which make use of the VRF range extension as described in \cite{vrf-range-extension}[4.1],
to re-use the single VRF value.

\begin{figure*}[htb]
  %
  \emph{Block Header Body}
  %
  \begin{equation*}
    \BHBody =
    \left(
      \begin{array}{r@{~\in~}lr}
        \var{prev} & \HashHeader^? & \text{hash of previous block header}\\
        \var{vk} & \VKey & \text{block issuer}\\
        \var{vrfVk} & \VKey & \text{VRF verification key}\\
        \var{blockno} & \BlockNo & \text{block number}\\
        \var{slot} & \Slot & \text{block slot}\\
        \hldiff{\var{vrfRes}} & \hldiff{\Seed} & \hldiff{\text{VRF result value}}\\
        \var{prf} & \Proof & \text{vrf proof}\\
        \var{bsize} & \N & \text{size of the block body}\\
        \var{bhash} & \HashBBody & \text{block body hash}\\
        \var{oc} & \OCert & \text{operational certificate}\\
        \var{pv} & \ProtVer & \text{protocol version}\\
      \end{array}
    \right)
  \end{equation*}
  %
  \emph{New Accessor Function}
  \begin{equation*}
    \begin{array}{r@{~\in~}l}
       \fun{bVrfRes} & \BHBody \to \Seed \\
       \fun{bVrfProof} & \BHBody \to \Proof \\
    \end{array}
  \end{equation*}
  %
  \emph{New Helper Functions}
    \begin{align*}
      & \fun{bleader} \in \BHBody \to \Seed \\
      & \fun{bleader}~(\var{bhb}) = \fun{hash}~(``L"~|~(\fun{bVrfRes}~\var{bhb}))\\
      \\
      & \fun{bnonce} \in \BHBody \to \Seed \\
      & \fun{bnonce}~(\var{bhb}) = \fun{hash}~(``N"~|~(\fun{bVrfRes}~\var{bhb}))\\
      \\
      & \fun{vrfChecks} \in \Seed \to \BHBody \to \Bool \\
      & \fun{vrfChecks}~\eta_0~\var{bhb} =
          \verifyVrf{\Seed}{\var{vrfK}}{(\slotToSeed{slot}~\XOR~\eta_0)}{(\var{proof},~\var{value}}) \\
      & ~~~~\where \\
      & ~~~~~~~~~~\var{slot} \leteq \bslot{bhb} \\
      & ~~~~~~~~~~\var{vrfK} \leteq \fun{bvkvrf}~\var{bhb} \\
      & ~~~~~~~~~~\var{value} \leteq \fun{bVrfRes}~\var{bhb} \\
      & ~~~~~~~~~~\var{proof} \leteq \fun{bVrfProof}~\var{bhb} \\
  \end{align*}
  %
  \caption{Block Definitions}
  \label{fig:defs:blocks}
\end{figure*}


\begin{figure}
  \emph{Protocol environments}
  \begin{equation*}
    \PrtclEnv =
    \left(
      \begin{array}{r@{~\in~}lr}
        \var{pd} & \PoolDistr & \text{pool stake distribution} \\
        \eta_0 & \Seed & \text{epoch nonce} \\
      \end{array}
    \right)
  \end{equation*}
  \caption{Protocol transition-system types}
  \label{fig:ts-types:prtcl}
\end{figure}

\begin{figure}[ht]
  \begin{equation}\label{eq:prtcl}
    \inference[PRTCL]
    {
      \var{bhb}\leteq\bhbody{bh} &
      \eta\leteq\fun{bnonce}~\var{bhb}
      \\~\\
      {
        \eta
        \vdash
        {\left(\begin{array}{c}
        \eta_v \\
        \eta_c \\
        \end{array}\right)}
        \trans{\hyperref[fig:rules:update-nonce]{updn}}{\fun{bslot}~\var{bhb}}
        {\left(\begin{array}{c}
        \eta_v' \\
        \eta_c' \\
        \end{array}\right)}
    }\\~\\
      {
        \vdash\var{cs}\trans{\hyperref[fig:rules:ocert]{ocert}}{\var{bh}}\var{cs'}
      }
      \\~\\
      \fun{praosVrfChecks}~\eta_0~\var{pd}~\ActiveSlotCoeff~\var{bhb}
    }
    {
      {\begin{array}{c}
         \var{pd} \\
         \eta_0 \\
       \end{array}}
      \vdash
      {\left(\begin{array}{c}
            \var{cs} \\
            \eta_v \\
            \eta_c \\
      \end{array}\right)}
      \trans{prtcl}{\var{bh}}
      {\left(\begin{array}{c}
            \varUpdate{cs'} \\
            \varUpdate{\eta_v'} \\
            \varUpdate{\eta_c'} \\
      \end{array}\right)}
    }
  \end{equation}
  \caption{Protocol rules}
  \label{fig:rules:prtcl}
\end{figure}

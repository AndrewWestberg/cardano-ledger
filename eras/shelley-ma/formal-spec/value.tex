\section{Coin and Multi-Asset Token algebras}
\label{sec:coin-ma}

In this chapter we introduce the concept of a \emph{Token algebra},
which is an abstraction used to generalize this specification to two
different eras depending on the Token algebra chosen.

\begin{definition}[Token algebra]
  A \emph{Token algebra} is a partially ordered commutative monoid
  $T$, written additively, (i.e. a commutative monoid together with a
  partial order, such that addition is monotonic in both variables)
  together with the functions and properties as described in Figure
  \ref{fig:TokenAlgebra}.
\end{definition}

%%
%% Figure ValMonoid and its Functions
%%
\begin{figure}[htb]
  %
  \emph{Functions}
  %
  \begin{align*}
      & \fun{coin} ~\in~ T \to \Coin\\
      & \text{Return the Ada contained inside the $\mathsf{ValType}$ element}
      \nextdef
      %
      & \fun{inject} ~\in~ \Coin \to T\\
      & \text{Create a $\mathsf{ValType}$ element containing only this amount of Ada}
      \nextdef
      %
      & \fun{policies} ~\in~ T \to \powerset{\PolicyID} \\
      & \text{The set of policy IDs required for minting}
      \nextdef
      %
      & \fun{size} ~\in~ T \to \MemoryEstimate\\
      & \text{Return the size, in words, of a $\mathsf{ValType}$ element}
  \end{align*}
  %
  \emph{Properties}
  %
  \begin{align*}
      & \fun{coin} \circ \fun{inject} = \fun{id}_{\Coin}
  \end{align*}
  \caption{Additional functions and properties required for a Token algebra}
  \label{fig:TokenAlgebra}
\end{figure}

A Token algebra is precisely the structure required to generalize the
$\Coin$ type of the Shelley specification in transaction outputs for
this ledger. We can then describe the ShelleyMA
transaction processing rules without fixing a concrete Token algebra.

Depending on the Token algebra chosen, we obtain distinct ledgers. In
particular, we get

\begin{itemize}
  \item the Allegra ledger rules with $\Coin$, and
  \item the Mary ledger rules with $\Value$ (defined below).
\end{itemize}

When multi-asset support on the ledger is introduced, Ada ($\Coin$) will still be
the most common type of asset on the ledger, as the ledger rules enforce that
some quantity of it (specified via
the $\fun{coinsPerUTxOWord}$ protocol parameter) must
be contained in every UTxO on the ledger.
It is the only
type of asset used for all non-UTxO ledger accounting, including deposits,
fees, rewards, treasury, and the proof of stake protocol. For this reason, not
all occurrences of $\Coin$ inside a transaction or in the ledger state can or
should be replaced by the chosen Token algebra.

Below we give the definitions of all the functions that must be defined on
$\Coin$ and $\Value$ in order for them to have the structure of a Token algebra.
In Section \ref{sec:other-valmonoids} we give several other types which we can meaningfully
support the definition of the required functions (addition, size, etc.), including
an optimized representation that more accurately represents the implementation
used in the Haskell implementation of the multi-asset type.
These types are convertible to and from the $\Value$ type, and appear in other
parts of the system as a multi-asset representation which is more suitable in particular
use cases.

\subsection{$\Coin$ as a Token algebra}

This section defines the Token algebra structure for the $\Coin$ type,
see Figure \ref{fig:coin}. The structure of a partially ordered monoid
is inherited from the (additive) integers.

For $\Coin$, no policies are associated with $\Coin$, since minting of Ada is not allowed.

%%
%% Figure Coin Functions
%%
\begin{figure}[htb]
  \begin{align*}
      & \fun{coin} = \fun{id}_{\Coin}
      \nextdef
      %
      & \fun{inject} = \fun{id}_{\Coin}
      \nextdef
      %
      & \fun{policies}~v = \emptyset
      \nextdef
      %
      & \fun{size}~v = 0
  \end{align*}
  \caption{The Token algebra structure of $\Coin$}
  \label{fig:coin}
\end{figure}


\subsection{Multi-assets and the Token algebra $\Value$}

Elements of $\Value$ represent heterogeneous collections of assets,
both user-defined and Ada. The Mary era ledger uses $\Value$ as its Token algebra
in order to support multi-assets.
$\Value$ and its Token algebra structure are given in Figure \ref{fig:defs:value}.

\begin{figure*}[t!]
  \emph{Derived types}
  %
  \begin{equation*}
    \begin{array}{r@{~\in~}l@{\qquad=\qquad}lr}
      \var{aname} & \AssetName & \mathsf{ByteString} \\
      \var{pid} & \PolicyID & \ScriptHash \\
      \var{adaID} & \AdaIDType & \{~\mathsf{AdaID}~\} \\
      \var{aid} & \AssetID & \AdaIDType \uniondistinct (\PolicyID \times \AssetName) \\
      \var{quan} & \Quantity & \Z \\
      \var{v}, \var{w} & \Value & \AssetID \mapsto_0 \Quantity
    \end{array}
  \end{equation*}
  %
  \emph{Token algebra structure of $\Value$}
  %
  \begin{align*}
    & \fun{coin}~\var{v} = \var{v}~\mathsf{AdaID}
    \nextdef
    %
    & \fun{inject}~c  = \mathsf{AdaID}~\mapsto_0~\var{c}
    \nextdef
    %
    & \fun{policies}~v = \{~\var{pid}~\vert~(\var{pid,~\wcard}) \in \supp v~\}
    \nextdef
    %
    & \fun{size} ~~~ \text{see Section \ref{sec:value-size}}
  \end{align*}
  \caption{$\Value$ and its Token algebra structure}
  \label{fig:defs:value}
\end{figure*}

\begin{itemize}
  \item $\PolicyID$ identifies monetary policies. A policy ID $\var{pid}$ is associated with a script
    $s$ such that $\fun{hashScript}~s=pid$. When a transaction attempts to create or destroy assets
    that fall under the policy ID $\var{pid}$,
    $s$ verifies that the transaction
    respects the restrictions that are imposed by the monetary policy.
    See sections \ref{sec:transactions} and \ref{sec:utxo} for details.

  \item $\AssetName$ is a byte string used to distinguish different assets with the same $\PolicyID$.
    Each $aname$ identifies a particular kind of asset out of all the assets under the
    $\var{pid}$ policy (but not necessarily among assets under other policies).
    The maximum length of this
    byte string is 32 bytes (this is not explicitly enforced in this specification).

  \item $\AssetID$ is either $\mathsf{AdaID}$ or a pair of a policy ID and an asset name.
  It is a unique and permanent
  identifier of an asset. That is, there are is no mechanism to change it or
  any part of it for any assets.

  Mary MA assets are fungible with each other if and only if they have to the same $\AssetID$.
  The reason the unique identifier is a pair of two elements (except for the non-mintable Ada case) is to allow
  minting arbitrary collections of unique assets under a single policy.

  \item $\mathsf{AdaID}$ is a special asset ID for Ada, different than all other asset IDs.
  It is a term of the single-term type $\AdaIDType$.
  It does not include a policy, so instead, the validation outcome in the presence
  of Ada in the $\fun{mint}$ field of the transaction is specified in the UTXO
  ledger rule. The rule disallows the $\fun{mint}$ field to contain Ada.

  \item $\Quantity$ is an integer type that represents an amount of a specific $\AssetName$. We associate
    a $q\in\Quantity$ with a specific asset to track how much of that asset is contained in a given asset value.

  \item $\Value$ is the multi-asset type that is used to represent
    a collection of assets, including Ada. This type is a finitely supported map.

    If $\var{aid}$ is an $\AssetID$ and $v \in \Value$,
    the quantity of assets with that assed ID contained in $v$ is $v~\var{aid}$.
    Elements of $\Value$ are sometimes also referred to as
    \emph{asset bundles}.
\end{itemize}

To give $\Value$ the structure of a partially ordered monoid, we define the
required operations
pointwise, in accordance with the usual definitions of these operations on
finitely supported maps. See Figure \ref{fig:pointwise}.

\begin{figure*}[t!]
  \begin{align*}
    v + w    &= \{~ aid \mapsto v~\var{aid} + w~\var{aid} ~\vert~ \var{aid} \in \fun{dom}~v \cup \fun{dom}~w ~\} \\
    v \leq w &\Leftrightarrow \forall~\var{aid} \in \AssetID, v~\var{aid} \leq w~\var{aid}
  \end{align*}
  \caption{Pointwise operations on Value}
  \label{fig:pointwise}
\end{figure*}

\subsection{Special Ada representation}
Although all assets are native on the Cardano ledger (ie. the accounting and
transfer logic for them is done directly by the ledger), Ada is still treated in a
special way by the ledger rules, and is the most common type of asset on the ledger.
It can
be used for special purposes (such as fees) for which other assets cannot be used.
The underlying consensus algorithm relies on Ada in a way that
cannot be extended to user-defined assets.
Ada can also neither be minted nor burned.

Note that in the $\Value$ definition above, we pick a special asset ID for Ada, that
is not part of the type which represents the asset IDs for all other assets.
Combining the asset name and policy for Ada gives a type-level guarantee that there is exactly
one kind of asset that is associated with it in any way, rather than
deriving this guarantee as an emergent property of minting rules.

Additionally, not giving Ada an actual policy ID
(that could have a hash-associated policy) eliminates the possibility
certain cryptographic attacks.
We sometimes refer to Ada as the primary or principal currency. Ada does not,
for the purposes of the Mary ledger specification, have a $\PolicyID$ or an $\AssetName$.

\subsection{Fixing a Token algebra}

For the remainder of this specification, let $\ValMonoid$ be an
arbitrary but fixed Token algebra. As described above, choosing
$\Coin$ results in the Allegra ledger, and choosing $\Value$ results
in the Mary ledger.

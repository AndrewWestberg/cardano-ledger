\documentclass[11pt,a4paper,dvipsnames,twosided]{article}

\usepackage{amsmath}
\usepackage[unicode=true,pdftex,pdfa,colorlinks=true]{hyperref}
\usepackage{xcolor}

\newcommand\pbar{\overline{p}}

\begin{document}

\title{Stake Pool Ranking in Cardano}

\author{Alexander Byaly  \\ {\small \texttt{alexander.byaly@iohk.io}} \\
   \and Jared Corduan  \\ {\small \texttt{jared.corduan@iohk.io}}}

%\date{}

\hypersetup{
  pdftitle={Stake Pool Ranking in Cardano},
  breaklinks=true,
  bookmarks=true,
  colorlinks=false,
  linkcolor={blue},
  citecolor={blue},
  urlcolor={blue},
  linkbordercolor={white},
  citebordercolor={white},
  urlbordercolor={white}
}

\setlength{\parindent}{0pt}

\maketitle

\section{Introduction}

The purpose of the stake pool ranking algorithm is to provide a systematic and unique ordering for stake pools,
based on the expected long-term rewards that each stake pool will produce.
The pool ranking is provided to users (wallets, people who own ADA, etc.), and can be used as a single
metric to guide delegation decisions.  The ranking is partly based on the past performance of each pool, but
also takes into account current levels of delegation etc.  It should thus be a useful predictor of future performance.


An obvious, but na\"{i}ve, approach to the ranking problem would order pools based on the rewards that are
currently being paid out to their delegators.  There are, however, a number of problems with this approach.

\begin{enumerate}
  \item
  The na\"{i}ve approach is \emph{short-sighted}, in the sense that it does not take into
  account the rewards that a newly formed stake pool would yield when it reaches saturation (\cite{bkks2018} describes this in detail).
  The specific relationship to the Cardano design is explained in the IOHK delegation design document \cite[Section 5.6]{delegation_design}.
  \item
    A stake pool will sometimes have the opportunity to produce blocks,
    but fails to do so due to issues such as a poor network connection or untimely downtime. The rate at
    which a pool uses their opportunities to make blocks, which we will call its \emph{hit rate}, affects rewards.
    It needs to be taken into account.
  \item
    The na\"{i}ve approach only considers a single set of rewards.  Delegating on this basis would create significant
    instability in the system, with delegators pursuing short-term rewards over the long-term health of the network.
    In order to avoid this issue and so create more stability, it is necessary to consider a pool's historic performance, rather than considering only a single set of rewards.
    This introduces some \emph{hysteris} into the system.
\end{enumerate}

Our aim with the pool ranking algorithm is to address all these concerns.

\section{Hit Rate}

As described above, the \emph{hit rate} for some time period $T$ is the ratio of expected to actual blocks
that the pool produces during that time period.

$$
  \begin{array}{lclcp{3in}}
  Hit~Rate_T &=& B_{Actual,T} / B_{Expected,T} \\
  & \textit{where} & B_{Actual,T} &=& \textit{The actual number of blocks that a pool has produced in time T}\\
  &  & B_{Expected,T} &=& \textit{The actual number of blocks that a pool was expected to produce in time T}
  \end{array}
$$

The pool operator is the only one who knows the actual pool
hit rate since other actors do not have access to their VRF calculation,
and  can only attempt to divine it using public knowledge of the
pool's relative stake and the number of blocks it has produced.

\subsection{Estimation}

Each slot, a stake pool with relative (active) stake $\sigma_a$ has probability
\[ 1 - (1-f)^{\sigma_a} \]
of producing a block, where $f$ is the active slot coefficient.
We can use this to estimate the hit rate.

We represent such an estimate as a probability distribution on the interval $[0,1]$.

After each epoch, we refine our estimate into a more accurate one as follows.

Let's say we
observe the output of a stake pool during an epoch. We have a previous estimate, $w$, the
epoch has $E$ slots, the pool produces $n$ blocks, and we calculate that on a given slot
the chance they're allowed to produce a block is $t$.

We compute the likelihood function, $L$, where $L(x)$ is the probability of producing exactly $n$ blocks,

\[ L(x) = \left( E \atop n \right) S(x)^{n} (1-S(x))^{E - n}, \]

and $S(x)$ is the probability that the pool produces a block at a given slot under the assumption that the
hit rate, $p$, is equal to $x$.

\[
\begin{array}{r@{=}l}
S(x) & P( \text{made block} | p = x) \\
     & t \cdot P( \text{made block} | p = x,~\text{can}) + (1-t) \cdot P( \text{made block} | p = x,~\text{can't}) \\
     & tx                              + (1-t) \cdot 0 \\
     & tx \\
\end{array}
\]

Our refined estimate is then $q(x) = \frac{w(x)L(x)}{\int w(x)L(x)}$.

By calculating a running product, $\mathcal{L}$, of likelihood functions, we can defer this computation.

At the time that a user wishes to produce a ranking, they will
supply a prior distribution. This is a default estimate applied to pools for which we don't have any historical data.

To make the rankings we use the posterior distribution,

\begin{equation}
\label{post}
q(x) = \frac{w_(x)\mathcal{L}(x)}{\int w(x)\mathcal{L}(x)}.
\end{equation}

\subsection{Implementation}

We approximate a probability density function, $g$, using a sequence of sample points,

\[ A(g) = \big(\ln(g(0.005)),~ \ln(g(0.015)),~ \ldots,~ \ln(g(0.995))\big). \]

This lets us use multiplication rather than exponentiation for most of our calculations.

Because of the normalization step in \eqref{post}, any scalar multiple of $g$
will yield the same result in the end. So any constant shift of $A(g)$ will as well.
To ensure that our numbers don't grow too large, we use the normal form

\[ A'(g) = \big(\ln(g(0.005)) - m,~ \ln(g(0.015)) - m,~ \ldots,~ \ln(g(0.995)) - m\big), \]
where $m$ is the minimal sample point. We use $g_i$ to denote the $i$'th term of $A'(g)$.

For each stake pool, every epoch we compute this approximation, add it to our previous sum, and convert it to this normal form.

To calculate the final pdf used in ranking, we approximate \eqref{post}.
If $\mathcal{L}$ is our cumulative likelihood
and $w$ is our prior, we use

\[ q_i = e^{w_i + \mathcal{L}_i} / S, \]
where
\[ S = \sum (0.01) e^{w_i + \mathcal{L}_i}. \]

In the ranking calculation, we use the $k$'th percentile of this distribution for some $k$.
We do so by computing the minimal $i$ such that the sum of $[q_0,~\ldots,q_i]$ is at least $k$.
Then $i/100$ is the $k$'th percentile.

Putting it all together, given some prior distribution,
a percentile, and a cumulative likelihood,
we estimate the hit rate to be $i/100$.


\subsection{Example}

Let's say Alice's pool has been active for two epochs.
There are 432,000 slots per epoch, and the active slot coefficient is 0.05
(the mainnet values at the time of writing).

During the first epoch, the relative stake was $0.001$ and 10 blocks were produced.

During the second epoch, the relative stake was $0.01$ and 200 blocks were produced.

We calculate $\ln(L_1(x))$ for the first epoch as follows.

With relative stake $0.001$, the probability of being allowed to produce a block on
each slot is

\[ t_1 = 1-(1-0.05)^{0.001}, \]
and $S_1(x)= t_1x$

Since 10 blocks were produced, the real likelihood function is

\[ L_1(x) =  \left( 432,000 \atop 10 \right) (t_1x)^{10} (1-t_1x)^{431,990} \]

and we will use
\[
\begin{array}{rl}
  \ell_1(x) &= \ln\left( (t_1x)^{10} (t_1x)^{431,990} \right) \\
            &= 10 \ln(t_1 x) + 431,990 \ln(1 - t_1 x) \\
\end{array}
\]
for storing information about Alice's pool's performance this epoch.
In particular, we store

\[ A'(L_1) = [\ell_1(0.005)-m_1,~\ell_1(0.015)-m_1,~\ldots,~\ell_1(.995)-m_1] \]
where $m_1=\min\{\ell_1(0.005),\ldots,\ell_1(0.995)\}$

For the second epoch,
\[
\begin{array}{rl}
  t_2 &= 1-(1-0.05)^{0.01} \\
  \ell_2(x) &= 200 \ln(t_2 x) + 431,800 \ln(1 - t_2 x).\\
\end{array}
\]
Then for the second epoch we store:
\[ A'(L_2) = [\ell_2(0.005)+\ell_1(0.005)-m_2,~\ldots,~\ell_2(.995)+\ell_1(.995)-m_2] \]
where $m_2=\min\{\ell_2(0.005)+\ell_1(0.005),\ldots,\ell_2(0.995)+\ell_1(0.995)\}$.

\section{Ranking and Saturation}

We now describe how to use the hit rate estimation to calculate the stake pool ranking
described in \cite{bkks2018} and \cite[Section 5.6]{delegation_design}.

First we must turn the sampling of the likelihood functions into a concrete value.
Given a percentage, $k$, we will calculate the $k$'th percentile of the hit rates.
Currently we're using the 10\%, but we aim for this to be user configurable.
For a given pool, let $h_k$ denote the $k$'th percentile.

Following \cite[Section 5.6.1]{delegation_design},
we can compute any given stake pool's rewards at saturation with:
\[
    \tilde{f}(s, \pbar) :=
    \hat{f}(s,z_0,\pbar)=
    \frac{\pbar R}{1 + a_0}
    \cdot
    \left(z_0 + \min(s,z_0)\cdot a_0\right).
\]

where $\bar{p}$ is the \textit{apparent performance} defined in
\cite[Section 5.5.2]{delegation_design}.
We will substitute our hit rate estimation $h_k$ for the apparent performance $\bar{p}$,
Given a pool with pledged owner stake \(s\), costs \(c\) and margin \(m\),
the desirability function of \cite[Section 5.6.1]{delegation_design} becomes:
\[
    d(c, m, s, h_k) :=
    \left\{
    \begin{array}{ll}
        \displaystyle 0 &
        \text{if $\tilde{f}(s,h_k)\leq c$,} \\
        \displaystyle\left(\tilde{f}(s,h_k) - c\right)\cdot(1-m) &
        \text{otherwise.}
    \end{array}
    \right.
\]
From here, we proceed exactly as in
\cite[Section 5.6.2]{delegation_design} and
\cite[Section 5.6.4]{delegation_design}.

The desirability function provides a ranking of stake pools.
For a stake pool with pledged owner stake $s$, total stake $\sigma$ and rank
$r$, we define its \emph{non-myopic stake} $\sigma_\mathrm{nm}$ as
\[
    \sigma_\mathrm{nm}(s,\sigma,r) :=
    \left\{
    \begin{array}{ll}
        \max(\sigma,z_0) &
        \text{if $r\leq k$,} \\
        s &
        \text{otherwise,}
    \end{array}
    \right.
\]
where $k$ is the parameter defined in \cite[Section 5.2]{delegation_design}
which represents the desired number of stake pools.

The non-myopic pool member rewards of a pool with costs $c$, margin $m$,
pledged owner stake $s$, stake $\sigma$, rank $r$, and apparent
performance $\pbar$, for a member contributing member stake $t$, are
\[
    r_\mathrm{member, nm}(c, m, s, \sigma, t, r, \pbar) :=
    r_\mathrm{member}\Bigl(\hat{f}\bigl(s,\sigma_\mathrm{nm}(s,\sigma, r),
    \pbar\bigr),
    c, m, t, \sigma_\mathrm{nm}(s,\sigma,r)\Bigr),
\label{eq:non-myopic-member-rewards}
\]
where
\[
    r_\mathrm{member}(\hat{f}, c, m, t, \sigma) :=
    \left\{
    \begin{array}{ll}
        \displaystyle 0 &
        \text{if $\hat{f}\leq c$,} \\
        \displaystyle (\hat{f} - c)\cdot(1-m)\cdot\frac{t}{\sigma} &
        \text{otherwise.}
    \end{array}
    \right.
\]

Finally, users can use $r_\mathrm{member, nm}$ to select a pool
that maximizes their rewards.

\addcontentsline{toc}{section}{References}
\bibliographystyle{habbrv}
\bibliography{references}

\end{document}

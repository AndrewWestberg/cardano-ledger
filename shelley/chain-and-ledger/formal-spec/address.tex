\section{Addresses}
\label{sec:addresses}

Addresses are described in section 4.2 of the delegation design document \cite{delegation_design}.
The types needed for the addresses are defined in Figure~\ref{fig:defs:addresses}.
There are four types of UTxO addresses:
\begin{itemize}
  \item Base addresses, $\AddrB$,
        containing the hash of a payment key and the hash of a staking key,
  \item Pointer addresses, $\AddrP$,
        containing the hash of a payment key and a pointer to a stake key registration certificate,
  \item Enterprise addresses, $\AddrE$,
        containing only the hash of a payment key (and which have no staking rights).
  \item Bootstrap addresses, $\AddrBS$,
        corresponding to the addresses in Byron, behaving exactly like enterprise addresses.
\end{itemize}
Together, these three address types make up the $\Addr$ type, which will be used
in transaction outputs in Section~\ref{sec:utxo}.

Note that for security, privacy and usability reasons, the staking (delegating)
key pair associated with an address should be different from its payment key pair.
Before the stake key is registered and delegated to an existing stake pool,
the payment key can be used for transactions, though it will not receive rewards from staking.
Once a stake key is registered, the shorter pointer addresses can generated.

Finally, there is an account style address $\AddrRWD$ which contains the hash of a staking key.
These account addresses will only be used for receiving rewards from the proof of
stake leader election.  Apendix A of \cite{delegation_design} explains this design choice.
The mechanism for transferring rewards from these accounts will be explained in
Section~\ref{sec:utxo} and follows \cite{chimeric}.

\begin{figure*}[hbt]
  \emph{Abstract types}
  %
  \begin{equation*}
    \begin{array}{r@{~\in~}lr}
      slot & \Slot & \text{absolute slot}\\
      ix & \Ix & \text{index}\\
    \end{array}
  \end{equation*}
  %
  \emph{Derived types}
  %
  \begin{equation*}
    \begin{array}{r@{~\in~}l@{\qquad=\qquad}lr}
      \var{(s,t,c)}
      & \Ptr
      & \Slot\times\Ix\times\Ix
      & \text{certificate pointer}
      \\
      \var{addr}
      & \AddrB
      & \KeyHash_{pay}\times\KeyHash_{stake}
      & \text{base address}
      \\
      \var{addr}
      & \AddrP
      & \KeyHash_{pay}\times\Ptr
      & \text{pointer address}
      \\
      \var{addr}
      & \AddrE
      & \KeyHash_{pay}
      & \text{enterprise address}
      \\
      \var{addr}
      & \AddrBS
      & \KeyHash_{pay}
      & \text{bootstrap address}
      \\
      \var{addr}
      & \Addr
      & \begin{array}{l@{~\uniondistinct}l}
          \AddrB & \AddrP \uniondistinct \AddrE
          \\
                 & \AddrBS
        \end{array}
      & \text{output address}
      \\
      \var{acct}
      & \AddrRWD
      & \KeyHash_{stake}
      & \text{reward account}
      \\
    \end{array}
  \end{equation*}
  %
  \emph{Accessor Functions}
  %
  \begin{equation*}
    \begin{array}{r@{~\in~}lr}
      \fun{paymentHK} & \Addr \to \KeyHash_{pay}
                      & \text{hash of payment key from addr}\\
      \fun{stakeHK_b} & \AddrB \to \KeyHash_{stake}
                      & \text{hash of stake key from base addr}\\
      \fun{stakeHK_r} & \AddrRWD \to \KeyHash_{stake}
                      & \text{hash of stake key from reward account}\\
      \fun{addrPtr} & \AddrP \to \Ptr
                    & \text{pointer from pointer addr}\\
    \end{array}
  \end{equation*}
  %
  \emph{Constructor Functions}
  %
  \begin{equation*}
    \begin{array}{r@{~\in~}lr}
      \fun{addr_{rwd}}
        & \KeyHash_{stake} \to \AddrRWD
        & \text{construct a reward account}
    \end{array}
  \end{equation*}
  %
  \emph{Constraints}
  %
  \begin{equation*}
    \var{hk_1} = \var{hk_2} \iff \fun{addr_{rwd}}~\var{hk_2} = \fun{addr_{rwd}}~\var{hk_2}
    ~~~ \left( \fun{addr_{rwd}} \text{ is injective} \right)
  \end{equation*}
  \caption{Definitions used in Addresses}
  \label{fig:defs:addresses}
\end{figure*}

\clearpage

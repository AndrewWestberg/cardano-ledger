\section{Addresses}
\label{sec:addresses}

Addresses are described in section 4.2 of the delegation design document \cite{delegation_design}.
The types needed for the addresses are defined in Figure~\ref{fig:defs:addresses}.
There are four types of UTxO addresses:
\begin{itemize}
\item Base addresses, $\AddrB$, containing the hash of a payment credential and
  the hash of a staking credential,
\item Pointer addresses, $\AddrP$, containing the hash of a payment credential
  and a pointer to a stake credential registration certificate,
\item Enterprise addresses, $\AddrE$,
  containing only the hash of a payment credential (and which have no staking rights).
\item Bootstrap addresses, $\AddrBS$, corresponding to the addresses in
  Byron, behaving exactly like enterprise addresses with a key hash
  payment credential.
\end{itemize}

\noindent Where a credential is either key or a multi-signature script. Together, these
three address types make up the $\Addr$ type, which will be used in transaction
outputs in Section~\ref{sec:utxo}.

For an explanation of why enterprise addresses do not have staking rights,
and how they can delegate to a stake pool, see \ref{sec:delegation-shelley}.
The boostrap addresses are needed for the Byron-Shelley transition in order to
accommodate having UTxO entries from the Byron era during the Shelley era.
Together, these address types make up the $\Addr$ type, which will be used
in transaction outputs in Section~\ref{sec:utxo}.

Note that for security, privacy and usability reasons, the staking (delegating)
credential associated with an address should be different from its payment
credential.  Before the stake credential is registered and delegated to an
existing stake pool, the payment credential can be used for transactions, though
it will not receive rewards from staking.  Once a stake credential is
registered, the shorter pointer addresses can be generated.

Finally, there is an account style address $\AddrRWD$ which contains the hash of
a staking credential. These account addresses will only be used for receiving
rewards from the proof of stake leader election. Appendix A
of~\cite{delegation_design} explains this design choice.  The mechanism for
transferring rewards from these accounts will be explained in
Section~\ref{sec:utxo} and follows~\cite{chimeric}.

Base, pointer and enterprise addresses contain a payment credential which is
either a key hash or a script hash. Base addresses contain a staking credential
which is also either a key hash or a script hash.

\begin{figure*}[hbt]
  \emph{Abstract types}
  %
  \begin{equation*}
    \begin{array}{r@{~\in~}lr}
      slot & \Slot & \text{absolute slot}\\
      ix & \Ix & \text{index}\\
    \end{array}
  \end{equation*}
  %
  \emph{Derived types}
  %
  \begin{equation*}
    \begin{array}{r@{~\in~}l@{\qquad=\qquad}lr}
      \var{cred} & \Credential & \KeyHash\uniondistinct\HashScr \\
      \var{(s,t,c)}
      & \Ptr
      & \Slot\times\Ix\times\Ix
      & \text{certificate pointer}
      \\
      \var{addr}
      & \AddrB
      & \KeyHash_{pay}\times\KeyHash_{stake}
      & \text{base address}
      \\
      \var{addr}
      & \AddrP
      & \KeyHash_{pay}\times\Ptr
      & \text{pointer address}
      \\
      \var{addr}
      & \AddrE
      & \KeyHash_{pay}
      & \text{enterprise address}
      \\
      \var{addr}
      & \AddrBS
      & \KeyHash_{pay}
      & \text{bootstrap address}
      \\
      \var{addr}
      & \Addr
      & \begin{array}{l@{~\uniondistinct}l}
          \AddrB & \AddrP \uniondistinct \AddrE
          \\
                 & \AddrBS
        \end{array}
      & \text{output address}
      \\
      \var{acct}
      & \AddrRWD
      & \KeyHash_{stake}
      & \text{reward account}
      \\
    \end{array}
  \end{equation*}
  %
  \emph{Accessor Functions}
  %
  \begin{equation*}
    \begin{array}{r@{~\in~}lr}
      \fun{paymentHK} & \AddrVKey \to \KeyHash_{pay}
      & \text{hash of payment key from addr}\\
      \fun{validatorHash} & \AddrScr \to \HashScr & \text{hash of validator
                                                    script} \\
            \fun{stakeCred_{b}} & (\AddrVKeyB \uniondistinct \AddrScrBase) \to
                          \StakeCredential & \text{stake credential from base
                                      addr}\\
      \fun{stakeCred_{r}} & \AddrRWD \to \StakeCredential & \text{stake credential
                                                   from reward addr}\\
      \fun{addrPtr} & \AddrP \to \Ptr
                    & \text{pointer from pointer addr}\\
    \end{array}
  \end{equation*}
  %
  \emph{Constructor Functions}
  %
  \begin{equation*}
    \begin{array}{r@{~\in~}lr}
      \fun{addr_{rwd}}
        & \Credential_{stake} \to \AddrRWD
        & \text{construct a reward account}
    \end{array}
  \end{equation*}
  %
  \emph{Constraints}
  %
  \begin{equation*}
    \var{hk_1} = \var{hk_2} \iff \fun{addr_{rwd}}~\var{hk_2} = \fun{addr_{rwd}}~\var{hk_2}
    ~~~ \left( \fun{addr_{rwd}} \text{ is injective} \right)
  \end{equation*}
  \caption{Definitions used in Addresses}
  \label{fig:defs:addresses}
\end{figure*}

\clearpage

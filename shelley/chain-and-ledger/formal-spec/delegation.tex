\section{Delegation}
\label{sec:delegation-shelley}

We briefly describe the motivation and context for delegation.
The full context is contained in \cite{delegation_design}.

Stake is said to be \textit{active} in the blockchain protocol when it is
eligible for participation in the leader election. In order for stake to become
active, the associated verification stake credential must be registered and its
staking rights must be delegated to an active stake pool. Individuals who wish
to participate in the protocol can register themselves as a stake pool.

Stake credentials are registered (or deregistered) through the use of
registration (or deregistration) certificates. Registered stake credentials are
delegated through the use of delegation certificates.  Finally, stake pools are
registered (or retired) through the use of registration (or retirement)
certificates.

Stake pool retirement is handled a bit differently than stake deregistration.
Stake credentials are considered inactive as soon as a deregistration
certificate is applied to the ledger state.  Stake pool retirement certificates,
however, specify the epoch in which it will retire.

Delegation requires the following to be tracked by the ledger state: the
registered stake credentials, the delegation map from registered stake
credentials to stake pools, pointers associated with stake credentials, the
registered stake pools and upcoming stake pool retirements.  Additionally, the
blockchain protocol rewards eligible stake and so we must also include a mapping
from active stake credentials to rewards.

Finally, there are two types of delegation certificates available only to the
genesis keys. The genesis keys will still be used for update proposals at the
begin of the Shelley era, and so there must be a way to maintain the delegation
of these keys to their cold keys.  This mapping is also maintained by the
delegation state. There is also a mechanism to transfer rewards directly from
the reserves pot to a reward address. While technically everybody can post such
a certificate, the transaction that contains it must be signed by genesis keys.

\subsection{Delegation Definitions}
\label{sec:deleg-defs}

In \cref{fig:delegation-defs} we give the delegation primitives.
Here we introduce the following primitive datatypes used in delegation:

\begin{itemize}
\item $\DCertRegKey$: a stake credential registration certificate.
\item $\DCertDeRegKey$: a stake credential de-registration certificate.
\item $\DCertDeleg$: a stake credential delegation certificate.
\item $\DCertRegPool$: a stake pool registration certificate.
\item $\DCertRetirePool$: a stake pool retirement certificate.
\item $\DCertGen$: a genesis key delegation certificate.
\item $\DCertMir$: a move instantaneous rewards certificate.
\item $\DCert$: any one of of the seven certificate types above.
\end{itemize}
The following derived types are introduced:
\begin{itemize}
\item $\type{StakeDelegs}$ represents registered stake delegations and is
  represented by a finite map from stake credentials to slot when it was
  registered.
\item$\type{StakePools}$ represents registered stake pools
\item $\PoolParam$ represents the parameters found in a stake pool registration certificate
  that must be tracked:
  \begin{itemize}
    \item the pool owners.
    \item the pool cost.
    \item the pool margin.
    \item the pool pledge.
    \item the pool reward account.
    \item the hash of the VRF verification key.
  \end{itemize}
  The idea of pool owners is explained in Section 4.4.4 of \cite{delegation_design}.
  The pool cost and margin indicate how much more of the rewards pool leaders
  get than the members.
  The pool pledge is explained in Section 5.1 of \cite{delegation_design}.
  The pool reward account is where all pool rewards go.
\end{itemize}

Accessor functions for certificates and pool parameters are also defined, but
only the $\cwitness{}$ accessor function needs explanation.
It does the following:
\begin{itemize}
  \item For a $\DCertRegKey$ certificate, $\fun{cwitness}$ returns the hashkey
  of the key being registered.
\item For a $\DCertDeRegKey$ certificate, $\fun{cwitness}$ returns the hashkey
  of the key being de-registered.
\item For a $\DCertDeleg$ certificate, $\fun{cwitness}$ returns the hashkey
  of the key that is delegating (and not the key to which the stake in being delegated to).
\item For a $\DCertRegPool$ certificate, $\fun{cwitness}$ returns the hashkey
  of the key of the pool operator.
\item For a $\DCertRetirePool$ certificate, $\fun{cwitness}$ returns the hashkey
  of the key of the pool operator.
\item For a $\DCertGen$ certificate, $\fun{cwitness}$ returns the hashkey
  of the genesis key.
\item For a $\DCertMir$ certificate, $\fun{cwitness}$ is not defined as there is
  no single core node or genesis key that posts the certificate.
\end{itemize}

%%
%% Figure - Delegation Definitions
%%
\begin{figure}[htb]
  \emph{Abstract types}
  %
  \begin{equation*}
    \begin{array}{r@{~\in~}lr}
    \end{array}
  \end{equation*}
  %
  \emph{Delegation Certificate types}
  %
  \begin{equation*}
  \begin{array}{r@{}c@{}l}
    \DCert &=& \DCertRegKey \uniondistinct \DCertDeRegKey \uniondistinct \DCertDeleg \\
                &\hfill\uniondistinct\;&
                \DCertRegPool \uniondistinct \DCertRetirePool \uniondistinct
                                         \DCertGen\\
           &\hfill\uniondistinct\;& \DCertMir
  \end{array}
  \end{equation*}
  %
  \emph{Derived types}
  \begin{equation*}
    \begin{array}{l@{\qquad=\qquad}lr}
      \StakeDelegs
      & \Credential \mapsto \Slot
      & \text{registered stake credential} \\
      %
      \StakePools
      & \KeyHash \mapsto \Slot
      & \text{registered stake pools} \\
      %
      \PoolParam
      & \powerset{\KeyHash} \times \Coin \times \unitInterval \times \Coin \times \AddrRWD
      & \text{stake pool parameters} \\
    \end{array}
  \end{equation*}
  %
  \emph{Certificate Accessor functions}
  %
  \begin{equation*}
    \begin{array}{r@{~\in~}lr}
      \cwitness{} & \DCert\setminus\DCertMir \to \StakeDelegs & \text{certificate witness} \\
      \fun{dpool} & \DCertDeleg \to \KeyHash
                                            & \text{pool being delegated to}
      \\
      \fun{poolParam} & \DCertRegPool \to \PoolParam
                                            & \text{stake pool}
      \\
      \fun{retire} & \DCertRetirePool \to \Epoch
                                            & \text{epoch of pool retirement}
      \\
      \fun{genDel} & \DCertGen \to (\VKeyGen,~\VKey)
                                            & \text{genesis delegation}
      \\
      \fun{moveRewards} & \DCertMir \to (\KeyHash \mapsto \Coin)
                                            & \text{moved instantaneous rewards}
    \end{array}
  \end{equation*}
  %
  \emph{Pool Parameter Accessor functions}
  %
  \begin{equation*}
  \begin{array}{r@{~\in~}lr}
    \fun{poolOwners} & \PoolParam \to \powerset{\KeyHash}
                     & \text{stake pool owners}
    \\
    \fun{poolCost} & \PoolParam \to \Coin
                     & \text{stake pool cost}
    \\
    \fun{poolMargin} & \PoolParam \to \unitInterval
                     & \text{stake pool margin}
    \\
    \fun{poolPledge} & \PoolParam \to \Coin
                     & \text{stake pool pledge}
    \\
    \fun{poolRAcnt} & \PoolParam \to \AddrRWD
                     & \text{stake pool reward account}
    \\
    \fun{poolVRF} & \PoolParam \to \KeyHash_{vrf}
                  & \text{stake pool VRF key hash}
    \\
  \end{array}
  \end{equation*}

  \caption{Delegation Definitions}
  \label{fig:delegation-defs}
\end{figure}

\clearpage

\subsection{Delegation Transitions}
\label{sec:deleg-trans}


In \cref{fig:delegation-transitions} we give the delegation and stake pool
state transition types. We define two separate parts of the ledger state.

\begin{itemize}
  \item $\DState$ keeps track of the delegation state, consisting of:
    \begin{itemize}
    \item $\var{stdelegs}$ tracks the registered stake credentials. It consists
      of a finite mapping from hashkeys to the slot of the registration.
    \item $\var{rewards}$ stores the rewards accumulated by stake credentials.
      These are represented by a finite map from reward addresses to the
      accumulated rewards.
    \item $\var{delegations}$ stores the delegation relation, mapping stake
      credentials to the pool to which is delegates.
    \item $\var{ptrs}$ maps stake credentials to the position of the
      registration certificate in the blockchain. This is needed to lookup the
      stake hashkey of a pointer address.
      \item $\var{fdms}$ are the future genesis keys delegations. This variable
      is needed because genesis keys can only update their delegation with a
      delay of $\SlotsPrior$ slots after submitting the certificate (this is
      necessary for header validation, see Section \ref{sec:chain})
      \item $\var{dms}$ maps genesis key hashes to hashes of the cold key
        delegates.
      \item $\var{i_{rwd}}$ stored the map of stake credentials to $\Coin$
        values for moving instantaneous rewards at the epoch boundary.
    \end{itemize}
  \item $\PState$ keeps track of the stake pool information:
    \begin{itemize}
      \item $\var{stpools}$ tracks the registered stake pools. It consists of a finite
        mapping from hashkeys to the slot of the registration.
      \item $\var{poolParams}$ tracks the parameters associated with each stake pool, such as
        their costs and margin.
      \item $\var{retiring}$ tracks stake pool retirements, using a map from hashkeys to
        the epoch in which it will retire.
      \item $\var{cs}$ stores the latest operational certificate issues numbers used for each pool.
        The numbers are used in the operation certificate transition
        in Figure~\ref{fig:rules:ocert}.
    \end{itemize}
\end{itemize}

The operational certificates counters $\var{cs}$ in the stake pool state are a
tool to ensure that blocks containing outdated certificates are rejected.
These certificates are part of the block header.
For a discussion of why this additional mechanism is needed,
see \cite{delegation_design}, and for
the relevant rules, see Section \ref{sec:oper-cert-trans}.

The environment for the state transition for $\DState$ contains the current slot number
and the index for the current certificate pointer.
The environment for the state transition for $\PState$ contains the current slot number
and the protocol parameters.

%%
%% Figure - Delegation Transitions
%%
\begin{figure}
  \emph{Delegation Types}
  \begin{equation*}
    \begin{array}{r@{~\in~}l@{\qquad=\qquad}lr}
      \var{stakeCred} & \StakeCredential & (\KeyHash_{stake} \uniondistinct
                                       \HashScr) \\
      \var{stakeDelegator} & \StakeDelegs & \StakeCredential \mapsto \Slot \\
    \end{array}
  \end{equation*}
  %
  \emph{Delegation States}
  %
  \begin{equation*}
    \begin{array}{l}
    \DState =
    \left(\begin{array}{r@{~\in~}lr}
            \var{stDelegs} & \StakeDelegs & \text{registered stake delegators}\\
            \var{rewards} & \AddrRWD \mapsto \Coin & \text{rewards}\\
            \var{delegations} & \StakeCredential \mapsto \KeyHash_{pool} & \text{delegations}\\
            \var{ptrs} & \Ptr \mapsto \KeyHash & \text{pointer to hashkey}\\
            \var{fdms} & (\Slot\times\KeyHashGen) \mapsto \KeyHash & \text{future genesis key delegations}\\
            \var{dms} & \KeyHashGen \mapsto \KeyHash & \text{genesis key delegations}\\
            \var{i_{rwd}} & \KeyHash \mapsto \Coin & \text{instantaneous rewards}\\
          \end{array}
      \right)
      \\
    \\
    \PState =
    \left(\begin{array}{r@{~\in~}lr}
      \var{stpools} & \StakePools & \text{registered pools to creation time}\\
      \var{poolParams} & \KeyHash_{pool} \mapsto \PoolParam
        & \text{registered pools to pool parameters}\\
      \var{retiring} & \KeyHash_{pool} \mapsto \Epoch & \text{retiring stake pools}\\
      \var{cs} & \KeyHash_{pool} \mapsto \N & \text{certificate issue numbers}\\
    \end{array}\right)
    \end{array}
  \end{equation*}
  %
  \emph{Delegation Environment}
  \begin{equation*}
    \DEnv =
    \left(
      \begin{array}{r@{~\in~}lr}
        \var{slot} & \Slot & \text{slot}\\
        \var{ptr} & \Ptr & \text{certificate pointer}\\
        \var{reserves} & \Coin & \text{total reserves available}
      \end{array}
    \right)
  \end{equation*}
  %
  \emph{Pool Environment}
  \begin{equation*}
    \PEnv =
    \left(
      \begin{array}{r@{~\in~}lr}
        \var{slot} & \Slot & \text{slot}\\
        \var{pp} & \PParams & \text{protocol parameters}\\
      \end{array}
    \right)
  \end{equation*}
  %
  \emph{Delegation Transitions}
  \begin{equation*}
    \_ \vdash \_ \trans{deleg}{\_} \_ \in
      \powerset (\DEnv \times \DState \times \DCert \times \DState)
  \end{equation*}
  %
  \begin{equation*}
    \_ \vdash \_ \trans{pool}{\_} \_ \in
    \powerset (\PEnv \times \PState \times \DCert \times \PState)
  \end{equation*}
  %
  \caption{Delegation Transitions}
  \label{fig:delegation-transitions}
\end{figure}

\clearpage

\subsection{Delegation Rules}
\label{sec:deleg-rules}


The rules for registering and delegating stake credentials are given in
\cref{fig:delegation-rules}.  Note that section 5.2 of \cite{delegation_design}
describes how a wallet would help a user choose a stake pool, though these
concerns are independent of the ledger rules.

\begin{itemize}
\item Stake credential registration is handled by \cref{eq:deleg-reg}, since it
  contains the precondition that the certificate has type $\DCertRegKey$.  All
  the equations in $\mathsf{DELEG}$ and $\mathsf{POOL}$ follow this same pattern
  of matching on certificate type.

  There is also a precondition on registration that the hashkey associated with
  the certificate witness of the certificate is not already found in the current
  list of stake credentials.

    Registration causes the following state transformation:
    \begin{itemize}
    \item The key is added to the set of registered stake credentials.
      \item A reward account is created for this key, with a starting balance of zero.
      \item The certificate pointer is mapped to the new stake credential.
    \end{itemize}

  \item Stake credential deregistration is handled by \cref{eq:deleg-dereg}.
    There is a precondition that the credential has been registered and that
    the reward balance is zero.  Deregistration causes the following state
    transformation:
    \begin{itemize}
      \item The key is removed from the collection of registered keys.
      \item The reward account is removed.
      \item The key is removed from the delegation relation.
      \item The certificate pointer is removed.
    \end{itemize}

  \item Stake credential delegation is handled by \cref{eq:deleg-deleg}.
    There is a precondition that the key has been registered.
    Delegation causes the following state transformation:
    \begin{itemize}
    \item The delegation relation is updated so that the stake credential is
      delegated to the given stake pool. The use of union override here allows
      us to use the same rule to perform both an initial delegation and an
      update to an existing delegation.
    \end{itemize}

  \item Genesis key delegation is handled by \cref{eq:deleg-gen}.
    There is a precondition that the genesis key is already in the mapping $\var{dms}$.
    Genesis delegation causes the following state transformation:
    \begin{itemize}
      \item The future genesis delegation relation is updated with the new delegate
        to be adopted in $\SlotsPrior$-many slots.
      \end{itemize}

    \item  Moving instantaneous rewards is handled by \cref{eq:deleg-mir}. There
      is a precondition that the current slot is early enough in the current
      epoch and that the available reserves are sufficient to pay for the
      instantaneous rewards.
\end{itemize}



%%
%% Figure - Delegation Rules
%%
\begin{figure}[hbt]
  \centering
  \begin{equation}\label{eq:deleg-reg}
    \inference[Deleg-Reg]
    {
    \var{c}\in\DCertRegKey & hk \leteq \cwitness{c} & hk \notin \dom \var{stdelegs}
    }
    {
      \begin{array}{r}
        \var{slot} \\
        \var{ptr} \\
        \var{reserves}
      \end{array}
      \vdash
      \left(
        \begin{array}{r}
        \var{stdelegs} \\
        \var{rewards} \\
        \var{delegations} \\
        \var{ptrs} \\
        \var{fdms} \\
        \var{dms} \\
        \var{i_{rwd}}
      \end{array}
      \right)
      \trans{deleg}{\var{c}}
      \left(
      \begin{array}{rcl}
        \varUpdate{\var{stdelegs}} & \varUpdate{\union} & \varUpdate{\{\var{hk} \mapsto slot\}} \\
        \varUpdate{\var{rewards}} & \varUpdate{\union} & \varUpdate{\{\addrRw \var{hk} \mapsto 0\}}\\
        \var{delegations} \\
        \varUpdate{\var{ptrs}} & \varUpdate{\union} & \varUpdate{\{ptr \mapsto \var{hk}\}} \\
        \var{fdms} \\
        \var{dms} \\
        \var{i_{rwd}}
      \end{array}
      \right)
    }
  \end{equation}

  \begin{equation}\label{eq:deleg-dereg}
    \inference[Deleg-Dereg]
    {
      \var{c}\in \DCertDeRegKey  & hk \leteq \cwitness{c} \\
    hk \in \dom \var{stdelegs} & \addrRw \var{hk} \mapsto 0 \in \var{rewards}
    }
    {
      \begin{array}{r}
        \var{slot} \\
        \var{ptr} \\
        \var{reserves}
      \end{array}
      \vdash
      \left(
      \begin{array}{r}
        \var{stdelegs} \\
        \var{rewards} \\
        \var{delegations} \\
        \var{ptrs} \\
        \var{fdms} \\
        \var{dms} \\
        \var{i_{rwd}}
      \end{array}
      \right)
      \trans{deleg}{\var{c}}
      \left(
      \begin{array}{rcl}
        \varUpdate{\{\var{hk}\}} & \varUpdate{\subtractdom} & \varUpdate{\var{stdelegs}} \\
        \varUpdate{\{\addrRw \var{hk}\}} & \varUpdate{\subtractdom} & \varUpdate{\var{rewards}} \\
        \varUpdate{\{\var{hk}\}} & \varUpdate{\subtractdom} & \varUpdate{\var{delegations}} \\
        \varUpdate{\var{ptrs}} & \varUpdate{\subtractrange} & \varUpdate{\{\var{hk}\}} \\
        \var{fdms} \\
        \var{dms} \\
        \var{i_{rwd}}
      \end{array}
      \right)
    }
  \end{equation}

  \begin{equation}\label{eq:deleg-deleg}
    \inference[Deleg-Deleg]
    {
      \var{c}\in \DCertDeleg & hk \leteq \cwitness{c} & hk \in \dom \var{stdelegs}
    }
    {
      \begin{array}{r}
        \var{slot} \\
        \var{ptr} \\
        \var{reserves}
      \end{array}
      \vdash
      \left(
      \begin{array}{r}
        \var{stdelegs} \\
        \var{rewards} \\
        \var{delegations} \\
        \var{ptrs} \\
        \var{fdms} \\
        \var{dms} \\
        \var{i_{rwd}}
      \end{array}
      \right)
      \trans{deleg}{c}
      \left(
      \begin{array}{rcl}
        \var{stdelegs} \\
        \var{rewards} \\
        \varUpdate{\var{delegations}} & \varUpdate{\unionoverrideRight}
                                      & \varUpdate{\{\var{hk} \mapsto \dpool c\}} \\
        \var{ptrs} \\
        \var{fdms} \\
        \var{dms} \\
        \var{i_{rwd}}
      \end{array}
      \right)
    }
  \end{equation}

  \begin{equation}\label{eq:deleg-gen}
    \inference[Deleg-Gen]
    {
      \var{c}\in \DCertGen
      & (\var{gkey},~\var{vk})\leteq\fun{genDel}~{c}
      \\
      \var{gkh} \leteq \hashKey{gkey}
      & \var{vkh} \leteq \hashKey{vk}
      \\
      s'\leteq\var{slot}+\SlotsPrior
      & \var{gkh}\in\dom{dms}
      & \var{vk}\notin\range{dms}
    }
    {
      \begin{array}{r}
        \var{slot} \\
        \var{ptr} \\
        \var{reserves}
      \end{array}
      \vdash
      \left(
      \begin{array}{r}
        \var{stdelegs} \\
        \var{rewards} \\
        \var{delegations} \\
        \var{ptrs} \\
        \var{fdms} \\
        \var{dms} \\
        \var{i_{rwd}}
      \end{array}
      \right)
      \trans{deleg}{c}
      \left(
      \begin{array}{rcl}
        \var{stdelegs} \\
        \var{rewards} \\
        \var{delegations} \\
        \var{ptrs} \\
        \varUpdate{\var{fdms}} & \varUpdate{\unionoverrideRight}
                               & \varUpdate{\{(\var{s'},~\var{gkh}) \mapsto \var{vkh}\}} \\
        \var{dms} \\
        \var{i_{rwd}}
      \end{array}
      \right)
    }
  \end{equation}

  \caption{Delegation Inference Rules}
  \label{fig:delegation-rules}
\end{figure}

\begin{figure}[htp]
  \centering
  \begin{equation}\label{eq:deleg-mir}
    \inference[Deleg-Mir]
    {
      \var{c}\in \DCertMir\\
      slot < \fun{firstSlot}((\fun{epoch}~\var{s} + 1)) - \fun{SlotsPrior}\\
      \left(
        \sum\limits_{\wcard\mapsto\var{val}\in(\var{i_{rwd}}\unionoverrideRight\var{i'_{rwd}})} val      \right)\leq\var{reserves}
    }
    {
      \begin{array}{r}
        \var{slot} \\
        \var{ptr} \\
        \var{reserves}
      \end{array}
      \vdash
      \left(
      \begin{array}{r}
        \var{stdelegs} \\
        \var{rewards} \\
        \var{delegations} \\
        \var{ptrs} \\
        \var{fdms} \\
        \var{dms} \\
        \var{i_{rwd}}
      \end{array}
      \right)
      \trans{deleg}{c}
      \left(
      \begin{array}{rcl}
        \var{stdelegs}\\
        \var{rewards} \\
        \var{delegations} \\
        \var{ptrs} \\
        \var{fdms}\\
        \var{dms} \\
        \varUpdate{\var{i_{rwd}}} & \varUpdate{\unionoverrideRight}
        & \varUpdate{\fun{moveRewards}~\var{c}}
      \end{array}
      \right)
    }
  \end{equation}
  \caption{Move Instantaneous Rewards Inference Rule}
  \label{fig:dcert-mir}
\end{figure}

The DELEG rule has seven possible predicate failures:
\begin{itemize}
\item In the case of a key registration certificate, if the staking credential
  is already registered, there is a a \emph{StakeKeyAlreadyRegistered} failure.
\item In the case of a key deregistration certificate, if the key is not
  registered, there is a \emph{StakeKeyNotRegistered} failure.
\item In the case of a non-existing stake pool key in a delegation certificate,
  there is a \emph{StakeDelegationImpossible} failure.
\item In the case of a pool delegation certificate, there is a
  \emph{WrongCertificateType} failure.
\item  In the case of a genesis key delegation certificate, if the genesis key is not
  in the domain of the genesis delegation mapping, there is a
  \emph{GenesisKeyNotInMapping} failure.
\item  In the case of a genesis key delegation certificate, if the delegate key is
  in the range of the genesis delegation mapping, there is a
  \emph{DuplicateGenesisDelegate} failure.
\item In the case of insufficient reserves to pay the instantaneous rewards,
  there is a \emph{InsufficientForInstantaneousRewards} failure.
\end{itemize}

\clearpage

\subsection{Stake Pool Rules}
\label{sec:pool-rules}


The rules for updating the part of the ledger state defining the current stake
pools are given in \cref{fig:pool-rules}. The calculation of stake distribution
is described in Section~\ref{sec:stake-dist-calc}.

In the pool rules, the stake pool is identified with the hashkey of the pool operator.
For each rule, again, we first check that a given certificate $c$ is of the correct type.

\begin{itemize}
  \item Stake pool registration is handled by \cref{eq:pool-reg}.
    It is required that the pool not be currently registered.
    Registration causes the following state transformation:
    \begin{itemize}
      \item The key is added to the set of registered stake pools.
      \item The pool's parameters are stored.
      \item The pool's operational certificate counter is set to zero.
    \end{itemize}
  \item Stake pool parameter updates are handled by \cref{eq:pool-rereg}.
    This rule, which also matches on the certificate type $\type{DCertRegPool}$,
    is distinguished from \cref{eq:pool-reg} by the requirement that
    the pool be registered. This rule also ends stake pool retirements.
    Reregistration causes the following state transformation:
    \begin{itemize}
      \item The pool's parameters are updated.
      \item The pool is removed from the collection of retiring pools.
      \item Note that $\var{stpools}$ is \textbf{not} updated.
        The registration creation slot does does not change.
    \end{itemize}
  \item Stake pool retirements are handled by \cref{eq:pool-ret}.
    Given a slot number $\var{slot}$, the application of this rule requires that the
    planned retirement epoch $\var{e}$ stated in the certificate is in the future,
    i.e.~after $\var{cepoch}$ (the epoch of the current slot number in this context) and
    that it is less than $\emax$ epochs after the current one.
    It is also required that the pool be registered.
    Note that imposing the $\emax$ constraint on the system is not strictly necessary.
    However, forcing stake pools to announce their retirement a shorter time in
    advance will curb the growth of the $\var{retiring}$ list in the ledger state.

    The pools scheduled for retirement must be removed from
    the $\var{retiring}$ state variable at the end of the epoch they are scheduled
    to retire in. This non-signaled transition (triggered, instead, directly by a
    change of current slot number in the environment), along with all other transitions
    that take place at the epoch boundary, are described in Section~\ref{sec:epoch}.

    Reregistration causes the following state transformation:
    \begin{itemize}
      \item The pool is marked to retire on the given epoch.
        If it was previously retiring, the retirement epoch is now updated.
    \end{itemize}
\end{itemize}

%%
%% Figure - Pool Rules
%%
\begin{figure}[hbt]
  \begin{equation}\label{eq:pool-reg}
    \inference[Pool-Reg]
    {
      \var{c}\in\DCertRegPool
      & \var{hk} \leteq \cwitness{c}
      & hk \notin \dom \var{stpools}
    }
    {
      \begin{array}{r}
        \var{slot} \\
        \var{pp} \\
      \end{array}
      \vdash
      \left(
      \begin{array}{r}
        \var{stpools} \\
        \var{poolParams} \\
        \var{retiring} \\
        \var{cs} \\
      \end{array}
      \right)
      \trans{pool}{c}
      \left(
      \begin{array}{rcl}
        \varUpdate{\var{stpools}} & \varUpdate{\union}
                                  & \varUpdate{\{\var{hk} \mapsto \var{slot}\}} \\
        \varUpdate{\var{poolParams}} & \varUpdate{\union}
                                    & \varUpdate{\{\var{hk} \mapsto \poolParam{c}\}} \\
       \var{retiring} \\
       \varUpdate{\var{cs}} & \varUpdate{\union}
                            & \varUpdate{\{\var{hk} \mapsto 0\}} \\
      \end{array}
      \right)
    }
  \end{equation}

  \begin{equation}\label{eq:pool-rereg}
    \inference[Pool-reReg]
    {
      \var{c}\in\DCertRegPool
      & \var{hk} \leteq \cwitness{c}
      & hk \in \dom \var{stpools}
    }
    {
      \begin{array}{r}
        \var{slot} \\
        \var{pp} \\
      \end{array}
      \vdash
      \left(
      \begin{array}{r}
        \var{stpools} \\
        \var{poolParams} \\
        \var{retiring} \\
        \var{cs} \\
      \end{array}
      \right)
      \trans{pool}{c}
      \left(
      \begin{array}{rcl}
        \var{stpools} \\
        \varUpdate{\var{poolParams}} & \varUpdate{\unionoverrideRight}
                                  & \varUpdate{\{\var{hk} \mapsto \poolParam{c}\}}\\
        \varUpdate{\{\var{hk}\}} & \varUpdate{\subtractdom} & \varUpdate{\var{retiring}} \\
        \var{cs} \\
      \end{array}
      \right)
    }
  \end{equation}

  \begin{equation}\label{eq:pool-ret}
    \inference[Pool-Retire]
    {
    \var{c} \in \DCertRetirePool
    & hk \leteq \cwitness{c}
    & \var{hk} \in \dom \var{stpools} \\
    \var{e} \leteq \retire{c}
    & \var{cepoch} \leteq \epoch{slot}
    & \var{cepoch} < \var{e} < \var{cepoch} + (\fun{emax}~{pp})
  }
  {
    \begin{array}{r}
      \var{slot} \\
      \var{pp} \\
    \end{array}
    \vdash
    \left(
      \begin{array}{r}
        \var{stpools} \\
        \var{poolParams} \\
        \var{retiring} \\
        \var{cs} \\
      \end{array}
    \right)
    \trans{pool}{c}
    \left(
      \begin{array}{rcl}
        \var{stpools} \\
        \var{poolParams} \\
        \varUpdate{\var{retiring}} & \varUpdate{\unionoverrideRight}
                                   & \varUpdate{\{\var{hk} \mapsto \var{e}\}} \\
        \var{cs} \\
      \end{array}
    \right)
  }
  \end{equation}

  \caption{Pool Inference Rule}
  \label{fig:pool-rules}

\end{figure}

The POOL rule has three predicate failures:
\begin{itemize}
\item In the case of a pool retirement certificate, if the pool key is not in
  the domain of the stake pools mapping, there is a
  \emph{StakePoolNotRegisteredOnKey} failure.
\item In the case of a pool retirement certificate, if the retirement epoch is
  not between the current epoch and the relative maximal epoch from the current
  epoch, there is a \emph{StakePoolRetirementWrongEpoch} failure.
\item If the delegation certificate is not of one of the pool types, there is a
  \emph{WrongCertificateType} failure.
\end{itemize}

\clearpage

\subsection{Delegation and Pool Combined Rules}
\label{sec:del-pool-rules}

We now combine the delegation and pool transition systems.
Figure~\ref{fig:defs:delpl} gives the state, environment and transition type for the
combined transition.

\begin{figure}[hbt]
  \emph{Delegation and Pool Combined Environment}
  \begin{equation*}
    \DPEnv =
    \left(
      \begin{array}{r@{~\in~}lr}
        \var{slot} & \Slot & \text{slot}\\
        \var{ptr} & \Ptr & \text{certificate pointer}\\
        \var{pp} & \PParams & \text{protocol parameters}\\
        \var{reserves} & \Coin & \text{total available reserves}
      \end{array}
    \right)
  \end{equation*}
  %
  \emph{Delegation and Pool Combined State}
  \begin{equation*}
    \DPState =
    \left(
      \begin{array}{r@{~\in~}lr}
        \var{dstate} & \DState & \text{delegation state}\\
        \var{pstate} & \PState & \text{pool state}\\
      \end{array}
    \right)
  \end{equation*}
  %
  \emph{Delegation and Pool Combined Transition}
  \begin{equation*}
    \_ \vdash \_ \trans{delpl}{\_} \_ \in
      \powerset (
        \DPEnv \times \DPState \times \DCert \times \DPState)
  \end{equation*}
  \caption{Delegation and Pool Combined Transition Type}
  \label{fig:defs:delpl}
\end{figure}

\clearpage

Figure~\ref{fig:rules:delpl}, gives the rules for the combined transition.
Note that for any given certificate, at most one of the two rules
(\cref{eq:delpl-d} and \cref{eq:delpl-p})
will be successful, since the pool certificates are disjoint from the delegation certificates.

\begin{figure}[hbt]
  \emph{Delegation and Pool Combined Rules}
  \begin{equation}
    \label{eq:delpl-d}
    \inference[Delpl-Del]
    {
      &
      {
        \begin{array}{r}
          \var{slot} \\
          \var{ptr} \\
          \var{reserves}
        \end{array}
      }
      \vdash \var{dstate} \trans{\hyperref[fig:delegation-rules]{deleg}}{c} \var{dstate'}
    }
    {
      \begin{array}{r}
        \var{slot} \\
        \var{ptr} \\
        \var{pp} \\
        \var{reserves}
      \end{array}
      \vdash
      \left(
      \begin{array}{r}
        \var{dstate} \\
        \var{pstate}
      \end{array}
      \right)
      \trans{delpl}{c}
      \left(
      \begin{array}{rcl}
        \varUpdate{\var{dstate'}} \\
        \var{pstate}
      \end{array}
      \right)
    }
  \end{equation}
  \begin{equation}
    \label{eq:delpl-p}
    \inference[Delpl-Pool]
    {
    &
    {
      \begin{array}{r}
        \var{slot} \\
        \var{pp} \\
      \end{array}
    }
    \vdash \var{pstate} \trans{\hyperref[fig:pool-rules]{pool}}{c} \var{pstate'}
    }
    {
      \begin{array}{r}
        \var{slot} \\
        \var{ptr} \\
        \var{pp} \\
        \var{reserves}
      \end{array}
      \vdash
      \left(
      \begin{array}{r}
        \var{dstate} \\
        \var{pstate}
      \end{array}
      \right)
      \trans{delpl}{c}
      \left(
      \begin{array}{rcl}
        \var{dstate} \\
        \varUpdate{\var{pstate'}}
      \end{array}
      \right)
    }
  \end{equation}
  \caption{Delegation and Pool Combined Transition Rules}
  \label{fig:rules:delpl}
\end{figure}

We now describe a transition system that processes the list of certificates inside a transaction.
It is defined recursively from the transition system in Figure~\ref{fig:rules:delpl} above.

Figure~\ref{fig:type:delegations} defines the types for the delegation certificate sequence
transition.

\begin{figure}[hbt]
  \emph{Certificate Sequence Environment}
  \begin{equation*}
    \DPSEnv =
    \left(
      \begin{array}{r@{~\in~}lr}
        \var{slot} & \Slot & \text{slot}\\
        \var{txIx} & \Ix & \text{transaction index}\\
        \var{pp} & \PParams & \text{protocol parameters}\\
        \var{tx} & \Tx & \text{transaction} \\
        \var{reserves} & \Coin & \text{total reserves}
      \end{array}
    \right)
  \end{equation*}
  %
  \begin{equation*}
    \_ \vdash \_ \trans{delegs}{\_} \_ \in
    \powerset (
    \DPSEnv \times \DPState \times \seqof{\DCert} \times \DPState)
  \end{equation*}
  \caption{Delegation sequence transition type}
  \label{fig:type:delegations}
\end{figure}

Figure~\ref{fig:rules:delegation-sequence} defines the transition system recursively.
This definition guarantees that a certificate list (and therefore, the transaction carrying it)
cannot be processed unless every certificate in it is valid. For example, if a transaction is
carrying a certificate that schedules a pool retirement in a past epoch, the whole transaction
will be invalid.

\begin{itemize}
\item The base case, when the list is empty, is captured by
  \cref{eq:delegs-base}.  In the base case we address one final accounting
  detail not yet covered by the UTxO transition, namely setting the reward
  account balance to zero for any account that made a withdrawal.  There is
  therefore a precondition that all withdrawals are correct, where correct means
  that there is a reward account for each stake credential and that the balance
  matches that of the reward being withdrawn.  The base case triggers the
  following state transformation:
    \begin{itemize}
      \item Reward accounts are set to zero for each corresponding withdrawal.
    \end{itemize}
  \item The inductive case, when the list is non-empty, is captured by \cref{eq:delegs-induct}.
    It constructs a certificate pointer given the current slot and transaction index,
    calls $\mathsf{DELPL}$ on the next certificate in the list and inductively
    calls $\mathsf{DELEGS}$ on the rest of the list.
    The inductive case triggers the following state transformation:
    \begin{itemize}
      \item The delegation and pool states are (inductively) updated by the results of
        $\mathsf{DELEGS}$, which is then updated according to $\mathsf{DELPL}$.
    \end{itemize}
\end{itemize}

\begin{figure}[hbt]
  \begin{equation}
    \label{eq:delegs-base}
    \inference[Seq-delg-base]
    {
      \var{wdrls} \leteq \txwdrls{tx}
      &
      \var{wdrls} \subseteq \var{rewards}
      \\
      \var{rewards'} \leteq \var{rewards} \unionoverrideRight \{(w, 0) \mid w \in \dom \var{wdrls}\}
    }
    {
      \begin{array}{c}
        \var{slot} \\
        \var{txIx} \\
        \var{pp} \\
        \var{tx} \\
        \var{reserves}
      \end{array}
      \vdash
      \left(
      \begin{array}{c}
        \left(
        \begin{array}{r}
          \var{stdelegs} \\
          \var{rewards} \\
          \var{delegations} \\
          \var{ptrs} \\
          \var{fdms} \\
          \var{dms} \\
          \var{i_{rwd}}
        \end{array}
        \right) \\~\\
        \left(
        \begin{array}{c}
          \var{stpools} \\
          \var{poolParams} \\
          \var{retiring} \\
          \var{cs} \\
        \end{array}
        \right) \\
      \end{array}
      \right)
      \trans{delegs}{\epsilon}
      \left(
      \begin{array}{c}
        \left(
        \begin{array}{c}
          \var{stdelegs} \\
          \varUpdate{\var{rewards'}} \\
          \var{delegations} \\
          \var{ptrs} \\
          \var{fdms} \\
          \var{dms} \\
          \var{i_{rwd}}
        \end{array}
        \right) \\~\\
        \left(
        \begin{array}{c}
          \var{stpools} \\
          \var{poolParams} \\
          \var{retiring} \\
          \var{cs} \\
        \end{array}
        \right) \\
      \end{array}
      \right)
    }
  \end{equation}

  \nextdef

  \begin{equation}
    \label{eq:delegs-induct}
    \inference[Seq-delg-ind]
    {
      \var{c}\in\DCertDeleg \Rightarrow \fun{dpool}~{c} \in \dom \var{stpools} \\
      ptr \leteq (\var{slot},~\var{txIx},~\mathsf{len}~\Gamma) \\~\\
        {
          \begin{array}{c}
            \var{slot}\\
            \var{txIx}\\
            \var{pp}\\
            \var{tx}\\
            \var{reserves}
          \end{array}
        }
      \vdash
      \var{dpstate}
      \trans{delegs}{\Gamma}
      \var{dpstate'}
    \\~\\~\\
    {
      \begin{array}{c}
        \var{slot}\\
        \var{ptr}\\
        \var{pp}\\
        \var{reserves}
      \end{array}
    }
    \vdash
      \var{dpstate'}
      \trans{\hyperref[fig:rules:delpl]{delpl}}{c}
      \var{dpstate''}
    }
    {
    {
      \begin{array}{c}
        \var{slot}\\
        \var{txIx}\\
        \var{pp}\\
        \var{tx}\\
        \var{reserves}
      \end{array}
    }
    \vdash
      \var{dpstate}
      \trans{delegs}{\Gamma; c}
      \varUpdate{\var{dpstate''}}
    }
  \end{equation}
  \caption{Delegation sequence rules}
  \label{fig:rules:delegation-sequence}
\end{figure}

The DELEGS rule has two predicate failures:
\begin{itemize}
\item In the case of a key delegation certificate, if the pool key is not
  registered, there is a \emph{DelegateeNotRegistered} failure.
\item If the withdrawals mapping of the transaction is not a subset of the
  rewards mapping of the delegation state, there is a
  \emph{WithdrawalsNotInRewards} failure.
\end{itemize}

\clearpage

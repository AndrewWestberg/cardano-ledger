\section{Bootstrap Witnesses}
\label{sec:bootstrap-witnesses}

\subsection{Bootstrap Witnesses CBOR Specification}

In the Byron era of Cardano, public keys are transmitted
on chain as extended public keys as specified in \cite{bip32}.
The Shelley era of Cardano does not use extended public keys,
but does use the same cryptographic signing algorithm,
namely Ed25519.

The extended public key consists of 64 bytes,
the first 32 of which corresponds to the public key,
the second 32 of which correpsonds to something called the chain code:

$$\mathsf{extended\_pubkey} = \mathsf{pubkey}|\mathsf{chain\_code}$$

The chaincode is required for constructing bootstrap witnesses.

The CBOR specification of bootstrap witnesses,
named $\mathsf{bootstrap\_witness}$,
is given in
\url{https://github.com/input-output-hk/cardano-ledger-specs/tree/master/shelley/chain-and-ledger/executable-spec/cddl-files}.

In the Shelley era, only pubkey address are supported,
and are named bootstrap addresses.

The bootstrap witness consists of the public key, the signature,
the chain code, and the address attributes.
As explained above, the public key and the signature format
are the same as for all other witnesses in the Shelley era.
The address attributes has the same format as from the Byron era address,
as specified by $\mathsf{addrattr}$ in
\url{https://github.com/input-output-hk/cardano-ledger-specs/blob/master/byron/cddl-spec/byron.cddl}.


\subsection{Bootstrap Address ID Recovery}

The bootstrap address ID, named $\mathsf{addressid}$ in the Byron CDDL
specification above, can be computed as follows.

First construct the following byte string:

$$\mathsf{b} =
  \mathsf{0x830082005840}
  | \mathsf{pubkey\_bytes}
  | \mathsf{chain\_code\_bytes}
  | \mathsf{attributes\_bytes}
$$

The address ID is then obtained by first applying
sha3 256 to $\mathsf{b}$ and then applying blake2b44 to the result.
$$\mathsf{address\_id} =
  \mathsf{blake2b44}(\mathsf{sha3\_256}~\mathsf{b})
$$

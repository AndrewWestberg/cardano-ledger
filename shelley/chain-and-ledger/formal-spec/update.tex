\section{Update Proposal Mechanism}
\label{sec:update}


The $\mathsf{UPDATE}$ transition is responsible for the federated governance model in Shelley.
The governance process includes a mechanism for core nodes to propose and vote on
updates. In this chapter we
outline rules for genesis keys \textit{proposing} both protocol parameter
and application version updates, as well as voting on whether a particular
software update is an acceptable future option.
For rules regarding the \textit{adoption} of protocol
parameter updates, see \ref{sec:pparam-update}. For rules regarding
adoption of new software versions see \ref{sec:software-updates}.

This chapter does not discuss authentication of update proposals.
The signature for the keys in the proposal will be checked in the
$\mathsf{UTXOW}$ transition, which checks all the necessary witnesses
for a transaction, see \ref{sec:witnesses-shelley}.

\textbf{Genesis Key Delegations.} The environment for both protocol parameter
and application version updates contains
the value $\var{genDelegs}$, which is a finite map indexed by genesis key hashes.
This is the genesis key delegations. During the Byron era, they are all
already delegated to some $\KeyHash$, and these delegations are inherited
through the Byron-Shelley transition (see \ref{sec:byron-to-shelley}).
These delegations can be updated as described in \ref{sec:delegation-shelley},
but there is no mechanism for them to un-delegate or for the keys to which they delegate
to retire (unlike regular stake pools).

Because the protocol parameter and application version update mechanisms are different, the
update proposals for these are handled by separate sets of rules. The main
difference is that the current application version on the ledger is updated
several slots ($\SlotsPrior$-slots) after concensus is reached, whereas protocol
parameter updates must happen at an epoch boundary.
For transparency in both voting mechanisms, the votes and proposals for updates
are stored
on the ledger in addition to the currently accepted protocol parameters and
application versions.

\subsection{Descriptions of the Data}
\label{sec:update-types}

The types were defined in Figure~\ref{fig:defs:utxo-shelley}.
In the derived types, $\Metadata$ and $\Applications$ are values
that contain versions of current and next versions of applications.
The $\ApName$ uniquely identifies a specific kind of application (e.g.
the wallet), and the associated $\ApVer$ is the specific version of that
application. The associated $\Metadata$ value gives the \textit{next possible}
versions of the given application. It is a mapping of system tags to hashes of
installer binaries, and is needed for the update mechanism.
The update proposal type $\Update$ is a pair of $\PPUpdate$ and $\AVUpdate$.
$\PPUpdate$ allows for changing protocol parameters,
and $\AVUpdate$ allows for updating the software versions.

Note that both of these finite maps are indexed by the hashes of the keys of
entities proposing the given updates, $\KeyHashGen$.
We use the abstract type $\KeyHashGen$ to represent hashes of genesis
(public verification) keys, which have type $\VKeyGen$.
Genesis keys are the keys belonging to the federated
nodes running the Cardano system currently (also referred to as core nodes).
The the regular user verification keys are of a type $\VKey$, distinct from the
genesis key type, $\VKeyGen$. Similarly, the type hashes of these
are distinct, $\KeyHash$ and $\KeyHashGen$ respectively.

Currently, updates can only be proposed and voted on by the owners of the genesis keys.
The process of decentralization will result in the core nodes gradually giving up
some of their priviledges and responsibilities to the network,
eventually give them \textit{all} up.
The update proposal mechanism will not be decentralization in the Shelley era, however.
For more on the decentralization process, see \ref{sec:new-epoch-trans}.

\subsection{Protocol Parameter Update Proposals}
\label{sec:pp-proposals}

The transition type $\mathsf{PPUP}$ is for proposing updates to protocol
parameters, see Figure \ref{fig:ts-types:pp-update} (for the corresponding rules,
see Figure \ref{fig:rules:pp-update}).
The signal for this transition is a finite map of proposed updates (indexed by
genesis keys). The purpose of the $\Seed$ type in the $\PPUpdate$ derived type is explained
in the protocol parameter adoption rules.

\begin{itemize}
  \item PP-Update-Empty : no new updates were proposed, do nothing
  \item PP-Update-Nonempty : some new updates $\var{pup}$ were proposed
  Add these to
  the existing proposals using a right override. That is, if a genesis key
  has previously submitted an update proposal, replace it with its new
  proposal in $\var{pup}$.
\end{itemize}

This rule has the following predicate failures:

\begin{enumerate}
\item In the case of \var{slot} being smaller than
  $\fun{firstSlot}~((\epoch{slot}) + 1) - \fun{SlotsPrior}$, there is
  a \emph{PPUpdateTooEarly} failure.
\item In the case of \var{pup} being non-empty, if the check $\dom pup \subseteq
  \dom genDelegs$ fails, there is a \emph{NonGenesisUpdate} failure as only genesis keys
  can be used in the protocol parameter update.
\item If a protocol parameter update in \var{pup} cannot follow the current
  protocol parameter, there is a \emph{PVCannotFollow} failure.
\end{enumerate}

Note that $\fun{pvCanFollow}$
was defined in \cite{byron_ledger_spec}. The constant $\SlotsPrior$
is used in defining the blockchain layer functionality (specifically,
properties of header validation),
and is explained in detail in the paper \cite{byron_chain_spec}.
It is used in this document in
Section \ref{sec:chain}.
Because of the header validation requirements, a protocol parameter proposal
must be submitted with at least $\SlotsPrior$-slots before the end of an epoch,
and will be rejected otherwise.

\begin{figure}[htb]
  \emph{Protocol Parameter Update environment}
  \begin{equation*}
    \PPUpdateEnv =
    \left(
      \begin{array}{r@{~\in~}lr}
        \var{slot} & \Slot & \text{current slot}\\
        \var{pp} & \PParams & \text{protocol parameters}\\
        \var{genDelegs} & \KeyHashGen\mapsto\KeyHash & \text{genesis key delegations} \\
      \end{array}
    \right)
  \end{equation*}
  %
  \emph{Protocol Parameter Update transitions}
  \begin{equation*}
    \_ \vdash
    \var{\_} \trans{ppup}{\_} \var{\_}
    \subseteq \powerset (\PPUpdateEnv \times \PPUpdate \times \PPUpdate \times \PPUpdate)
  \end{equation*}
  %
  \caption{Protocol Parameter Update Transition System Types}
  \label{fig:ts-types:pp-update}
\end{figure}

\begin{figure}[htb]
  \begin{equation}\label{eq:pp-update-Empty}
    \inference[PP-Update-Empty]
    {
      \var{pup} = \emptyset
    }
    {
      \begin{array}{r}
        \var{slot}\\
        \var{pp}\\
        \var{genDelegs}\\
      \end{array}
      \vdash \var{pup_s}\trans{ppup}{pup}\var{pup_s}
    }
  \end{equation}

  \nextdef

  \begin{equation}\label{eq:update-nonempty}
    \inference[PP-Update-Nonempty]
    {
      \var{pup}\neq\emptyset
      &
      \dom{pup}\subseteq\dom{genDelegs}
      \\
      \forall\var{ps}\in\range{pup},~
        \var{pv}\mapsto\var{v}\in\var{ps}\implies\fun{pvCanFollow}~(\fun{pv}~\var{pp})~\var{v}
      \\
      \var{slot} < \fun{firstSlot}~((\epoch{slot}) + 1) - \SlotsPrior
    }
    {
      \begin{array}{r}
        \var{slot}\\
        \var{pp}\\
        \var{genDelegs}\\
      \end{array}
      \vdash
      \var{pup_s}
      \trans{ppup}{pup}
      \varUpdate{pup_s\unionoverrideRight pup}
    }
  \end{equation}

  \caption{Protocol Parameter Update Inference Rules}
  \label{fig:rules:pp-update}
\end{figure}

\clearpage

Figure~\ref{fig:funcs:helper-updates} gives some helper functions for the
application version update transition.
The function $\fun{votedValue_T}$ returns
the consensus value of update proposals in the event that at least five
genesis keys agree (see \cite{delegation_design}). Note that this is greater
than majority (there are seven core nodes), so there can
be \text{at most one value} with five or more votes.
This function will also be used for the protocol parameters later in Section~\ref{sec:epoch}.
Note that $\type{T}$ is an arbitrary type. The specific choice of this type
determines what is being voted on, e.g. the application version.
Recall here that $\type{T}^?$ is the option type, see \ref{sec:notation-shelley},
and a term of this type can have a value of type $\type{T}$ or no value.

The function $\fun{validAV}$ uses three functions from \cite{byron_ledger_spec}, namely
$\fun{apNameValid}$, $\fun{svCanFollow}$, and $\fun{sTagValid}$. It determines
whether an application version is valid, given its name, version number,
metadata, and a set of applications to compare against.
The function $\fun{newAVs}$ adds the most recent valid application
versions to a finite map of applications using right override.
This helper function will be used in the ledger update.

\textbf{Note that} $\fun{votedValue}$
\textbf{is only well-defined if } $\Quorum$
\textbf{is greater than half the number of core nodes, i.e.}
$\Quorum > |\var{genDelegs}|/2$ \textbf{.}
%%
%% Figure - Helper Function for Consensus of Update Proposals
%%
\begin{figure}[htb]
  \begin{align*}
      & \fun{votedValue_T} \in (\KeyHashGen\mapsto\type{T}) \to \type{T}^?\\
      & \fun{votedValue_T}~\var{vs} =
        \begin{cases}
          t & \exists t\in\range{vs}~(|vs\restrictrange t|\geq \Quorum) \\
          \Nothing & \text{otherwise} \\
        \end{cases}
  \end{align*}

  \begin{align*}
      & \fun{maxVer} \in \ApName \to \Applications \to (\Slot\mapsto\Applications)
        \to (\ApVer, \Metadata)\\
      & \fun{maxVer}~\var{an}~\var{avs}~\var{favs} = \\
      & ~~~~\fun{max_{first}} \Bigg(
        \Big\{(0, \emptyset)\Big\}
        \cup
        \range{avs}
        \cup
        \bigcup_{\var{fav}\in\var{favs}}
        \range{favs}
        \Bigg)
  \end{align*}

  \begin{align*}
      & \fun{svCanFollow} \in \Applications \to (\Slot\mapsto\Applications)
        \to \ApName \to \ApVer \to \Bool\\
      & \fun{svCanFollow}~\var{avs}~\var{favs}~\var{an}~\var{av}= av = 1 + v' \\
      & ~~~~\where (v',~\wcard) = \fun{maxVer}~\var{an}~\var{avs}~\var{favs}
  \end{align*}

  \begin{align*}
      & \fun{newAVs} \in \Applications \to (\Slot\mapsto\Applications) \to \Applications \\
      & \fun{newAVs}~\var{avs}~\var{favs} = \\
      & ~~~~\Bigg\{\var{an}\mapsto\fun{maxVer}~\var{an}~\var{avs}~\var{favs}
            ~\mid~
            \var{an}\in\Big(\dom{avs}\cup\bigcup_{\var{fav}\in\var{favs}}\dom{fav}\Big)
            \Bigg\}
  \end{align*}

  \caption{Epoch Helper Functions}
  \label{fig:funcs:helper-updates}
\end{figure}

\subsection{Application Version Update Proposals}
\label{sec:app-proposals}

The application (or software) version update is defined by the $\mathsf{AVUP}$
transition type (see Figure \ref{fig:ts-types:av-update}), and the rules in
Figure \ref{fig:rules:av-update}.
The state contains a finite map of update proposals
$\var{aup} \in \AVUpdate$ ($= \VKeyGen \mapsto \Applications$) that the genesis keys have
made previously.
The \textit{voting} and \textit{proposing an application version update} is done through the same
mechanism - the $\var{aup}$ signal carried by a transaction. The difference
is that a proposal is not in the
range of the $\var{aup}$, but a vote is for an existing proposal
in $\var{aup}$.

Recall here that a single application version update proposal can be
proposing changes to multiple parts of the Cardano system software simultaneously.
This is why it is stored as the finite map $\Applications$, associating the
specific bit of software (referred to by name) with the proposed version.
A proposal for several versions of the same application cannot be made in
a single transaction by the same genesis key.

The ledger state variable $\var{avs}$ stores the current application version
that the users \textit{should} be running.
Now, the state variable $\var{favs}$ contains update proposals that
were voted on by at least five genesis keys.
The particular update $\var{favs} = s \mapsto ap$ will become
the current application version at slot $s$.
Updating the current application version is part of the rules in the ledger transitionn,
see Section \ref{sec:ledger-trans}.

At most one application version update
can be selected for a given slot, and some slots do not have any
updates selected by voting - either because none were proposed, or the vote
did not reach a concensus. Finally, the $\var{avs}$ variable stores
the versions of the software that are currently in use. The update
proposal rules are as follows:

\begin{itemize}
  \item AV-Update-Empty: rule represents an update with
and empty set of application version proposals, when nothing happens.
  \item AV-Update-No-Consensus: rule
is for the case where a voting consensus was not reached on a proposed software
update after combining the set of new (signal) and existing (state) proposals
by a right override. In this case, we simply use this combined new value
of update proposal (votes) as the state UP value.
  \item AV-Update-Consensus: applies when voting on update proposals by genesis keys
reaches consensus. This update is stored associated with a slot which is
$\SlotsPrior$ in the future from the current one, because this is when
this update will become the current AV.
All other update proposals are scrapped.
\end{itemize}

The AVUP rule has three predicate failures:
\begin{itemize}
\item In the case of \var{aup} being non-empty, if the check $\dom aup \subseteq
  \dom genDelegs$ fails, there is a \emph{NonGenesisUpdate} failure as only genesis keys
  can be used in the application version update.
\item In the case of \var{aup} being non-empty, if any of the application names
  in the proposal are invalid, there is a \emph{InvalidName} failure.
\item In the case of \var{aup} being non-empty, if any of the application versions
  in the proposal are not exactly one more than the greatest version for the given
  application name in the current application versions or the future application
  versions, there is a \emph{CannotFollow} failure.
\item In the case of \var{aup} being non-empty, if any of the metadata in
  in the proposal contains an invalid system tag, there is a \emph{InvalidSystemTags} failure.
\end{itemize}

\begin{figure}[htb]
  \emph{Application Version Update environment}
  \begin{equation*}
    \AVUpdateEnv =
    \left(
      \begin{array}{r@{~\in~}lr}
        \var{slot} & \Slot & \text{current slot}\\
        \var{genDelegs} & \KeyHashGen\mapsto\KeyHash & \text{genesis key delegations} \\
      \end{array}
    \right)
  \end{equation*}
  %
  \emph{Application Version Update states}
  \begin{equation*}
    \AVUpdateState =
    \left(
      \begin{array}{r@{~\in~}lr}
        \var{aup} & \AVUpdate & \text{application versions proposals} \\
        \var{favs} & \Slot\mapsto\Applications & \text{future application versions} \\
        \var{avs} & \Applications & \text{current application versions} \\
      \end{array}
    \right)
  \end{equation*}
  %
  \emph{Application Version Update transitions}
  \begin{equation*}
    \_ \vdash
    \var{\_} \trans{avup}{\_} \var{\_}
    \subseteq \powerset (\AVUpdateEnv \times \AVUpdateState \times \AVUpdate \times \AVUpdateState)
  \end{equation*}
  %
  \caption{Application Version Update transition-system types}
  \label{fig:ts-types:av-update}
\end{figure}

\begin{figure}[htb]
  \begin{equation}\label{eq:av-update-Empty}
    \inference[AV-Update-Empty]
    {
      \var{aup} = \emptyset
    }
    {
      \begin{array}{l}
        \var{slot}\\
        \var{genDelegs}\\
      \end{array}
      \vdash
      \left(
      \begin{array}{l}
        \var{aup_s}\\
        \var{favs}\\
        \var{avs}\\
      \end{array}
      \right)
      \trans{avup}{aup}
      \left(
      \begin{array}{l}
        \var{aup_s}\\
        \var{favs}\\
        \var{avs}\\
      \end{array}
      \right)
    }
  \end{equation}

  \nextdef

  \begin{equation}\label{eq:update-no-consensus}
    \inference[AV-Update-No-Consensus]
    {
      \var{aup}\neq\emptyset
      &
      \dom{\var{aup}}\subseteq\dom{\var{genDelegs}}
      \\
      \forall \wcard\mapsto\var{vote}\in\var{aup},\forall n\in\dom{vote},~
        \fun{apNameValid}~\var{v}
      \\
      \forall \wcard\mapsto\var{vote}\in\var{aup},\forall n\mapsto(v,~\wcard)\in\var{vote},~
      \fun{svCanFollow}~\var{avs}~\var{favs}~\var{n}~\var{v}
      \\
      \forall \wcard\mapsto\var{vote}\in\var{aup},\forall \wcard\mapsto(\wcard,~m)\in\var{vote},~
      \forall \var{st}\mapsto\wcard\in\var{m},~ \fun{sTagValid}~\var{st}
      \\
      \var{aup'}\leteq\var{aup_s}\unionoverrideRight\var{aup}
      \\
      \var{fav}\leteq\fun{votedValue_{Applications}}~\var{aup'}
      &
      \var{fav}=\Nothing
    }
    {
      \begin{array}{l}
        \var{slot}\\
        \var{genDelegs}\\
      \end{array}
      \vdash
      \left(
      \begin{array}{l}
        \var{aup_s}\\
        \var{favs}\\
        \var{avs}\\
      \end{array}
      \right)
      \trans{avup}{aup}
      \left(
      \begin{array}{l}
        \varUpdate{\var{aup'}}\\
        \var{favs}\\
        \var{avs}\\
      \end{array}
      \right)
    }
  \end{equation}

  \nextdef

  \begin{equation}\label{eq:update-consensus}
    \inference[AV-Update-Consensus]
    {
      \var{aup}\neq\emptyset
      &
      \dom{\var{aup}}\subseteq\dom{\var{genDelegs}}
      \\
      \forall \wcard\mapsto\var{vote}\in\var{aup},\forall n\in\dom{vote},~
        \fun{apNameValid}~\var{v}
      \\
      \forall \wcard\mapsto\var{vote}\in\var{aup},\forall n\mapsto(v,~\wcard)\in\var{vote},~
        \fun{svCanFollow}~\var{avs}~\var{favs}~\var{n}~\var{v}
      \\
      \forall \wcard\mapsto\var{vote}\in\var{aup},\forall \wcard\mapsto(\wcard,~m)\in\var{vote},~
      \forall \var{st}\mapsto\wcard\in\var{m},~ \fun{sTagValid}~\var{st}
      \\
      \var{aup'}\leteq\var{aup_s}\unionoverrideRight\var{aup}
      \\
      \var{fav}\leteq\fun{votedValue_{Applications}}~\var{aup'}
      &
      \var{fav}\neq\Nothing
      \\
      s\leteq\var{slot}+\SlotsPrior
    }
    {
      \begin{array}{l}
        \var{slot}\\
        \var{genDelegs}\\
      \end{array}
      \vdash
      \left(
      \begin{array}{l}
        \var{aup_s}\\
        \var{favs}\\
        \var{avs}\\
      \end{array}
      \right)
      \trans{avup}{aup}
      \left(
      \begin{array}{l}
        \varUpdate{\emptyset}\\
        \varUpdate{\var{favs}\unionoverrideRight\{\var{s}\mapsto\var{fav}\}}\\
        \var{avs}\\
      \end{array}
      \right)
    }
  \end{equation}

  \caption{Application Version Update inference rules}
  \label{fig:rules:av-update}
\end{figure}
\clearpage

The $\mathsf{UP}$ transition type combines processing update
proposals for protocol parameters and applications. The signal for this
transition is a pair of update proposals for parameters and applications,
and each of these is a signal for updating the corresponding data structure.
Both get updated simultaneously by this rule according to a relevant
update rules for each.

\begin{figure}[htb]
  \emph{Update environment}
  \begin{equation*}
    \UpdateEnv =
    \left(
      \begin{array}{r@{~\in~}lr}
        \var{slot} & \Slot & \text{current slot}\\
        \var{pp} & \PParams & \text{protocol parameters}\\
        \var{genDelegs} & \KeyHashGen\mapsto\KeyHash & \text{genesis key delegations} \\
      \end{array}
    \right)
  \end{equation*}
  %
  \emph{Update states}
  \begin{equation*}
    \UpdateState =
    \left(
      \begin{array}{r@{~\in~}lr}
        \var{pup} & \PPUpdate & \text{protocol parameter proposals} \\
        \var{aup} & \AVUpdate & \text{application versions proposals} \\
        \var{favs} & \Slot\mapsto\Applications & \text{future application versions} \\
        \var{avs} & \Applications & \text{current application versions} \\
      \end{array}
    \right)
  \end{equation*}
  %
  \emph{Update transitions}
  \begin{equation*}
    \_ \vdash
    \var{\_} \trans{up}{\_} \var{\_}
    \subseteq \powerset (\UpdateEnv \times \UpdateState \times \Update \times \UpdateState)
  \end{equation*}
  %
  \caption{Application Version Update transition-system types}
  \label{fig:ts-types:update}
\end{figure}

\begin{figure}[htb]
  \begin{equation}\label{eq:update}
    \inference[Update]
    {
      (\var{pup_{u}},~\var{aup_{u}})\leteq\var{up}
      \\~\\
      {
        \begin{array}{r}
          \var{slot} \\
          \var{pp} \\
          \var{genDelegs} \\
        \end{array}
      }
      \vdash
      \left(\var{pup}\right)
      \trans{\hyperref[fig:rules:pp-update]{ppup}}{\var{pup_{u}}}
      \left(\var{pup'}\right)
      &
      {
        \begin{array}{r}
          \var{slot} \\
          \var{genDelegs} \\
        \end{array}
      }
      \vdash
      {
        \left(
          \begin{array}{r}
            \var{aup}\\
            \var{favs}\\
            \var{avs}\\
          \end{array}
        \right)
      }
      \trans{\hyperref[fig:rules:av-update]{avup}}{\var{aup_{u}}}
      {
        \left(
          \begin{array}{r}
            \var{aup'}\\
            \var{favs'}\\
            \var{avs'}\\
          \end{array}
        \right)
      }
    }
    {
      \begin{array}{r}
        \var{slot}\\
        \var{pp} \\
        \var{genDelegs}\\
      \end{array}
      \vdash
      \left(
      \begin{array}{l}
        \var{pup}\\
        \var{aup}\\
        \var{favs}\\
        \var{avs}\\
      \end{array}
      \right)
      \trans{up}{up}
      \left(
      \begin{array}{l}
        \varUpdate{\var{pup}'} \\
        \varUpdate{\var{aup}'} \\
        \varUpdate{\var{favs}'} \\
        \varUpdate{\var{avs}'} \\
      \end{array}
      \right)
    }
  \end{equation}

  \caption{Update inference rules}
  \label{fig:rules:update}
\end{figure}

%%% Local Variables:
%%% mode: latex
%%% TeX-master: t
%%% End:

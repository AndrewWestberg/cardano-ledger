\section{Update Proposal Mechanism}
\label{sec:update}


The $\mathsf{UPDATE}$ transition is responsible for the federated governance model in Shelley.
The governance process includes a mechanism for core nodes to propose and vote on
protocol parameter updates. In this chapter we
outline rules for genesis keys \textit{proposing} protocol parameter updates.
For rules regarding the \textit{adoption} of protocol
parameter updates, see Section~\ref{sec:pparam-update}.

This chapter does not discuss authentication of update proposals.
The signature for the keys in the proposal will be checked in the
$\mathsf{UTXOW}$ transition, which checks all the necessary witnesses
for a transaction, see Section\ref{sec:witnesses-shelley}.

\textbf{Genesis Key Delegations.} The environment for the protocol parameter
update transition contains the value $\var{genDelegs}$,
which is a finite map indexed by genesis key hashes,
and which maps to a pair consisting of a delegate key hash
(corresponding to the cold key used for producing blocks) and
a VRF key hash.

During the Byron era, the genesis nodes are all
already delegated to some $\KeyHash$, and these delegations are inherited
through the Byron-Shelley transition (see Section~\ref{sec:byron-to-shelley}).
The VRF key hashes in this mapping will be new to the Shelley era.

The delegations mapping can be updated as described in
Section~\ref{sec:delegation-shelley},
but there is no mechanism for them to un-delegate or for the keys to which they delegate
to retire (unlike regular stake pools).

The types $\ProposedPPUpdates$ and $\Update$ were defined in
Figure~\ref{fig:defs:utxo-shelley}.
The update proposal type $\Update$ is a pair of $\ProposedPPUpdates$ and $\Epoch$.
The epoch in the update specifies the epoch in which the proposal is valid.
$\ProposedPPUpdates$ is a finite maps which is indexed by the hashes of the keys of
entities proposing the given updates, $\KeyHashGen$.
We use the abstract type $\KeyHashGen$ to represent hashes of genesis
(public verification) keys, which have type $\VKeyGen$.
Genesis keys are the keys belonging to the federated
nodes running the Cardano system currently (also referred to as core nodes).
The the regular user verification keys are of a type $\VKey$, distinct from the
genesis key type, $\VKeyGen$. Similarly, the type hashes of these
are distinct, $\KeyHash$ and $\KeyHashGen$ respectively.

Currently, updates can only be proposed and voted on by the owners of the genesis keys.
The process of decentralization will result in the core nodes gradually giving up
some of their priviledges and responsibilities to the network,
eventually give them \textit{all} up.
The update proposal mechanism will not be decentralization in the Shelley era, however.
For more on the decentralization process, see Section~\ref{sec:new-epoch-trans}.

\subsection{Protocol Parameter Update Proposals}
\label{sec:pp-proposals}

The transition type $\mathsf{PPUP}$ is for proposing updates to protocol
parameters, see Figure \ref{fig:ts-types:pp-update} (for the corresponding rules,
see Figure \ref{fig:rules:pp-update}).
The signal for this transition is an optional update.

Protocol updates are only allowed up until ($2\cdot\StabilityWindow$)-many slots before the
end of the epoch. The reason for this involves how we safely predict hard forks.
Changing the protocol version can result in a hard fork, and we would like an
entire stability period between when we know that a hard fork will necessarily happen
and when the current epoch ends.

The transition $\mathsf{PPUP}$ has two rules:
\begin{itemize}
  \item PP-Update-Empty : No new updates were proposed, do nothing.
  \item PP-Update-Nonempty : Some new updates $\var{up}$ were proposed.
  Add these to the existing proposals using a right override. That is, if a genesis key
  has previously submitted an update proposal, replace it with its new
  proposal in $\var{pup}$.
\end{itemize}

This rule has the following predicate failures:

\begin{enumerate}
\item In the case of \var{slot} being greater than or equal to
  $\fun{firstSlot}~((\epoch{slot}) + 1) - 2\cdot\fun{StabilityWindow}$, there is
  a \emph{PPUpdateTooLate} failure.
\item In the case that the epoch number in the signal does not match the current epoch,
  there is a \emph{PPUpdateWrongEpoch} failure.
\item In the case of \var{pup} being non-empty, if the check $\dom pup \subseteq
  \dom genDelegs$ fails, there is a \emph{NonGenesisUpdate} failure as only genesis keys
  can be used in the protocol parameter update.
\item If a protocol parameter update in \var{pup} cannot follow the current
  protocol parameter, there is a \emph{PVCannotFollow} failure.
  Note that $\fun{pvCanFollow}$ is defined in Figure~\ref{fig:ts-types:pp-update}.
\end{enumerate}

\begin{figure}[htb]
  \emph{Derived types}
  \begin{equation*}
    \begin{array}{lclr}
      \GenesisDelegation
      & ~=~
      & \KeyHashGen\mapsto(\KeyHash\times\KeyHash_{vrf})
      & \text{genesis delegations} \\
    \end{array}
  \end{equation*}
  %
  \emph{Protocol Parameter Update environment}
  %
  \begin{equation*}
    \PPUpdateEnv =
    \left(
      \begin{array}{r@{~\in~}lr}
        \var{slot} & \Slot & \text{current slot}\\
        \var{pp} & \PParams & \text{protocol parameters}\\
        \var{genDelegs} & \GenesisDelegation
                        & \text{genesis key delegations} \\
      \end{array}
    \right)
  \end{equation*}
  %
  \emph{Protocol Parameter Update transitions}
  \begin{equation*}
    \_ \vdash
    \var{\_} \trans{ppup}{\_} \var{\_}
    \subseteq \powerset (
    \PPUpdateEnv \times \ProposedPPUpdates \times \Update^? \times \ProposedPPUpdates)
  \end{equation*}
  %
  \emph{Helper Functions}
  \begin{align*}
      & \fun{pvCanFollow} \in \ProtVer \to \ProtVer \to \Bool\\
      & \fun{pvCanFollow}~(m,~n)~(m',~n') = \\
      & ~~~~(m + 1, 0) = (m', n') \lor (m, n + 1) = (m', n')
  \end{align*}
  %
  \caption{Protocol Parameter Update Transition System Types}
  \label{fig:ts-types:pp-update}
\end{figure}

\begin{figure}[htb]
  \begin{equation}\label{eq:pp-update-Empty}
    \inference[PP-Update-Empty]
    {
      \var{up} = \Nothing
    }
    {
      \begin{array}{r}
        \var{slot}\\
        \var{pp}\\
        \var{genDelegs}\\
      \end{array}
      \vdash \var{pup_s}\trans{ppup}{up}\var{pup_s}
    }
  \end{equation}

  \nextdef

  \begin{equation}\label{eq:update-nonempty}
    \inference[PP-Update-Nonempty]
    {
      (\var{pup},~\var{e})\leteq\var{up}
      &
      \dom{pup}\subseteq\dom{genDelegs}
      \\
      \forall\var{ps}\in\range{pup},~
        \var{pv}\mapsto\var{v}\in\var{ps}\implies\fun{pvCanFollow}~(\fun{pv}~\var{pp})~\var{v}
      \\
      \var{slot} < \fun{firstSlot}~((\epoch{slot}) + 1) - 2\cdot\StabilityWindow
      \\
      \epoch{slot} = e
    }
    {
      \begin{array}{r}
        \var{slot}\\
        \var{pp}\\
        \var{genDelegs}\\
      \end{array}
      \vdash
      \var{pup_s}
      \trans{ppup}{up}
      \varUpdate{pup_s\unionoverrideRight pup}
    }
  \end{equation}

  \caption{Protocol Parameter Update Inference Rules}
  \label{fig:rules:pp-update}
\end{figure}

\clearpage

%%% Local Variables:
%%% mode: latex
%%% TeX-master: t
%%% End:

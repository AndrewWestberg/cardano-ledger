\section{Implementation of \fun{txSize}}
\label{sec:txSize}

The minimum fee calculation in Figure~\ref{fig:defs:protocol-parameters-helpers}
depends on an abstract $\fun{txSize}$ function, which we describe here.

We define $\fun{txSize}$ as:
$$\fun{txSize}~tx~=~\var{numInputs} \cdot 40 + \var{numOutputs} \cdot 65 + \var{rest},$$
where
\begin{itemize}
  \item $\var{numInputs}$ is the number of transaction inputs in $\var{tx}$,
  \item $\var{numOutputs}$ is the number of transaction outputs in $\var{tx}$,
  \item $\var{tx'}$ is identical to $\var{tx}$, except that it has
    \begin{itemize}
      \item no inputs,
      \item no outputs,
      \item a fee of zero
    \end{itemize}
  \item $\var{rest}$ is the number of serialized bytes in $\var{tx'}$,
    as defined in Appendix~\ref{sec:cddl},
\end{itemize}


We now justify this calculation.
Using the number of bytes in the serialized transaction is problematic for a couple of reasons.
First, the fee is listed in the transaction, so there is a circularity problem.
Second, algorithms implementing coin selection
(choosing which unspent transaction outputs to consume)
would have to make heavy use of serialization.
Besides these two issues, however, the number of serialized bytes
does exactly what we want.

Therefore we calculate the transaction size by first computing
the number of bytes in a modifed version of the transaction
that has no inputs, no outputs, and has a fee of zero,
and then we adjust accordingly by the number of inputs and outputs.

As given by the CDDL spec in Appendix~\ref{sec:cddl},
a transaction input is serialized as:
\begin{lstlisting}[backgroundcolor = \color{lightgray}]
transaction_input = [ transaction_id : $hash32 , index : uint ]
\end{lstlisting}
which is bounded by 40 bytes.

Similarly, a transaction output is serialized as:
\begin{lstlisting}[backgroundcolor = \color{lightgray}]
transaction_output = [address, amount : uint]
\end{lstlisting}
which is bounded by 65 bytes.

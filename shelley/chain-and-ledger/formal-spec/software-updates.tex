\section{Software Updates}
\label{sec:software-updates}


The distinction between protocol parameter and application updates is as
follows: \newline

\noindent \textbf{Protocol parameters}
\begin{itemize}
  \item All parameters excluding $\ProtVer$: constants
currently used in ledger calculations performed according to the rules described
in this document
\begin{itemize}
\item[$\circ$] Updated only at epoch boundary
\item[$\circ$] All nodes automatically adopt new values (this mechanism is an explicit
 part of the ledger rules, and is included in the epoch boundary transition)
\item[$\circ$] Has no effect on the logic of the ledger transition rules
\end{itemize}
\item Protocol version parameter ($\ProtVer$): a special value which
corresponds to a specific version of the \textit{ledger rules}
\begin{itemize}
\item[$\circ$] I.e. if $\var{pv}$ changes, this document may have to be updated with
the new rules and types if there is a change in the logic
\item[$\circ$] E.g. the change may be that the new rules now allow \textit{all} nodes
to vote on update proposals
\item[$\circ$] Whether the $\var{pv}$ must change with any change of protocol
parameters when the \textit{rules do not change} is to be decided
\item[$\circ$] Mechanism for updating is inherited from the general protocol
parameter update and voting rules
\item[$\circ$] If there is a change in transition rules, nodes must have
software installed that can implement these rules at the epoch boundary
when the protocol parameter adoption occurs
\item[$\circ$] Switching to using these new rules is mandatory in the sense that
if the nodes do not have the applications implementing them, this
will prevent a user from reading and
writing to the ledger
\end{itemize}
\end{itemize}
\textbf{Applications} The version of the software the nodes run,
as well as the related system tags and metadata
\begin{itemize}
\item[$\circ$] We cannot force the users to actually upgrade their software
\item[$\circ$] Any application version that is capable of implementing the protocol version
currently on the ledger can be used by a node
\item[$\circ$] Users can update applications as soon as update is agreed upon, and should
do so before their current application becomes incompatible with the
current protocol version (due to the update), however
\item[$\circ$] A voted on application version update has a recommended adoption date,
which applications will automatically follow
\end{itemize}

Applications must sometimes support \textit{several different versions}
of ledger rules in order to accommodate the timely switch of the $\ProtVer$ at the
epoch boundary. Usually, the currently accepted protocol version, and next the
version they are ready to upgrade to (that their application versions can
implement).
The newest protocol version a node is ready to use is included in the block
header of the blocks it produces, see \ref{sec:defs-blocks}. This is either

\begin{itemize}
\item the current version (if there is no protocol version update pending or the node
has not updated to an upcoming software version capable of of implementing a
newer protocol version), or
\item the next protocol version,
(if the node has updated its software, but the current protocol version on the
ledger has not been updated yet).
\end{itemize}

So, users have some agency in the process of adoption of
new protocol versions. They may refuse to download and install updates.
Since software updates cannot be \textit{forced} on the users, if the majority of
users do not perform an update which allows to switch to the next $\ProtVer$,
it cannot happen.

There is no data in blocks or transactions that says what application
versions a user has, or what protocol version they are using (this always has to
be the version recorded on the ledger).
Having the wrong version of an application
may potentially become problematic (when it is not able to follow the current
ledger rules dictated by $\ProtVer$), however, the update mechanism implemented
in the node software should
ensure this does not happen often.
The process of upgrading the system a new version consists of:

\begin{enumerate}
  \item New software is ready for downloading.
  \item The core nodes propose changing
  the application version to this
  new software. The the voting and proposal of application version updates is discussed
    in Section \ref{sec:update}.
  \item If there is concensus, this application version becomes the current application version,
  see Section \ref{sec:ledger-trans}.
  \item All nodes see a new application version on the ledger and update their
  software.
\end{enumerate}

Note that if there is a \textit{new protocol version} implemented by this new
software, the core nodes monitor how many nodes are ready to use the new
protocol version via the block headers.
Once enough nodes are ready for the new protocol version, this
may now be updated as well (by the mechanism in described in 
Section \ref{sec:update}).

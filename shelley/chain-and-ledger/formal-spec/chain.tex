\section{Blockchain layer}
\label{sec:chain}

\newcommand{\Seed}{\type{Seed}}
\newcommand{\seedOp}{\star}
\newcommand{\Proof}{\type{Proof}}
\newcommand{\Seedl}{\mathsf{Seed}_\ell}
\newcommand{\Seede}{\mathsf{Seed}_\eta}
\newcommand{\SlotsPrior}{\ensuremath{\mathsf{SlotsPrior}}}
\newcommand{\StartRewards}{\ensuremath{\mathsf{StartRewards}}}
\newcommand{\activeSlotCoeff}[1]{\fun{activeSlotCoeff}~ \var{#1}}
\newcommand{\slotToSeed}[1]{\fun{slotToSeed}~ \var{#1}}

\newcommand{\Bool}{\type{Bool}}
\newcommand{\T}{\type{T}}
\newcommand{\vrf}[3]{\fun{vrf}_{#1} ~ #2 ~ #3}
\newcommand{\verifyVrf}[4]{\fun{verifyVrf}_{#1} ~ #2 ~ #3 ~#4}

\newcommand{\HashHeader}{\type{HashHeader}}
\newcommand{\bhHash}[1]{\fun{bhHash}~ \var{#1}}
\newcommand{\bHeaderSize}[1]{\fun{bHeaderSize}~ \var{#1}}
\newcommand{\bSize}[1]{\fun{bSize}~ \var{#1}}
\newcommand{\bBodySize}[1]{\fun{bBodySize}~ \var{#1}}
\newcommand{\OCert}{\type{OCert}}
\newcommand{\BHeader}{\type{BHeader}}
\newcommand{\BHBody}{\type{BHBody}}

\newcommand{\bheader}[1]{\fun{bheader}~\var{#1}}
\newcommand{\hsig}[1]{\fun{hsig}~\var{#1}}
\newcommand{\bprev}[1]{\fun{bprev}~\var{#1}}
\newcommand{\bhash}[1]{\fun{bhash}~\var{#1}}
\newcommand{\bsig}[1]{\fun{bsig}~\var{#1}}
\newcommand{\bissuer}[1]{\fun{bissuer}~\var{#1}}
\newcommand{\bseedl}[1]{\fun{bseed}_{\ell}~\var{#1}}
\newcommand{\bprfn}[1]{\fun{bprf}_{n}~\var{#1}}
\newcommand{\bseedn}[1]{\fun{bseed}_{n}~\var{#1}}
\newcommand{\bprfl}[1]{\fun{bprf}_{\ell}~\var{#1}}
\newcommand{\bocert}[1]{\fun{bocert}~\var{#1}}

\newcommand{\PraosEnv}{\type{PraosEnv}}
\newcommand{\PraosState}{\type{PraosState}}
\newcommand{\BHeaderEnv}{\type{BHeaderEnv}}
\newcommand{\BHeaderState}{\type{BHeaderState}}
\newcommand{\VRFEnv}{\type{VRFEnv}}
\newcommand{\DSliderEnv}{\type{DSliderEnv}}
\newcommand{\VRFState}{\type{VRFState}}
\newcommand{\OCertState}{\type{OCertState}}
\newcommand{\NewEpochEnv}{\type{NewEpochEnv}}
\newcommand{\NewEpochState}{\type{NewEpochState}}
\newcommand{\PoolDistr}{\type{PoolDistr}}
\newcommand{\BBodyEnv}{\type{BBodyEnv}}
\newcommand{\BBodyState}{\type{BBodyState}}
\newcommand{\RUpdEnv}{\type{RUpdEnv}}
\newcommand{\ChainEnv}{\type{ChainEnv}}
\newcommand{\ChainState}{\type{ChainState}}
\newcommand{\ChainSig}{\type{ChainSig}}


This chapter introduces the view of the blockchain layer as required for the
ledger. This includes in particular the information required for the epoch
boundary and its rewards calculation as described in Section~\ref{sec:epoch}. It
also covers the transitions that keep track of produced blocks in order to
calculate rewards and penalties for stake pools.

The main transition rule is $\mathsf{CHAIN}$ which calls the subrules
$\mathsf{NEWEPOCH}$ and $\mathsf{UPDN}$, $\mathsf{VRF}$ and $\mathsf{BBODY}$.

\subsection{Verifiable Random Functions (VRF)}
\label{sec:defs-vrf}

\begin{figure}[htb]
  %
  \emph{Abstract types}
  \begin{equation*}
    \begin{array}{r@{~\in~}lr}
      \var{seed} & \Seed  & \text{seed for pseudo-random number generator}\\
      \var{prf} & \Proof  & \text{VRF proof}\\
    \end{array}
  \end{equation*}
  %
  \emph{Abstract functions ($T$ an arbitrary type)}
  %
  \begin{equation*}
    \begin{array}{r@{~\in~}lr}
      \seedOp & \Seed \to \Seed \to \Seed & \text{binary seed operation} \\
      \vrf{\T}{}{} & \SKey \to \Seed \to \T\times\Proof
                   & \text{verifiable random function} \\
                   %
      \verifyVrf{\T}{}{}{} & \VKey \to \Seed \to \T\times\Proof \to \Bool
                           & \text{verify vrf proof} \\
                           %
    \end{array}
  \end{equation*}
  %
  \emph{Derived Types}
  \begin{align*}
    \PoolDistr = \HashKey \mapsto [0, 1] \text{ \hspace{3cm}stake pool distribution}
  \end{align*}
  %

  \emph{Constraints}
  \begin{align*}
    & \forall (sk, vk) \in \KeyPair,~ seed \in \Seed,~
    \verifyVrf{T}{vk}{seed}{\left(\vrf{T}{sk}{seed}\right)}
  \end{align*}
  %
  \emph{Constants}
  \begin{align*}
    & \Seedl \in \Seed & \text{leader seed constant} \\
    & \Seede \in \Seed & \text{nonce seed constant}\\
    & \SlotsPrior \in \Duration & \tau\text{ in \cite{ouroboros_praos}}\\
    & \StartRewards \in \Duration & \text{duration to start reward calculations}\\
  \end{align*}

  \caption{VRF definitions}
  \label{fig:defs-vrf}
\end{figure}

\clearpage

\subsection{Block Definitions}
\label{sec:defs-blocks}

\begin{figure*}[htb]
  \emph{Abstract types}
  %
  \begin{equation*}
    \begin{array}{r@{~\in~}lr}
      \var{h} & \HashHeader & \text{hash of a block header}\\
    \end{array}
  \end{equation*}
  %
  \emph{Operational Certificate}
  %
  \begin{equation*}
    \OCert =
    \left(
      \begin{array}{r@{~\in~}lr}
        \var{vk_{hot}} & \VKeyEv & \text{operational (hot) key}\\
        \var{n} & \N & \text{counter}\\
        c_0 & \KESPeriod & \text{start KES period}\\
        \sigma & \Sig & \text{cold key signature}\\
      \end{array}
    \right)
  \end{equation*}
  %
  \emph{Block Header Body}
  %
  \begin{equation*}
    \BHBody =
    \left(
      \begin{array}{r@{~\in~}lr}
        \var{prev} & \HashHeader & \text{hash of previous block body}\\
        \var{vk} & \VKey & \text{block issuer}\\
        \var{slot} & \Slot & \text{block slot}\\
        \eta & \Seed & \text{nonce}\\
        \var{prf}_{\eta} & \Proof & \text{nonce proof}\\
        \ell & \unitInterval & \text{leader election value}\\
        \var{prf_{\ell}} & \Proof & \text{leader election proof}\\
        \var{sig} & \Sig & \text{block body signature}\\
        \var{oc} & \OCert & \text{operational certificate}\\
      \end{array}
    \right)
  \end{equation*}
  %
  \emph{Block Types}
  %
  \begin{equation*}
    \begin{array}{r@{~\in~}l@{\qquad=\qquad}l}
      \var{bh}
      & \BHeader
      & \BHBody \times \Sig
      \\
      \var{b}
      & \Block
      & \BHeader \times \seqof{\Tx}
    \end{array}
  \end{equation*}
  \emph{Abstract functions}
  %
  \begin{equation*}
    \begin{array}{r@{~\in~}lr}
      \bhHash{} & \BHeader \to \HashHeader
                   & \text{hash of a block header} \\
      \bHeaderSize{} & \BHeader \to \N
                   & \text{size of a block header} \\
      \bBodySize{} & \seqof{\Tx} \to \N
                   & \text{size of a block body} \\
      \slotToSeed{} & \Slot \to \Seed
                    & \text{convert a slot to a seed} \\
    \end{array}
  \end{equation*}
  %
  \emph{Accessor Functions}
  \begin{equation*}
    \begin{array}{r@{~\in~}lr}
      \fun{bheader} & \Block \to \BHeader \\
      \fun{bhbody} & \BHeader \to \BHBody \\
      \fun{hsig} & \BHeader \to \Sig\\
      \fun{bbody} & \Block \to \seqof{\Tx} \\
      \fun{bprev} & \BHBody \to \HashHeader\\
      \fun{bissuer} & \BHBody \to \VKey\\
      \fun{bslot} & \BHBody \to \Slot\\
      \fun{bnonce} & \BHBody \to \Seed\\
      \fun{\bprfn{}} & \BHBody \to \Proof\\
      \fun{bleader} & \BHBody \to \unitInterval\\
      \fun{\bprfl{}} & \BHBody \to \Proof\\
      \fun{bsig} & \BHBody \to \Sig \\
      \fun{bocert} & \BHBody \to \OCert \\
    \end{array}
  \end{equation*}
  %
  \caption{Block Definitions}
  \label{fig:defs:blocks}
\end{figure*}

\clearpage

\subsection{New Epoch Transition}
\label{sec:new-epoch-trans}

For the transition to a new epoch ($\mathsf{NEWEPOCH}$), the environment is
given in Figure~\ref{fig:ts-types:newepoch}, it consists of

\begin{itemize}
\item The epoch nonce.
\item The current slot.
\end{itemize}
The new epoch state is given in Figure~\ref{fig:ts-types:newepoch}, it consists
of

\begin{itemize}
\item The number of the last epoch.
\item The old epoch nonce.
\item The information about produced blocks for each stake pool.
\item The old epoch state.
\item An optional rewards update.
\item The stake pool distribution of the epoch.
\item The OBFT leader schedule.
\end{itemize}

Figure~\ref{fig:ts-types:newepoch} also defines an abstract pseudorandom function
$\fun{obftSchedule}$ for creating the OBFT leader schedule for each new epoch,
as explained in section 3.9.2 of~\cite{delegation_design}.
The function takes a seed, the decentralization parameter $d$, and the active slot coeffient $f$.
It must create $(d\cdot\SlotsPerEpoch)$-many OBFT slots, $(f\cdot d\cdot \SlotsPerEpoch)$
of which are active.

\begin{figure}
  \emph{New Epoch environments}
  \begin{equation*}
    \NewEpochEnv =
    \left(
      \begin{array}{r@{~\in~}lr}
        \eta_1 & \Seed & \text{epoch nonce} \\
        \var{s} & \Slot & \text{current slot}
      \end{array}
    \right)
  \end{equation*}
  %
  \emph{New Epoch states}
  \begin{equation*}
    \NewEpochState =
    \left(
      \begin{array}{r@{~\in~}lr}
        \var{e_\ell} & \Epoch & \text{last epoch} \\
        \eta_0 & \Seed & \text{epoch nonce} \\
        \var{b} & \BlocksMade & \text{blocks made} \\
        \var{es} & \EpochState & \text{epoch state} \\
        \var{ru} & \RewardUpdate^? & \text{reward update} \\
        \var{pd} & \PoolDistr & \text{pool stake distribution} \\
        \var{dsched} & \Slot\mapsto\VKeyGen^? & \text{OBFT schedule} \\
      \end{array}
    \right)
  \end{equation*}
  %
  \emph{Abstract pseudorandom schedule function}
  \begin{align*}
    & \fun{obftSchedule} \in \Seed \to \unitInterval \to \unitInterval \to
      (\Slot\mapsto\VKeyGen^?) \\
  \end{align*}
  %
  \emph{Constraints}
  \begin{align*}
    \text{ given: }~\var{dsched}\leteq\fun{obftSchedule}~\eta~\var{d}~\var{f} \\
    |\var{dsched}| = \floor{d\cdot\SlotsPerEpoch} \\
    |\{s\mapsto k\in\var{dsched}~\mid~k\neq\Nothing\}| = \floor{f\cdot d\cdot\SlotsPerEpoch} \\
  \end{align*}
  %
  \emph{New Epoch Transitions}
  \begin{equation*}
    \_ \vdash \var{\_} \trans{newepoch}{\_} \var{\_} \subseteq
    \powerset (\NewEpochEnv \times \NewEpochState \times \Epoch \times \NewEpochState)
  \end{equation*}
  \caption{NewEpoch transition-system types}
  \label{fig:ts-types:newepoch}
\end{figure}

Figure~\ref{fig:rules:not-new-epoch} defines the new epoch state transition. It
has two rules: the first one deals with the case when the epoch signal $e$ is before the
current epoch \var{e_\ell}. This rule does not change the state. The second rule
describes the change in the case of $e$ being equal to the next epoch
$e_\ell+ 1$. It also calls the $\mathsf{EPOCH}$ rule.

In the second case, the new epoch state is updated as follows:

\begin{itemize}
\item The epoch is set to the new epoch $e$.
\item The epoch nonce is replaced with the new one from the new epoch
  environment.
\item The mapping for the blocks produced by each stake pool is set to the empty
  map.
\item The epoch state is updated first with the rewards update \var{ru} and then
  via the call to $\mathsf{EPOCH}$.
\item The rewards update is set to \Nothing.
\item The pool distribution is updated according to the changes in the stake pools.
\item A new OBFT schedule is created.
\end{itemize}

The new pool distribution \var{pd}' is calculated from the delegation map and
stake allocation of the previous epoch. It associates every stake pool key
($\HashKey_{pool}$) to which a staking key ($\HashKey_{stake}$) delegates with
the stake that $\HashKey_{stake}$ has. The sum of stake for all staking keys
delegating to $\HashKey_{pool}$ then becomes the new entry of $\HashKey_{pool}$
in the pool distribution.

\begin{figure}[ht]
  \begin{equation}\label{eq:new-epoch}
    \inference[New-Epoch]
    {
      e = e_\ell + 1
      &
      (\wcard,~\wcard,~(\wcard,~\wcard,~\var{us}))\leteq\var{es}
      \\
      {
        \var{s}\vdash\var{us} \trans{upiec}{} \var{us'}
      }
      &
      \var{pp_n}\leteq\pps{us'}
     \\
     {
        \left(
          {\begin{array}{c}
              \var{pp}_n \\
              \var{b} \\
           \end{array}}
       \right)
       \vdash
       (\fun{applyRUpd}~\var{ru}~\var{es})\trans{epoch}{\var{e}}\var{es'}
     }
     \\~\\~\\
     {\begin{array}{r@{~\leteq~}l}
        (\wcard,~\wcard,~\var{ss},~(\wcard,~\wcard,~\var{us})) & \var{es} \\
        \var{pp} & \pps{us} \\
        (\wcard,~\var{pstake_{set}},~\wcard,~\wcard,~\wcard,~\wcard) & \var{ss} \\
        (\var{stake}, \var{delegs}) & \var{pstake_{set}} \\
        total & \sum_{\_ \mapsto c\in\var{stake}} c \\
        \var{pd'} & \fun{aggregate_{+}}~\var{delegs}^{-1}\circ
                    \left\{
                    (\var{hk}, \frac{\var{c}}{\var{total}}) \vert (\var{hk},
                    \var{c}) \in \var{stake}
                    \right\} \\
         \var{dsched'} & \fun{obftSchedule}~\eta_1~(\activeSlotCoeff{pp})~(\fun{d}~{pp})\\
        \var{es''} & \var{es'}\unionoverrideRight\{us'\}
      \end{array}}
    }
    {
      \left(
        {\begin{array}{c}
            \eta_1 \\
            \var{s} \\
        \end{array}}
      \right)
      \vdash
      {\left(\begin{array}{c}
            \var{e_\ell} \\
            \eta_0 \\
            \var{b} \\
            \var{es} \\
            \var{ru} \\
            \var{pd} \\
            \var{dsched} \\
      \end{array}\right)}
      \trans{newepoch}{\var{e}}
      {\left(\begin{array}{c}
            \varUpdate{\var{e}} \\
            \varUpdate{\eta_1} \\
            \varUpdate{\emptyset} \\
            \varUpdate{\var{es}''} \\
            \varUpdate{\Nothing} \\
            \varUpdate{\var{pd}'} \\
            \varUpdate{\var{dsched}'} \\
      \end{array}\right)}
    }
  \end{equation}

  \nextdef

  \begin{equation}\label{eq:not-new-epoch}
    \inference[Not-New-Epoch]
    {
      e_\ell \neq e + 1
    }
    {
      \left(
        {\begin{array}{c}
            \eta_1 \\
            \var{s} \\
        \end{array}}
      \right)
      \vdash
      {\left(\begin{array}{c}
            \var{e_\ell} \\
            \eta_0 \\
            \var{b} \\
            \var{es} \\
            \var{ru} \\
            \var{pd} \\
      \end{array}\right)}
      \trans{newepoch}{\var{e}}
      {\left(\begin{array}{c}
            \var{e_\ell} \\
            \eta_0 \\
            \var{b} \\
            \var{es} \\
            \var{ru} \\
            \var{pd} \\
      \end{array}\right)}
    }
  \end{equation}
  \caption{New Epoch rules}
  \label{fig:rules:not-new-epoch}
\end{figure}

\subsection{Update Nonce Transition}
\label{sec:update-nonces-trans}

The Update Nonce Transition updates the nonces until the randomness gets
fixed. The environment is shown in Figure~\ref{fig:ts-types:updnonce} and
consists of the epoch nonce $\eta$. The update nonce state is shown in
Figure~\ref{fig:rules:update-nonce} and consists of the candidate nonce
$\eta_c$ and the evolving nonce $\eta_v$.

\begin{figure}
  \emph{Update Nonce Transitions}
  \begin{equation*}
    \_ \vdash \var{\_} \trans{updn}{\_} \var{\_} \subseteq
    \powerset (\Seed \times (\Seed\times\Seed) \times \Slot \times (\Seed\times\Seed))
  \end{equation*}
  \caption{UpdNonce transition-system types}
  \label{fig:ts-types:updnonce}
\end{figure}

The transition rule $\mathsf{UPDN}$ takes the slot \var{s} as signal.
There are two different cases for $\mathsf{UPDN}$, one where \var{s} is not yet
\SlotsPrior{} slots from the beginning of the epoch and one where \var{s} is at least
\SlotsPrior{} slots after the first slot of the epoch.

\begin{figure}[ht]
  \begin{equation}\label{eq:update-both}
    \inference[Update-Both]
    {
      s < \firstSlot{((\epoch{s}) + 1) - \SlotsPrior}
    }
    {
      \left(
        {\begin{array}{c}
            \eta \\
        \end{array}}
      \right)
      \vdash
      {\left(\begin{array}{c}
            \eta_v \\
            \eta_c \\
      \end{array}\right)}
      \trans{updn}{\var{s}}
      {\left(\begin{array}{c}
            \varUpdate{\eta_v\seedOp\eta} \\
            \varUpdate{\eta_c\seedOp\eta} \\
      \end{array}\right)}
    }
  \end{equation}

  \nextdef

  \begin{equation}\label{eq:only-evolve}
    \inference[Only-Evolve]
    {
      s \geq \firstSlot{((\epoch{s}) + 1) - \SlotsPrior}
    }
    {
      \left(
        {\begin{array}{c}
            \eta \\
        \end{array}}
      \right)
      \vdash
      {\left(\begin{array}{c}
            \eta_v \\
            \eta_c \\
      \end{array}\right)}
      \trans{updn}{\var{s}}
      {\left(\begin{array}{c}
            \varUpdate{\eta_v\seedOp\eta} \\
            \eta_c \\
      \end{array}\right)}
    }
  \end{equation}
  \caption{Update Nonce rules}
  \label{fig:rules:update-nonce}
\end{figure}

\subsection{Operational Certificate Transition}
\label{sec:oper-cert-trans}

The Operational Certificate Transition contains the genesis keys as its environment.
Its signal is a block header body, its state is shown in
Figure~\ref{fig:ts-types:ocert}. The state consists of the pool state and the genesis
key delegations:

\begin{itemize}
\item The stake pools map \var{stpools}.
\item The pool parameters \var{poolParams}.
\item The map of retiring stake pools to epochs \var{retiring}.
\item The mapping \var{cs} of operational certificate counters.
\item The mapping of genesis keys to operational keys.
\end{itemize}

\begin{figure}
  %
  \emph{Operational Certificate states}
  \begin{equation*}
    \OCertState =
    \left(
      \begin{array}{r@{~\in~}lr}
        \var{ps} & \PState & \text{pool state} \\
        \var{dms} & \VKeyGen\mapsto\VKey & \text{genesis key delegations} \\
      \end{array}
    \right)
  \end{equation*}
  %
  \emph{Operational Certificate Transitions}
  \begin{equation*}
    \_ \vdash \var{\_} \trans{ocert}{\_} \var{\_} \subseteq
    \powerset (\powerset{\VKeyGen} \times \OCertState \times \BHBody \times \OCertState)
  \end{equation*}
  \caption{OCert transition-system types}
  \label{fig:ts-types:ocert}
\end{figure}

The transition rule is shown in Figure~\ref{fig:rules:ocert}. From the block
header body \var{bhb} we first extract the following:

\begin{itemize}
\item The operational certificate, consisting of the hot key \var{vk_{hot}}, the
  KES period number \var{n}, the KES period start \var{c_0} and the cold key
  signature.
\item The block header issuer verification key \var{vk_{cold}}.
\item The hash \var{hk} of \var{vk_{cold}}.
\end{itemize}

Using this we verify the preconditions of the operational certificate state
transition which are the following:

\begin{itemize}
\item The KES period start \var{c_0} is greater than or equal to the KES period of
  the slot of the block header body and less than 90 KES periods after \var{c_0}.
\item \var{hk} exists as key in the mapping of certificate counters to a KES
  period \var{m} and that period is less than or equal to \var{n}.
\item The signature $sigma$ can be verified with the cold verification key
  \var{vk_{cold}}.
\end{itemize}

After this, the transition system updates the operational certificate state by
updating the mapping of operational certificates where it overwrites the entry
of the key \var{hk} with the KES period \var{n}.
If the cold key was a genesis key, then the \var{dms} map is updated with the new
operational hot key.

\begin{figure}[ht]
  \begin{equation}\label{eq:ocert}
    \inference[OCert]
    {
      (\var{vk_{hot}},~n,~c_{0},~\sigma) \leteq \bocert{bhb}
      &
      \var{vk_{cold}} \leteq \bissuer{bhb}
      &
      \var{hk} \leteq \hashKey{vk_{cold}}
      \\~\\
      c_0 \leq \kesPeriod{\bslot{bhb}} < c_0 + 90
      \\
      \var{hk}\mapsto m\in\var{cs}
      &
      m \leq n
      &
      \mathcal{V}_{\var{vk_{cold}}}{\serialised{(\var{vk_{hot}},~n,~c_0)}}_{\sigma}
      \\
      d = \var{gkeys} \restrictdom \{\var{vk_{cold}}\mapsto \var{vk_{hot}}\}
    }
    {
      \var{gkeys}
      \vdash
      \left(
      \begin{array}{r}
        \var{stpools} \\
        \var{poolParams} \\
        \var{retiring} \\
        \var{cs} \\
        \\
        \var{dms} \\
      \end{array}
      \right)
      \trans{ocert}{\var{bhb}}
      \left(
      \begin{array}{rcl}
        \var{stpools} \\
        \var{poolParams} \\
        \var{retiring} \\
        \varUpdate{\var{cs}\unionoverrideRight\{\var{hk}\mapsto n\}} \\
        \\
        \varUpdate{\var{dms}\unionoverrideRight d} \\
      \end{array}
      \right)
    }
  \end{equation}
  \caption{OCert rules}
  \label{fig:rules:ocert}
\end{figure}

\subsection{Block Header Transition}
\label{sec:block-header-trans}

The Block Header Transition updates the block header state. The environment is
given in Figure~\ref{fig:ts-types:bheader} contains the current slot (in
wall-clock time) and the protocol parameters. The block header state is shown in
Figure~\ref{fig:ts-types:bheader}, it contains the hash of the last header and
the last slot.

\begin{figure}
  \emph{Block Header environments}
  \begin{equation*}
    \BHeaderEnv =
    \left(
      \begin{array}{r@{~\in~}lr}
        \var{s_{now}} & \Slot & \text{current slot (wall-clock)} \\
        \var{pp} & \PParams & \text{protocol parameters} \\
        \var{stpools} & \StakePools & \text{stake pools} \\
      \end{array}
    \right)
  \end{equation*}
  %
  \emph{Block Header states}
  \begin{equation*}
    \BHeaderState =
    \left(
      \begin{array}{r@{~\in~}lr}
        \var{h} & \HashHeader & \text{latest header hash} \\
        \var{s_\ell} & \Slot & \text{last slot} \\
      \end{array}
    \right)
  \end{equation*}
  %
  \emph{Block Header Transitions}
  \begin{equation*}
    \_ \vdash \var{\_} \trans{bhead}{\_} \var{\_} \subseteq
    \powerset (\BHeaderEnv \times \BHeaderState \times \BHeader \times \BHeaderState)
  \end{equation*}
  \caption{BHeader transition-system types}
  \label{fig:ts-types:bheader}
\end{figure}

The $\mathsf{BHEAD}$ transition rule is shown in
Figure~\ref{fig:rules:bheader}. The signal is a new block header \var{bh}, from
its body \var{bhb} we extract:

\begin{itemize}
\item The slot \var{slot} of the block header.
\item The hash of the block header \var{h} of the previous block header.
\item The hot verification key \var{vk_{hot}} of the operational certificate of
  the block header.
\item The signature $\sigma$ of the block header.
\item The block header body \var{bhb}.
\end{itemize}

The transition rule is executed, if the following conditions are satisfied:

\begin{itemize}
\item The size of \var{bh} is less than or equal to the maximal size that the
  protocol parameters allow for block headers.
\item The \var{slot} of the block header is greater than the last slot and less
  than or equal to the current slot \var{s_{now}}.
\item The hash \var{h} of the previous block header in the block header state
  matches the hash in the body of the block header.
\item The verification key validates the signature $\sigma$ for the block
  header body.
\item The hot verification key \var{vk_{hot}} maps to the slot \var{s_0} in
  \var{stpools}.
\end{itemize}

The state update finally sets \var{slot} as new block header slot and hash of
\var{bh} as the new block header hash in the block header state.

This transition makes sure that the block can be applied at the current time,
that the block header is in the right sequence and that the block header has
been signed with the correct signing key according to the KES.

\begin{figure}[ht]
  \begin{equation}\label{eq:bheader}
    \inference[BHead]
    {
      (\var{bhb},~\sigma) \leteq \var{bh}
      &
      \var{slot} \leteq \bslot{bhb}
      \\
      (\var{vk_{hot}},~\wcard,~\wcard,~\wcard) \leteq \bocert{bhb}
      & \var{hk} \leteq \hashKey{vk_{hot}}
      \\~\\
      \var{hk}\mapsto s_0\in\var{stpools}
      &
      t \leteq \kesPeriod{slot} - \kesPeriod{s_0}
      \\~\\
      \var{s_\ell} < \var{slot} \leq \var{s_{now}}
      &
      \var{h} = \bprev{bhb}
      \\
      \mathcal{V}_{vk_{hot}}{\serialised{bhb}}_{\sigma}^{t}
      &
      \bHeaderSize{bh} \leq \fun{maxBHSize}~\var{pp}
    }
    {
      \left(
        {\begin{array}{c}
            \var{s_{now}} \\
            \var{pp} \\
            \var{stpools} \\
        \end{array}}
      \right)
      \vdash
      {\left(\begin{array}{c}
            \var{h} \\
            \var{s_\ell} \\
      \end{array}\right)}
      \trans{bhead}{\var{bh}}
      {\left(\begin{array}{c}
            \varUpdate{\bhHash{bh}} \\
            \varUpdate{slot} \\
      \end{array}\right)}
    }
  \end{equation}
  \caption{BHeader rules}
  \label{fig:rules:bheader}
\end{figure}

\subsection{Verifiable Random Function}
\label{sec:verif-rand-funct}

The verifiable random function (VRF) transition rule validates that a block
header is correctly produced, i.e., the block issuer is a selected leader for
that slot and has the right to produce the block.

The verifiable random function transition uses the environment shown in
Figure~\ref{fig:ts-types:vrf} which consists of

\begin{itemize}
\item The current slot number \var{s_{now}}.
\item The protocol parameters.
\item The epoch nonce $\eta_0$.
\item The pool stake distribution.
\item The stake pool mapping \var{stpools}.
\end{itemize}

The VRF state is $\BHeaderState$.

\begin{figure}
  \emph{VRF environments}
  \begin{equation*}
    \VRFEnv =
    \left(
      \begin{array}{r@{~\in~}lr}
        \var{s_{now}} & \Slot & \text{current slot} \\
        \var{pp} & \PParams & \text{protocol parameters} \\
        \eta_0 & \Seed & \text{epoch nonce} \\
        \var{pd} & \PoolDistr & \text{pool stake distribution} \\
        \var{stpools} & \StakePools & \text{stake pools} \\
      \end{array}
    \right)
  \end{equation*}
  %
  \emph{VRF Transitions}
  \begin{equation*}
    \_ \vdash \var{\_} \trans{vrf}{\_} \var{\_} \subseteq
    \powerset (\VRFEnv \times \BHeaderState \times \BHeader \times \BHeaderState)
  \end{equation*}
  \caption{VRF transition-system types}
  \label{fig:ts-types:vrf}
\end{figure}

The $\mathsf{VRF}$ rule calls $\mathsf{BHEAD}$ as a subrule, its signal is a
block header \var{bh}. For the transition it extracts the following:

\begin{itemize}
\item The block header body \var{bhb} of the block header.
\item The verification key \var{vk} of the block issuer.
\item The slot seed \var{ss} of the slot of the block.
\end{itemize}

This is used to validate the preconditions of the transition rule via
verification of the following:

\begin{itemize}
\item The verification key is associated to a non-zero relative stake
  $\sigma$ in \var{stpools}.
\item The $\fun{bleader}$ value of \var{bhb} indicates a possible leader for
  this slot.
\item The verification of the proofs for the seed leader constant \var{\Seedl}
  and the nonce seed constant $\Seede$ succeeds.
\end{itemize}

When these preconditions are fulfilled, the transition updates the VRF states by
replacing the hash header with \var{h'} and the slot with \var{s_\ell}, as
returned by the application of the $\mathsf{BHEAD}$ transition rule.

\begin{figure}[ht]
  \begin{equation}\label{eq:vrf}
    \inference[VRF]
    {
      \var{bhb} \leteq \fun{bhbody}~{\var{bh}}
      &
      f \leteq \activeSlotCoeff{pp}
      \\
      \var{vk} \leteq \bissuer bhb
      &
      \var{ss} \leteq \slotToSeed{\bslot{bhb}}
      &
      \var{hk} \leteq \hashKey{vk}
      \\~\\
      \verifyVrf{\Seed}{vk}{((\eta_0\seedOp ss)\seedOp\Seede)}{(\bprfn{bhb})}
      \\
      \verifyVrf{\unitInterval}{vk}{((\eta_0\seedOp ss)\seedOp\Seedl)}{(\bprfl{bhb})}
      \\
      \var{hk}\mapsto \sigma\in\var{pd}
      &
      \fun{bleader}~\var{bhb} < 1 - (1 - f)^{\sigma}
      \\~\\
      {
        \left(
          {\begin{array}{c}
             \var{s_{now}} \\
             \var{pp} \\
             \var{stpools} \\
           \end{array}}
        \right)
        \vdash
        \left(
          {\begin{array}{c}
             \var{h} \\
             \var{s_{\ell}} \\
           \end{array}}
        \right)
        \trans{bhead}{\var{bh}}
        \left(
          {\begin{array}{c}
             \var{h}' \\
             \var{s_{\ell}}' \\
           \end{array}}
        \right)
      }
    }
    {
      \left(
        {\begin{array}{c}
            \var{s_{now}} \\
            \var{pp} \\
            \eta_0 \\
            \var{pd} \\
            \var{stpools} \\
        \end{array}}
      \right)
      \vdash
      {\left(\begin{array}{c}
            \var{h} \\
            \var{s_\ell} \\
      \end{array}\right)}
      \trans{vrf}{\var{bh}}
      {\left(\begin{array}{c}
            \varUpdate{\var{h}'} \\
            \varUpdate{\var{s_\ell}'} \\
      \end{array}\right)}
    }
  \end{equation}
  \caption{VRF rules}
  \label{fig:rules:vrf}
\end{figure}

\clearpage

\subsection{Decentralization Slider}
\label{sec:decentralization-slider}

The transition from the boostrap era to a fully decentralized network is explained in
section 3.9.2 of~\cite{delegation_design}.
Key to this transition is a protocol parameter $d$ which controls how many slots are governed by
the genesis nodes via OBFT, and which slots are open to any registered stake pool.
The transition system introduced in this section, $\type{DSLIDE}$, covers this mechanism.

The environments for this transition is:
\begin{itemize}
  \item The environment for the $\type{VRF}$ transition $\var{ve}$.
  \item A mapping $\var{dsched}$ of slots to an optional genesis key.
    In the terminology of \cite{delegation_design},
    the slots in $\var{dsched}$ are the ``OBFT slots''.
    A slot in this map with a value of $\Nothing$ is a non-active slots,
    otherwise it is an active slot and its value designates the genesis key
    responsible for producing the block.
\end{itemize}

\begin{figure}
  \emph{DSlider environments}
  \begin{equation*}
    \DSliderEnv =
    \left(
      \begin{array}{r@{~\in~}lr}
        \var{ve} & \VRFEnv & \text{VRF environment} \\
        \var{dsched} & \Slot\mapsto\VKeyGen^? & \text{OBFT schedule} \\
      \end{array}
    \right)
  \end{equation*}
  %
  \emph{DSlider Transitions}
  \begin{equation*}
    \_ \vdash \var{\_} \trans{dslider}{\_} \var{\_} \subseteq
    \powerset (\DSliderEnv \times \BHeaderState \times \BHeader \times \BHeaderState)
  \end{equation*}
  \caption{DSlider transition-system types}
  \label{fig:ts-types:decent-slider}
\end{figure}

\begin{figure}[ht]
  \begin{equation}\label{eq:active-pbft}
    \inference[Active-OBFT]
    {
      \{\bslot \bhbody bh \mapsto \var{gkey}\} \in \var{dsched}
      \\
      (\var{s_{now}},~\var{pp},~\wcard,~\wcard,~\var{stpools})\leteq ve
      \\~\\
      {
        \left(
          {\begin{array}{c}
             \var{s_{now}} \\
             \var{pp} \\
             \var{stpools} \\
           \end{array}}
        \right)
        \vdash
        \left(
          {\begin{array}{c}
             \var{h} \\
             \var{s_{\ell}} \\
           \end{array}}
        \right)
        \trans{bhead}{\var{bh}}
        \left(
          {\begin{array}{c}
             \var{h}' \\
             \var{s_{\ell}}' \\
           \end{array}}
        \right)
      }
    }
    {
      \left(
        {\begin{array}{c}
            \var{ve} \\
            \var{dsched} \\
        \end{array}}
      \right)
      \vdash
      {\left(\begin{array}{c}
            \var{h} \\
            \var{s_\ell} \\
      \end{array}\right)}
      \trans{dslider}{\var{bh}}
      {\left(\begin{array}{c}
            \varUpdate{\var{h}'} \\
            \varUpdate{\var{s_\ell}'} \\
      \end{array}\right)}
    }
  \end{equation}

  \nextdef

  \begin{equation}\label{eq:decentralized}
    \inference[Decentralized]
    {
      \bslot \bhbody bh \notin \dom{\var{dsched}}
      \\~\\
      {
        \var{ve}\vdash
        \left(
          {\begin{array}{c}
             \var{h} \\
             \var{s_{\ell}} \\
           \end{array}}
        \right)
        \trans{vrf}{\var{bh}}
        \left(
          {\begin{array}{c}
             \var{h}' \\
             \var{s_{\ell}}' \\
           \end{array}}
        \right)
      }
    }
    {
      \left(
        {\begin{array}{c}
            \var{ve} \\
            \var{dsched} \\
        \end{array}}
      \right)
      \vdash
      {\left(\begin{array}{c}
            \var{h} \\
            \var{s_\ell} \\
      \end{array}\right)}
      \trans{dslider}{\var{bh}}
      {\left(\begin{array}{c}
            \varUpdate{\var{h}'} \\
            \varUpdate{\var{s_\ell}'} \\
      \end{array}\right)}
    }
  \end{equation}

  \caption{DSlider rules}
  \label{fig:rules:decent-slider}
\end{figure}

\subsection{Block Body Transition}
\label{sec:block-body-trans}

The Block Body Transition updates the block body state which comprises the
ledger state and the map describing the produced blocks. The environment of the
$\mathsf{BBODY}$ transition are the genesis key delegations.

\begin{figure}
  \emph{BBody Transitions}

  \begin{equation*}
    \_ \vdash \var{\_} \trans{bbody}{\_} \var{\_} \subseteq
    \powerset ((\VKeyGen\mapsto\VKey) \times (\LState\times\BlocksMade)
    \times \Block \times (\LState\times\BlocksMade))
  \end{equation*}
  \caption{BBody transition-system types}
  \label{fig:ts-types:bbody}
\end{figure}

The $\mathsf{BBODY}$ transition rule is shown in Figure~\ref{fig:rules:bbody},
its sub-rule is $\mathsf{LEDGERS}$ which does the update of the ledger
state. The signal is a block from which we extract:

\begin{itemize}
\item The block header body \var{bhb}.
\item The sequence of transactions \var{txs} of the block.
\item The verification key \var{vk} of the issuer of the \var{block} and its
  hash \var{hk}.
\end{itemize}

The transition is executed if the following preconditions are met:

\begin{itemize}
\item The size of the block body is less than or equal to the maximal body size
  described in the protocol parameters.
\item Either \var{hk} to \var{n} is in the mapping of produced blocks, or
  \var{n} is equal to 0.
\item The transactions \var{txs} were correctly signed by \var{vk}, producing
  the signature $\sigma$.
\end{itemize}

After this, the transition system updates the mapping of the hashed stake pool
keys to the incremented value of produced blocks (\var{n} + 1).

\begin{figure}[ht]
  \begin{equation}\label{eq:bbody}
    \inference[Block-Body]
    {
      \var{txs} \leteq \bbody{block}
      &
      \var{vk} \leteq \bissuer{bhb}
      &
      \var{hk} \leteq \hashKey{vk}
      \\
      \sigma \leteq \bsig{bhb}
      &
      \var{bh} \leteq \bheader{block}
      &
      \var{bhb} \leteq \bhbody{bh}
      \\
      (\wcard,~\wcard,~\var{us})\leteq ls
      &
      \var{pp}\leteq\pps{us}
      \\~\\
      \bBodySize{txs} < \fun{maxBBSize}~\var{pp}
      &
      \var{hk}\mapsto n\in (b \unionoverrideLeft \{\var{kh}\mapsto 0\})
      &
      \mathcal{V}_{\var{vk}}{\serialised{txs}}_{\sigma}
      \\~\\
      {
        \left(
          {\begin{array}{c}
             \bslot{bhb} \\
             \var{dms} \\
           \end{array}}
        \right)
        \vdash
             \var{ls} \\
        \trans{ledgers}{\var{txs}}
             \var{ls}' \\
      }
    }
    {
      \var{dms} \\
      \vdash
      {\left(\begin{array}{c}
            \var{ls} \\
            \var{b} \\
      \end{array}\right)}
      \trans{bbody}{\var{block}}
      {\left(\begin{array}{c}
            \varUpdate{\var{ls}'} \\
            \varUpdate{b \unionoverrideRight \{\var{hk}\mapsto n+1\}} \\
      \end{array}\right)}
    }
  \end{equation}
  \caption{BBody rules}
  \label{fig:rules:bbody}
\end{figure}

\subsection{Reward Update Transition}
\label{sec:reward-update-trans}

The Reward Update Transition calculates a new $\RewardUpdate$ to apply in a
$\mathsf{NEWEPOCH}$ transition. The environment is shown in
Figure~\ref{fig:ts-types:reward-update}, it consists of the produced blocks
mapping \var{b} and the epoch state \var{es}. Its state is an optional reward
update.

\begin{figure}
  \emph{Reward Update environments}
  \begin{equation*}
    \RUpdEnv =
    \left(
      \begin{array}{r@{~\in~}lr}
        \var{b} & \BlocksMade & \text{blocks made} \\
        \var{es} & \EpochState & \text{epoch state} \\
      \end{array}
    \right)
  \end{equation*}
  %
  \emph{Reward Update Transitions}
  \begin{equation*}
    \_ \vdash \var{\_} \trans{rupd}{\_} \var{\_} \subseteq
    \powerset (\RUpdEnv \times \RewardUpdate^? \times \Slot \times \RewardUpdate^?)
  \end{equation*}
  \caption{Reward Update transition-system types}
  \label{fig:ts-types:reward-update}
\end{figure}

The transition rules are shown in Figure~\ref{fig:rules:reward-update}. There
are three cases, one which computes a new reward update, one which leaves the
rewards update unchanged as it has not yet been applied and finally one that
leaves the reward update unchanged as the transition was started too early.

The signal of the transition rule $\mathsf{RUPD}$ is the slot \var{s}. The
execution of the transition role is as follows:

\begin{itemize}
\item If \var{s} is less than or equal to the sum of the first slot of its epoch
  and the duration to start rewards \StartRewards, then the state is not
  updated.
\item If the current reward update \var{ru} is not \Nothing, i.e., a reward
  update has already been calculated but not yet applied, then the state is not
  updated.
\item If the current reward update \var{ru} is empty and \var{s} is greater than
  the sum of the first slot of its epoch and the duration \StartRewards, then a
  new rewards update is calculated and the state is updated.
\end{itemize}

\begin{figure}[ht]
  \begin{equation}\label{eq:reward-update}
    \inference[Reward-Update]
    {
      ru' \leteq \createRUpd{b}{es}
      \\~\\
      s > (\firstSlot{\epoch{s}}) + \StartRewards
      &
      ru = \Nothing
    }
    {
      \left(
        {\begin{array}{c}
            \var{b} \\
            \var{es} \\
        \end{array}}
      \right)
      \vdash
      \var{ru}\trans{rupd}{\var{s}}\varUpdate{\var{ru}'}
    }
  \end{equation}

  \nextdef

  \begin{equation}\label{eq:no-reward-update}
    \inference[No-Reward-Update]
    {
      ru \neq \Nothing
    }
    {
      \left(
        {\begin{array}{c}
            \var{b} \\
            \var{es} \\
        \end{array}}
      \right)
      \vdash
      \var{ru}\trans{rupd}{\var{s}}\var{ru}
    }
  \end{equation}

  \nextdef

  \begin{equation}\label{eq:reward-too-early}
    \inference[Reward-Too-Early]
    {
      s \leq (\firstSlot{\epoch{s}}) + \StartRewards
    }
    {
      \left(
        {\begin{array}{c}
            \var{b} \\
            \var{es} \\
        \end{array}}
      \right)
      \vdash
      \var{ru}\trans{rupd}{\var{s}}\var{ru}
    }
  \end{equation}

  \caption{Reward Update rules}
  \label{fig:rules:reward-update}
\end{figure}

\subsection{Chain Transition}
\label{sec:chain-trans}

The $\mathsf{CHAIN}$ transition rule is the main rule of the blockchain layer
part of the STS. It calls $\mathsf{BBODY}$, $\mathsf{VRF}$, $\mathsf{RUPD}$,
$\mathsf{UPDN}$, $\mathsf{OCERT}$ and $\mathsf{NEWEPOCH}$ as sub-rules.

The environment for the chain rule is shown in Figure~\ref{fig:ts-types:chain},
it is comprised of the current slot number \var{s_{now}} and the set of genesis keys.

The chain state is shown in Figure~\ref{fig:ts-types:chain}, it consists of the
following:

\begin{itemize}
\item The three nonces for the $\mathsf{VRF}$.
\item The mapping of produced blocks.
\item The last slot \var{s_\ell}.
\item The last epoch \var{e_\ell}.
\item The epoch state \var{es}.
\item The reward update \var{ru}.
\item The stake pool distribution \var{pd}.
\item The genesis key delegations.
\end{itemize}

\begin{figure}
  \emph{Chain environments}
  \begin{equation*}
    \ChainEnv =
    \left(
      \begin{array}{r@{~\in~}lr}
        \var{s_{now}} & \Slot & \text{current slot} \\
        \var{gkeys} & \VKeyGen & \text{genesis keys} \\
      \end{array}
    \right)
  \end{equation*}
  %
  \emph{Chain states}
  \begin{equation*}
    \ChainState =
    \left(
      \begin{array}{r@{~\in~}lr}
        (\eta_0,~\eta_c,~\eta_v) & \Seed\times\Seed\times\Seed & \text{nonces} \\
        \var{b} & \BlocksMade & \text{blocks made} \\
        \var{h} & \HashHeader & \text{latest header hash} \\
        \var{s_\ell} & \Slot & \text{last slot} \\
        \var{e_\ell} & \Epoch & \text{lats epoch} \\
        \var{es} & \EpochState & \text{epoch state} \\
        \var{ru} & \RewardUpdate^? & \text{potential reward update} \\
        \var{pd} & \PoolDistr & \text{pool stake distribution} \\
        \var{dsched} & \Slot\mapsto\VKeyGen^? & \text{OBFT schedule} \\
        \var{dms} & \VKeyGen\mapsto\VKey & \text{genesis key delegations} \\
      \end{array}
    \right)
  \end{equation*}
  %
  \emph{Chain Transitions}
  \begin{equation*}
    \_ \vdash \var{\_} \trans{chain}{\_} \var{\_} \subseteq
    \powerset (\ChainEnv \times \ChainState \times \Block \times \ChainState)
  \end{equation*}
  \caption{Chain transition-system types}
  \label{fig:ts-types:chain}
\end{figure}

The $\mathsf{CHAIN}$ transition rule is shown in
Figure~\ref{fig:rules:chain}. Its signal is a \var{block} from the \var{block}
we extract the block header body \var{bhb}. From \var{bhb} we extract the block
nonce $\eta$ and the block slot \var{s}. The transition rule itself has no
preconditions, instead it calls all its subrules.
Note that $\mathsf{UPIEC}$ is defined in \cite{byron_ledger_spec}.

\begin{figure}[ht]
  \begin{equation}\label{eq:chain}
    \inference[Chain]
    {
      \var{bh} \leteq \bheader{block}
      &
      \var{bhb} \leteq \bhbody{bh}
      &
      \eta \leteq \fun{bnonce}~\var{bhb}
      &
      \var{s} \leteq \bslot{bhb}
      \\~\\
      {
        \left(
          {\begin{array}{c}
              \eta_c \\
              \var{s} \\
          \end{array}}
        \right)
        \vdash
        {\left(\begin{array}{c}
              \var{e_\ell} \\
              \eta_0 \\
              \var{b} \\
              \var{es} \\
              \var{ru} \\
              \var{pd} \\
              \var{dsched} \\
        \end{array}\right)}
        \trans{newepoch}{\var{e}}
        {\left(\begin{array}{c}
              \var{e_\ell}' \\
              \eta_0' \\
              \var{b}' \\
              \var{es}' \\
              \var{ru}' \\
              \var{pd}' \\
              \var{dsched}' \\
        \end{array}\right)}
      }
      &
      {
        \eta_{0} \vdash
        {\left(\begin{array}{c}
              \eta_v \\
              \eta_c \\
        \end{array}\right)}
        \trans{updn}{\var{s}}
        {\left(\begin{array}{c}
              \eta_v' \\
              \eta_c' \\
        \end{array}\right)}
      }
      \\~\\
      (\var{acnt}',~\var{ss}',~(\var{utxoSt}',~(\var{ds}',~\var{ps'}),~\var{us}'))\leteq\var{es}'
      \\
      \var{gkeys}
      \vdash
      {\left(\begin{array}{c}
      \var{ps}' \\
      \var{dms} \\
      \end{array}\right)}
      \trans{ocert}{bhb}
      {\left(\begin{array}{c}
      \var{ps}'' \\
      \var{dms}' \\
      \end{array}\right)}
      &
      (\var{stpools},~\wcard,~\wcard,~\wcard,~\wcard) \leteq \var{ps}''
      \\
      {
        \left(
          {\begin{array}{c}
              \var{s_{now}} \\
              \var{pp}' \\
              \eta_0' \\
              \var{pd}' \\
              \var{stpools} \\
              \\
              \var{dsched'} \\
          \end{array}}
        \right)
        \vdash
        {\left(\begin{array}{c}
              \var{h} \\
              \var{s_\ell} \\
        \end{array}\right)}
        \trans{dslider}{\var{bh}}
        {\left(\begin{array}{c}
              \var{h}' \\
              \var{s_\ell}' \\
        \end{array}\right)}
      }
      &
      {
        \left(
          {\begin{array}{c}
              \var{b} \\
              \var{es}' \\
          \end{array}}
        \right)
        \vdash \var{ru}' \trans{rupd}{\var{s}} \var{ru}''
      }
      \\
      \var{ls} \leteq (\var{utxoSt}',~(\var{ds}',~\var{ps}''),~\var{us}')
      \\
      {
        \var{dms}'\vdash
        {\left(\begin{array}{c}
              \var{ls} \\
              \var{b'} \\
        \end{array}\right)}
        \trans{bbody}{\var{block}}
        {\left(\begin{array}{c}
              \var{ls}' \\
              b'' \\
        \end{array}\right)}
      }
      &
      \var{es}'' \leteq (\var{acnt}',~\var{ss}',~\var{ls}')
    }
    {
      \left(
        {\begin{array}{c}
            \var{s_{now}} \\
            \var{gkeys} \\
        \end{array}}
      \right)
      \vdash
      {\left(\begin{array}{c}
            (\eta_0,~\eta_c,~\eta_v) \\
            \var{b} \\
            \var{h} \\
            \var{s_\ell} \\
            \var{e_\ell} \\
            \var{es} \\
            \var{ru} \\
            \var{pd} \\
            \var{dsched} \\
            \var{dms} \\
      \end{array}\right)}
      \trans{chain}{\var{block}}
      {\left(\begin{array}{c}
            \varUpdate{(\eta_0',~\eta_c',~\eta_v')} \\
            \varUpdate{\var{b}''} \\
            \varUpdate{\var{h}'} \\
            \varUpdate{\var{s_\ell}'} \\
            \varUpdate{\var{e_\ell}'} \\
            \varUpdate{\var{es}''} \\
            \varUpdate{\var{ru}''} \\
            \varUpdate{\var{pd}'} \\
            \varUpdate{\var{dsched}'} \\
            \varUpdate{\var{dms}'} \\
      \end{array}\right)}
    }
  \end{equation}
  \caption{Chain rules}
  \label{fig:rules:chain}
\end{figure}

%%% Local Variables:
%%% mode: latex
%%% TeX-master: "ledger-spec"
%%% End:

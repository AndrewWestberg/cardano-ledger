\section{Output Size}
\label{sec:value-size}

In Shelley, the protocol parameter $\var{minUTxOValue}$ is used to
disincentivize attacking nodes by putting many small outputs on the
UTxO, permanently blocking memory. Because members of the $\Value$
type can be arbitrary large, the ledger requires a $\UTxO$ entry to
contain a minimum amount of Ada, proportional to its size.

How much memory is used exactly is an implementation detail, but some facts are:
\begin{itemize}
  \item Every output contains and address and Ada, which take a constant amount of space.
  \item The size will scale linearly with the number of $\PolicyID$s and tokens in the $\Value$.
  \item The number of $\PolicyID$s is less than or equal to the number of tokens.
\end{itemize}

Because of the first two items in that list, we could use the following definition for $\fun{outputSize}$:

\begin{figure*}[h]
  \begin{align*}
    & \fun{outputSize} \in \PParams \to \TxOut \to \MemoryEstimate \\
    & \fun{outputSize}~{pp}~\var{out} = k_0 + k_1 * |\{ (\var{pid}, \var{aid}) : \var{v}~\var{pid}~\var{aid} \neq 0
            \land (\var{pid}, \var{aid}) \neq (\mathsf{adaPolicy}, \mathsf{adaName}) \}| \\
    & \phantom{\fun{outputSize}~{pp}~\var{out} = k_0} + k_2 * |\{ \var{pid} : \exists aid : v~pid~aid \neq 0 \}| \\
    & ~~\where \\
    & ~~~~~~(k_0, k_1, k_2) = \fun{outputSizeConstants}~{pp}
  \end{align*}
  \caption{Value Size}
  \label{fig:test}
\end{figure*}

Here, $k_0$ reflects the constant part, i.e. Ada, the key and the address, $k_1$ reflects the
size of non-ada tokens, and $k_2$ the size of $\PolicyID$s.

There are some trade-offs here. Because of the third item in the above
list, $k_2$ could be set to zero without changing the asymptotic
behaviour, somewhat simplifying this function. This is important,
because this function influences how much users have to send each
other for transactions to be valid, so having a function that a user
can easily evaluate by themselves is important.

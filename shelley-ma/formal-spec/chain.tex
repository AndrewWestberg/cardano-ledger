\section{Blockchain layer}
\label{sec:chain}

\newcommand{\Proof}{\type{Proof}}
\newcommand{\Seedl}{\mathsf{Seed}_\ell}
\newcommand{\Seede}{\mathsf{Seed}_\eta}
\newcommand{\activeSlotCoeff}[1]{\fun{activeSlotCoeff}~ \var{#1}}
\newcommand{\slotToSeed}[1]{\fun{slotToSeed}~ \var{#1}}

\newcommand{\T}{\type{T}}
\newcommand{\vrf}[3]{\fun{vrf}_{#1} ~ #2 ~ #3}
\newcommand{\verifyVrf}[4]{\fun{verifyVrf}_{#1} ~ #2 ~ #3 ~#4}

\newcommand{\HashHeader}{\type{HashHeader}}
\newcommand{\HashBBody}{\type{HashBBody}}
\newcommand{\bhHash}[1]{\fun{bhHash}~ \var{#1}}
\newcommand{\bHeaderSize}[1]{\fun{bHeaderSize}~ \var{#1}}
\newcommand{\bSize}[1]{\fun{bSize}~ \var{#1}}
\newcommand{\bBodySize}[1]{\fun{bBodySize}~ \var{#1}}
\newcommand{\OCert}{\type{OCert}}
\newcommand{\BHeader}{\type{BHeader}}
\newcommand{\BHBody}{\type{BHBody}}

\newcommand{\bheader}[1]{\fun{bheader}~\var{#1}}
\newcommand{\hsig}[1]{\fun{hsig}~\var{#1}}
\newcommand{\bprev}[1]{\fun{bprev}~\var{#1}}
\newcommand{\bhash}[1]{\fun{bhash}~\var{#1}}
\newcommand{\bvkcold}[1]{\fun{bvkcold}~\var{#1}}
\newcommand{\bseedl}[1]{\fun{bseed}_{\ell}~\var{#1}}
\newcommand{\bprfn}[1]{\fun{bprf}_{n}~\var{#1}}
\newcommand{\bseedn}[1]{\fun{bseed}_{n}~\var{#1}}
\newcommand{\bprfl}[1]{\fun{bprf}_{\ell}~\var{#1}}
\newcommand{\bocert}[1]{\fun{bocert}~\var{#1}}
\newcommand{\bnonce}[1]{\fun{bnonce}~\var{#1}}
\newcommand{\bleader}[1]{\fun{bleader}~\var{#1}}
\newcommand{\hBbsize}[1]{\fun{hBbsize}~\var{#1}}
\newcommand{\bbodyhash}[1]{\fun{bbodyhash}~\var{#1}}
\newcommand{\overlaySchedule}[4]{\fun{overlaySchedule}~\var{#1}~\var{#2}~{#3}~\var{#4}}

\newcommand{\PrtclState}{\type{PrtclState}}
\newcommand{\PrtclEnv}{\type{PrtclEnv}}
\newcommand{\OverlayEnv}{\type{OverlayEnv}}
\newcommand{\VRFState}{\type{VRFState}}
\newcommand{\NewEpochEnv}{\type{NewEpochEnv}}
\newcommand{\NewEpochState}{\type{NewEpochState}}
\newcommand{\PoolDistr}{\type{PoolDistr}}
\newcommand{\BBodyEnv}{\type{BBodyEnv}}
\newcommand{\BBodyState}{\type{BBodyState}}
\newcommand{\RUpdEnv}{\type{RUpdEnv}}
\newcommand{\ChainEnv}{\type{ChainEnv}}
\newcommand{\ChainState}{\type{ChainState}}
\newcommand{\ChainSig}{\type{ChainSig}}

In Figure~\ref{fig:functions:to-ma}, we give the functions that will be used
to transition from a Shelley chain state into a chain state that provides multi-asset support.
The only part of the state that is affected by the transition is the UTxO. For ease of
reading, we use different notation here to define the UTxO update, and specify
explicitly only the change to the UTxO, implying that every other part of the
state is unaffected.

%%
%% Figure - Shelley to MA Transition
%%
\begin{figure}[htb]
  \begin{align*}
      & \fun{mkUTxO} ~\in~ \ShelleyUTxO  \to \UTxO  \\
      & \fun{mkUTxO}~\var{utxo} ~=~ \{~ \var{txin} \mapsto (a,\fun{coinToValue}~c) ~\vert~
      \var{txin} \mapsto \var{(a,c)}\in ~\var{utxo}~\} \\
      & \text{make UTxO with MA}
      \nextdef
      & \fun{toMA} \in ~ \ShelleyChainState \to \ChainState \\
      & \fun{toMA}~cs = cs' \\
      & \where \\
      & ~~~~ cs'_{utxo} = \fun{mkUTxO}~\var{utxo} \\
      & \text{transform Shelley chain state to a MA-supporting state}
  \end{align*}
  \caption{Shelley to Multi-Asset Chain State Transtition}
  \label{fig:functions:to-ma}
\end{figure}

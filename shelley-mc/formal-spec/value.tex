\section{Coin and Multi-Asset Tokens}
\label{sec:coin-ma}

This section describes the type $\Value$, the operations required on
it, and its relation to $\Coin$.

When multi-asset support on the ledger is introduced, Ada is still expected to be
the most common type of token on the ledger.
The $\Coin$ type is used to represent an amount of Ada.
It is the only
type of token that can be used for all non-UTxO ledger accounting, including deposits,
fees, rewards, treasury, and the proof of stake protocol. Under no circumstances
are these administrative fields and calculations ever expected to operate on
any types of tokens besides Ada, and this will continue to have the type $\Coin$.

\begin{figure*}[t!]
  \emph{Abstract Types}
  %
  \begin{align*}
    \var{aid} \in& ~\AssetID & \text{Asset IDs}
  \end{align*}
  %
  \emph{Derived types}
  %
  \begin{equation*}
    \begin{array}{r@{~\in~}l@{\qquad=\qquad}lr}
      \var{pid} & \PolicyID & \ScriptHash \\
      \var{quan} & \Quantity & \Z \\
      %\text{quantity of a token}\\
      \var{v}, \var{w} & \Value
      & \PolicyID \to_0 ( \AssetID \to_0 \Quantity )
%      & \text{a collection of tokens}
    \end{array}
  \end{equation*}
  %
  \emph{Abstract Functions and Values}
  %
  \begin{align*}
    \mathsf{adaID} \in& ~\PolicyID
    & \text{Ada asset ID} \\
    \mathsf{adaToken} \in& ~\AssetID
    & \text{Ada Token}
  \end{align*}
  \emph{Helper Functions}
  %
  \begin{align*}
    & \fun{coinToValue} \in \Coin\to \Value \\
    & \fun{coinToValue}~c = \lambda~\var{pid}~\var{aid}.
      \begin{cases}
        c & \var{pid} = \mathsf{adaID}~\text{and}~\var{aid} = \mathsf{adaToken} \\
        0 & \text{otherwise}
      \end{cases}
    \nextdef
    & \fun{valueToCoin} \in \Value \to \Coin \\
    & \fun{valueToCoin}~v = v~\mathsf{adaID}~\mathsf{adaToken}
    \nextdef
    & \fun{valueSize} \in \Value \to \N \\
    & \fun{valueSize}~\var{v} = k + k' * |\{ (\var{pid}, \var{aid}) : \var{v}~\var{pid}~\var{aid} \neq 0
      \land (\var{pid}, \var{aid}) \neq (\mathsf{adaID}, \mathsf{adaToken}) \}|
  \end{align*}
  \caption{Type Definitions and auxiliary functions for Value}
  \label{fig:defs:value}
\end{figure*}

\subsection*{Representing Multi-asset Types and Values}
An \emph{Asset} comprises a set of different \emph{Asset Classes}, each of which has
a unique identifier, $\AssetID$. We will informally refer to a pair $(\var{pid}, \var{aid})$ of a Policy ID and an Asset ID as a ``token''. The token for Ada is $(\mathsf{adaID}, \mathsf{adaToken})$.

The set of tokens that are referred to by the underlying monetary policy represents the coinage that the asset supports.  A multi-asset value, $\Value$ is a map over zero or more assets
to single asset values.  A single asset value is then a finite map from
$\AssetID$s to quantities.

For convenience, here and in the rest of the document, we
will treat values of type $\Value$ also as non-partial functions where
any omitted tokens in the domain are assumed to be zero.

\begin{itemize}
  \item $\PolicyID$ identifies monetary policies. A policy ID $\var{pid}$ is associated with a script
    $s$ such that $\fun{hashScript}~s~=~pid$. When a transaction attempts to create or destroy tokens
    that fall under the policy ID $\var{pid}$,
    $s$ verifies that the transaction
    respects the restrictions that are imposed by the monetary policy.
    See sections \ref{sec:transactions} and \ref{sec:utxo} for details.

  \item $\AssetID$ is a type used to distinguish different tokens with the same $\PolicyID$.
    Each $aid$ identifies a unique kind of token in $\var{pid}$.

  \item $\Quantity$ is an integer type that represents an amount of a specific $\AssetID$. We associate
    a term $q\in\Quantity$ with a specific token to track how much of that token is contained in a given asset value.

  \item $\Value$ is the multi-asset type that is used to represent
    an amount of a collection of tokens, including Ada. If $(\var{pid}, \var{aid})$ is a token and $v \in \Value$,
    the amount in $v$ belonging to that token is $v~\var{pid}~\var{aid}$ if defined, or zero otherwise.
    Token amounts are fungible with each other if and only if they belong to the same token,
    i.e. they have the same $\PolicyID$ and $\AssetID$. Terms of type $\Value$ are sometimes also referred to as
    \emph{token bundles}.

  \item $\mathsf{adaID}$ and $\mathsf{adaToken}$ are the $\PolicyID$ and $\AssetID$ of Ada respectively.

  \item $\fun{coinToValue}$ and $\fun{valueToCoin}$ convert between the two types,
  by taking a $\Coin$ to a $\Value$ that only contains Ada, and extracting the amount of Ada from a $\Value$ respectively.

  \item The $\fun{valueSize}$ function provides a rough estimate of
    the amount of memory the storage of a certain value will consume.
\end{itemize}

\subsection*{Value Operations and Partial Order}
We require some operations on $\Value$, including equality, addition and $\leq$ comparison.

Addition and binary relations are extended pointwise from $\Coin$ to $\Value$, so if $R$ is a binary relation defined on $\Coin$, like $=$ or $\leq$, and $v, w$ are values, we define

\[ v~R~w :\Leftrightarrow \forall~\var{pid}~\var{aid}, (v~\var{pid}~\var{aid})~R~(w~\var{pid}~\var{aid}) \]
\[ (v + w)~\var{pid}~\var{aid} := (v~\var{pid}~\var{aid}) + (w~\var{pid}~\var{aid}). \]

\section{UTxO}
\label{sec:utxo}


\subsection{UTxO Transitions}
\label{sec:utxo-trans}

We have added the following helper functions, which are used in defining the
UTxO transition system, see Figure~\ref{fig:functions:insouts}. These include:

\begin{itemize}
  \item the function $\fun{getOut}$ builds a UTxO-type output out of a
  transaction output
  \item the function $\fun{outs}$ builds the MC UTxO entries from the outputs
  of a transaction
\end{itemize}

For calculating the minimum size of an output, we also need the
function $\fun{valueSize}$ that computes the size of a $\Value$. It is
defined as the size of the serialization of the $\Value$, in analogy
to $\fun{txSize}$.

\begin{figure}[htb]
  \begin{align*}
    & \fun{getOut} \in \TxOut \to \UTxOOut \\
    & \text{tx outputs transformed to UTxO outputs} \\
    & \fun{getOut} ~{txout}~= (\fun{getAddr}~\var{txout}, \fun{getValue}~\var{txout})
    \nextdef
    & \fun{outs} \in \TxBody \to \UTxO \\
    & \text{tx outputs as UTxO} \\
    & \fun{outs} ~\var{txb} =
        \left\{
          (\fun{txid} ~ \var{txb}, \var{ix}) \mapsto \fun{getOut}~\var{txout} ~
          \middle|~
          \var{ix} \mapsto \var{txout} \in \txouts{txb}
        \right\} \\
  \end{align*}
  \caption{Functions on Tx Outputs}
  \label{fig:functions:insouts}
\end{figure}

\textbf{UTxO Helper Functions.}

Figure~\ref{fig:functions:utxo} defines additional calculations needed for the
UTxO transition system with MC:

\begin{itemize}

  \item $\fun{getCoin}$ sums all the Ada in a given output and returns it as a
  $\Coin$ value

  \item
    The $\fun{ubalance}$ function calculates the (aggregated by $\PolicyID$ and
    $\AssetID$) sum total of all the value in a given UTxO.

  \item
    The $\fun{valueSize}$ function estimates an upper bound on the size of a value
    as stored in the UTxO.

  \item The $\fun{consumed}$ calculation is still the sum of the reward address
   value consumed, the values of the UTxO entries consumed,
   and the value consumed from the deposit pot due
   to the transaction collecting deposit refunds. There is an additional
   summand in this calculation, the value forged by a transaction.
   This calculation now returns a $\Value$.

  \item The $\fun{produced}$ calculation sums up the same things
  as the corresponding $\fun{produced}$ calculation in Shelley.
  This calculation also returns a $\Value$.
\end{itemize}

\textbf{Produced and Consumed Calculations and Preservation of Value.}
Note that
the $\fun{consumed}$ and $\fun{produced}$ calculations both produce a $\Value$.
The reason for this is that the outputs of a transaction, as well as UTxO outputs,
are of the $\Value$ type. The administrative amounts (of the $\Coin$ type)
are converted into MC values for these summations.

While the preservation of value is a single
equality, it is really a comparison of token quantities aggregated by
$\AssetID$ and by $\PolicyID$. In particular, ensuring that the produced
amount equals the consumed amount also implies that the total quantity of
Ada tokens is preserved.

\textbf{Forging and the Preservation of Value.}
What does it mean to preserve the value of non-Ada tokens, since they
are put in and taken out of circulation by the users themselves?
This is expressed by including the $\fun{forge}$ value of the transaction
in the preservation of value equation.

The \textit{produced} side of the equation adds up, among other things, the
values in the
outputs that will be added to the ledger UTxO by the transaction. These outputs are
where the
forged value is "put into of circulation", i.e. how it ends up in the UTxO.
Suppose a transaction $tx$ contains a single output $(a, pid \mapsto tkns)$. Suppose
also that it does not
have any inputs spending any UTxO outputs with policy ID $pid$.

A valid transaction $tx$ satisfies the preservation of value
condition by adding the value $pid \mapsto tkns$ to the \textit{consumed} side as well.
To do this, the $tx$ declares that it is forging the tokens $pid \mapsto tkns$
via the $\fun{forge}$ field, i.e. $tx$ must have

\[pid \mapsto tkns\in\fun{forge}~tx\]

The forge field value is then added to the consumed side. This approach
to balancing the POV equation extends
to cases where the transaction might also be consuming some existing $pid$ tokens,
or taking the out of circulation with negative quantities in the forge field.

The forge field value represents the change in total existing tokens of each given currency
as a result of processing the transaction. It is always added to the
\textit{consumed} side of the POV equation because of this side, the signs of the
quantities in the forge field match the signs of the change. That is,
when tokens are added into the UTxO, their quantities are positive, and when they are
taken out of circulation via the forge field, the signs are negative.

Note also that the UTXO rule only checks that the transaction is forging the
amount it has declared using the forge field (and that no Ada is forged).
The forging scripts themselves are not evaluated in this transition rule.
That step is part of witnessing, i.e. the UTXOW rule, see below.

\begin{figure}[htb]
  \emph{Helper Functions}
  \begin{align*}
    & \fun{getCoin} \in \UTxOOut \to \Coin \\
    & \fun{getCoin}~{\var{out}} ~=~\fun{co}~(\var{out}~\mathsf{adaID}~\mathsf{adaToken}) \\
    \nextdef
    & \fun{ubalance} \in \UTxO \to \Value \\
    & \fun{ubalance} ~ utxo = \sum_{\wcard\mapsto\var{u}\in~\var{utxo}}
    \fun{getValue}~\var{u} \\
    & \text{UTxO balance} \\
    \nextdef
    & \fun{valueSize} \in \Value \to \N \\
    & \fun{valueSize}~\var{v} = k + k' * |\{ (\var{pid}, \var{aid}) : \var{v}~\var{pid}~\var{aid} \neq 0
      \land (\var{pid}, \var{aid}) \neq (\mathsf{adaID}, \mathsf{adaToken}) \}|
  \end{align*}
  %
  \emph{Produced and Consumed Calculations}
  \begin{align*}
    & \fun{consumed} \in \PParams \to \UTxO \to \StakeCreds \to \Wdrl \to \TxBody \to \Value \\
    & \consumed{pp}{utxo}{stkCreds}{rewards}~{txb} = \\
    & ~~\ubalance{(\txins{txb} \restrictdom \var{utxo})} + \\
    &~~  \fun{coinToValue}(\fun{wbalance}~(\fun{txwdrls}~{txb})~\\
        &~~+~ \keyRefunds{pp}{stkCreds}{txb}) +
        ~\fun{forge}~\var{txb} \\
    & \text{\emph{-- value consumed}} \\
    \nextdef
    & \fun{produced} \to \PParams \to \StakePools \to \TxBody \to \Value \\
    & \fun{produced}~\var{pp}~\var{stpools}~\var{txb} = \\
    &~~\ubalance{(\fun{outs}~{txb})}  + \fun{coinToValue}(\txfee{txb} \\
    &~~+ \deposits{pp}{stpools}~{(\txcerts{txb})})\\
    & \text{\emph{-- value produced}} \\
  \end{align*}
  \caption{UTxO Calculations}
  \label{fig:functions:utxo}
\end{figure}

\clearpage

\textbf{The UTXO Transition Rule.}
In Figure \ref{fig:rules:utxo-shelley}, we give the UTXO transition rule,
updated for MC support. There are the following changes to the preconditions
of this rule as compared to the original Shelley UTXO rule:

\begin{itemize}
  \item The transaction is not forging any Ada

  \item All outputs of the transaction contain only non-negative quantities
  (this is the $\Value$-type version to the corresponding rule about non-negative
  $\Coin$ amounts in the Shelley ledger rules)

  \item In the preservation of value calculation (which looks the same as in
  Shelly), the value in the $\fun{forge}$ field is taken into account
\end{itemize}

Note that updating the $\UTxO$ with the inputs and the outputs of the transaction
looks the same as in the Shelley rule, however, there is a type-level difference.
Recall that the outputs of a transaction contain a $\Value$ term, rather than
$\Coin$. Moreover, the $\fun{outs}$ map converts $\TxOut$ terms into $\UTxOOut$.


\begin{figure}[htb]
  \begin{equation}\label{eq:utxo-inductive-shelley}
    \inference[UTxO-inductive]
    { \var{txb}\leteq\txbody{tx}
      & \txttl txb \geq \var{slot}
      \\ \txins{txb} \neq \emptyset
      & \minfee{pp}{tx} \leq \txfee{txb}
      & \txins{txb} \subseteq \dom \var{utxo}
      \\
      \consumed{pp}{utxo}{stkCreds}{rewards}~{txb} = \produced{pp}{stpools}~{txb}
      \\
      ~
      \\
      {
        \begin{array}{r}
          \var{slot} \\
          \var{pp} \\
          \var{genDelegs} \\
        \end{array}
      }
      \vdash \var{ups} \trans{\hyperref[fig:rules:update]{up}}{\fun{txup}~\var{tx}} \var{ups'}
      \\
      ~
      \\
      \mathsf{adaID}~\notin \dom~{\fun{forge}~tx} \\
      \forall txout \in \txouts{txb}, ~ \fun{getValue}~txout  ~\geq ~ 0 \\~
      \forall txout \in \txouts{txb}, ~ \fun{getCoin}~txout ~\geq \\
      \fun{valueSize}~(\fun{getValue}~txout) * \fun{minUTxOValue}~pp \\~
      \\
      \fun{txsize}~{tx}\leq\fun{maxTxSize}~\var{pp}
      \\
      ~
      \\
      \var{refunded} \leteq \keyRefunds{pp}{stkCreds}~{txb}
      \\
      \var{depositChange} \leteq
        \fun{totalDeposits}~{pp}~{stpools}~{(\txcerts{txb})} - \var{refunded}
    }
    {
      \begin{array}{r}
        \var{slot}\\
        \var{pp}\\
        \var{stkCreds}\\
        \var{stpools}\\
        \var{genDelegs}\\
      \end{array}
      \vdash
      \left(
      \begin{array}{r}
        \var{utxo} \\
        \var{deposits} \\
        \var{fees} \\
        \var{ups}\\
      \end{array}
      \right)
      \trans{utxo}{tx}
      \left(
      \begin{array}{r}
        \varUpdate{(\txins{txb} \subtractdom \var{utxo}) \cup \fun{outs}{txb}}  \\
        \varUpdate{\var{deposits} + \var{depositChange}} \\
        \varUpdate{\var{fees} + \txfee{txb}} \\
        \varUpdate{ups'}\\
      \end{array}
      \right)
    }
  \end{equation}
  \caption{UTxO inference rules}
  \label{fig:rules:utxo-shelley}
\end{figure}


\clearpage

\textbf{Witnessing.}

We have changed the definition of the function
$\fun{scriptsNeeded}$, see Figure~\ref{fig:functions-witnesses}. There is
now an additional category of scripts that are needed for transaction validation,
the forging scripts.

Note that there are no restrictions on the use of forging scripts. Their hashes may
be used as credentials in UTxO entries, certificates, and withdrawals.
Non-MPS type scripts can also be used for forging, e.g. MSig scripts.

Note also that UTxO entries containing MC tokens, just like Shelley UTxO entries,
can be locked by a script. This script will add an additional set of
restrictions to the use of MC tokens (additional to the forging script
requirements, but enforced at spending time). This output-locking script can itself
also be a forging script.

\begin{figure}[htb]
  \begin{align*}
    & \hspace{-1cm}\fun{scriptsNeeded} \in \UTxO \to \Tx \to
      \powerset{\ScriptHash}
    & \text{required script hashes} \\
    &  \hspace{-1cm}\fun{scriptsNeeded}~\var{utxo}~\var{tx} = \\
    & ~~\{ \fun{validatorHash}~a \mid i \mapsto (a, \wcard) \in \var{utxo},\\
    & ~~~~~i\in\fun{txinsScript}~{(\fun{txins~\var{txb}})}~{utxo}\} \\
    \cup & ~~\{ \fun{stakeCred_{r}}~\var{a} \mid a \in \dom (\AddrRWDScr
           \restrictdom \fun{txwdrls}~\var{txb}) \} \\
      \cup & ~~(\AddrScr \cap \fun{certWitsNeeded}~{txb}) \\
      \cup & ~~\dom~(\fun{forge}~{txb}) \\
      & \where \\
      & ~~~~~~~ \var{txb}~=~\txbody{tx} \\
  \end{align*}
  \caption{Scripts Needed}
  \label{fig:functions-witnesses}
\end{figure}



\clearpage

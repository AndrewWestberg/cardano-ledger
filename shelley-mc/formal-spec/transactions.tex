\begin{figure*}[t!]
  %
  \emph{Derived types}
  %
  \begin{equation*}
    \begin{array}{r@{~\in~}l@{\qquad=\qquad}lr}
      \var{txout} & \TxOut & \Addr \times \hldiff{\Value}
%      & \text{tx outputs}
      \\
    \end{array}
  \end{equation*}
  %
  \emph{Transaction Type}
  %
  \begin{equation*}
    \begin{array}{r@{~~}l@{~~}l@{\qquad}l}
      \var{txbody} ~\in~ \TxBody ~=~
      & \powerset{\TxIn} & \fun{txinputs}& \text{inputs}\\
      &\times ~(\Ix \mapsto \TxOut) & \fun{txouts}& \text{outputs}\\
      & \times~ \seqof{\DCert} & \fun{txcerts}& \text{certificates}\\
       & \times ~\hldiff{\Value}  & \hldiff{\fun{forge}} &\text{value forged}\\
       & \times ~\Coin & \fun{txfee} &\text{non-script fee}\\
       & \times ~\Slot & \fun{txttl} & \text{time to live}\\
       & \times~ \Wdrl  & \fun{txwdrls} &\text{reward withdrawals}\\
       & \times ~\Update  & \fun{txUpdates} & \text{update proposals}\\
       & \times ~\MetaDataHash^? & \fun{txMDhash} & \text{metadata hash}
    \end{array}
  \end{equation*}
  %
  \emph{Accessor Functions}
  \begin{equation*}
    \begin{array}{r@{~\in~}lr}
      \fun{getValue} & \TxOut \to \Value & \text{output value} \\
      \fun{getAddr} & \TxOut \to \Addr & \text{output address}
    \end{array}
  \end{equation*}
  \caption{Type Definitions used in the UTxO transition system}
  \label{fig:defs:utxo-shelley}
\end{figure*}

\section{Transactions}
\label{sec:transactions}

This section describes the changes that are necessary to the
transaction structure to support native multi-asset functionality in
Cardano. The main differences are that $\TxOut$ contains a $\Value$
rather than a $\Coin$, and the $\fun{forge}$ field as part of the
transaction body.

\subsection*{The Forge Field}

The body of a transaction with multi-asset support contains one additional
field, the $\fun{forge}$ field.
The $\fun{forge}$ field is a term of type $\Value$, which contains
tokens the transaction is putting into or taking out of
circulation. Here, by "circulation", we mean specifically "the UTxO on the
ledger". Since the administrative fields cannot contain tokens other than Ada,
and Ada cannot be forged (this is enforced by the UTxO rule, see Figure~\ref{fig:rules:utxo-shelley}),
they are not affected in any way by forging.

Putting tokens into circulation is done with positive values in the $\Quantity$
fields $\fun{forge}$ field, and taking tokens out of circulation can be done
with negative quantities.

A transaction cannot simply forge arbitrary tokens. Restrictions on
Multi-Asset tokens are imposed, for each asset with ID $\var{pid}$, by a script
with the hash $\var{pid}$. Whether a given transaction adheres to the restrictions
prescribed by its script is verified as part of the processing of the transaction.
The forging mechanism is detailed in Section~\ref{sec:utxo}.

\subsection*{Transaction Body}

Besides the addition of the $\fun{forge}$ field to the transaction body,
note that the $\TxOut$ type in the body is not the same as
the $\TxOut$ in the system without multi-asset support. Instead of
$\Coin$, the transaction outputs now have type $\Value$.

The only change to the types related to transaction witnessing is the addition
of monetary policy scripts to the $\Script$ type, so we do not include the whole $\Tx$ type here.

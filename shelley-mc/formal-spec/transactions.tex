\section{Transactions}
\label{sec:transactions}

In this chapter, we outline the changes necessary to transaction and
UTxO type structure to introduce native multicurrency functionality
into the Cardano
system.

\textbf{Representing Multicurrency.}
Some new types (and changes to existing types) are required for
an MC-supporting ledger, see Figure~\ref{fig:defs:utxo-shelley-1}.
The types not appearing in the Figure below are consistent with the
current Shelley ledger design.

\begin{itemize}

  \item $\Language$ is a natural number-valued language tag used to represent
  the type scripting language, e.g.
  $\mathsf{nativeMSigTag}$ and $\mathsf{nativeMCTag}$

  \item $\CurrencyId$ is the identifier of a specific currency. The monetary
  policy of a currency with the ID $cid$ is given by the monetary policy script
  (MPS) $s$, such that $\fun{hashScript}~s~=~cid$. The script is called a
  monetary policy script because when run, it verifies that the tokens of a currency
  with the corresponding $cid$ are according to the restrictions imposed by the
  monetary policy. If the restrictions are complied with, it returns
  $\mathsf{True}$, and
  $\mathsf{False}$ otherwise. More on MPS below.

  \item $\Token$ : for each currency ID $cid$, in the MC model, we can associate
  some tokens with this currency. A term $t\in\Token$ is an identifier of a
  kind of token in a given currency $cid$. Within the currency $cid$,
  tokens of the same type (with the same identifier $t\in\Token$) are fungible
  with each other. That is, they are effectively the "same" token, in the same way that
  every Ada coin is indistinguishable from another Ada coin. Ada
  has the currency ID $\mathsf{adaID}$. The statement \textit{all of the tokens
  of this currency must
  be the same}, in a MC scheme, means that there is only one kind of term of
  type $\Token$ associated
  with $\mathsf{adaID}$, and we call it $\mathsf{adaToken}$.
  See below for a more detailed discussion of Ada representation.

  Note that currencies with different IDs may have tokens with the same names.
  For example, a new currency $\mathsf{NotAda}$ can also have an $\mathsf{adaToken}$.
  These are not the same tokens and are not fungible with each other.

  \item $\Quantity$ is an integer type representing an amount. This type is used
  specifically to represent an amount of a specific $\Token$. We associate
  a term $q\in\Quantity$ with a specific token to keep track of how much of
  this token there is in a given value.

  \item $\Value$ is the multicurrency type used to represent
  all tokens, including Ada. The key of this finite map type is
  the hash of the monetary policy script for this currency.
  A value $v\in\Value$ is a finite map that assciates to each of its
  currency ID keys a finite map $tkns\in\Token \mapsto \Quantity$,
  which represents all the tokens (of that particular
  currency) that are contained in $v$.

  The map $tkns$ associates to each key $t\in\Token$ the quantity of tokens of
  this kind (i.e. with this identifier).

  \item $\TxOut$ : The type of outputs carried by a transaction. This type
  is different from the Shelley $\TxOut$ type because it contains a $\Value$
  instead of a $\Coin$

  \item $\UTxOOut$ is the type of UTxO entry that gets created when a transaction
  output is processed. In this specification, it has the same structure as
  the transaction output $\TxOut$, but we give it a different name, taking into
  account that $\Value$ is stored differently in the outputs of $\UTxO$ and $\Tx$
  (due to optimization in the $\UTxO$)

  \item $\UTxO$ entries are stored in the finite map $\TxIn\mapsto \UTxOOut$
  (this type is also different than the Shelley $\UTxO$ type)

\end{itemize}


\begin{figure*}[htb]
  %
  \emph{Scripts}
  %
  \begin{equation*}
    \begin{array}{r@{~\in~}l@{\qquad=\qquad}lr}

    \end{array}
  \end{equation*}
%
  \emph{Derived types}
  %
  \begin{equation*}
    \begin{array}{r@{~\in~}l@{\qquad=\qquad}lr}
      \var{lng} & \Language & \N \\
      %\text{script language}
      \var{cid} & \CurrencyId & \ScriptHash\\
       % \text{currency ID}\\
      \var{tok} & \Token & \mathbb{H}\\
       % \text{token identifier}\\
      \var{quan} & \Quantity & \Z \\
      %\text{quantity of a token}\\
      \var{val} & \Value
      & \ScriptHash \mapsto (\Token \mapsto \Quantity) \\
%      & \text{a collection of tokens}
      \var{txout}
      & \TxOut
      & \Addr \times \Value
%      & \text{tx outputs}
      \\
      \var{utxoout}
      & \UTxOOut
      & \Addr \times \Value \\
%      & \text{utxo outputs}
      \var{utxo}
      & \UTxO
      & \TxIn \to \UTxOOut
%      & \text{unspent tx outputs}
    \end{array}
  \end{equation*}
  %
  \emph{Language Tags}
  %
  \begin{align*}
    & s_{mc} & ~\in~ \ScriptMPS \\
    & \mathsf{nativeMCTag} & ~=~ \fun{language}~s_{mc}
  \end{align*}
  \caption{Definitions used in the UTxO transition system}
  \label{fig:defs:utxo-shelley-1}
\end{figure*}

\textbf{The Monetary Policy Scripting Language.}
Recall that the ID of a currency is the hash of its MPS.
In Figure~\ref{fig:defs:tx-mc-script}, we give the types related to monetary
policy scripts. The $\Script$ type
is made up of multisig scripts and monetary policy scripts.
The monetary policy script type, $\ScriptMPS$, has several constructors, each of which
is used in a different way during script evaluation (see MPS script evaluation
discussion below).

The abstract
function $\fun{language}$ returns the tag corresponding to
the type of the language of a given script.

\begin{figure*}[htb]
  \emph{Script Types}
  %
  \begin{equation*}
    \begin{array}{r@{~\in~}l@{~}lr}
      \var{smc} & \ScriptMPS & \text{monetary policy script} \\
      \var{scr} & \Script &=~ \ScriptMSig \uniondistinct \ScriptMPS
    \end{array}
  \end{equation*}
    %
  \emph{Helper Functions}
  %
  \begin{align*}
    \fun{language} ~\in~& \Script \to \Language \\
    &\text{returns the language tag, e.g. $\mathsf{mcTag}$ for the MPS language} \\
  \caption{Languages}
  \end{align*}
  %
  \emph{MPS Script definition}
  %
  \begin{equation*}
    \begin{array}{rll}
      \ScriptMPS & \subseteq & \Script \\
      \\~\\
      \var{s_{mc}}\in\ScriptMPS & = & \type{JustMSig}~ \ScriptMSig\\
      & \uniondistinct &
         \type{MaxMinTimedTotal}~(\Token \mapsto (\Slot\times\Slot\times\Quantity\times\Quantity)) \\
      & \uniondistinct&
        \type{RequireAll}~\powerset{\ScriptMPS}
    \end{array}
  \end{equation*}
  \caption{The Multicurrency Scripting Language}
  \label{fig:defs:tx-mc-script}
\end{figure*}

\textbf{Multicurrency Script Evaluation.}
A monetary policy is a collection of restrictions on the tokens of a
specific multicurrency. MP scripts are evaluated for the purpose of checking that
the given currency adheres to its monetary policy. The monetary policy scripting
language is a basic scripting language that
allows for expressing some of the most common restrictions, e.g.
the maximum total number of different kinds of tokens of a given currency.
A suggestion for $\ScriptMPS$ and the implementation of the function
$\fun{evalMPSScript}$, which evaluates MPS scripts, is given in
Appendix~\ref{sec:mps-lang}. As inputs, $\fun{evalMPSScript}$ takes

\begin{itemize}
\item the script getting evaluated
\item the currency ID of the currency being forged
\item the current slot number,
\item a set of key hashes (needed to use MSig scripts as MPS scripts)
\item the transaction body
\item the inputs of the transaction as a UTxO finite map (with addresses and values),
i.e. the outputs it is spending
\end{itemize}


\textbf{MPS Script Validation.}

In the Shelley spec, there is a script validation function which
is used to evaluate all types of native (ledger-rule-defined) scripts.
In Figure~\ref{fig:defs:tx-mc-valid}, we modify this function to also call the
evaluation function specific to the MPS script type being added in this spec.

The arguments passed to the $\fun{validateScript}$ function include all those
needed for MPS and MSig script evaluation. Because of the extra arguments
(the slot number and the UTxO), we also modify the call to this function, made
by the UTXOW rule.

\begin{figure*}[htb]
    \begin{align*}
      \fun{validateScript} & \in\Script\to\ScriptHash\to\Slot\to
      \powerset{\KeyHash}\to\TxBody\to\UTxO\to\Bool \\
      \fun{validateScript} & ~s~\var{cid}~\var{slot}~\var{vhks}
       ~\var{txb}~\var{spentouts} =
                             \begin{cases}
                               \fun{evalMultiSigScript}~s~vhks & \text{if}~s \in\MSig \\
                               \fun{evalMPSScript}~s~\var{cid}~\var{slot}~\var{vhks} \\
                                ~~~~txb~\var{utxo} & \text{if}~s \in\ScriptMPS \\
                               \mathsf{False} & \text{otherwise}
                             \end{cases} \\
    \end{align*}
  \caption{Script Validation}
  \label{fig:defs:tx-mc-valid}
\end{figure*}

\begin{note}
  give right arguments to validateScript in UTXOW rule

  check about validation interval - ok with just ttl?

  Fresh tokens - can’t unforge, quantity 1
\end{note}

\textbf{The Forge Field.}

The body of a transaction with multicurrency support contains one additional
field, the $\fun{forge}$ field (see Figure~\ref{fig:defs:utxo-shelley-2}).
The $\fun{forge}$ field is a term of type $\Value$, which contains
tokens the transaction is putting into circulation or taking out of
circulation. Here, by "circulation", we mean specifically "the UTxO on the
ledger". Since the administrative fields cannot contain tokens other than Ada,
and Ada cannot be forged, they are not affected in any way by forging.

Putting tokens into circulation is done with positive values in the $\Quantity$
fields of the tokens forged, and taking tokens out of circulation can be done
with negative quantities.

A transaction cannot simply forge arbitrary tokens. Recall that restrictions on
MC tokens are imposed, for each currency with ID $cid$, by the script
with the hash $cid$. Whether a given currency adheres to the restrictions
prescribed by its script is verified at forging time (i.e. when the transaction
forging it is being processed). Another restriction on forging is imposed by
the preservation of value conditition. Also, no forging Ada
is permitted. In Section~\ref{sec:utxo}, we specify the mechanism by which
forging is done, and rules which enforce these restrictions.

\textbf{Transaction Body.}

Besides the addition of the $\fun{forge}$ field to the transaction body,
note that the $\TxOut$ type in the body is not the same as
the $\TxOut$ in the system without multicurrency support. Instead of
$\Coin$, the transaction outputs now have type $\Value$.

The only change to the types related to transaction witnessing is the addition
of MPS scripts to the $\Script$ type, so we do not include the whole $\Tx$ type here.

\begin{figure*}[htb]
  \emph{Transaction Type}
  %
  \begin{equation*}
    \begin{array}{r@{~~}l@{~~}l@{\qquad}l}
      \var{txbody} ~\in~ \TxBody ~=~
      & \powerset{\TxIn} & \fun{txinputs}& \text{inputs}\\
      &\times ~(\Ix \mapsto \TxOut) & \fun{txouts}& \text{outputs}\\
      & \times~ \seqof{\DCert} & \fun{txcerts}& \text{certificates}\\
       & \times ~\Value  & \fun{forge} &\text{value forged}\\
       & \times ~\Coin & \fun{txfee} &\text{non-script fee}\\
       & \times ~\Slot & \fun{txttl} & \text{time to live}\\
       & \times~ \Wdrl  & \fun{txwdrls} &\text{reward withdrawals}\\
       & \times ~\Update  & \fun{txUpdates} & \text{update proposals}\\
       & \times ~\MetaDataHash^? & \fun{txMDhash} & \text{metadata hash}\\
    \end{array}
  \end{equation*}
  %
  \emph{Accessor Functions}
  \begin{equation*}
    \begin{array}{r@{~\in~}lr}
      \fun{getValue} & \TxOut \uniondistinct \UTxOOut \to \Value & \text{output value} \\
      \fun{getAddr} & \TxOut \uniondistinct \UTxOOut \to \Addr & \text{output address} \\
    \end{array}
  \end{equation*}
  \caption{Definitions used in the UTxO transition system, cont.}
  \label{fig:defs:utxo-shelley-2}
\end{figure*}

\textbf{Coin and Multicurrency Tokens}
When multicurrency is introduced, Ada is still expected to be
the most common type of token on the ledger.
The $\Coin$ type is used to represent an amount of Ada.
It is the only
type of token which can be used for all non-UTxO ledger accounting, including deposits,
fees, rewards, treasury, and the proof of stake protocol. Under no circumstances
are these administrative fields and calculations ever expected to operate on
any types of tokens besides Ada. These fiels will continue to have the type $\Coin$.

The exact representation of tokens in the UTxO and inside transactions
is an implementation detail, which we omit here.
Note that it necessarily is equivalent to $\Value$, optimized
for Ada-only cases, has a unique representation for Ada tokens,
and does not allow Ada to have tokens denoted by anything other than $\mathsf{adaToken}$.

In Figure \ref{fig:defs:functions-helper} we give the following helper functions
and constants, needed for using Ada in a multicurrency setting.

\begin{itemize}
  \item $\mathsf{adaID}$ is a random script hash value with no known associated
  script. It is the currency ID of Ada. Even if a
  script that hashes to this value
  and validates is found, the UTXO rule forbids forging Ada
  \item $\mathsf{adaToken}$ is a byte string representation of the word "Ada".
  The ledger should never allow the use of any other token name associated
  with Ada's currency ID
  \item $\fun{qu}$ and $\fun{co}$ are type conversions from quantity to
  coin. Both of these types are synonyms for $\Z$, so they are
  type re-naming conversions that are mutual inverses, with

  $\fun{qu} ~(\fun{co} ~q )~= ~q$, and

  $\fun{co}~ (\fun{qu}~ c) ~=~ c$, for $c \in \Coin,~q \in \Quantity$.

  \item $\fun{coinToValue}$ takes a coin value and generates a $\Value$ type representation
  of it
\end{itemize}

An amount of Ada can also be represented as a multicurrency value
using the notation in Figure \ref{fig:defs:functions-helper}, as
$\fun{coinToValue}~c$ where $c \in \Coin$. We must use this representation
when adding or subtracting Ada and other tokens as $\Value$, e.g. in the
preservation of value calculations.

\begin{figure*}[htb]
  \emph{Abstract Functions and Values}
  %
  \begin{align*}
    \mathsf{adaID} \in& ~\ScriptHash
    & \text{Ada currency ID} \\
    \mathsf{adaToken} \in& ~\Token
    & \text{Ada Token} \\
    \mathsf{co} \in& ~\Quantity \to \Coin
    & \text{type conversion} \\
    \mathsf{qu} \in& ~\Coin \to \Quantity
    & \text{type conversion} \\
  \end{align*}
  \emph{Helper Functions}
  %
  \begin{align*}
    \fun{coinToValue} \in & ~\Coin\to \Value \\
    \fun{coinToValue}~ c = & \{\mathsf{adaID} \mapsto \{\mathsf{adaToken} \mapsto \fun{qu}~c\}\} \\
    &\text{convert a Coin amount to a Value} \\
  \end{align*}
  \caption{Multicurrency}
  \label{fig:defs:functions-helper}
\end{figure*}

\clearpage

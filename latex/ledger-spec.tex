\documentclass[11pt,a4paper]{article}
\usepackage[margin=2.5cm]{geometry}
\usepackage{iohk}
\usepackage{microtype}
\usepackage{mathpazo} % nice fonts
\usepackage{amsmath}
\usepackage{amssymb}
\usepackage{amsthm}
\usepackage{latexsym}
\usepackage{mathtools}
\usepackage{stmaryrd}
\usepackage{extarrows}
\usepackage{slashed}
\usepackage[colon]{natbib}
\usepackage[unicode=true,pdftex,pdfa]{hyperref}
\usepackage{xcolor}
\usepackage[capitalise,noabbrev,nameinlink]{cleveref}
\usepackage{float}
\floatstyle{boxed}
\restylefloat{figure}

%%
%% Package `semantic` can be used for writing inference rules.
%%
\usepackage{semantic}
%% Setup for the semantic package
\setpremisesspace{20pt}

%%
%% Types
%%
\newcommand{\Tx}{\type{Tx}}
\newcommand{\Ix}{\type{Ix}}
\newcommand{\TxId}{\type{TxId}}
\newcommand{\Addr}{\type{Addr}}
\newcommand{\UTxO}{\type{UTxO}}
\newcommand{\Value}{\type{Value}}
\newcommand{\Coin}{\type{Coin}}
\newcommand{\PrtclConsts}{\type{PrtclConsts}}
\newcommand{\Slot}{\type{Slot}}
\newcommand{\Duration}{\type{Duration}}
\newcommand{\Allocs}{\type{Allocs}}

\newcommand{\DCert}{\type{DCert}}
\newcommand{\DCertRegKey}{\type{DCert_{regkey}}}
\newcommand{\DCertDeRegKey}{\type{DCert_{deregkey}}}
\newcommand{\DCertDeleg}{\type{DCert_{delegate}}}
\newcommand{\DCertRegPool}{\type{DCert_{regpool}}}
\newcommand{\DCertRetirePool}{\type{DCert_{retirepool}}}
\newcommand{\ledgerState}{\ensuremath{\type{ledgerState}}}

\newcommand{\AddrRWD}{\type{Addr_{rwd}}}
\newcommand{\DState}{\type{DState}}
\newcommand{\DWState}{\type{DWState}}
\newcommand{\DWEnv}{\type{DWEnv}}
\newcommand{\PState}{\type{PState}}
\newcommand{\DCertBody}{\type{DCertBody}}
\newcommand{\DPoolReap}{\ensuremath{\type{poolreap}}}

%% Adding witnesses
\newcommand{\TxIn}{\type{TxIn}}
\newcommand{\TxOut}{\type{TxOut}}
\newcommand{\VKey}{\type{VKey}}
\newcommand{\SKey}{\type{SKey}}
\newcommand{\Hash}{\type{Hash}}
\newcommand{\SkVk}{\type{SkVk}}
\newcommand{\Sig}{\type{Sig}}
\newcommand{\Data}{\type{Data}}
%% Adding delegation
\newcommand{\Epoch}{\type{Epoch}}
\newcommand{\VKeyGen}{\type{VKeyGen}}
%% Blockchain
\newcommand{\Gkeys}{\var{G_{keys}}}
\newcommand{\Block}{\type{Block}}
\newcommand{\SlotId}{\type{SlotId}}
\newcommand{\UTxOEnv}{\type{UTxOEnv}}
\newcommand{\CEEnv}{\type{CEEnv}}
\newcommand{\CEState}{\type{CEState}}
\newcommand{\BDEnv}{\type{BDEnv}}
\newcommand{\BDState}{\type{BDState}}
\newcommand{\LEnv}{\type{LEnv}}
\newcommand{\LState}{\type{LState}}

%%
%% Functions
%%
\newcommand{\txins}[1]{\fun{txins}~ \var{#1}}
\newcommand{\txid}[1]{\fun{txid}~ \var{#1}}
\newcommand{\txouts}[1]{\fun{txouts}~ \var{#1}}
\newcommand{\values}[1]{\fun{values}~ #1}
\newcommand{\balance}[1]{\fun{balance}~ \var{#1}}
\newcommand{\ttl}[1]{\fun{ttl}~ \var{#1}}
\newcommand{\deposits}[2]{\fun{deposits}~ \var{#1} ~ \var{#2}}
\newcommand{\refund}[4]{\fun{refund}~ \var{#1}~ \var{#2}~ \var{#3}~ \var{#4}}
\newcommand{\refunds}[4]{\fun{refunds}~ \var{#1}~ \var{#2}~ \var{#3}~ \var{#4}}
\newcommand{\created}[5]{\fun{created}~ \var{#1}~ \var{#2}~ \var{#3}~ \var{#4}~ \var{#5}}
\newcommand{\destroyed}[2]{\fun{destroyed}~ \var{#1}~ \var{#2}}
\newcommand{\applyFun}[2]{\fun{#1}~\var{#2}}

\newcommand{\RegKey}[1]{\textsc{RegKey}(#1)}
\newcommand{\DeregKey}[1]{\textsc{DeregKey}(#1)}
\newcommand{\Delegate}[1]{\textsc{Delegate}(#1)}
\newcommand{\RegPool}[1]{\textsc{RegPool}(#1)}
\newcommand{\RetirePool}[1]{\textsc{RetirePool}(#1)}
\newcommand{\cauthor}[1]{\fun{author}~ \var{#1}}
\newcommand{\pool}[1]{\fun{pool}~ \var{#1}}
\newcommand{\retire}[1]{\fun{retire}~ \var{#1}}
\newcommand{\addrRw}[1]{\fun{addr_{rwd}}~ \var{#1}}
\newcommand{\epoch}[1]{\fun{epoch}~ \var{#1}}
\newcommand{\dcerts}[1]{\fun{dcerts}~ \var{#1}}

%% UTxO witnesses
\newcommand{\inputs}[1]{\fun{inputs}~ \var{#1}}
\newcommand{\wits}[1]{\fun{wits}~ \var{#1}}
\newcommand{\verify}[3]{\fun{verify} ~ #1 ~ #2 ~ #3}
\newcommand{\sign}[2]{\fun{sign} ~ #1 ~ #2}
\newcommand{\serialised}[1]{\llbracket \var{#1} \rrbracket}
\newcommand{\addr}[1]{\fun{addr}~ \var{#1}}
\newcommand{\hash}[1]{\fun{hash}~ \var{#1}}
\newcommand{\txbody}[1]{\fun{txbody}~ \var{#1}}
\newcommand{\txfee}[1]{\fun{txfee}~ \var{#1}}
\newcommand{\minfee}[2]{\fun{minfee}~ \var{#1}~ \var{#2}}
\newcommand{\slotminus}[2]{\var{#1}~-_{s}~\var{#2}}
\DeclarePairedDelimiter\floor{\lfloor}{\rfloor}
% wildcard parameter
\newcommand{\wcard}[0]{\underline{\phantom{a}}}
%% Adding ledgers...
\newcommand{\utxo}[1]{\fun{utxo}~ #1}
%% Delegation
\newcommand{\delegatesName}{\fun{delegates}}
\newcommand{\delegates}[3]{\delegatesName~#1~#2~#3}
\newcommand{\dwho}[1]{\fun{dwho}~\var{#1}}
\newcommand{\depoch}[1]{\fun{depoch}~\var{#1}}
%% Delegation witnesses
\newcommand{\dbody}[1]{\fun{dbody}~\var{#1}}
\newcommand{\dwit}[1]{\fun{dwit}~\var{#1}}
%% Blockchain
\newcommand{\bwit}[1]{\fun{bwit}~\var{#1}}
\newcommand{\bslot}[1]{\fun{bslot}~\var{#1}}
\newcommand{\bbody}[1]{\fun{bbody}~\var{#1}}
\newcommand{\bdlgs}[1]{\fun{bdlgs}~\var{#1}}
%% ledgerstate constants
\newcommand{\genesisId}{\ensuremath{Genesis_{Id}}}
\newcommand{\genesisTxOut}{\ensuremath{Genesis_{Out}}}
\newcommand{\genesisUTxO}{\ensuremath{Genesis_{UTxO}}}
\newcommand{\emax}{\mathsf{E_{max}}}

\theoremstyle{definition}
\newtheorem{definition}{Definition}[section]

\theoremstyle{definition}
\newtheorem{property}{Property}[section]

\begin{document}

\hypersetup{
  pdftitle={A Formal Specification of the Cardano Ledger},
  breaklinks=true,
  bookmarks=true,
  colorlinks=false,
  linkcolor={blue},
  citecolor={blue},
  urlcolor={blue},
  linkbordercolor={white},
  citebordercolor={white},
  urlbordercolor={white}
}

\title{A Formal Specification of the Cardano Ledger}

\author{Jared Corduan  \\ {\small \texttt{jared.corduan@iohk.io}} \\
   \and Polina Vinogradova \\ {\small \texttt{polina.vinogradova@iohk.io}} \\
   \and Matthias G\"udemann  \\ {\small \texttt{matthias.gudemann@iohk.io}}}

%\date{}

\maketitle

\begin{abstract}
This documents defines the rules for extending a ledger with transactions.
The transactions will affect both UTxO and stake delegation.
It is intended to serve as the specification for random generators of transactions
which adhere to the rules presented here.
\end{abstract}

\section*{List of Contributors}
\label{acknowledgements}

Nicol\'as Arqueros,
Nicholas Clarke,
Duncan Coutts,
Ruslan Dudin,
Sebastien Guillemot,
Vincent Hanquez,
Ru Horlick,
Michael Hueschen,
Philipp Kant,
Jean-Christophe Mincke,
Damian Nadales,
Nicolas Di Prima.


\tableofcontents
\listoffigures

\section{Introduction}
\label{sec:introduction}
This document is a formal specification of the functionality of the ledger
on the blockchain. The blockchain layer of the
protocol and the interaction between the ledger and the blockchain
layer is presented in a separate document, see~\cite{shelley_consensus}. The details of the
background and the larger context
for the design decisions formalized in this document are presented
in~\cite{delegation_design}

In this work,
we present important properties any implementation of the ledger must have.
Specifically, we model the following aspects
of the functionality of the ledger on the blockchain:

\begin{description}
\item[Preservation of value] relationship between the total value of input and
  outputs in a new transaction, and the unspent outputs.
\item[Witnesses] authentication of parts of the transaction data by means of
  cryptographic entities (such as signatures and private keys) contained in
  these transactions.
\item[Delegation] validity of delegation certificates, which delegate
  block-signing rights.
\item[Stake] staking rights associated to an address.
\end{description}

While the blockchain protocol is a reactive system driven by the arrival
of blocks causing updates to the ledger, the formal description is a collection
of rules which is a
static description of what a \textit{valid ledger} is. The specifics of the
semantics we use to define and apply
the rules we present in this document are explained in detail in
\cite{small_step_semantics}. A valid ledger state can only
reached by applying a sequence of inference rules, and any valid ledger state
is reachable by applying some sequence of these rules.

The structure of the rules we give here is such that their application is
deterministic. That is, given a specific initial state and relevant environmental
constants, there is no ambiguity
about which rule should be applied at any given time (i.e. which state
transition is allowed take place). This is an important property which reflects
the reality of the implementation - the blockchain evolves in a particular way
given some user activity and the passage of time, and its behaviour is
never unexpected.


\section{Notation}\label{sec:notation}

The transition system is explained in \cite{small_step_semantics}.

\begin{description}
\item[Powerset] Given a set $\type{X}$, $\powerset{\type{X}}$ is the set of all
  the subsets of $X$.
\item[Sequences] Given a set $\type{X}$, $\seqof{\type{X}}$ is the set of
  sequences having elements taken from $\type{X}$. The empty sequence is
  denoted by $\epsilon$, and given a sequence $\Lambda$, $\Lambda; \type{x}$ is
  the sequence that results from appending $\type{x} \in \type{X}$ to
  $\Lambda$.
\item[Functions] $A \to B$ denotes a \textbf{total function} from $A$ to $B$.
  Given a function $f$ we write $f~a$ for the application of $f$ to argument
  $a$.
\item[Fibre] Given a function $f: A \to B$ and $b\in B$, we write
  $f^{-1}~b$ for the \textbf{fibre} of $f$ at $b$, which is defined by
  $\{a \mid\ f a =  b\}$.
\item[Maps and partial functions] $A \mapsto B$ denotes a \textbf{partial
    function} from $A$ to $B$, which can be seen as a map (dictionary) with
  keys in $A$ and values in $B$. Given a map $m \in A \mapsto B$, notation
  $a \mapsto b \in m$ is equivalent to $m~ a = b$.
\end{description}

\section{Cryptographic primitives}
\label{sec:crypto-primitives}

Figure~\ref{fig:crypto-defs} introduces the cryptographic abstractions used in
this document.

\begin{figure}
  \emph{Abstract types}
  %
  \begin{equation*}
    \begin{array}{r@{~\in~}lr}
      \var{vk} & \SKey & \text{private signing key}\\
      \var{vk} & \VKey & \text{public verifying key}\\
      \var{hk} & \Hash & \text{hash of a key}\\
      \sigma & \Sig  & \text{signature}\\
      \var{d} & \Data  & \text{data}\\
    \end{array}
  \end{equation*}
  \emph{Derived types}
  \begin{equation*}
    \begin{array}{r@{~\in~}lr}
      (sk, vk) & \SkVk & \text{signing-verifying key pairs}
    \end{array}
  \end{equation*}
  \emph{Abstract functions}
  %
  \begin{equation*}
    \begin{array}{r@{~\in~}lr}
      \hash{} & \VKey \to \Hash
      & \text{hash function} \\
      %
      \fun{verify} & \powerset{\left(\VKey \times \Data \times \Sig\right)}
      & \text{verification relation}\\
    \end{array}
  \end{equation*}
  \emph{Constraints}
  \begin{align*}
    & \forall (sk, vk) \in \SkVk,~ m \in \Data,~ \sigma \in \Sig \cdot
      \verify{vk}{m}{\sigma} \iff \sign{sk}{m} = \sigma
  \end{align*}
  \emph{Notation for serialized and verified data}
  \begin{align*}
    & \serialised{x} & \text{serialised representation of } x\\
    & \mathcal{V}_{\var{vk}}{\serialised{m}}_{\sigma} = \verify{vk}{m}{\sigma}
      & \text{shorthand notation for } \fun{verify}
  \end{align*}
  \caption{Cryptographic definitions}
  \label{fig:crypto-defs}
\end{figure}

\section{Serialization}
\label{sec:serialization}

\begin{todo}
  Discuss here serialization and
  \href{https://iohk.myjetbrains.com/youtrack/issue/CDEC-628}{composable
    serialization}
\end{todo}

\section{Delegation}
\label{sec:delegation}
%%
%% Types
%%
\newcommand{\AddrRWD}{\type{Addr_{rwd}}}
\newcommand{\DState}{\type{DState}}
\newcommand{\DWState}{\type{DWState}}
\newcommand{\DWEnv}{\type{DWEnv}}
\newcommand{\PState}{\type{PState}}
\newcommand{\DCertBody}{\type{DCertBody}}

%%
%% Functions
%%
\newcommand{\RegKey}[1]{\textsc{RegKey}(#1)}
\newcommand{\DeregKey}[1]{\textsc{DeregKey}(#1)}
\newcommand{\Delegate}[1]{\textsc{Delegate}(#1)}
\newcommand{\RegPool}[1]{\textsc{RegPool}(#1)}
\newcommand{\RetirePool}[1]{\textsc{RetirePool}(#1)}
\newcommand{\cauthor}[1]{\fun{author}~ \var{#1}}
\newcommand{\pool}[1]{\fun{pool}~ \var{#1}}
\newcommand{\retire}[1]{\fun{retire}~ \var{#1}}
\newcommand{\addrRw}[1]{\fun{addr_{rwd}}~ \var{#1}}

%%
%% Constants
%%
\newcommand{\emax}{\mathsf{E_{max}}}

We briefly describe the motivation and context for delegation.
The full context is contained in \cite{delegation_design}.

For stake to be active in the blockchain protocol,
(i.e. eligible for participation in the leader election)
the associated verification stake key must be registered
and its staking rights must be delegated to an active stake pool.
\footnote{Individuals who wish to participate in the protocol can
register themselves as a stake pool.}

Stake keys are registered (deregistered) through the use of
registration (deregistration) certificates.
Registered stake keys are delegated through the use of delegation certificates.
Finally, stake pools are registered (retired) through the use of
registration (retirement) certificates.

Stake pool retirement is handled a bit differently than stake key deregistration.
Stake keys are considered inactive as soon as a deregistration certificate
is applied to the ledger state.
Stake pool retirement certificates, however, specify the epoch in
which it will retire.

Delegation requires the following to be tracked by the ledger state:
the registered stake keys, the delegation map from registered stake keys to stake
pools, the registered stake pools, and upcoming stake pool retirements.
Additionally, the blockchain protocol rewards eligible stake, and so we must
also include a mapping from active stake keys to rewards.

In \cref{fig:delegation-definitons} we give the delegation primitives,
and in \cref{fig:delegation-transitions} we give the delegation transition rule.

The rules for registering and delegating stake keys are given in \cref{fig:delegation-rules}.
The rules for registering stake pools are given in \cref{fig:pool-rules}.

\begin{note}
  The current rules allow for delegation to a non-existent pool.
  Such stake is not eligible for leader election.
  This allows for a clean separation between the rules in
  \cref{fig:delegation-rules} and \cref{fig:pool-rules}.
  We may, however, later choose to enforce that delegation certificates
  target a registered pool. It would then make sense to remove
  mappings in $\var{delegations}$ when stake pools retire.
\end{note}

%%
%% Figure - Delegation Definitions
%%
\begin{figure}
  \emph{Abstract types}
  %
  \begin{equation*}
    \begin{array}{r@{~\in~}lr}
      a & \AddrRWD & \text{reward address} \\
      epoch & \Epoch & \text{epoch}
    \end{array}
  \end{equation*}
  %
  \emph{Constants}
  \begin{equation*}
    \begin{array}{r@{~\in~}lr}
      \emax & \Epoch & \text{epoch bound on pool retirement}
    \end{array}
  \end{equation*}
  %
  \emph{Delegation Certificate types}
  %
  \begin{equation*}
  \begin{array}{r@{}c@{}l}
    \DCert &=& \DCertRegKey \uniondistinct \DCertDeRegKey \uniondistinct \DCertDeleg \\
                &\hfill\uniondistinct\;& \DCertRegPool \uniondistinct \DCertRetirePool \\
    \RegKey{c} \in \DCert &\iff& c \in \DCertRegKey \\
    \DeregKey{c} \in \DCert &\iff& c \in \DCertDeRegKey \\
    \Delegate{c} \in \DCert &\iff& c \in \DCertDeleg \\
    \RegPool{c} \in \DCert &\iff& c \in \DCertRegPool\\
    \RetirePool{c} \in \DCert &\iff& c \in \DCertRetirePool \\
  \end{array}
  \end{equation*}
  %
  \emph{Abstract functions}
  %
  \begin{equation*}
  \begin{array}{r@{~\in~}lr}
  \fun{hash} & \VKey \to \Hash
  & \text{hashing a key}
  \\
  \fun{addr_{rwd}} & \Hash \to \AddrRWD
  & \text{address of a hashkey}
  \\
  \fun{author} & \DCert \to \Hash
  & \text{certificate author}
  \\
  \fun{pool} & \DCertDeleg \to \Hash
  & \text{pool being delegated to}
  \\
  \fun{retire} & \DCertRetirePool \to \Epoch
  & \text{epoch of pool retirement}
  \end{array}
  \end{equation*}
  %

  \caption{Delegation Definitions}
  \label{fig:delegation-definitons}
\end{figure}

%%
%% Figure - Delegation Transitions
%%
\begin{figure}
  \emph{Delegation States}
  %
  \begin{equation*}
    \begin{array}{l}
    \DState =
    \left(\begin{array}{r@{~\in~}lr}
      \var{stkeys} & \powerset (\Hash) & \text{registered stake keys}\\
      \var{rewards} & \AddrRWD \mapsto \Coin & \text{rewards}\\
      \var{delegations} & \Hash \mapsto \Hash & \text{delegations}\\
    \end{array}\right)
    \\
    \\
    \PState =
    \left(\begin{array}{r@{~\in~}lr}
      \var{stpools} & \Hash \mapsto \DCertRegPool & \text{registered stake pools}\\
      \var{retiring} & \Hash \mapsto \Epoch & \text{retiring stake pools}\\
    \end{array}\right)
    \end{array}
  \end{equation*}
  %
  \emph{Delegation Transitions}
  \begin{equation*}
    \_ \trans{deleg}{\_} \_ \in
      \powerset (\DState \times \DCert \times \DState)
  \end{equation*}
  %
  \begin{equation*}
    \_ \vdash \_ \trans{pool}{\_} \_ \in
      \powerset (\Epoch \times \DState \times \DCert \times \DState)
  \end{equation*}
  %
  \caption{Delegation Transitions}
  \label{fig:delegation-transitions}
\end{figure}

%%
%% Figure - Delegation Rules
%%
\begin{figure}
  \centering
  \begin{equation}\label{eq:deleg-reg}
    \inference[Deleg-Reg]
    {
      \RegKey{c} & \cauthor{c} = hk & hk \notin \var{stkeys}
    }
    {
      \left(
      \begin{array}{r}
        \var{stkeys} \\
        \var{rewards} \\
        \var{delegations}
      \end{array}
      \right)
      \trans{deleg}{\var{c}}
      \left(
      \begin{array}{rcl}
        \var{stkeys} & \union & \{\var{hk}\}\\
        \var{rewards} & \unionoverride & \{\addrRw \var{hk} \mapsto 0\}\\
        \var{delegations}
      \end{array}
      \right)
    }
  \end{equation}

  \begin{equation}\label{eq:deleg-dereg}
    \inference[Deleg-Dereg]
    {
      \DeregKey{c} & \cauthor{c} = hk & hk \in \var{stkeys}
    }
    {
      \left(
      \begin{array}{r}
        \var{stkeys} \\
        \var{rewards} \\
        \var{delegations}
      \end{array}
      \right)
      \trans{deleg}{\var{c}}
      \left(
      \begin{array}{rcl}
        \var{stkeys} & \setminus & \{\var{hk}\}\\
        \{\addrRw \var{hk}\} & \subtractdom & \var{rewards} \\
        \{\var{hk}\} & \subtractdom & \var{delegations}
      \end{array}
      \right)
    }
  \end{equation}

  \begin{equation}\label{eq:deleg-deleg}
    \inference[Deleg-Deleg]
    {
      \Delegate{c} & \cauthor{c} = hk & hk \in \var{stkeys}
    }
    {
      \left(
      \begin{array}{r}
        \var{stkeys} \\
        \var{rewards} \\
        \var{delegations}
      \end{array}
      \right)
      \trans{deleg}{c}
      \left(
      \begin{array}{rcl}
        \var{stkeys} \\
        \var{rewards} \\
        \var{delegations} & \unionoverride & \{\var{hk} \mapsto \pool c\}
      \end{array}
      \right)
    }
  \end{equation}
  \caption{Delegation Inference Rules}
  \label{fig:delegation-rules}
\end{figure}

%%
%% Figure - Pool Rules
%%
\begin{figure}
  \begin{equation}\label{eq:pool-reg}
    \inference[Pool-Reg]
    {
      \RegPool{c} & \cauthor{c} = hk
    }
    {
      \var{cepoch} \vdash
      \left(
      \begin{array}{r}
        \var{stpools} \\
        \var{retiring}
      \end{array}
      \right)
      \trans{pool}{c}
      \left(
      \begin{array}{rcl}
        \var{stpools} & \unionoverride & \{\var{hk} \mapsto c\} \\
        \{\var{hk}\} & \subtractdom & \var{retiring} \\
      \end{array}
      \right)
    }
  \end{equation}


  \begin{equation}\label{eq:pool-ret}
    \inference[Pool-Retire]
    {
    \RetirePool{c} & \cauthor{c} = hk & \var{hk} \in \dom \var{stpools} \\
    \var{e} = \retire{c} & \var{cepoch} < \var{e} < \var{cepoch} + \emax
  }
  {
    \var{cepoch} \vdash
    \left(
      \begin{array}{r}
        \var{stpools} \\
        \var{retiring}
      \end{array}
    \right)
    \trans{pool}{c}
    \left(
      \begin{array}{rcl}
        \var{stpools} \\
        \var{retiring} & \unionoverride & \{\var{hk} \mapsto \var{e}\} \\
      \end{array}
    \right)
  }
  \end{equation}

  \begin{equation}\label{eq:pool-reap}
    \inference[Pool-Reap]
    {
      \var{retired} = \var{retiring}^{-1}~\var{cepoch}
      & \var{retired} \neq \emptyset
    }
    {
      \var{cepoch} \vdash
      \left(
      \begin{array}{r}
        \var{stpools} \\
        \var{retiring}
      \end{array}
      \right)
      \trans{pool}{c}
      \left(
      \begin{array}{rcl}
        \var{retired} & \subtractdom & \var{stpools} \\
        \var{retired} & \subtractdom & \var{retiring} \\
      \end{array}
      \right)
    }
  \end{equation}
  \caption{Pool Inference Rule}
  \label{fig:pool-rules}

\end{figure}

\subsection{Witnesses}
\label{sec:delegation-witnesses}

The rule for certificate witnesses is given in
Figure~\ref{fig:rules:delegationw}. The new definitions introduced in this rule
are given in Figure~\ref{fig:defs:delegationw}.

\begin{figure}
  \emph{Abstract types}
  \begin{equation*}
    \begin{array}{r@{~\in~}lr}
      tx & \Tx & \text{transaction}\\
      cb & \DCertBody & \text{certificate body}\\
    \end{array}
  \end{equation*}
  %
  \emph{Abstract functions}
  \begin{equation*}
    \begin{array}{r@{~\in~}lr}
      \fun{dbody} & \DCert \mapsto \DCertBody
      & \text{body of the delegation certificate}\\
      \fun{dwit} & \DCert \mapsto (\VKey \times \Sig)
      & \text{witness for the delegation certificate}
    \end{array}
  \end{equation*}
  %
  \emph{Delegation Witnesses environment}
  \begin{equation*}
    \DWEnv =
    \left(
      \begin{array}{r@{~\in~}lr}
        \var{tx} & \Tx & \text{transaction}\\
        \var{epoch} & \Epoch & \text{epoch}\\
      \end{array}
    \right)
  \end{equation*}
  %
  \emph{Delegation Witnesses state}
  \begin{equation*}
    \DWState =
    \left(
      \begin{array}{r@{~\in~}lr}
        \var{dstate} & \DState & \text{delegation state}\\
        \var{pstate} & \PState & \text{pool state}\\
      \end{array}
    \right)
  \end{equation*}
  %
  \emph{Delegation Witnesses Transition}
  \begin{equation*}
    \_ \vdash \_ \trans{delegw}{\_} \_ \in
      \powerset (
        \DWEnv \times \DWState \times \DCert \times \DWState)
  \end{equation*}
  \caption{Delegation witnesses definitions}
  \label{fig:defs:delegationw}
\end{figure}

\begin{figure}
  \emph{Delegation with witness rule}
  \begin{equation}
    \label{eq:deleg-witnesses}
    \inference[Deleg-wit]
    { \dwit{c} = (\var{vk_s}, \sigma)
      & \var{dstate} \trans{deleg}{c} \var{dstate'}
      & \var{cepoch} \vdash \var{pstate}
      \trans{pool}{c} \var{pstate'}
      \\ ~ \\
      \verify{vk_s}{\serialised{(\dbody{c},~\txbody \var{tx})}}{\sigma}
    }
    { \left(
      \begin{array}{r}
        \var{tx} \\
        \var{cepoch}
      \end{array}
      \right)
      \vdash
      \left(
      \begin{array}{r}
        \var{dstate} \\
        \var{pstate}
      \end{array}
      \right)
      \trans{delegw}{c}
      \left(
      \begin{array}{rcl}
        \var{dstate'} \\
        \var{pstate'}
      \end{array}
      \right)
    }
  \end{equation}
  \caption{Delegation witnesses inference rules}
  \label{fig:rules:delegationw}
\end{figure}



\section{UTxO}
\label{sec:state-trans-utxo-1}

In order to define the \textit{preservation of value} conditon,
we define the calculations for deposits and refunds in
Figure~\ref{fig:defs:deposits}.

\begin{note}
  We define $\fun{refund}$ by cases on whether or not
  the refunding certificate has a corresponding
  resource creating certificate.
  If our rules are correct, then $\fun{refund}$
  is only called in the case where such a matching
  certificate exists.
\end{note}

The transition rules for unspent outputs are presented in
Figure~\ref{fig:rules:utxo}. The states of the UTxO transition system,
along with their types are defined in Figure~\ref{fig:defs:utxo}.
Functions on these types are defined in Figure~\ref{fig:derived-defs:utxo}.



\begin{figure*}
  \emph{Abstract types}
  \begin{equation*}
    \begin{array}{r@{~\in~}lr}
      pc & \PrtclConsts & \text{protocol constants}
    \end{array}
  \end{equation*}
  %
  \emph{Derived types}
  \begin{equation*}
    \begin{array}{r@{~\in~}l@{\qquad=\qquad}r@{~\in~}lr}
      \var{allocs}
      & \Allocs
      & hkeys \mapsto slot
      & \Hash \to \Slot
      & \text{resource allocations}
    \end{array}
  \end{equation*}
  %
  \emph{Abstract Functions}
  \begin{equation*}
    \begin{array}{r@{~\in~}lr}
      \fun{dvalue} & \PrtclConsts \to \DCert \to \Coin
        & \text{deposit amount of a certificate}\\

      \fun{decay} & \PrtclConsts \to \mathbb{N}\times\mathbb{Q}^{+}
        & \text{decay constants}\\

      \fun{dresource} & \Tx \to \powerset{(\DCertRegKey \uniondistinct \DCertRegPool)}
        & \text{resource allocating certificates}\\

      \fun{dderegister} & \Tx \to \powerset{\DCertDeRegKey}
        & \text{resource releasing certificates}\\

      \fun{dretire} & \Tx \to \powerset{\DCertRetirePool}
        & \text{resource releasing certificates}\\

      \fun{ttl} & \Tx \to \Slot
        & \text{time to live}\\
    \end{array}
  \end{equation*}
  \caption{Definitions used in Deposits}
  \label{fig:defs:deposits}
\end{figure*}

\begin{figure}
  \begin{align*}
      & \fun{deposits} \in \PrtclConsts \to \Tx \to \Coin
      & \text{total deposits for transaction} \\
      & \fun{deposits}~{pc}~{tx} = \sum\limits_{c \in \fun{dresource}~tx} (\fun{d}~pc~c)
      \nextdef
      & \fun{poolAllocs} \in (\Hash \mapsto (\DCertRegPool \times \Slot)) \to \Allocs
      & \text{pool allocations} \\
      & \fun{poolAllocs}~\var{stpool} =
          \{hash \mapsto slot \mid hash \mapsto (\_, slot) \in \var{stpool}\}
      \nextdef
      & \fun{refund} \in \PrtclConsts \to \Allocs \to \Slot \to \DCert \to \Coin
      & \text{total refund for a certificate} \\
      & \refund{allocs}{pc}{slot}{c} =\\
      & \begin{cases}
        0 & \text{if not}~(\fun{releasing}~c)\\
            0 & \text{if}~\cauthor c \notin allocs\\
            \floor*{
              \left(\fun{d}~pc~c\right) \cdot
            \left(d_{\min}+(1-d_{\min})\cdot e^{-\lambda\cdot\delta}\right)}
            & \text{otherwise}
        \end{cases}\\
      &
      \begin{array}{lr@{~=~}l}
        \where &\fun{releasing}~\var{c} & \DeregKey{c} \lor \RetirePool{c}\\
        & d_{\min},~\lambda & \fun{decay}~pc\\
        &\delta & \slotminus{slot}{(allocs~(\cauthor c))}\\
      \end{array}\\
      \nextdef
      & \fun{refunds} \in \PrtclConsts \to \Allocs \to \Allocs \to \Tx \to \Coin
      & \text{total refunds for transaction} \\
      & \refunds{pc}{dallocs}{pallocs}{tx} =\\
      &   \sum\limits_{c \in \fun{dderegister}~tx} \refund{pc}{dallocs}{(\ttl{tx})}{c}\\
      &   ~~~+ \sum\limits_{c \in \fun{dretire}~tx} \refund{pc}{pallocs}{(\retire{c})}{c}
  \end{align*}
  \caption{Functions used in Deposits}
  \label{fig:functions:deposits}
\end{figure}


\begin{figure*}
  \emph{Abstract types}
  %
  \begin{equation*}
    \begin{array}{r@{~\in~}lr}
      \var{txid} & \TxId & \text{transaction id}\\
      %
      ix & \Ix & \text{index}\\
      %
      \var{addr} & \Addr & \text{address}\\
      %
      c & \Coin & \text{currency value}\\
      %
      slot & \Slot & \text{slot}
    \end{array}
  \end{equation*}
  \emph{Derived types}
  %
  \begin{equation*}
    \begin{array}{r@{~\in~}l@{\qquad=\qquad}r@{~\in~}lr}
      \var{txin}
      & \TxIn
      & (\var{txid}, \var{ix})
      & \TxId \times \Ix
      & \text{transaction input}
      \\
      \var{txout}
      & \type{TxOut}
      & (\var{addr}, c)
      & \Addr \times \Coin
      & \text{transaction output}
      \\
      \var{utxo}
      & \UTxO
      & \var{txin} \mapsto \var{txout}
      & \TxIn \mapsto \TxOut
      & \text{unspent tx outputs}
    \end{array}
  \end{equation*}
  %
  \emph{Abstract Functions}
  \begin{equation*}
    \begin{array}{r@{~\in~}lr}
      \txid{} & \Tx \to \TxId & \text{compute transaction id}\\
      %
      \fun{txbody} & \Tx \to \powerset{\TxIn} \times (\Ix \mapsto \TxOut)
                                  & \text{transaction body}\\
      %
      \fun{txfee} & \Tx \to \Coin & \text{transaction fee}\\
      %
      \fun{minfee} & \PrtclConsts \to \Tx \to \Coin & \text{minimum fee}
    \end{array}
  \end{equation*}
  \caption{Definitions used in the UTxO transition system}
  \label{fig:defs:utxo}
\end{figure*}

\begin{figure}
  \begin{align*}
    & \fun{txins} \in \Tx \to \powerset{\TxIn}
    & \text{transaction inputs} \\
    & \txins{tx} = \var{inputs} \where \txbody{tx} = (\var{inputs}, ~\wcard)
    \nextdef
    & \fun{txouts} \in \Tx \to \UTxO
    & \text{transaction outputs as UTxO} \\
    & \fun{txouts} ~ \var{tx} =
      \left\{ (\fun{txid} ~ \var{tx}, \var{ix}) \mapsto \var{txout} ~
      \middle| \begin{array}{l@{~}c@{~}l}
                 (\_, \var{outputs}) & = & \txbody{tx} \\
                 \var{ix} \mapsto \var{txout} & \in & \var{outputs}
               \end{array}
      \right\}
    \nextdef
    & \fun{balance} \in \UTxO \to \Coin
    & \text{UTxO balance} \\
    & \fun{balance} ~ utxo = \sum_{(~\wcard ~ \mapsto (\wcard, ~c)) \in \var{utxo}} c
    \nextdef
    & \fun{created} \in \PrtclConsts \to \UTxO \to \Allocs \to \Allocs \to \Tx \to \Coin
    & \text{value created} \\
    & \created{pc}{utxo}{dallocs}{pallocs}{tx} = \\
    & ~~\balance{(\txins{tx} \restrictdom \var{utxo})} + \refunds{pc}{dallocs}{pallocs}{tx}
    \nextdef
    & \fun{destroyed} \in \PrtclConsts \to \Tx \to \Coin
    & \text{value destroyed} \\
    & \fun{destroyed} ~ pc ~ tx =
      \balance{(\txouts{tx})}  + \txfee{tx} + \deposits{pc}{tx}\\
  \end{align*}

  \begin{align*}
    \var{ins} \restrictdom \var{utxo}
    & = \{ i \mapsto o \mid i \mapsto o \in \var{utxo}, ~ i \in \var{ins} \}
    & \text{domain restriction}
    \\
    \var{ins} \subtractdom \var{utxo}
    & = \{ i \mapsto o \mid i \mapsto o \in \var{utxo}, ~ i \notin \var{ins} \}
    & \text{domain exclusion}
    \\
    \var{utxo} \restrictrange \var{outs}
    & = \{ i \mapsto o \mid i \mapsto o \in \var{utxo}, ~ o \in \var{outs} \}
    & \text{range restriction}
  \end{align*}
  \caption{Functions used in UTxO rules}
  \label{fig:derived-defs:utxo}
\end{figure}

\begin{figure}
  \emph{UTxO environment}
  \begin{equation*}
    \UTxOEnv =
    \left(
      \begin{array}{r@{~\in~}lr}
        \var{slot} & \Slot & \text{current slot}\\
        \var{pc} & \PrtclConsts & \text{protocol constants}\\
        \var{dallocs} & \Allocs & \text{stake key allocations}\\
        \var{pallocs} & \Allocs & \text{stake pool allocations}\\
      \end{array}
    \right)
  \end{equation*}
  %
  \emph{UTxO transitions}
  \begin{equation*}
    \_ \vdash
    \var{\_} \trans{utxo}{\_} \var{\_}
    \subseteq \powerset (\UTxOEnv \times \UTxO \times \Tx \times \UTxO)
  \end{equation*}
  \caption{UTxO transition-system types}
  \label{fig:ts-types:utxo}
\end{figure}

\begin{figure}
  \begin{equation}\label{eq:utxo-inductive}
    \inference[UTxO-inductive]
    { \ttl tx \leq \var{slot}
        & \minfee{pc}{tx} \leq \txfee{tx}
        & \txins{tx} \subseteq \dom \var{utxo}
        \\
        \created{pc}{utxo}{dallocs}{pallocs}{tx}
          = \destroyed{pc}{tx}
    }
    {
      \begin{array}{l}
        \var{slot}\\
        \var{pc}\\
        \var{dallocs}\\
        \var{pallocs}\\
      \end{array}
      \vdash \var{utxo} \trans{utxo}{tx}
      (\txins{tx} \subtractdom \var{utxo}) \cup \txouts{tx}
    }
  \end{equation}
  \caption{UTxO inference rules}
  \label{fig:rules:utxo}
\end{figure}

Rule~\ref{eq:utxo-inductive} specifies under which conditions a transaction can
be applied to a set of unspent outputs, and how the set of unspent output
changes with a transaction:
\begin{itemize}
\item Each input spent in the transaction must be in the set of unspent
  outputs.
\item The balance of the unspent outputs in a transaction (i.e. the total
  amount paid in a transaction) must be equal or less than the amount of spent
  inputs.
\item If the above conditions hold, then the new state will not have the inputs
  spent in transaction $\var{tx}$ and it will have the new outputs in
  $\var{tx}$.
\end{itemize}

\begin{note}
  $\Coin$ is defined as a primitive type, but there is a difference
  between implementing it with $\mathbb{N}$ versus $\mathbb{Z}$.
  Since this is a pure UTxO ledger, $\mathbb{N}$ suffices.
  If, however, $\mathbb{Z}$ is used, then extra validation is required
  to ensure that all $\TxOut$ are non-negative.
  This extra condition would be added to \cref{eq:utxo-inductive}.
\end{note}

\subsection{Properties}
\label{sec:utxo-properties}

\begin{todo}
  Can we prove properties of the transition system of this section? For
  instance we might like to formalize ``double spending'' and prove that these
  rules prevent it. Do we want it?
\end{todo}

\subsection{Witnesses}
\label{sec:witnesses}

The rules for witnesses are presented in Figure~\ref{fig:rules:utxow}.
The definitions used in Rule~\ref{eq:utxo-witness-inductive} are given in
Figure~\ref{fig:defs:utxow}. Note that
Rule~\ref{eq:utxo-witness-inductive} uses the transition relation defined in
Figure~\ref{fig:rules:utxo}. The main reason for doing this is to define
the rules incrementally, modeling different aspects in isolation to keep the
rules as simple as possible. Also note that the $\trans{utxo}{}$ relation could
have been defined in terms of $\trans{utxow}{}$ (thus composing the rules in a
different order). The choice here is arbitrary.

\begin{figure}
  \emph{Abstract functions}
  %
  \begin{equation*}
    \begin{array}{r@{~\in~}lr}
      \fun{wits} & \Tx \to \powerset{(\VKey \times \Sig)}
      & \text{witnesses of a transaction}\\
      \fun{hash_{spend}} & \Addr \mapsto \Hash
      & \text{hash of a spending key in an address}\\
    \end{array}
  \end{equation*}
  \caption{Definitions used in the UTxO transition system with witnesses}
  \label{fig:defs:utxow}
\end{figure}

\begin{figure}
  \begin{align*}
    & \addr{}{} \in \UTxO \to \TxIn \mapsto \Addr & \text{address of an input}\\
    & \addr{utxo} = \{ i \mapsto a \mid i \mapsto (a, \wcard) \in \var{utxo} \} \\
    \nextdef
    & \fun{addr_h} \in \UTxO \to \TxIn \mapsto \Hash & \text{hash of an input address}\\
    & \fun{addr_h}~utxo = \{ i \mapsto h \mid i \mapsto (a, \wcard) \in \var{utxo}
      \wedge a \mapsto h \in \fun{hash_{spend}} \}
  \end{align*}
  \caption{Functions used in rules witnesses}
  \label{fig:derived-defs:utxow}
\end{figure}

\begin{figure}
  \emph{UTxO with witness transitions}
  \begin{equation*}
    \var{\_} \vdash
    \var{\_} \trans{utxow}{\_} \var{\_}
    \subseteq \powerset (\UTxOEnv \times \UTxO \times \Tx \times \UTxO)
  \end{equation*}
  \caption{UTxO with witness transition-system types}
  \label{fig:ts-types:utxow}
\end{figure}

\begin{figure}
  \begin{equation}
    \label{eq:utxo-witness-inductive}
    \inference[UTxO-wit]
    {
      {
        \begin{array}{l}
        \var{slot}\\
        \var{pc}
        \end{array}
      }
      \vdash \var{utxo} \trans{utxo}{tx} \var{utxo'}\\ ~ \\
      & \forall i \in \txins{tx} \cdot \exists (\var{vk}, \sigma) \in \wits{\var{tx}}
      \cdot
      \mathcal{V}_{\var{vk}}{\serialised{\txbody{tx}}}_{\sigma}
      \wedge  \fun{addr_h}~{utxo}~i = \hash{vk}\\
    }
    {
      \begin{array}{l}
        \var{slot}\\
        \var{pc}\\
      \end{array}
      \vdash \var{utxo} \trans{utxow}{tx} \var{utxo'}
    }
  \end{equation}
  \caption{UTxO with witnesses inference rules}
  \label{fig:rules:utxow}
\end{figure}

\begin{figure}
  \emph{Ledger environment}
  \begin{equation*}
    \LEnv =
    \left(
      \begin{array}{r@{~\in~}lr}
        \var{slot} & \Slot & \text{current slot}\\
        \var{pc} & \PrtclConsts & \text{protocol constants}\\
      \end{array}
    \right)
  \end{equation*}
  %
  \emph{Ledger state}
  \begin{equation*}
    \LState =
    \left(
      \begin{array}{r@{~\in~}lr}
        \var{utxo} & \UTxO & \text{UTxO}\\
        \var{dwstate} & \DWState & \text{delegation witnesess state}\\
      \end{array}
    \right)
  \end{equation*}
  %
  \emph{Ledger transitions}
  \begin{equation*}
    \_ \vdash
    \var{\_} \trans{ledger}{\_} \var{\_}
    \subseteq \powerset (\LEnv \times \LState \times \Tx \times \LState)
  \end{equation*}
  \caption{Ledger transition-system types}
  \label{fig:ts-types:ledger}
\end{figure}

\begin{figure}
  \begin{equation}
    \label{eq:ledger}
    \inference[ledger]
    {
      \Gamma = \dcerts{tx}
      & pallocs = \fun{poolAllocs}~stpools\\~\\
      {
        \begin{array}{r}
        slot\\
        pc\\
        stkeys\\
        pallocs\\
        \end{array}
      }
      \vdash \var{utxo} \trans{utxow}{tx} \var{utxo'}\\~\\~\\
      %
      {
        \begin{array}{l}
          tx \\
          slot \\
        \end{array}
      }
      \vdash
      dwstate \trans{delegs}{\Gamma} dwstate'
    }
    {
      \begin{array}{l}
        \var{slot}\\
        \var{pc}\\
      \end{array}
      \vdash
      \left(
        \begin{array}{ll}
          utxo \\
          dwstate \\
        \end{array}
      \right)
      \trans{ledger}{certs}
      \left(
        \begin{array}{ll}
          utxo' \\
          dwstate' \\
        \end{array}
      \right)
    }
  \end{equation}
  \caption{Ledger inference rule}
  \label{fig:rules:ledger}
\end{figure}

\section{Properties}
\label{sec:properties}
In this section we discuss the properties which we want the ledger to have. One
goal is to include these properties in the executable specification for doing
property-based testing or formal verification.

\subsection{Validity of a Ledger State}
\label{sec:valid-ledg-state}

Many properties only make sense when applied to a valid ledger state. In
informal terms, a valid ledger state $l$ can only be reached when starting from
an initial state $l_{0}$ (genesis state) and only executing state transition
rules as specified in Section~\ref{sec:state-trans-utxo-1} for UTxO or
Section~\ref{sec:delegation} for delegation.

\begin{figure}[ht]
  \centering
  \begin{align*}
    \genesisId & \in & \TxId \\
    \genesisTxOut & \in & \TxOut \\
    \genesisUTxO & \coloneqq & \{\genesisId, \emptyset\} \mapsto \genesisTxOut
    \\
    \ledgerState & \in & \left(
                         \begin{array}{c}
                           \UTxO \\
                           \DPState
                         \end{array}
    \right)\\
               && \\
    \fun{getUTxO} & \in & \ledgerState \to \UTxO \\
    \fun{getUTxO} & \coloneqq & (\var{utxo}, \wcard) \to \var{utxo}
  \end{align*}
  \caption{Definitions and Functions for Valid Ledger State}
  \label{fig:valid-ledger}
\end{figure}

In Figure~\ref{fig:valid-ledger} \genesisId{} marks the transaction identifier
of the initial coin distribution, where \genesisTxOut{} represents the initial
UTxO. It should be noted that no corresponding inputs exists, i.e., the
transaction inputs are the empty set for the initial transaction. An element of
\ledgerState{} is a tuple of UTxO and delegation witness state (\DPState).

\begin{definition}[\textbf{Valid Ledger State}]
  \begin{multline*}
    \forall l_{0},\ldots,l_{n} \in \ledgerState, l_{0} =
    \left(
      \begin{array}{c}
        \left\{
        \genesisUTxO
        \right\} \\
        \left(
        \begin{array}{c}
          \emptyset\\
          \emptyset
        \end{array}
        \right)
      \end{array}
    \right)  \\
    \implies \forall 0 < i \leq n, (\exists tx_{i} \in \Tx, l_{i-1}
    \trans{ledger}{tx_{i}} l_{i}) \implies \applyFun{validLedgerState} l_{n}
  \end{multline*}
  \label{def:valid-ledger-state}
\end{definition}

Definition~\ref{def:valid-ledger-state} defines a valid ledger state reachable
from the genesis state via valid UTxO, stake delegation or stake pool
transactions. This gives a constructive rule how to reach a valid ledger state.

\subsection{Ledger Properties}
\label{sec:ledger-properties}

The following properties state the desired features of updating a valid ledger
state.

\begin{property}[\textbf{Preserve Balance Modulo Fee}]
  \begin{multline*}
    \forall \var{l}, \var{l'} \in \ledgerState: \applyFun{validLedgerstate}{l}\\
    \implies \forall \var{tx} \in \Tx, \var{l} \trans{utxow}{tx} \var{l'} \\
    \implies \applyFun{destroyed}{pc utxo stKeys rewards tx} =
    \applyFun{created}{pc stPools tx}
  \end{multline*}
  \label{prop:ledger-properties-1}
\end{property}

Property~\ref{prop:ledger-properties-1} states that for each valid ledger $l$,
if a transaction $tx$ is added to the ledger via the state transition rule
$utxow$ to the new ledger state $l'$, the balance of the UTxOs in $l$ equals the
balance of the UTxOs in $l'$ in the sense that the amount of created value in
$l'$ equals the amount of destroyed value in $l$. This means that the total
amount of value is left unchanged by a transaction.

\begin{property}[\textbf{Preserve Balance Restricted to TxIns in Balance of
    TxOuts}]
  \begin{multline*}
    \forall \var{l}, \var{l'} \in \ledgerState: \applyFun{validLedgerstate}{l}\\
    \implies \forall \var{tx} \in \Tx, \var{l} \trans{utxow}{tx} \var{l'}
    \implies \fun{balance}(\applyFun{txins}{tx} \restrictdom
    \applyFun{getUTxO}{l}) = \fun{balance}(\applyFun{outs}{tx}) +
    \applyFun{txfee}{tx}
  \end{multline*}
  \label{prop:ledger-properties-2}
\end{property}

Property~\ref{prop:ledger-properties-2} states the more detailed relation of the
balances change. For ledgers $l, l'$ and a transaction $tx$ as above, the
balance of the UTxOs of $l$ restricted to those whose domain is in the set of
transaction inputs of $tx$ equals the balance of the transaction outputs of $tx$
minus the transaction fees.

\begin{property}[\textbf{Preserve Outputs of Transaction}]
  \begin{multline*}
    \forall \var{l}, \var{l'} \in \ledgerState: \applyFun{validLedgerstate}{l}\\
    \implies \forall \var{tx} \in \Tx, \var{l} \trans{utxow}{tx} \var{l'}
    \implies \forall \var{out} \in \applyFun{outs}{tx}, out \in
    \applyFun{getUTxO}{l'}
  \end{multline*}
  \label{prop:ledger-properties-3}
\end{property}

Property~\ref{prop:ledger-properties-3} states that for every ledger states
$l, l'$ and transaction $tx$ as above, all output UTxOs of $tx$ are in the UTxO
set of $l'$, i.e., they are now available as unspent transaction output.

\begin{property}[\textbf{Eliminate Inputs of Transaction}]
  \begin{multline*}
    \forall \var{l}, \var{l'} \in \ledgerState: \applyFun{validLedgerstate}{l}\\
    \implies \forall \var{tx} \in \Tx, \var{l} \trans{utxow}{tx} \var{l'}
    \implies \forall \var{in} \in \applyFun{txins}{tx}, in \not\in
    \fun{dom}(\applyFun{getUTxO}{l'})
  \end{multline*}
  \label{prop:ledger-properties-4}
\end{property}

Property~\ref{prop:ledger-properties-4} states that for every ledger states
$l, l'$ and transaction $tx$ as above, all transaction inputs $in$ of $tx$ are
not in the domain of the UTxO set of $l'$, i.e., these are no longer available
to spend.

\begin{property}[\textbf{Completeness and Collision-Freeness of new Transaction
    Ids}]
  \begin{multline*}
    \forall \var{l}, \var{l'} \in \ledgerState: \applyFun{validLedgerstate}{l}\\
    \implies \forall \var{tx} \in \Tx, \var{l} \trans{utxow}{tx} \var{l'}
    \implies \forall utxo' \in \applyFun{outs}{tx}, \var{utxo'} \in
    \applyFun{getUTxO}{l'} \wedge \\(\var{utxo'} = ((\var{txId'}, \wcard) \mapsto
    \wcard) \implies \forall \var{utxo} \in \applyFun{getUTxO}{l}, \var{utxo} =
    ((\var{txId}, \wcard) \mapsto \wcard) \implies \var{txId'} \neq \var{txId}
  \end{multline*}
  \label{prop:ledger-properties-5}
\end{property}

Property~\ref{prop:ledger-properties-5} states that for ledger states $l, l'$
and a transaction $tx$ as above, the UTxOs of $l'$ contain all newly created
UTxOs and the referred transaction id of each new UTxO is not used in the UTxO
set of $l$.

\begin{property}[\textbf{Absence of Double-Spend}]
  \begin{multline*}
    \forall l_{0},\ldots,l_{n} \in \ledgerState, l_{0} =
    \left(
      \begin{array}{c}
        \left\{
        \genesisUTxO
        \right\} \\
        \left(
        \begin{array}{c}
          \emptyset\\
          \emptyset
        \end{array}
        \right)
      \end{array}
    \right) \wedge \applyFun{validLedgerState} l_{n} \\
    \implies \forall 0 < i \leq n, tx_{i} \in \Tx, l_{i-1}
    \trans{ledger}{tx_{i}} l_{i} \wedge \applyFun{validLedgerState} l_{i}
    \\ \implies \forall j < i, \applyFun{txins}{tx_{j}} \cap
    \applyFun{txins}{tx_{i}} = \emptyset
  \end{multline*}
  \label{prop:ledger-properties-no-double-spend}
\end{property}

Property~\ref{prop:ledger-properties-no-double-spend} states that for each valid
ledger state $l_{n}$ reachable from the genesis state, each transaction $t_{i}$
does not share any input with any previous transaction $t_{j}$. This means that
each output of a transition is spent at most once.

\subsection{Ledger State Properties for Delegation Transitions}
\label{sec:ledg-prop-deleg}

\begin{figure}[ht]
  \centering
  \begin{align*}
    \fun{getStKeys} & \in & \ledgerState \to \powerset \HashKey \\
    \fun{getStKeys} & \coloneqq & (\wcard, (\var{stKeys}, \wcard, \wcard),
                                  \wcard) \to \var{stkeys} \\
                    &&\\
    \fun{getRewards} & \in & \ledgerState \to \AddrRWD \mapsto \Coin \\
    \fun{getRewards} & \coloneqq & (\wcard, (\wcard, \var{rewards}, \wcard),
                                   \wcard) \to \var{rewards} \\
                    &&\\
    \fun{getDelegations} & \in & \ledgerState \to \HashKey \mapsto \HashKey \\
    \fun{getDelegations} & \coloneqq & (\wcard, (\wcard, \wcard,
                                       \var{delegations}), \wcard) \to
                                       \var{delegations} \\
                    &&\\
    \fun{getStPools} & \in & \ledgerState \to \HashKey \mapsto \DCertRegPool \\
    \fun{getStPools} & \coloneqq & (\wcard, \wcard,
                                   (\var{stpools}, \wcard)) \to \var{stpools} \\
                    &&\\
    \fun{getRetiring} & \in & \ledgerState \to \HashKey \mapsto \Epoch \\
    \fun{getRetiring} & \coloneqq & (\wcard, \wcard,
                                    (\wcard, \var{retiring})) \to \var{retiring} \\
  \end{align*}
  \caption{Definitions and Functions for Stake Delegation in Ledger States}
  \label{fig:stake-delegation-functions}
\end{figure}


\begin{property}[\textbf{Registered Staking Key with Zero Rewards}]
  \begin{multline*}
    \forall \var{l}, \var{l'} \in \ledgerState: \applyFun{validLedgerstate}{l}\\
    \implies \forall \var{c} \in \DCertRegKey, \var{l} \trans{delegw}{c} \var{l'}
    \implies \applyFun{author}{c} = \var{hk}\\ \implies
    \var{hk} \in \applyFun{getStKeys}{l'} \wedge
    (\applyFun{getRewards}var{rewards})[hk] = 0
  \end{multline*}
  \label{prop:ledger-properties-6}
\end{property}

Property~\ref{prop:ledger-properties-6} states that for each valid ledger state
$l$, if a delegation transaction of type $\DCertRegKey$ is executed, then in the
resulting ledger state $l'$, the set of staking keys of $l'$ includes the key
$hk$ associated with the key registration certificate and the associated reward
is set to 0 in $l'$.

\begin{property}[\textbf{Deregistered Staking Key}]
  \begin{multline*}
    \forall \var{l}, \var{l'} \in \ledgerState: \applyFun{validLedgerstate}{l}\\
    \implies \forall \var{c} \in \DCertDeRegKey, \var{l} \trans{delegw}{c} \var{l'}
    \implies \applyFun{author}{c} = \var{hk}\\ \implies
    \var{hk} \not\in \applyFun{getStKeys}{l'} \wedge
    (\fun{dom}(\applyFun{getRewards}{l'}) \cup
    \fun{dom}(\applyFun{getDelegations}{l'})) \cap \{hk\} = \emptyset
  \end{multline*}
  \label{prop:ledger-properties-7}
\end{property}

Property~\ref{prop:ledger-properties-7} states that for $l, l'$ as above but
with a delegation transition of type $\DCertDeRegKey$, the staking key $hk$
associated with the deregistration certificate is not in the set of staking keys
of $l'$ and is not in the domain of neither the rewards nor the delegation map
of $l'$.

\begin{property}[\textbf{Delegated Stake}]
  \begin{multline*}
    \forall \var{l}, \var{l'} \in \ledgerState: \applyFun{validLedgerstate}{l}\\
    \implies \forall \var{c} \in \DCertDeleg, \var{l} \trans{delegw}{c} \var{l'}
    \implies \applyFun{author}{c} = \var{hk}\\ \implies
    \var{hk} \in \applyFun{getStKeys}{l} \wedge
    (\applyFun{getDelegations}{l'})[hk] = \applyFun{pool}{c}
  \end{multline*}
  \label{prop:ledger-properties-8}
\end{property}

Property~\ref{prop:ledger-properties-8} states that for $l, l'$ as above but
with a delegation transition of type $\DCertDeleg$, the staking key $hk$
associated with the deregistration certificate is in the set of staking keys of
$l$ and delegates to the staking pool associated with the delegation
certificate in $l'$.

\subsection{Ledger State Properties for Staking Pool Transitions}
\label{sec:ledg-state-prop}

\begin{property}[\textbf{Registered Staking Pool}]
  \begin{multline*}
    \forall \var{l}, \var{l'} \in \ledgerState: \applyFun{validLedgerstate}{l}\\
    \implies \forall \var{c} \in \DCertRegPool, \var{l} \trans{pool}{c} \var{l'}
    \implies \applyFun{author}{c} = \var{hk}\\ \implies
    (\applyFun{getStPools}{l'})[\var{hk}] = c \wedge \var{hk} \not\in
    \applyFun{getRetiring}{l'}
  \end{multline*}
  \label{prop:ledger-properties-9}
\end{property}

Property~\ref{prop:ledger-properties-9} states that for $l, l'$ as above but
with a delegation transition of type $\DCertRegPool$, the key $hk$ is associated
with the author of the pool registration certificate in $\var{stpools}$ of $l'$
and that $hk$ is not in the set of retiring stake pools in $l'$.

\begin{property}[\textbf{Start Staking Pool Retirement}]
  \begin{multline*}
    \forall \var{l}, \var{l'} \in \ledgerState, \var{cepoch} \in \Epoch:
    \applyFun{validLedgerstate}{l} \\
    \implies \forall \var{c} \in \DCertRetirePool, \var{l} \trans{pool}{c} \var{l'}
    \\ \implies e = \applyFun{retire}{c} \wedge
    \var{cepoch} < e < \var{cepoch} + \emax \wedge \applyFun{author}{c} =
    \var{hk}\\ \implies (\applyFun{getRetiring}{l'})[\var{hk}] = e \wedge
    \var{hk} \in \fun{dom}(\applyFun{getStPools}{l})
  \end{multline*}
  \label{prop:ledger-properties-10}
\end{property}

Property~\ref{prop:ledger-properties-10} states that for $l, l'$ as above but
with a delegation transition of type $\DCertRetirePool$, the key $hk$ is
associated with the author of the pool registration certificate in
$\var{stpools}$ of $l'$ and that $hk$ is not in the set of retiring stake pools
in $l'$.

\begin{property}[\textbf{Stake Pool Reaping}]
  \begin{multline*}
    \forall \var{l}, \var{l'} \in \ledgerState, \var{cepoch} \in \Epoch:
    \applyFun{validLedgerstate}{l} \\
    \implies \var{l} \trans{poolreap}{} \var{l'} \implies \forall \var{retire} =
    retiring^{-1} cepoch, retired \neq \emptyset \\ \wedge \var{retire}
    \subseteq \fun{dom}(\applyFun{getStPool}{l}) \wedge
    \var{retire} \cap \fun{dom}(\applyFun{getStPool}{l'}) = \emptyset \\
    \wedge \var{retire} \subseteq \fun{dom}(\applyFun{getRetiring}{l}) \wedge
    \var{retire} \cap \fun{dom}(\applyFun{getRetiring}{l'}) = \emptyset
  \end{multline*}
  \label{prop:ledger-properties-11}
\end{property}

Property~\ref{prop:ledger-properties-11} states that for $l, l'$ as above but
with a delegation transition of type $\DPoolReap$, there exist registered stake
pools in $l$ which are associated to stake pool registration certificates and
which are to be retired at the current epoch $\var{cepoch}$. In $l'$ all those
stake pools are removed from the maps $stpools$ and $retiring$.


\subsection{Properties of Numerical Calculations}
\label{sec:prop-numer-calc}

The numerical calculations for refunds and rewards calculation in
(see Section~\ref{sec:epoch}) are also required to have certain properties. In
particular we need to make sure that the functions that use non-integral
arithmetic have properties which guarantee consistency of the system. Here, we
state those properties and formulate them in a way that makes them possible to
use properties-based testing for validation in the executable spec.

\begin{property}[\textbf{Minimal Refund}]
  \label{prop:minimal-refund}

  The function $\fun{refund}$ takes a value, a minimal percentage, a decay
  parameter and a duration. It must guarantee that the refunded amount is within
  the minimal refund (off-by-one for rounding / floor) and the original value.

  \begin{multline*}
    \forall d_{val} \in \mathbb{N}, d_{min} \in [0,1], \lambda \in (0, \infty),
    \delta \in \mathbb{N} \\
    \implies \max(0,d_{val}\cdot d_{min} - 1) \leq \floor*{d_{val}\cdot(d_{min} +
      (1-d_{min})\cdot e^{-\lambda\cdot\delta})} \leq d_{val}
  \end{multline*}
\end{property}


\begin{property}[\textbf{Exponential Moving Average}]
  \label{prop:exponential-moving-average}

  The function $\fun{movingAvg}$ calculates the exponential moving average,
  dividing the number of blocks created by the pool by the expected number of
  slots the pool is elected leader (or 1 if the expected number is below 1). It
  guarantees that the result is (i) non-negative and (ii) if a previous moving
  average has already been calculated, the new moving average lies between the
  minimum and maximum of the old and new calculated value. With
  $current := \frac{n}{\max(\overline{N}, 1)}$ this is trivial for (i), for (ii)
  it is

  \begin{multline*}
    \forall \alpha \in [0, 1], n \in \mathbb{N}, \overline{N} \in \Rnn, prev \in
    \Rnn\\
    \implies 0\leq \min(prev, current) \leq
    \alpha\cdot current + (1 - \alpha)\cdot prev
    \leq \max(prev, current)
  \end{multline*}
\end{property}

\begin{property}[\textbf{Actual Reward}]
  \label{prop:actual-reward}

  The actual reward for a stake pool in an epoch is calculated by the function
  $\fun{poolReward}$. The actual reward per stake pool is non-negative and
  bounded by the maximal reward for the stake pool, with $avg$ being the
  calculated moving average of the stake pool and $maxP$ being the maximal
  reward for the stake pool, we get:

  \begin{equation*}
    \forall \gamma \in [0,1] \implies 0\leq \floor*{avg^{\gamma}\cdot maxP} \leq maxP
  \end{equation*}
\end{property}

\begin{todo}
  The property~(\ref{prop:actual-reward}) requires that $avg \in [0,1]$, else
  the actual reward can exceed the maximal reward. This is not true, we need to
  take into account the rewards for all stake pools.
\end{todo}


%%% Local Variables:
%%% mode: latex
%%% TeX-master: "ledger-spec"
%%% End:


\addcontentsline{toc}{section}{References}
\bibliographystyle{plainnat}
\bibliography{references}

\end{document}

\section{Cryptographic primitives}
\label{sec:crypto-primitives-shelley}

Figure~\ref{fig:crypto-defs-shelley} introduces the cryptographic abstractions used in
this document. we begin by listing the abstract types, which are meant to
represent the corresponding concepts in cryptography. Only the functionality
explicitly stated in the figures below is assumed within the scope of this paper.
That is, their exact
implementation remains open to interpretation, and we do not rely on
any additional properties derived from the study or implementation of public key
cryptography outside this work. The types and rules we give here are needed in
order to guarantee certain security properties of the delegation process, which
we discuss later.

The cryptographic concepts required for the formal definition
of witnessing include public-private key pairs, one-way functions
and signatures. The constraint we introduce states that a signature of
some data signed with a (private) key is only correct whenever we can verify
it using the corresponding public key.

Besides basic cryptographic abstractions, we also make use of some abstract
data storage properties in this document in order to build necessary definitions
and make judgement calls about them.

Abstract data types in this paper are essentially placeholders with names
indicating the data types they are meant to represent in an implementation.
Derived types are made up of data structures (i.e.~products, lists, finite
maps, etc.) built from abstract types. The underlying structure of a data type
is implementation-dependent, and furthermore, the way the data is stored on
physical storage can vary as well.

Serialization is a physical manifestation of data on a given storage device.
In this document, the properties and rules we state involving serialization are
assumed to hold true independently of the storage medium and style of data
organization chosen for an implementation.

\begin{figure}[htb]
  \emph{Abstract types}
  %
  \begin{equation*}
    \begin{array}{r@{~\in~}lr}
      \var{sk} & \SKey & \text{private signing key}\\
      \var{vk} & \VKey & \text{public verifying key}\\
      \var{hk} & \HashKey & \text{hash of a key}\\
      \sigma & \Sig  & \text{signature}\\
      \var{d} & \Data  & \text{data}\\
    \end{array}
  \end{equation*}
  \emph{Derived types}
  \begin{equation*}
    \begin{array}{r@{~\in~}lr}
      (sk, vk) & \KeyPair & \text{signing-verifying key pairs}
    \end{array}
  \end{equation*}
  \emph{Abstract functions}
  %
  \begin{equation*}
    \begin{array}{r@{~\in~}lr}
      \hashKey{} & \VKey \to \HashKey
                 & \text{hashKey function} \\
                 %
      \fun{verify} & \powerset{\left(\VKey \times \Data \times \Sig\right)}
                   & \text{verification relation}\\
                   %
      \fun{sign} & \SKey \to \Data \to \Sig
                 & \text{signing function}\\
    \end{array}
  \end{equation*}
  \emph{Constraints}
  \begin{align*}
    & \forall (sk, vk) \in \KeyPair,~ d \in \Data,~ \sigma \in \Sig \cdot
    \sign{sk}{d} = \sigma \implies \verify{vk}{d}{\sigma}
  \end{align*}
  \emph{Notation for serialized and verified data}
  \begin{align*}
    & \serialised{x} & \text{serialised representation of } x\\
    & \mathcal{V}_{\var{vk}}{\serialised{d}}_{\sigma} = \verify{vk}{d}{\sigma}
    & \text{shorthand notation for } \fun{verify}
  \end{align*}
  \caption{Cryptographic definitions}
  \label{fig:crypto-defs-shelley}
\end{figure}

\clearpage


We briefly describe the motivation and context for delegation.
The full context is contained in \cite{delegation_design}.

Stake is said to be \textit{active} in the blockchain protocol
when it is eligible for participation in the leader election. In order for
stake to become active,
the associated verification stake key must be registered
and its staking rights must be delegated to an active stake pool.
Individuals who wish to participate in the protocol can
register themselves as a stake pool.

Stake keys are registered (deregistered) through the use of
registration (deregistration) certificates.
Registered stake keys are delegated through the use of delegation certificates.
Finally, stake pools are registered (retired) through the use of
registration (retirement) certificates.

Stake pool retirement is handled a bit differently than stake key deregistration.
Stake keys are considered inactive as soon as a deregistration certificate
is applied to the ledger state.
Stake pool retirement certificates, however, specify the epoch in
which it will retire.

Delegation requires the following to be tracked by the ledger state:
the registered stake keys, the delegation map from registered stake keys to stake
pools, pointers associated with stake keys,
the registered stake pools, and upcoming stake pool retirements.
Additionally, the blockchain protocol rewards eligible stake, and so we must
also include a mapping from active stake keys to rewards.

\subsection{Delegation Definitions}
\label{sec:deleg-defs}

In \cref{fig:delegation-definitons} we give the delegation primitives.
Here we introduce the following primitive datatypes used in delegation:

\begin{itemize}
\item reward addresses: different from base addresses introduced later
\item indexes: this type is used to index certificates in a transaction,
index of a transaction inside a slot, outputs inside a transaction,
and inputs in a UTxO or a transaction
\item epochs
\item slot numbers
\item duration: the difference between two slot numbers
\item $\StakePool$: constants found in a stake pool registration certificate
(we must continue to keep track of these after registration is complete)
\end{itemize}

The constant $\emax$ gives the number of epochs a stake pool will take to retire.
The type $\DCert$ is a generic certificate type, which can be a registration,
deregistration or delegation certificate for a key, or a registration/retirement
 certificate for a stake pool. It is denoted as disjoint union in the figure,
one should, however, think of a term of this type as a term of a specific
one of these five subtypes.

Note that the reason for combining the different types of
certificates into a common type is that it allows us to use that type to later
define a single type for all
ledger state transitions having to do with delegation (i.e. $\DState$
transitions),
and, in a very similarly
way, a type for transitions describing stake pool-related ledger updates
(i.e. $\PState$ transitions).

The type $\Allocs$ is a general datatype used to represent both individual and stake
pool resource allocations. It stores, as a finite map, the hash key to which the resource is
allocated, and the corresponding slot number in which this allocation
(registration) was made. Here we also introduce the pointer structure
$\Ptr$, which we will use to index stake keys based on
when and by what transaction they were registered.

A variable of this type consists of
the slot number when transaction (which carried the key registration certificate)
was processed, the index of the transaction inside that slot, and the index
of the certificate inside the transaction (i.e.\ its position in the list of
the transaction's certificates).

The maps $\fun{hash}$, $\fun{addr_{rwd}}$, $\fun{author}$, $\fun{dpool}$,
$\fun{stakepool}$,
$\fun{retire}$, $\fun{epoch}$, and $\fun{slot}$ are all used to
retrieve specific information about the origin type. The subtraction $\slotminus{}{}$
of slots gives the number of slots in between (referred to here as
$\Duration$ here by $\var{dur}$).

%%
%% Figure - Delegation Definitions
%%
\begin{figure}
  \emph{Abstract types}
  %
  \begin{equation*}
    \begin{array}{r@{~\in~}lr}
      ix & \Ix & \text{index}\\
      a & \AddrRWD & \text{reward address} \\
      slot & \Slot & \text{absolute slot}\\
      dur & \Duration & \text{duration}\\
      epoch & \Epoch & \text{epoch} \\
      stakepool & \StakePool & \text{stake pool constants}
    \end{array}
  \end{equation*}
  %
  \emph{Constants}
  \begin{equation*}
    \begin{array}{r@{~\in~}lr}
      \slotsPer & \mathbb{N} & \text{slots per epoch} \\
      \emax & \Epoch & \text{epoch bound on pool retirement} \\
    \end{array}
  \end{equation*}
  %
  \emph{Delegation Certificate types}
  %
  \begin{equation*}
  \begin{array}{r@{}c@{}l}
    \DCert &=& \DCertRegKey \uniondistinct \DCertDeRegKey \uniondistinct \DCertDeleg \\
                &\hfill\uniondistinct\;& \DCertRegPool \uniondistinct \DCertRetirePool
  \end{array}
  \end{equation*}
  %
  \emph{Derived types}
  \begin{equation*}
    \begin{array}{r@{~\in~}l@{\qquad=\qquad}lr}
      \var{allocs}
      & \Allocs
      & \HashKey \mapsto \Slot
      & \text{resource allocations} \\
      \var{(s,t,c)}
      & \Ptr
      & \Slot*\Ix*\Ix
      & \text{certificate pointer}
    \end{array}
  \end{equation*}
  %
  \emph{Abstract functions}
  %
  \begin{equation*}
  \begin{array}{r@{~\in~}lr}
  \fun{addr_{rwd}} & \HashKey \to \AddrRWD
  & \text{address of a hashKey}
  \\
  \fun{author} & \DCert \to \HashKey
  & \text{certificate author}
  \\
  \fun{txSlotIx} & \Slot \to \Tx \to \Ix
  & \text{tx index in slot}
  \\
  \fun{cIx} & \DCert \to \Ix
  & \text{certificate index in tx}
  \\
  \fun{dpool} & \DCertDeleg \to \HashKey
  & \text{pool being delegated to}
  \\
  \fun{stakepool} & \DCertRegPool \to \StakePool
  & \text{stake pool constants}
  \\
  \fun{retire} & \DCertRetirePool \to \Epoch
  & \text{epoch of pool retirement}
  \\
  \fun{epoch} & \Slot \to \Epoch
  & \text{epoch of a slot}
  \\
  \fun{slot} & \Epoch \to \Slot
  & \text{first slot of an epoch}
  \\
    (\slotminus{}{}) & \Slot \to \Slot \to \Duration
  & \text{slot duration}
  \end{array}
  \end{equation*}
  %

  \caption{Delegation Definitions}
  \label{fig:delegation-definitons}
\end{figure}

\clearpage

\subsection{Delegation Transitions}
\label{sec:deleg-trans}


In \cref{fig:delegation-transitions} we give the delegation and stake pool
state transition types. We define two separate parts of the ledger state.
The part of the ledger state keeping track of current delegations, $\DState$,
consists of four variables. The first, $\var{stkeys}$, keeps track of individual
stake key resource allocations using the $\Allocs$ type.

Note that for security, privacy, and usability reasons, the staking (delegating)
key pair associated with an address must be different from its paying key pair.
\textit{Every address} must have a paying key \textit{and} a stake key
associated to it in order to participate in using the currency. Before the
stake key is registered and delegates to an existing stake pool,
the money at its address can be used for
transactions, but does not accumulate additional stake (i.e. rewards/money).
Once the stake key is registered and delegating, either the base address or
the pointer address can be used to generate outputs
in a transaction. However, it makes more sense to use the shorter pointer
address at this point.

Note also that individual key registration
and deregistration certificates contain no important data beyond a declaration
of registration, and thus are not stored as part of the state variables once
they have been processed.

The second, $\var{rewards}$, stores the
rewards accumulated by stake keys. These are represented by
a finite
map that matches addresses with the coin value of the rewards belonging to the
key to which this address is associated. The third
variable in $\DState$ stores currently registered delegations.
Delegations ($\var{delegations}$) are also expressed by a finite map, which
associates a stake key with the hash key of the pool to which it delegates.
Finally, the finite map $\var{ptrs}$ variable stores,
indexed by pointers,
the stake keys associated with these pointers. This data is needed for the
epoch boundary reward calculations, see Section~\ref{sec:epoch}.

The ledger additionally keeps track of the stake pools in a separate state variable
$\PState$.
This state contains a list of registered stake pools, $\var{stpools}$.
These are again stored as pairs of type $\Allocs$. The stake pool certificates
do contain useful data that must be stored. However, they are stored in a
separate variable in order to allow us to use a signle datatype for
all deposit and refund calculations for both individual and pool key registrations
and deregistrations or retirement. The variable that stores other necessary pool
registration data is $\var{pparams}$, and is a finite map that stores $\StakePool$
constants indexed by the hash keys to which they correspond.
Finally, $\var{avgs}$ stores the latest value of the pool's moving average
used in the reward calculation, see Section~\ref{sec:epoch}. This value
quantifies, as a fraction, the desirability of delegating to the given pool, calculated
based on past performance.

The state $\PState$ also keeps track of
stake pools scheduled to retire via the variable $\var{retiring}$,
which associates
a stake pool hash key with the epoch in which it is supposed to retire.

The environment for state transitions for both $\DState$ and $\PState$ contains
only the current slot number. The $\DState$ transition DELEG as well as
the $\PState$ transition POOL are both triggered by a
certificate (contained in a signal transaction).


%%
%% Figure - Delegation Transitions
%%
\begin{figure}
  \emph{Delegation States}
  %
  \begin{equation*}
    \begin{array}{l}
    \DState =
    \left(\begin{array}{r@{~\in~}lr}
      \var{stkeys} & \Allocs & \text{registered stake keys to creation time}\\
      \var{rewards} & \AddrRWD \mapsto \Coin & \text{rewards}\\
      \var{delegations} & \HashKey \mapsto \HashKey & \text{delegations}\\
      \var{ptrs} & \Ptr \mapsto \HashKey & \text{pointer to hashkey}\\
    \end{array}\right)
    \\
    \\
    \PState =
    \left(\begin{array}{r@{~\in~}lr}
      \var{stpools} & \Allocs & \text{registered pools to creation time}\\
      \var{pparams} & \HashKey \mapsto \StakePool
        & \text{registered pools to pool parameters}\\
      \var{retiring} & \HashKey \mapsto \Epoch & \text{retiring stake pools}\\
      \var{avgs} & \HashKey \mapsto [0, 1] & \text{performance moving average}\\
    \end{array}\right)
    \end{array}
  \end{equation*}
  %
  \emph{Delegation Transitions}
  \begin{equation*}
    \_ \vdash \_ \trans{deleg}{\_} \_ \in
      \powerset (\Slot \times \DState \times \DCert \times \DState)
  \end{equation*}
  %
  \begin{equation*}
    \_ \vdash \_ \trans{pool}{\_} \_ \in
      \powerset (\Slot \times \PState \times \DCert \times \PState)
  \end{equation*}
  %
  \caption{Delegation Transitions}
  \label{fig:delegation-transitions}
\end{figure}

\clearpage

\subsection{Delegation Rules}
\label{sec:deleg-rules}


The rules for registering and delegating stake keys are given in
\cref{fig:delegation-rules}. The preconditions for the registration of a stake
key ensure that a given certificate $c$ is of the correct type
(i.e. $\DCertRegKey$),
and that the hash key associated with the author of the certificate is not
already found in the current list of stake keys.

In order to delegate to a stake pool, a stake key must first be registered.
When registering a new stake key (the \cref{eq:deleg-reg} inference rule),

\begin{itemize}
\item the key must be
added to the set of stake keys ($\var{stkeys}$),
\item the rewards for the address
corresponding to that key set to 0,
\item the pointers consisting of the current slot number, the
index of the transaction in this slot, and the index of the certificate, is
added to the $\var{pointers}$ finite map, mapping to the new key
\item no new delegations are added
\end{itemize}

Note that
since the hash key corresponding to the address must not have been previously
registered,
it should not have any rewards in its associated address. Thus, it is safe
to set the rewards to 0 using union override (i.e. replace any value previously
associated with this address with 0).

When deregistering a key (the \cref{eq:deleg-dereg} rule), we again
require that the certificate is of the correct type. We also require
that the key of the author is indeed a registered stake key
in order to be able to retrieve its address for the rewards update.

As a result of this rule, the author's key must be removed from the $\var{stkeys}$ list,
and all the rewards, pointers, and delegations associated with this key must be removed
from the $\var{rewards}$, $\var{pointers}$, and $\var{delegations}$ parameters as well.

Note that we must only allow performing a key deregistration when there
are no active stake pools associated with this key. This check requires additional
environment variables. For this reason
it (and several other checks that need
additional environment variables to specify a valid delegation state transition)
are performed in the DELEGT rule, discussed in Section~\ref{sec:utxo-rewards}.

Finally, for creating a delegation (the \cref{eq:deleg-deleg} rule),
given that the certificate $c$ is of the correct type, we add to the
$\var{delegations}$ finite map the pair of
the author's hash key and hash key of the pool being delegated to.
Again, we require the author's key be a registered key, as it does not make
sense to allow delegation otherwise.

There is another check which we perform when processing a delegation certificate:
we must check that the certificate is for delegating to an
existing stake pool. This, again, is verified in the DELEGT rule, see
Section~\ref{sec:utxo-rewards}.

The $\var{stkeys}$, $\var{rewards}$, and $\var{pointers}$ parameters are kept constant by
this rule. The use of union override here allows us to use the same rule
to perform an update on an existing delegation while keeping the rewards
associated with the key accounted for.

The rules of the distribution of rewards to eligible stake holders are described
in Section~\ref{sec:epoch}, since the amount of rewards they can claim in the next
epoch is calculated at the epoch boundary. However, the claiming of the rewards
(i.e.\ conversion of reward value to unspent outputs available to the key
holder), can be requested by the key holder at any time as part of a transaction.
This process is decribed in Section~\ref{sec:utxo-rewards}.

We would like to point out, here, that this document does not describe
how the wallet makes a decision about which stake pool a stake key will
delegate to. This decision, however, is influenced by some parameters in the
protocol. These parameters determine which stake pools are more profitable
to delegate to, as well as the optimal number of stake pools in the system,
by means of regulating reward distribution.
This avoids forming a monopoly of a single large stake pool constantly
being delegated to.


%%
%% Figure - Delegation Rules
%%
\begin{figure}
  \centering
  \begin{equation}\label{eq:deleg-reg}
    \inference[Deleg-Reg]
    {
      \var{c}\in\DCertRegKey & \cauthor{c} = hk \\
      hk \notin \dom \var{stkeys} & ptr = (slot,~\fun{txSlotIx}~\var{tx},~\fun{cIx}~\var{c})
    }
    {
      \var{slot} \vdash
      \left(
      \begin{array}{r}
        \var{stkeys} \\
        \var{rewards} \\
        \var{delegations} \\
        \var{ptrs}
      \end{array}
      \right)
      \trans{deleg}{\var{c}}
      \left(
      \begin{array}{rcl}
        \varUpdate{\var{stkeys}} & \varUpdate{\union} & \varUpdate{\{\var{hk} \mapsto slot\}} \\
        \varUpdate{\var{rewards}} & \varUpdate{\unionoverrideRight} & \varUpdate{\{\addrRw \var{hk} \mapsto 0\}}\\
        \var{delegations} \\
        \varUpdate{\var{ptrs}} & \varUpdate{\union} & \varUpdate{\{ptr \mapsto \var{hk}\}}
      \end{array}
      \right)
    }
  \end{equation}

  \begin{equation}\label{eq:deleg-dereg}
    \inference[Deleg-Dereg]
    {
      \var{c}\in \DCertDeRegKey  & \cauthor{c} = hk \\
    hk \in \dom \var{stkeys} & ptr = (slot,\fun{txSlotIx}~\var{tx}, \fun{cIx}~\var{c})
    }
    {
      \var{slot} \vdash
      \left(
      \begin{array}{r}
        \var{stkeys} \\
        \var{rewards} \\
        \var{delegations} \\
        \var{ptrs}
      \end{array}
      \right)
      \trans{deleg}{\var{c}}
      \left(
      \begin{array}{rcl}
        \varUpdate{\{\var{hk}\}} & \varUpdate{\subtractdom} & \varUpdate{\var{stkeys}} \\
        \varUpdate{\{\addrRw \var{hk}\}} & \varUpdate{\subtractdom} & \varUpdate{\var{rewards}} \\
        \varUpdate{\{\var{hk}\}} & \varUpdate{\subtractdom} & \varUpdate{\var{delegations}} \\
        \varUpdate{\{ptr\}} & \varUpdate{\subtractdom} & \varUpdate{\var{ptrs}} \\
      \end{array}
      \right)
    }
  \end{equation}

  \begin{equation}\label{eq:deleg-deleg}
    \inference[Deleg-Deleg]
    {
      \var{c}\in \DCertDeleg & \cauthor{c} = hk & hk \in \dom \var{stkeys}
    }
    {
      \var{slot} \vdash
      \left(
      \begin{array}{r}
        \var{stkeys} \\
        \var{rewards} \\
        \var{delegations} \\
        \var{ptrs}
      \end{array}
      \right)
      \trans{deleg}{c}
      \left(
      \begin{array}{rcl}
        \var{stkeys} \\
        \var{rewards} \\
        \varUpdate{\var{delegations}} & \varUpdate{\unionoverrideRight}
                                      & \varUpdate{\{\var{hk} \mapsto \dpool c\}} \\
        \var{ptrs}
      \end{array}
      \right)
    }
  \end{equation}
  \caption{Delegation Inference Rules}
  \label{fig:delegation-rules}
\end{figure}

\clearpage

\subsection{Stake Pool Rules}
\label{sec:pool-rules}


The rules for updating the part of the ledger state defining the current stake
pools are given in \cref{fig:pool-rules}. The calculation of stake distribution
is described in Section~\ref{sec:stake-dist}.

In the pool rules, the stake key is used directly to register a stake pool.
For each rule, again, we first check that a given certificate $c$ is of
the correct type.

The first rule, \cref{eq:pool-reg}, can only be used to register a stake pool
with a new key (to which no stake pool was previously registered).
When this rule is invoked,
 the author's key and current slot number are added to $\var{stpools}$, and the
key and $\var{stakepool}$ constants in the certificate are added to the $\var{pparams}$
finite map.

The second rule, \cref{eq:pool-rereg}, is used to re-register a
stake pool with a new certificate ($\var{c}$), or stop the retirement
process for that key. It does not update the slot number recorded previously
for the registration of this key, but it does update the associated pool constants
in $\var{pparams}$. The author's hash key and associated data
are also removed from the $\var{retiring}$ parameter to cancel any pending
retirement for that key.

The third rule, \cref{eq:pool-ret}, starts the pool retirement process. Given a
slot number $\var{slot}$, the application of this rule requires that the
planned retirement epoch $\var{e}$ stated in the certificate is in the future,
i.e. after
$\var{cepoch}$, the epoch of the current slot number in this context, as well as
that it is
less than $\emax$ epochs after the current one. Another precondition for this
rule is that
the certificate author's key is in the set of the registered pools' keys.
This rule simply adds the pair of the
certificate author's key and the planned retirement epoch $e$ to the $\var{retiring}$
parameter, overrwiting any existing retirement schedule for the key.

Note that imposing the $\emax$ constraint on the system is not strictly necessary.
However, forcing stake pools to announce their retirement a shorter time in
advance will curb the growth of the $\var{retiring}$ list in the ledger state
(this is a finite map used to keep track of what stake pool is retiring when,
discussed in more detail later).
Allowing pools to make retirement announcements arbitrarily far in advance
could result in a linear (in the number of pools) growth of storage space needed
for this datatype.

The pools scheduled for retirement must be removed from
the $\var{retiring}$ state variable at the end of the epoch they are scheduled
to retire in. This non-signaled transition (triggered, instead, directly by a
change of current slot number in the environment), along with all other transitions
that take place at the epoch boundary, are described in Section~\ref{sec:epoch}.


%%
%% Figure - Pool Rules
%%
\begin{figure}
  \begin{equation}\label{eq:pool-reg}
    \inference[Pool-Reg]
    {
      \var{c}\in\DCertRegPool
      & \cauthor{c} = \var{hk}
      & hk \notin \dom \var{stpools}
      & \stakepool{c} = \var{stakepool}
    }
    {
      \var{slot} \vdash
      \left(
      \begin{array}{r}
        \var{stpools} \\
        \var{pparams} \\
        \var{retiring} \\
        \var{avgs}
      \end{array}
      \right)
      \trans{pool}{c}
      \left(
      \begin{array}{rcl}
        \varUpdate{\var{stpools}} & \varUpdate{\union}
                                  & \varUpdate{\{\var{hk} \mapsto \var{slot}\}} \\
        \varUpdate{\var{pparams}} & \varUpdate{\union}
                                  & \varUpdate{\{\var{hk} \mapsto \var{stakepool}\}} \\
       \var{retiring} \\
       \var{avgs}
      \end{array}
      \right)
    }
  \end{equation}

  \begin{equation}\label{eq:pool-rereg}
    \inference[Pool-reReg]
    {
      \var{c}\in\DCertRegPool
      & \cauthor{c} = \var{hk}
      & hk \in \dom \var{stpools}
      & \stakepool{c} = \var{stakepool}
    }
    {
      \var{slot} \vdash
      \left(
      \begin{array}{r}
        \var{stpools} \\
        \var{pparams} \\
        \var{retiring} \\
        \var{avgs}
      \end{array}
      \right)
      \trans{pool}{c}
      \left(
      \begin{array}{rcl}
        \var{stpools} \\
        \varUpdate{\var{pparams}} & \varUpdate{\unionoverrideRight}
                                  & \varUpdate{\{\var{hk} \mapsto \var{stakepool}\}}\\
        \varUpdate{\{\var{hk}\}} & \varUpdate{\subtractdom} & \varUpdate{\var{retiring}} \\
        \var{avgs}
      \end{array}
      \right)
    }
  \end{equation}

  \begin{equation}\label{eq:pool-ret}
    \inference[Pool-Retire]
    {
    \var{c} \in \DCertRetirePool
    & \cauthor{c} = hk
    & \var{hk} \in \dom \var{stpools} \\
    \var{e} = \retire{c}
    & \var{cepoch} = \epoch{slot}
    & \var{cepoch} < \var{e} < \var{cepoch} + \emax
  }
  {
    \var{slot} \vdash
    \left(
      \begin{array}{r}
        \var{stpools} \\
        \var{pparams} \\
        \var{retiring} \\
        \var{avgs}
      \end{array}
    \right)
    \trans{pool}{c}
    \left(
      \begin{array}{rcl}
        \var{stpools} \\
        \var{pparams} \\
        \varUpdate{\var{retiring}} & \varUpdate{\unionoverrideRight}
                                   & \varUpdate{\{\var{hk} \mapsto \var{e}\}} \\
        \var{avgs}
      \end{array}
    \right)
  }
  \end{equation}

  \caption{Pool Inference Rule}
  \label{fig:pool-rules}

\end{figure}


Next, we give the state, environment, and transition type for the full delegation
state transition, which includes the delegations ($\DState$) and the stake pools
($\PState$), see
Figure~\ref{fig:defs:delpl}.

\begin{figure}
  \emph{Delegation and Pool Combined Environment}
  \begin{equation*}
    \DWEnv =
    \left(
      \begin{array}{r@{~\in~}lr}
        \var{tx} & \Tx & \text{transaction}\\
        \var{slot} & \Slot & \text{slot}\\
      \end{array}
    \right)
  \end{equation*}
  %
  \emph{Delegation and Pool Combined State}
  \begin{equation*}
    \DWState =
    \left(
      \begin{array}{r@{~\in~}lr}
        \var{dstate} & \DState & \text{delegation state}\\
        \var{pstate} & \PState & \text{pool state}\\
      \end{array}
    \right)
  \end{equation*}
  %
  \emph{Delegation and Pool Combined Transition}
  \begin{equation*}
    \_ \vdash \_ \trans{delpl}{\_} \_ \in
      \powerset (
        \DWEnv \times \DWState \times \DCert \times \DWState)
  \end{equation*}
  \caption{Delegation and Pool Combined Transition Type}
  \label{fig:defs:delpl}
\end{figure}


In the following Figure~\ref{fig:rules:delpl}, we give the rules for
defining a valid state transition for the combined delegation and pool
state, $\DWState$, signaled by a certificate. There are two separate rules,
one where only the
delegation state changes, and one where only the pool state changes.
This is because a certificate can never trigger a transition of both kinds,
and can only be applied to \textit{either} register (deregister) a key,
\textit{or} to register (retire) a pool.

\begin{figure}
  \emph{Delegation and Pool Combined Rules}
  \begin{equation}
    \label{eq:delpl-d}
    \inference[Delpl-Del]
    {
      & \var{slot} \vdash \var{dstate} \trans{deleg}{c} \var{dstate'}
    }
    { \left(
      \begin{array}{r}
        \var{tx} \\
        \var{slot}
      \end{array}
      \right)
      \vdash
      \left(
      \begin{array}{r}
        \var{dstate} \\
        \var{pstate}
      \end{array}
      \right)
      \trans{delpl}{c}
      \left(
      \begin{array}{rcl}
        \varUpdate{\var{dstate'}} \\
        \var{pstate}
      \end{array}
      \right)
    }
  \end{equation}
    \begin{equation}
    \label{eq:delpl-p}
    \inference[Delpl-Pool]
    {
    & \var{slot} \vdash \var{pstate} \trans{pool}{c} \var{pstate'}
    }
    { \left(
      \begin{array}{r}
        \var{tx} \\
        \var{slot}
      \end{array}
      \right)
      \vdash
      \left(
      \begin{array}{r}
        \var{dstate} \\
        \var{pstate}
      \end{array}
      \right)
      \trans{delpl}{c}
      \left(
      \begin{array}{rcl}
        \var{dstate} \\
        \varUpdate{\var{pstate'}}
      \end{array}
      \right)
    }
  \end{equation}
  \caption{Delegation and Pool Combined Transition Rules}
  \label{fig:rules:delpl}
\end{figure}


Now, since a transaction can carry more than one certificate at a time, we need
to give the specifics of processing the entire list of certificates (contained
in a transaction).

We may do so by defining a new type of transition, where the signal type is a
list of certificates, as illustrated in Figure~\ref{fig:rules:delegation-sequence}.
The transition itself is defined recursively, so that
an empty list results in a state transition to itself, and given
a list $\Gamma$ of certificates signaling a valid DELEGS transition, as well
as a DELPL transition signaled by certificate $c$, the transition
composed of these two transitions in sequence, signaled by the list $c; \Gamma$
is also valid.

This definition guarantees a certificate list (and therefore, the transaction
carrying it) cannot be processed unless every
certificate in it is valid. For example, if a transaction is carrying a
certificate that schedules a pool retirement in a past epoch, the
whole transaction will be invalid.


\begin{figure}
\begin{equation*}
  \_ \vdash \_ \trans{delegs}{\_} \_ \in
    \powerset (
      \DWEnv \times \DWState \times (\text{list~} \DCert) \times \DWState)
\end{equation*}
\caption{Delegation with a certificate list transition type}
\label{fig:type:delegations}
\end{figure}


\begin{figure}
  \begin{equation}
    \inference[Seq-delg-base]
    {}
    {
      \begin{array}{r}
        \var{tx}\\
        \var{slot}
      \end{array}
      \vdash
      \var{dwstate}
      \trans{delegs}{\epsilon}
      \var{dwstate}
    }
    \label{eq:rule:sequence-delegation-base}
  \end{equation}

  \begin{equation}
    \inference[Seq-delg-ind]
    {
      {
        \begin{array}{r}
          \var{tx}\\
          \var{slot}\\
        \end{array}
      }
      \vdash
      \var{dwstate}
      \trans{delegs}{\Gamma}
      \var{dwstate'}
    \\ ~ \\
    {
      \begin{array}{r}
        \var{tx}\\
        \var{slot}\\
      \end{array}
    }
    \vdash
      \var{dwstate'}
      \trans{delpl}{c}
      \var{dwstate''}
    }
    {
    {
      \begin{array}{r}
        \var{tx}\\
        \var{slot}\\
      \end{array}
    }
    \vdash
      \var{dwstate}
      \trans{delegs}{\Gamma; c}
      \varUpdate{\var{dwstate''}}
    }
    \label{eq:rule:sequence-delegation-inductive}
  \end{equation}
  \caption{Delegation sequence rules}
  \label{fig:rules:delegation-sequence}
\end{figure}

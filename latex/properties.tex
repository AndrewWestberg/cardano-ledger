In this section we discuss the properties which we want the ledger to have. One
goal is to include these properties in the executable specification for doing
property-based testing or formal verification.

\subsection{Validity of a Ledger State}
\label{sec:valid-ledg-state}

Many properties only make sense when applied to a valid ledger state. In
informal terms, a valid ledger state $l$ can only be reached when starting from
an initial state $l_{0}$ (genesis state) and only executing state transition
rules as specified in Section~\ref{sec:state-trans-utxo-1} for UTxO or
Section~\ref{sec:delegation} for delegation.

\begin{figure}[ht]
  \centering
  \begin{align*}
    \genesisId & \in & \TxId \\
    \genesisTxOut & \in & \TxOut \\
    \genesisUTxO & \coloneqq & \{\genesisId, \emptyset\} \mapsto \genesisTxOut
    \\
    \ledgerState & \in & \left(
                         \begin{array}{c}
                           \UTxO \\
                           \DWState
                         \end{array}
    \right)\\
               && \\
    \fun{getUTxO} & \in & \ledgerState \to \UTxO \\
    \fun{getUTxO} & \coloneqq & (\var{utxo}, \wcard) \to \var{utxo}
  \end{align*}
  \caption{Definitions and Functions for Valid Ledger State}
  \label{fig:valid-ledger}
\end{figure}

In Figure~\ref{fig:valid-ledger} \genesisId{} marks the transaction identifier
of the initial coin distribution, where \genesisTxOut{} represents the initial
UTxO. It should be noted that no corresponding inputs exists, i.e., the
transaction inputs are the empty set for the initial transaction. An element of
\ledgerState{} is a triplet of UTxO, stake delegation state (\DState) and
delegation pool state (\PState).

\begin{definition}[\textbf{Valid Ledger State}]
  \begin{multline*}
    \label{eq:2}
    \forall l_{0},..,l_{n} \in \ledgerState, l_{0} =
    \left(
      \begin{array}{c}
        \left\{
        \genesisUTxO
        \right\} \\
        \left(
        \begin{array}{c}
          \emptyset\\
          \emptyset
        \end{array}
        \right)
      \end{array}
    \right)  \\
    \implies \forall 0 < i \leq n, (\exists certs \in \Tx, l_{i-1}
    \trans{ledger}{c} l_{i}) \implies \applyFun{validLedgerState}(l_{n})
  \end{multline*}
  \label{def:valid-ledger-state}
\end{definition}

Definition~\ref{def:valid-ledger-state} defines a valid ledger state reachable
from the genesis state via valid UTxO, stake delegation or stake pool
transactions. This gives a constructive rule how to reach a valid ledger state.

\subsection{Ledger Properties}
\label{sec:ledger-properties}

The following properties state the desired features of updating a valid ledger
state.

\begin{property}[\textbf{Preserve Balance Modulo Fee}]
  \begin{multline*}
    \forall \var{l}, \var{l'} \in \ledgerState: \applyFun{validLedgerstate}{l}\\
    \implies \forall \var{tx} \in \Tx, \var{l} \trans{utxow}{tx} \var{l'} \\
    \implies \fun{balance}(\applyFun{getUTxO}{l}) +
    \fun{refunds}~\var{pc}~\var{dallocs}~\var{pallocs}~\var{tx} =
    \fun{balance}(\applyFun{getUTxO}{l'}) + \applyFun{txfee}{tx} +
    \fun{refunds}~\var{pc}~\var{tx}
  \end{multline*}
  \label{prop:ledger-properties-1}
\end{property}

Property~\ref{prop:ledger-properties-1} states that for each valid ledger $l$,
if a transaction $tx$ is added to the ledger via the state transition rule
$utxow$ to the new ledger state $l'$, the balance of the UTxOs in $l$ equals the
balance of the UTxOs in $l'$ minus the transaction fees.

\begin{property}[\textbf{Preserve Balance Restricted to TxIns in Balance of
    TxOuts}]
  \begin{multline*}
    \forall \var{l}, \var{l'} \in \ledgerState: \applyFun{validLedgerstate}{l}\\
    \implies \forall \var{tx} \in \Tx, \var{l} \trans{utxow}{tx} \var{l'}
    \implies \fun{balance}(\applyFun{txins}{tx} \restrictdom
    \applyFun{getUTxO}{l}) = \fun{balance}(\applyFun{txouts}{tx}) +
    \applyFun{txfee}{tx}
  \end{multline*}
  \label{prop:ledger-properties-2}
\end{property}

Property~\ref{prop:ledger-properties-2} states the more detailed relation of the
balances change. For ledgers $l, l'$ and a transaction $tx$ as above, the
balance of the UTxOs of $l$ restricted to those whose domain is in the set of
transaction inputs of $tx$ equals the balance of the transaction outputs of $tx$
minus the transaction fees.

\begin{property}[\textbf{Preserve Outputs of Transaction}]
  \begin{multline*}
    \forall \var{l}, \var{l'} \in \ledgerState: \applyFun{validLedgerstate}{l}\\
    \implies \forall \var{tx} \in \Tx, \var{l} \trans{utxow}{tx} \var{l'}
    \implies \forall \var{out} \in \applyFun{txouts}{tx}, out \in
    \applyFun{getUTxO}{l'}
  \end{multline*}
  \label{prop:ledger-properties-3}
\end{property}

Property~\ref{prop:ledger-properties-3} states that for every ledger states
$l, l'$ and transaction $tx$ as above, all output UTxOs of $tx$ are in the UTxO
set of $l'$, i.e., they are now available as unspent transaction output.

\begin{property}[\textbf{Eliminate Inputs of Transaction}]
  \begin{multline*}
    \forall \var{l}, \var{l'} \in \ledgerState: \applyFun{validLedgerstate}{l}\\
    \implies \forall \var{tx} \in \Tx, \var{l} \trans{utxow}{tx} \var{l'}
    \implies \forall \var{in} \in \applyFun{txins}{tx}, in \not\in
    \fun{dom}(\applyFun{getUTxO}{l'})
  \end{multline*}
  \label{prop:ledger-properties-4}
\end{property}

Property~\ref{prop:ledger-properties-4} states that for every ledger states
$l, l'$ and transaction $tx$ as above, all transaction inputs $in$ of $tx$ are
not in the domain of the UTxO set of $l'$, i.e., these are no longer available
to spend.

\begin{property}[\textbf{Completeness and Collision-Freeness of new Transaction
    Ids}]
  \begin{multline*}
    \forall \var{l}, \var{l'} \in \ledgerState: \applyFun{validLedgerstate}{l}\\
    \implies \forall \var{tx} \in \Tx, \var{l} \trans{utxow}{tx} \var{l'}
    \implies \forall utxo' \in \applyFun{txouts}{tx}, \var{utxo'} \in
    \applyFun{getUTxO}{l'} \wedge \\(\var{utxo'} = ((\var{txId'}, \wcard) \mapsto
    \wcard) \implies \forall \var{utxo} \in \applyFun{getUTxO}{l}, \var{utxo} =
    ((\var{txId}, \wcard) \mapsto \wcard) \implies \var{txId'} \neq \var{txId}
  \end{multline*}
  \label{prop:ledger-properties-5}
\end{property}

Property~\ref{prop:ledger-properties-5} states that for ledger states $l, l'$
and a transaction $tx$ as above, the UTxOs of $l'$ contain all newly created
UTxOs and the referred transaction id of each new UTxO is not used in the UTxO
set of $l$.

\subsection{Ledger State Properties for Delegation Transitions}
\label{sec:ledg-prop-deleg}

\begin{figure}[ht]
  \centering
  \begin{align*}
    \fun{getStKeys} & \in & \ledgerState \to \powerset \Hash \\
    \fun{getStKeys} & \coloneqq & (\wcard, (\var{stKeys}, \wcard, \wcard),
                                  \wcard) \to \var{stkeys} \\
                    &&\\
    \fun{getRewards} & \in & \ledgerState \to \AddrRWD \mapsto \Coin \\
    \fun{getRewards} & \coloneqq & (\wcard, (\wcard, \var{rewards}, \wcard),
                                   \wcard) \to \var{rewards} \\
                    &&\\
    \fun{getDelegations} & \in & \ledgerState \to \Hash \mapsto \Hash \\
    \fun{getDelegations} & \coloneqq & (\wcard, (\wcard, \wcard,
                                       \var{delegations}), \wcard) \to
                                       \var{delegations} \\
                    &&\\
    \fun{getStPools} & \in & \ledgerState \to \Hash \mapsto \DCertRegPool \\
    \fun{getStPools} & \coloneqq & (\wcard, \wcard,
                                   (\var{stpools}, \wcard)) \to \var{stpools} \\
                    &&\\
    \fun{getRetiring} & \in & \ledgerState \to \Hash \mapsto \Epoch \\
    \fun{getRetiring} & \coloneqq & (\wcard, \wcard,
                                    (\wcard, \var{retiring})) \to \var{retiring} \\
  \end{align*}
  \caption{Definitions and Functions for Stake Delegation in Ledger States}
  \label{fig:stake-delegation-functions}
\end{figure}


\begin{property}[\textbf{Registered Staking Key with Zero Rewards}]
  \begin{multline*}
    \forall \var{l}, \var{l'} \in \ledgerState: \applyFun{validLedgerstate}{l}\\
    \implies \forall \var{c} \in \DCert, \var{l} \trans{delegw}{c} \var{l'}
    \wedge \RegKey{c} \implies \applyFun{author}{c} = \var{hk}\\ \implies
    \var{hk} \in \applyFun{getStKeys}{l'} \wedge
    (\applyFun{getRewards}var{rewards})[hk] = 0
  \end{multline*}
  \label{prop:ledger-properties-6}
\end{property}

Property~\ref{prop:ledger-properties-6} states that for each valid ledger state
$l$, if a delegation transaction of type $\DCertRegKey$ is executed, then in the
resulting ledger state $l'$, the set of staking keys of $l'$ includes the key
$hk$ associated with the key registration certificate and the associated reward
is set to 0 in $l'$.

\begin{property}[\textbf{Deregistered Staking Key}]
  \begin{multline*}
    \forall \var{l}, \var{l'} \in \ledgerState: \applyFun{validLedgerstate}{l}\\
    \implies \forall \var{c} \in \DCert, \var{l} \trans{delegw}{c} \var{l'}
    \wedge \DeregKey{c} \implies \applyFun{author}{c} = \var{hk}\\ \implies
    \var{hk} \not\in \applyFun{getStKeys}{l'} \wedge
    (\fun{dom}(\applyFun{getRewards}{l'}) \cup
    \fun{dom}(\applyFun{getDelegations}{l'})) \cap \{hk\} = \emptyset
  \end{multline*}
  \label{prop:ledger-properties-7}
\end{property}

Property~\ref{prop:ledger-properties-7} states that for $l, l'$ as above but
with a delegation transition of type $\DCertDeRegKey$, the staking key $hk$
associated with the deregistration certificate is not in the set of staking keys
of $l'$ and is not in the domain of neither the rewards nor the delegation map
of $l'$.

\begin{property}[\textbf{Delegated Stake}]
  \begin{multline*}
    \forall \var{l}, \var{l'} \in \ledgerState: \applyFun{validLedgerstate}{l}\\
    \implies \forall \var{c} \in \DCert, \var{l} \trans{delegw}{c} \var{l'}
    \wedge \Delegate{c} \implies \applyFun{author}{c} = \var{hk}\\ \implies
    \var{hk} \in \applyFun{getStKeys}{l} \wedge
    (\applyFun{getDelegations}{l'})[hk] = \applyFun{pool}{c}
  \end{multline*}
  \label{prop:ledger-properties-8}
\end{property}

Property~\ref{prop:ledger-properties-8} states that for $l, l'$ as above but
with a delegation transition of type $\DCertDeleg$, the staking key $hk$
associated with the deregistration certificate is in the set of staking keys of
$l$ and delegates to the staking pool associated with the delegation
certificate in $l'$.

\subsection{Ledger State Properties for Staking Pool Transitions}
\label{sec:ledg-state-prop}

\begin{property}[\textbf{Registered Staking Pool}]
  \begin{multline*}
    \forall \var{l}, \var{l'} \in \ledgerState: \applyFun{validLedgerstate}{l}\\
    \implies \forall \var{c} \in \DCertRegPool, \var{l} \trans{pool}{c} \var{l'}
    \wedge \RegPool{c} \implies \applyFun{author}{c} = \var{hk}\\ \implies
    (\applyFun{getStPools}{l'})[\var{hk}] = c \wedge \var{hk} \not\in
    \applyFun{getRetiring}{l'}
  \end{multline*}
  \label{prop:ledger-properties-9}
\end{property}

Property~\ref{prop:ledger-properties-9} states that for $l, l'$ as above but
with a delegation transition of type $\DCertRegPool$, the key $hk$ is associated
with the author of the pool registration certificate in $\var{stpools}$ of $l'$
and that $hk$ is not in the set of retiring stake pools in $l'$.

\begin{property}[\textbf{Start Staking Pool Retirement}]
  \begin{multline*}
    \forall \var{l}, \var{l'} \in \ledgerState, \var{cepoch} \in \Epoch:
    \applyFun{validLedgerstate}{l} \\
    \implies \forall \var{c} \in \DCertRegPool, \var{l} \trans{pool}{c} \var{l'}
    \wedge \RetirePool{c} \\ \implies e = \applyFun{retire}{c} \wedge
    \var{cepoch} < e < \var{cepoch} + \emax \wedge \applyFun{author}{c} =
    \var{hk}\\ \implies (\applyFun{getRetiring}{l'})[\var{hk}] = e \wedge
    \var{hk} \in \fun{dom}(\applyFun{getStPools}{l})
  \end{multline*}
  \label{prop:ledger-properties-10}
\end{property}

Property~\ref{prop:ledger-properties-10} states that for $l, l'$ as above but
with a delegation transition of type $\DCertRetirePool$, the key $hk$ is
associated with the author of the pool registration certificate in
$\var{stpools}$ of $l'$ and that $hk$ is not in the set of retiring stake pools
in $l'$.

\begin{property}[\textbf{Stake Pool Reaping}]
  \begin{multline*}
    \forall \var{l}, \var{l'} \in \ledgerState, \var{cepoch} \in \Epoch:
    \applyFun{validLedgerstate}{l} \\
    \implies \var{l} \trans{poolreap}{} \var{l'} \implies \forall \var{retire} =
    retiring^{-1} cepoch, retired \neq \emptyset \\ \wedge \var{retire}
    \subseteq \fun{dom}(\applyFun{getStPool}{l}) \wedge
    \var{retire} \cap \fun{dom}(\applyFun{getStPool}{l'}) = \emptyset \\
    \wedge \var{retire} \subseteq \fun{dom}(\applyFun{getRetiring}{l}) \wedge
    \var{retire} \cap \fun{dom}(\applyFun{getRetiring}{l'}) = \emptyset
  \end{multline*}
  \label{prop:ledger-properties-11}
\end{property}

Property~\ref{prop:ledger-properties-11} states that for $l, l'$ as above but
with a delegation transition of type $\DPoolReap$, there exist registered stake
pools in $l$ which are associated to stake pool registration certificates and
which are to be retired at the current epoch $\var{cepoch}$. In $l'$ all those
stake pools are removed from the maps $stpools$ and $retiring$.

%%% Local Variables:
%%% mode: latex
%%% TeX-master: "ledger-spec"
%%% End:
